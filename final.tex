% The first command in your LaTeX source must be the \documentclass command.

\documentclass[sigconf]{acmart}
 % Do not change for CCS'19

\settopmatter{printacmref=true}
  % mandatory for CCS'19

\usepackage{balance}
  % for creating a balanced last page (usually last page with references)

% header.tex
%
% Formatting and common macros for crypto papers. Include this first.
\usepackage{graphics}
\usepackage[font={small}]{caption}
\usepackage{hyperref}
\usepackage{xspace}
\iffull
\usepackage{amsthm}
\newtheorem{lemma}{Lemma}
\newtheorem{theorem}{Theorem}
\newtheorem{corollary}{Corollary}
\newtheorem{definition}{Definition}
\theoremstyle{remark}
%\newtheorem*{remark}{Remark}
\newtheorem{remark}{Remark}
\newcommand{\missingqed}{}
\else
\documentclass[runningheads]{llncs}
\newcommand{\missingqed}{\hfill\qed}
\fi
\usepackage{amsmath}
\usepackage{pifont}
\usepackage{amsfonts}
\usepackage{parskip}
\usepackage{multirow}
\usepackage{array}
%\usepackage{framed}

\hypersetup{
    colorlinks,%
    citecolor=black,%
    filecolor=black,%
    linkcolor=black,%
    urlcolor=black
}

\def\dashuline{\bgroup
  \ifdim\ULdepth=\maxdimen  % Set depth based on font, if not set already
    \settodepth\ULdepth{(j}\advance\ULdepth.4pt\fi
  \markoverwith{\kern.15em
  \vtop{\kern\ULdepth \hrule width .3em}%
  \kern.15em}\ULon}

\newcounter{foot}
\setcounter{foot}{1}
\setlength\parindent{2em}

% Editorial
\renewcommand{\paragraph}[1]{\noindent\textsc{#1}}
\newcommand{\heading}[1]{\paragraph{#1}}
\newcommand{\ala}{\textit{a la}\xspace}
\newcommand{\etal}{\textit{et al.}\xspace}
\newcommand{\viceversa}{\textit{vice versa}\xspace}

% Fonts for various types
\newcommand{\notionfont}[1]{\textnormal{#1}\xspace}
% FIXME(cjpatton) I added the "\xspace" because it's supposed to add a space
% after the macro in text mode. But this doesn't seem to be working!
\newcommand{\varfont}[1]{\textit{#1}}
\newcommand{\flagfont}[1]{\mathsf{#1}}
\newcommand{\vectorfont}[1]{\vec{#1}}
\newcommand{\oraclefont}[1]{\cryptofont{#1}}
\newcommand{\schemefont}[1]{\textsc{#1}}
\newcommand{\procfont}[1]{\mathsf{#1}}
\newcommand{\algorithmfont}[1]{\mathcal{#1}}
\newcommand{\adversaryfont}[1]{\mathit{#1}}
\newcommand{\setfont}[1]{\mathcal{#1}}
\newcommand{\cryptofont}[1]{\mathbf{#1}\hspace{0.5pt}}
\newcommand{\capgreekfont}[1]{\mathrm{#1}}

% Crypto functions
\newcommand{\Exp}[1]{\cryptofont{Exp}^{{\tiny \MakeLowercase{#1}}}}
\newcommand{\Adv}[1]{\cryptofont{Adv}^{{\tiny \MakeLowercase{#1}}}}

% Math
\DeclareMathAlphabet\mathbfcal{OMS}{cmsy}{b}{n}
\newcommand{\dqed}{\hfill$\Diamond$}
% FIXME What's the deal with this command and nested parans? This and also
% substr
\def\ceil(#1){\lceil #1 \rceil}
\def\floor(#1){\lfloor #1 \rfloor}
\newcommand{\goesto}{{\rightarrow}}

% - Sets
\newcommand{\setify}[1]{\procfont{set}\left(#1\right)}
\newcommand{\setlen}[1]{|#1|}
\newcommand{\multisetlen}[1]{\|#1\|}
\newcommand{\Z}{\mathbb{Z}}
\newcommand{\N}{\mathbb{N}}
\newcommand{\R}{\mathbb{R}}
\newcommand{\bits}{\{0,1\}}
\newcommand*\bigunion{\bigcup}
\newcommand*\bigintersection{\bigcap}
\newcommand*\union{\cup}
\newcommand{\multiunion}{\uplus}
\newcommand*\intersection{\cap}
\newcommand*\cross{\times}
\newcommand*\by{\cross}
\newcommand{\getsr}{\mathrel{\leftarrow\mkern-14mu\leftarrow}}
%\newcommand{\getsr}{\xleftarrow{\text{\tiny{\$}}}}
%\newcommand{\getsr}{{\:{\leftarrow{\hspace*{-3pt}\raisebox{.75pt}{$\scriptscriptstyle\$$}}}\:}}
\newcommand{\setop}[1]{\mathsf{set}(#1)} %^ \procfont
\def\str(#1){\procfont{set}\left(#1\right)}
\def\bydef{\stackrel{\rm def}{=}}

\newcommand{\undef}{\mathtt{undefined}}

% - String operations
\newcommand{\emptystr}{\varepsilon}
\newcommand{\cat}{\, \| \,}
\def\str(#1){\langle #1 \rangle}
\def\substr(#1,#2,#3){#1[#2\mbox{\,:\,}#3]}
\def\toint{\procfont{int}}
\def\tostr{\procfont{str}}
\def\byte(#1){[#1]}

% - Boolean operators
\newcommand*\AND{\wedge}
\newcommand*\OR{\vee}
\newcommand*\NOT{\neg}
\newcommand*\IMPLIES{\implies}
\newcommand*\XOR{\mathbin{\oplus}}
\newcommand*\xor{\XOR}

% - Asymptotics
\newcommand{\negl}{\procfont{negl}}
\newcommand{\poly}{\procfont{poly}}

% - Probablity
\newcommand{\E}{\mathrm{E}}
\newcommand{\Prob}[1]{\Pr\hspace{-1pt}\left[\,#1\,\right]}
\newcommand{\given}{\mid}

% Games
\newcommand{\halt}{\bot}
\newcommand{\game}{\cryptofont{G}}
\newcommand{\G}{\game}
\newcommand{\foreach}[3]{$\text{for }#1 \gets #2\text{ to }#3\text{ do}$}
\newcommand{\tab}{\hspace*{10pt}}
\newcommand{\outputs}{=}
\newcommand{\sets}{\,\mathrm{sets}\,}
\newcommand{\bad}{\flagfont{bad}}
\newcommand{\true}{1}
\newcommand{\false}{0}
\newcommand{\invalid}{\bot}
\newcommand{\exception}{\invalid}
\newcommand{\experimentv}[1]{\underline{#1}}
\newcommand{\oraclev}[1]{\underline{{oracle} #1}:}
\newcommand{\adversaryv}[1]{\underline{{adv.} #1}:}
\newcommand{\algorithmv}[1]{\underline{{alg.} #1}:}

% - Inline comment
\definecolor{CommentColor}{RGB}{125,175,230}
\newcommand{\comment}[1]{\textcolor{CommentColor}{\,\textbf{\#}\,#1}}

\newcommand{\gamesfontsize}{\footnotesize}
\newcommand{\gamespadleft}{\hskip 1pt}
\newcommand{\gamespad}{\hskip 4pt}

% - One game
\newcommand{\oneCol}[2]{
  \makebox[\textwidth][c]{
    \begin{tabular}{|@{\gamespadleft}l@{\gamespad}|}
    \hline
    \rule{0pt}{1\normalbaselineskip}
    \begin{minipage}[t]{#1\textwidth}\gamesfontsize
      #2 \vspace{6pt}
    \end{minipage} \\
    \hline
  \end{tabular}
  }
}

\newcommand{\twoCols}[3]{
  \makebox[\textwidth][c]{
    \begin{tabular}{|@{\gamespadleft}l@{\gamespad}|@{}@{\gamespad}l@{\gamespad}|}
    \hline
    \rule{0pt}{1\normalbaselineskip}
    \begin{minipage}[t]{#1\textwidth}\gamesfontsize
      #2 \vspace{6pt}
    \end{minipage} &
    \begin{minipage}[t]{#1\textwidth}\gamesfontsize
      #3 \vspace{6pt}
    \end{minipage} \\
    \hline
  \end{tabular}
  }
}

\newcommand{\twoColsUnbalanced}[4]{
  \makebox[\textwidth][c]{
    \begin{tabular}{|@{\gamespadleft}l@{\gamespad}|@{}@{\gamespad}l@{\gamespad}|}
    \hline
    \rule{0pt}{1\normalbaselineskip}
    \begin{minipage}[t]{#1\textwidth}\gamesfontsize
      #3 \vspace{6pt}
    \end{minipage} &
    \begin{minipage}[t]{#2\textwidth}\gamesfontsize
      #4 \vspace{6pt}
    \end{minipage} \\
    \hline
  \end{tabular}
  }
}

\newcommand{\twoColsNoDivide}[3]{
  \makebox[\textwidth][c]{
    \begin{tabular}{|@{\gamespadleft}l@{\gamespad}@{}@{\gamespad}l@{\gamespad}|}
    \hline
    \rule{0pt}{1\normalbaselineskip}
    \begin{minipage}[t]{#1\textwidth}\gamesfontsize
      #2 \vspace{6pt}
    \end{minipage} &
    \begin{minipage}[t]{#1\textwidth}\gamesfontsize
      #3 \vspace{6pt}
    \end{minipage} \\
    \hline
  \end{tabular}
  }
}

\newcommand{\twoColsTwoRows}[5]{
  \makebox[\textwidth][c]{
  \begin{tabular}{|@{\gamespadleft}l@{\gamespad}|@{}@{\gamespad}l@{\gamespad}|}
    \hline
    \rule{0pt}{1\normalbaselineskip}
    \begin{minipage}[t]{#1\textwidth}\gamesfontsize
      #2 \vspace{6pt}
    \end{minipage} &
    \begin{minipage}[t]{#1\textwidth}\gamesfontsize
      #3 \vspace{6pt}
    \end{minipage} \\
    \hline
    \rule{0pt}{1\normalbaselineskip}
    \begin{minipage}[t]{#1\textwidth}\gamesfontsize
      #4 \vspace{6pt}
    \end{minipage} &
    \begin{minipage}[t]{#1\textwidth}\gamesfontsize
      #5 \vspace{6pt}
    \end{minipage} \\
    \hline
  \end{tabular}
  }
}

\newcommand{\threeCols}[4]{
  \makebox[\textwidth][c]{
    \begin{tabular}{|@{\gamespadleft}l@{\gamespad}|@{}@{\gamespad}l@{\gamespad}|@{}@{\gamespad}l@{\gamespad}|}
    \hline
    \rule{0pt}{1\normalbaselineskip}
    \begin{minipage}[t]{#1\textwidth}\gamesfontsize
      #2 \vspace{6pt}
    \end{minipage} &
    \begin{minipage}[t]{#1\textwidth}\gamesfontsize
      #3 \vspace{6pt}
    \end{minipage} &
    \begin{minipage}[t]{#1\textwidth}\gamesfontsize
      #4 \vspace{6pt}
    \end{minipage} \\
    \hline
  \end{tabular}
  }
}

\newcommand{\threeColsOneDivide}[5]{
  \makebox[\textwidth][c]{
    \begin{tabular}{|@{\gamespadleft}l@{\gamespad}|@{}@{\gamespad}l@{\gamespad}@{}@{\gamespad}l@{\gamespad}|}
    \hline
    \rule{0pt}{1\normalbaselineskip}
    \begin{minipage}[t]{#1\textwidth}\gamesfontsize
      #3 \vspace{6pt}
    \end{minipage} &
    \begin{minipage}[t]{#2\textwidth}\gamesfontsize
      #4 \vspace{6pt}
    \end{minipage} &
    \begin{minipage}[t]{#2\textwidth}\gamesfontsize
      #5\vspace{6pt}
    \end{minipage} \\
    \hline
  \end{tabular}
  }
}

\newcommand{\fourColsOneDivide}[8]{
  \makebox[\textwidth][c]{
    \begin{tabular}{|@{\gamespadleft}l@{\gamespad}|@{}@{\gamespad}l@{\gamespad}@{}@{\gamespad}l@{\gamespad}@{}@{\gamespad}l@{\gamespad}|}
    \hline
    \rule{0pt}{1\normalbaselineskip}
    \begin{minipage}[t]{#1\textwidth}\gamesfontsize
      #5 \vspace{6pt}
    \end{minipage} &
    \begin{minipage}[t]{#2\textwidth}\gamesfontsize
      #6 \vspace{6pt}
    \end{minipage} &
    \begin{minipage}[t]{#3\textwidth}\gamesfontsize
      #7\vspace{6pt}
    \end{minipage} &
    \begin{minipage}[t]{#4\textwidth}\gamesfontsize
      #8\vspace{6pt}
    \end{minipage}\\
    \hline
  \end{tabular}
  }
}

\newcommand{\boxThmBFSaltCorrect}[5]{
  \makebox[\textwidth][c]{
  \begin{tabular}{|@{\gamespadleft}l@{}@{}@{\gamespad}l|}
    \hline
    \rule{0pt}{1\normalbaselineskip}
    \begin{minipage}[t]{#1\textwidth}\gamesfontsize
      #2 \vspace{6pt}
    \end{minipage} \vline &
    \begin{minipage}[t]{#1\textwidth}\gamesfontsize
      #3 \vspace{6pt}
    \end{minipage} \\
    \hline
    \rule{0pt}{1\normalbaselineskip}
    \begin{minipage}[t]{#1\textwidth}\gamesfontsize
      #4 \vspace{6pt}
    \end{minipage} &
    \begin{minipage}[t]{#1\textwidth}\gamesfontsize
      #5 \vspace{6pt}
    \end{minipage} \\
    \hline
  \end{tabular}
  }
}

% Notes
\newcounter{notectr}[section]
\newcommand{\getnotectr}{\stepcounter{notectr}\thesection.\thenotectr}
\newcommand{\basenote}[4]{{
  \textrm{\textcolor{#1}{(\getnotectr: #2 #3: #4)}}
}}

% Uncomment to mute notes.
%\renewcommand{\basenote}[4]{\ignorespaces}

\iffull
\newcommand{\note}[3]{\basenote{#1}{#2}{says}{#3}}
\else
\renewcommand{\note}[3]{\basenote{#1}{#2}{says}{#3}}
\fi

\newcommand{\todo}[2]{\basenote{red}{#1}{to-do}{#2}}
\newcommand{\tsnote}[1]{\note{cyan}{TS}{#1}}
\newcommand{\jnote}[1]{\note{green}{Jon}{#1}}
\definecolor{darkgreen}{RGB}{50,127,0}
\newcommand{\cpnote}[1]{\note{darkgreen}{Chris}{#1}}
\newcommand{\dctodo}[1]{\todo{David}{#1}}
\newcommand{\ignore}[1]{\if{0} #1 \fi}
\newcommand{\oldstuff}[1]{\textcolor{gray}{#1}}

\newcommand{\Func}{\mathrm{Func}}
\newcommand{\cmark}{\ding{51}}
\newcommand{\xmark}{\ding{55}}
% Override the CCS mathcal font.
\DeclareMathAlphabet{\mathcal}{OMS}{cmsy}{m}{n}

\newcommand{\proc}[1]{}
\newcommand{\full}[1]{#1}

% defining the \BibTeX command - from Oren Patashnik's original BibTeX documentation.
\def\BibTeX{{\rm B\kern-.05em{\sc i\kern-.025em b}\kern-.08emT\kern-.1667em\lower.7ex\hbox{E}\kern-.125emX}}
    
% Rights management information. 
% This information is sent to you when you complete the rights form.
% These commands have SAMPLE values in them; it is your responsibility as an author to replace
% the commands and values with those provided to you when you complete the rights form.
%
% These commands are for a PROCEEDINGS abstract or paper.

\copyrightyear{2019}
\acmYear{2019}
\acmConference[CCS '19]{2019 ACM SIGSAC Conference on Computer and Communications Security}{November 11--15, 2019}{London, United Kingdom}
\acmBooktitle{2019 ACM SIGSAC Conference on Computer and Communications Security (CCS '19), November 11--15, 2019, London, United Kingdom}
\acmPrice{15.00}
\acmDOI{10.1145/3319535.3354235}
\acmISBN{978-1-4503-6747-9/19/11}


% Submission ID. 
% Use this when submitting an article to a sponsored event. You'll receive a unique submission ID from the organizers
% of the event, and this ID should be used as the parameter to this command.
%\acmSubmissionID{123-A56-BU3}


% end of the preamble, start of the body of the document source.

\begin{document}

\fancyhead{}
  % do not delete this code.

% The "title" command has an optional parameter, allowing the author to define a "short title" to be used in page headers.
\title{Probabilistic Data Structures in Adversarial Environments}


% The "author" command and its associated commands are used to define the authors and their affiliations.
% Of note is the shared affiliation of the first two authors, and the "authornote" and "authornotemark" commands
% used to denote shared contribution to the research.
\author{David Clayton}
\affiliation{%
  \institution{University of Florida}
  \city{Gainesville}
  \state{Florida}}
\email{davidclayton@ufl.edu}

\author{Christopher Patton}
\affiliation{%
  \institution{University of Florida}
  \city{Gainesville}
  \state{Florida}
}
\email{cjpatton@ufl.edu}

\author{Thomas Shrimpton}
\affiliation{%
 \institution{University of Florida}
 \city{Gainesville}
 \state{Florida}
 }
\email{teshrim@ufl.edu}

%
% By default, the full list of authors will be used in the page headers. Often, this list is too long, and will overlap
% other information printed in the page headers. This command allows the author to define a more concise list
% of authors' names for this purpose.
\renewcommand{\shortauthors}{Trovato and Tobin, et al.}

%
% The abstract is a short summary of the work to be presented in the article.
\begin{abstract}
Probabilistic data structures use space-efficient representations of data in order to
(approximately) respond to queries about the data. 
Traditionally, these structures are accompanied by probabilistic bounds on
query-response errors. These bounds implicitly assume benign attack
models, in which the data and the queries are inputs are chosen
non-adaptively, and independent of the randomness used to
construct the representation. Yet probabilistic data structures are
increasingly used in settings where these assumptions may be violated.

This work provides a provable security treatment of probabilistic data
structures in adversarial environments. We give a syntax that captures a wide
variety of in-use structures, and our security notions support
development of error bounds in the presence of powerful attacks.
%
Concretely, we examine the widely used Bloom filter, counting (Bloom)
filter, and count-min sketch data structures.  For the traditional
version of these, our security findings are largely negative; however,
we show that simple embellishments (e.g., using salts, or secret keys)
yields structures that provide provable security, and with little overhead.


\ignore{
  This work initiates the study of abstract data structures from a cryptographic
  perspective.  We first establish a precise syntax that captures a broad
  class of real-world data structures.  We then treat the
  \emph{correctness} and \emph{privacy} of data structures
  as security properties, and establish formal security notions for
  each.  Loosely, our notion of correctness captures an (adaptive) adversary's ability to cause
  a data structure to err in the course of responding to a set of supported
  queries, and our two privacy notions neatly capture what a data structure leaks about
  the data it represents.

  We use our formalisms to explore the security of the widely used
  Bloom filter~\cite{bloom1970space} and some important variants.
  %
  We find, for example, that the security of Bloom filters depends
  crucially on whether or not the underlying hash functions are known
  by the adversary prior to the filter being constructed.
  %
  We also study a real-world mechanism for privacy-preserving record
  linkage (over hospital databases).  Our notions provide a crisp view of the
  (in)security of this Bloom-filter-based mechanism.
  %
  To demonstrate the broader applicability of our definitions, we move
  from data structures supporting set-membership queries (only) to
  dictionary data structures.  Concretely, we analyze the ``Bloomier
  filter''~\cite{chazelle2004bloomier}, which provides a compact
  representation of a key/value store.
}

\end{abstract}

%
% The code below is generated by the tool at http://dl.acm.org/ccs.cfm.
% Please copy and paste the code instead of the example below.
%
\begin{CCSXML}
<ccs2012>
<concept>
<concept_id>10002978.10002979.10002982</concept_id>
<concept_desc>Security and privacy~Symmetric cryptography and hash functions</concept_desc>
<concept_significance>300</concept_significance>
</concept>
</ccs2012>
\end{CCSXML}

\ccsdesc[300]{Security and privacy~Symmetric cryptography and hash functions}

%
% Keywords. The author(s) should pick words that accurately describe the work being
% presented. Separate the keywords with commas.
\keywords{Data structures, Bloom filters, Count min-sketches, counting Bloom filters}

%
% A "teaser" image appears between the author and affiliation information and the body 
% of the document, and typically spans the page. 
%\begin{teaserfigure}
%  \includegraphics[width=\textwidth]{sampleteaser}
%  \caption{Seattle Mariners at Spring Training, 2010.}
%  \Description{Enjoying the baseball game from the third-base seats. Ichiro Suzuki preparing to bat.}
%  \label{fig:teaser}
%\end{teaserfigure}

%
% This command processes the author and affiliation and title information and builds
% the first part of the formatted document.
\maketitle

%\newif\iffull
%\fulltrue

%\iffull
%\documentclass{article}
%\else
%\documentclass{llncs}
%\fi

%\usepackage{epsf}

%% macros.tex
%
% Macros for this paper. (Include build/headers.tex, then this.)
%\usepackage{times}
%\usepackage[dvipdf]{graphicx}
\usepackage{pifont} % \ding{109}
\usepackage{afterpage} % \afterpage{ ... }

% Query spaces
\newcommand{\qrysp}{\capgreekfont{\Gamma}}
\newcommand{\univ}{\setfont{U}}
\newcommand{\results}{\setfont{R}}
\newcommand{\queries}{\setfont{Q}}

% Data structures
\newcommand{\struct}{\capgreekfont{\Pi}}
\newcommand{\Gen}{\schemefont{Gen}}
\newcommand{\Rep}{\schemefont{Rep}}
\newcommand{\Qry}{\schemefont{Qry}}
\newcommand{\ky}{K}
\newcommand{\pub}{\procfont{repr}} % TODO Change to \rep?
\newcommand{\qry}{\procfont{qry}}
\newcommand{\res}{a}
\newcommand{\keys}{\setfont{K}}
\newcommand{\col}{\setfont{S}}
\newcommand{\elts}{\setfont{X}} % TODO What does "elts" mean?



% Constructions
\newcommand{\BF}{\schemefont{BF}}
\newcommand{\SBF}{\schemefont{SBF}}
\newcommand{\SKBF}{\schemefont{KBF}}
\newcommand{\PRLBF}{\schemefont{PPRL}}
\newcommand{\DICT}{\schemefont{DICT}}
\newcommand{\bi}{\mathrm{bi}}
\newcommand{\bloom}{\mathrm{bf}}
\newcommand{\saltybloom}{\mathrm{sbf}}
\newcommand{\prfbloom}{\mathrm{kbf}}
\newcommand{\dict}{\mathrm{dict}}
\newcommand{\hash}{\schemefont{Hash}}
\newcommand{\hashbf}{\schemefont{2Hash}}
\newcommand{\hashlin}{\schemefont{K2Hash}}
\newcommand{\hashprf}{\schemefont{stupid}} % FIXME remove

% Adversaries
\newcommand{\advA}{\adversaryfont{A}}
\newcommand{\advB}{\adversaryfont{B}}
\newcommand{\advD}{\adversaryfont{D}}
\newcommand{\dist}{\advD}

% Oracles
\newcommand{\OO}{\oraclefont{O}}
\newcommand{\REPO}{\oraclefont{Rep}}
\newcommand{\QRYO}{\oraclefont{Qry}}
\newcommand{\PRFO}{\oraclefont{F}}
\newcommand{\HASHO}{\oraclefont{Hash}}

% Notions
\newcommand{\prf}{\notionfont{PRF}}
\newcommand{\errep}{\notionfont{ER\mbox{-}REP}}
\newcommand{\ssrep}{\notionfont{SS\mbox{-}REP}}
\def\ssrepX#1{\mbox{\ssrep-#1}}
\newcommand{\owrep}{\notionfont{OW\mbox{-}REP}}
\newcommand{\errepone}{\notionfont{ER\mbox{-}REP1}}

% Sets
\newcommand{\setA}{\setfont{A}}
\newcommand{\setB}{\setfont{B}}
\newcommand{\setC}{\setfont{C}}
\newcommand{\setI}{\setfont{I}}
\newcommand{\setK}{\setfont{K}}
\newcommand{\setM}{\setfont{M}}
\newcommand{\setN}{\setfont{N}}
\newcommand{\setP}{\setfont{P}}
\newcommand{\setQ}{\setfont{Q}}
\newcommand{\setR}{\setfont{R}}
\newcommand{\setS}{\setfont{S}}
\newcommand{\setT}{\setfont{T}}
\newcommand{\setV}{\setfont{V}}
\newcommand{\setX}{\setfont{X}}
\newcommand{\setW}{\setfont{W}}
\newcommand{\setZ}{\setfont{Z}}

% Variables
\newcommand{\ct}{\varfont{ct}}
\newcommand{\err}{\varfont{err}}
\newcommand{\Ans}{\varfont{Ans}}
\newcommand{\coins}{r}
\newcommand{\undef}{\texttt{undefined}}

% Vectors
\newcommand{\vv}{\vectorfont{v}}
\newcommand{\hh}{\vectorfont{h}}

% Misc.
\def\ticks(#1,#2){\procfont{T}_{\hspace*{-1.5pt}#1}({#2})}
\newcommand{\leak}{\procfont{lk}}
\newcommand{\Sim}{\procfont{Sim}}
\newcommand{\salt}{Z}
\newcommand{\bigram}{\procfont{bigram}}
\newcommand{\tableM}{\vectorfont{M}}
\newcommand{\tableT}{\vectorfont{T}}

% Authors' comments.
\definecolor{darkgreen}{RGB}{50,127,0}
\newcommand{\cpnote}[1]{\note{darkgreen}{Chris}{#1}}
\newcommand{\cptodo}[1]{\todo{red}{Chris}{#1}}
\newcommand{\tsnote}[1]{\note{blue}{Tom}{#1}}
\newcommand{\jnote}[1]{\note{cyan}{Jon}{#1}}
\newcommand{\review}[2]{\note{red}{Reviewer #1}{#2}}
\newcommand{\anytodo}[1]{\todo{red}{\ignorespaces}{#1}}

% Boxes
\newcommand{\boxPrivacyNotions}[6]{
  \makebox[\textwidth][c]{
  \begin{tabular}{|@{\gamespadleft}l@{}@{\gamespad}l@{}|@{\gamespad}l|}
    \hline
    \rule{0pt}{1\normalbaselineskip}
    \begin{minipage}[t]{#1\textwidth}\gamesfontsize
      #4 \vspace{6pt}
    \end{minipage} &
    \begin{minipage}[t]{#2\textwidth}\gamesfontsize
      #5 \vspace{6pt}
    \end{minipage} &
    \begin{minipage}[t]{#3\textwidth}\gamesfontsize
      #6 \vspace{6pt}
    \end{minipage} \\
    \hline
  \end{tabular}
  }
}

\newcommand{\boxThmBFSaltCorrect}[5]{
  \makebox[\textwidth][c]{
  \begin{tabular}{|@{\gamespadleft}l@{}@{}@{\gamespad}l|}
    \hline
    \rule{0pt}{1\normalbaselineskip}
    \begin{minipage}[t]{#1\textwidth}\gamesfontsize
      #2 \vspace{6pt}
    \end{minipage} \vline &
    \begin{minipage}[t]{#1\textwidth}\gamesfontsize
      #3 \vspace{6pt}
    \end{minipage} \\
    \hline
    \rule{0pt}{1\normalbaselineskip}
    \begin{minipage}[t]{#1\textwidth}\gamesfontsize
      #4 \vspace{6pt}
    \end{minipage} &
    \begin{minipage}[t]{#1\textwidth}\gamesfontsize
      #5 \vspace{6pt}
    \end{minipage} \\
    \hline
  \end{tabular}
  }
}

\newcommand{\boxThmBFPRFCorrect}[7]{
  \makebox[\textwidth][c]{
  \begin{tabular}{|@{\gamespadleft}l@{}@{}@{\gamespad}l|}
    \hline
    \rule{0pt}{1\normalbaselineskip}
    \begin{minipage}[t]{#1\textwidth}\gamesfontsize
      #2 \vspace{6pt}
    \end{minipage} \vline &
    \begin{minipage}[t]{#1\textwidth}\gamesfontsize
      #3 \vspace{6pt}
    \end{minipage} \\
    \hline
    \rule{0pt}{1\normalbaselineskip}
    \begin{minipage}[t]{#1\textwidth}\gamesfontsize
      #4 \vspace{6pt}
    \end{minipage} &
    \begin{minipage}[t]{#1\textwidth}\gamesfontsize
      #5 \vspace{6pt}
    \end{minipage} \\
    \hline
    \rule{0pt}{1\normalbaselineskip}
    \begin{minipage}[t]{#1\textwidth}\gamesfontsize
      #6 \vspace{6pt}
    \end{minipage} \vline &
    \begin{minipage}[t]{#1\textwidth}\gamesfontsize
      #7 \vspace{6pt}
    \end{minipage} \\
    \hline
  \end{tabular}
  }
}

\definecolor{lightgreen}{RGB}{200,255,200}
\definecolor{lightred}{RGB}{255,215,215}
\definecolor{lightgray}{RGB}{230,230,230}
\newcommand{\diff}[1]{\colorbox{grey}{\parbox{\dimexpr\linewidth-2\fboxsep-2\fboxrule\relax}{#1}}}
%\newcommand{\diffplus}[1]{\colorbox{lightgreen}{#1}}
%\newcommand{\diffplusbox}[1]{\colorbox{lightgreen}{\parbox{\dimexpr\textwidth-2\fboxsep-2\fboxrule\relax}{#1}}}
%\newcommand{\diffminus}[1]{\colorbox{lightred}{#1}}
%\newcommand{\diffminusbox}[1]{\colorbox{lightred}{\parbox{\dimexpr\textwidth-2\fboxsep-2\fboxrule\relax}{#1}}}
\newcommand{\diffplus}[1]{\fbox{#1}}
\newcommand{\diffplusbox}[1]{\fbox{\parbox{\dimexpr\textwidth-2\fboxsep-2\fboxrule\relax}{#1}}}
\newcommand{\diffminus}[1]{\colorbox{lightgray}{#1}}
\newcommand{\diffminusbox}[1]{\colorbox{lightgray}{\parbox{\dimexpr\textwidth-2\fboxsep-2\fboxrule\relax}{#1}}}

\newcommand{\Func}{\mathrm{Func}}

% Notions
\newcommand{\errep}{\notionfont{ERR\mbox{-}UP}}
\newcommand{\erreps}{\notionfont{ERR\mbox{-}UP(S)}}
\newcommand{\indrep}{\notionfont{IND\mbox{-}UP}}
\newcommand{\indrepr}{\notionfont{IND\mbox{-}UPR}}
\newcommand{\prf}{\notionfont{PRF}}
\newcommand{\ssrep}{\notionfont{SS\mbox{-}REP}}
\def\ssrepX#1{\mbox{\ssrep-#1}}
\newcommand{\owrep}{\notionfont{OW\mbox{-}REP}}
\newcommand{\errepone}{\notionfont{ER\mbox{-}REP1}}
\def\indrepX#1{\indrep\mbox{-}#1}
\def\indreprX#1{\indrepr\mbox{-}#1}
\def\prfX#1{\prf\mbox{-}#1}

% Structures
\newcommand{\struct}{\capgreekfont{\Pi}}
\newcommand{\Init}{\schemefont{Init}}
\newcommand{\Up}{\schemefont{Up}}
\newcommand{\Qry}{\schemefont{Qry}}
\newcommand{\Rep}{\schemefont{Rep}}
\newcommand{\qry}{\procfont{qry}}
\newcommand{\lk}{\procfont{lk}}
\newcommand{\up}{\procfont{up}}
%\newcommand{\pub}{\procfont{pub}}
\newcommand{\pub}{\procfont{repr}}
\newcommand{\param}{\procfont{par}}
\newcommand{\key}{K}
\newcommand{\pp}{\mathit{pp}}
\renewcommand{\sp}{\mathit{sp}}

\newcommand{\ky}{K}
\newcommand{\res}{a}
\newcommand{\elts}{\setfont{X}} % TODO What does "elts" mean?

% Constructions
\newcommand{\BF}{\schemefont{BF}}
\newcommand{\SBF}{\schemefont{SBF}}
\newcommand{\SKBF}{\schemefont{KBF}}
\newcommand{\PRLBF}{\schemefont{PPRL}}
\newcommand{\DICT}{\schemefont{DICT}}
\newcommand{\bi}{\mathrm{bi}}
\newcommand{\bloom}{\mathrm{bf}}
\newcommand{\saltybloom}{\mathrm{sbf}}
\newcommand{\prfbloom}{\mathrm{kbf}}
\newcommand{\dict}{\mathrm{dict}}
%\newcommand{\hashprf}{\schemefont{stupid}} % FIXME remove

% Other schemes
\newcommand{\hashbf}{\schemefont{2Hash}}
\newcommand{\hash}{\schemefont{Hash}}
\newcommand{\hashlin}{\schemefont{2Hash}}
\newcommand{\tinyhash}{\schemefont{Tiny}}
\newcommand{\kbf}{\schemefont{KBF}}

% Sets
\newcommand{\univ}{\setfont{U}}
\newcommand{\queries}{\setfont{Q}}
\newcommand{\results}{\setfont{R}}
\newcommand{\mutants}{\setfont{M}}
\newcommand{\keys}{\setfont{K}}
\newcommand{\col}{\setfont{S}}

\newcommand{\setA}{\setfont{A}}
\newcommand{\setB}{\setfont{B}}
\newcommand{\setC}{\setfont{C}}
\newcommand{\setD}{\setfont{D}}
\newcommand{\setI}{\setfont{I}}
\newcommand{\setK}{\setfont{K}}
\newcommand{\setM}{\setfont{M}}
\newcommand{\setN}{\setfont{N}}
\newcommand{\setP}{\setfont{P}}
\newcommand{\setQ}{\setfont{Q}}
\newcommand{\setR}{\setfont{R}}
\newcommand{\setS}{\setfont{S}}
\newcommand{\setT}{\setfont{T}}
\newcommand{\setU}{\setfont{U}}
\newcommand{\setV}{\setfont{V}}
\newcommand{\setX}{\setfont{X}}
\newcommand{\setW}{\setfont{W}}
\newcommand{\setZ}{\setfont{Z}}
% Adversaries
\newcommand{\advA}{\adversaryfont{A}}
\newcommand{\advB}{\adversaryfont{B}}
\newcommand{\dist}{\adversaryfont{D}}

% Variables
\newcommand{\err}{\varfont{err}}
\newcommand{\ct}{\varfont{ct}}
\renewcommand{\st}{\varfont{st}}
\newcommand{\salt}{Z}

% Oracles
\newcommand{\REPO}{\oraclefont{Rep}}
\newcommand{\UPO}{\oraclefont{Up}}
\newcommand{\QRYO}{\oraclefont{Qry}}
\newcommand{\PRFO}{\oraclefont{F}}

% Vectors
\newcommand{\xx}{\vectorfont{x}}
\newcommand{\vv}{\vectorfont{v}}
\newcommand{\REVO}{\mathbf{Reveal}}
\newcommand{\HASHO}{\oraclefont{Hash}}
\newcommand{\INTO}{\oraclefont{Int}}
\newcommand{\diffplus}[1]{\fbox{#1}}
\newcommand{\diffplusbox}[1]{\fbox{\parbox{\dimexpr\textwidth-2\fboxsep-2\fboxrule\relax}{#1}}}
\newcommand{\diffminus}[1]{\colorbox{lightgray}{#1}}
\newcommand{\diffminusbox}[1]{\colorbox{lightgray}{\parbox{\dimexpr\textwidth-2\fboxsep-2\fboxrule\relax}{#1}}}
\newcommand{\hh}{\vectorfont{h}}
\newcommand{\fff}{\schemefont{Fn}}
\newcommand{\Rnd}{\schemefont{Rand}}
\newcommand{\Repx}{\Rep1}
\newcommand{\Qryx}{\Qry1}
\newcommand{\Upx}{\Up1}
\newcommand{\Ans}{\varfont{Ans}}
\newcommand{\setE}{\mathcal{E}}
\newcommand{\Resp}{\varfont{Resp}}
\def\ticks(#1,#2){\procfont{T}_{\hspace*{-1.5pt}#1}({#2})}
\newcommand{\highlighto}[1]{\colorbox{lightgray}{$\scriptstyle #1$}}
\newcommand{\highlight}[1]{\colorbox{lightgray}{$\displaystyle #1$}}

\newcommand{\emptystring}{\varepsilon}

\tsnote{Intro structure, thoughts...}
\begin{itemize}
\item Point to NY, quickly pivot to mutable DS
\item Our syntax and what it captures
\item Our correctness and what it captures (don't forget: immutability as an
  adversarial restriction, not a syntactic one) 
\item Loosely speaking, what the results say (see picture of
  chalkboard) --~rough ``structure'' of all results: non-adaptive
  term, plus ``(informed) guessing'' term)
\item Data structures we consider: BF, counting filter, cuckoo filter,
  CMS,... what about Bloomier filter results from old paper?
\item future directions
\end{itemize}

\section{Introduction}
Data structures are fundamental to essentially all areas of computer science.
The traditional approach to analyzing the correctness of a data structure is to
assume that all inputs, and all queries, are independent of any internal
randomness used to construct it.  But as highlighted by Naor and
Yogev (CRYPTO '15~\cite{naor2015bloom}), there are important use-cases in which the inputs
and queries may be chosen \emph{adversarially} and \emph{adaptively}, based on
partial information and prior observations about the data structure. Attacks of
this sort can be used to disrupt or reduce the availability of real systems
\cite{crosby2003denial,gerbet2015power,lipton1993clocked}.

Naor and Yogev (NY) formalized a notion of adversarial correctness for
Bloom-filter-like structures. Recall that a Bloom filter provides a compact
representation, $\pub$, of a set~$\col$. The representation is a length~$m$
bit-array (initally all zeros), and elements $x \in \col$ are added to it by
computing hash values $h_1(x),h_2(x),\ldots,h_k(x)\in [m]$, then setting the
indicated array positions to~$1$.  A Bloom filter supports set membership
queries, i.e., ``is $x\in\col$?'', by hashing~$x$ and responding positively
iff all of the indicated positions hold a 1-bit.  When $h_1,\ldots,h_k$ are
modeled as random functions, and~$\col$ is independent of these, classical
results relate $|\col|,m,k$ to the probability of false-positive query
responses~\cite{broder2004network,kirsch2006less}.
%
NY revisited these results from a security perspective, by
formalizing an attack model in which the adversary specifies a
(fixed) set~$\col$ that may
depend on the hash functions, and is then allowed to adaptively query the
(immutable) representation~$\pub$ in an effort to induce errors.

This work expands upon NY in several, practically relevant ways.  To begin, our
attack model allows the adversary to adaptively \emph{update} the
collection~$\col$, thereby capturing settings in which the target data may
change over time, e.g., streaming data applications.  Correspondingly, we
consider data structures that natively support \emph{mutable} representations.
Concrete examples of these include the counting filter~\cite{xxx}, count-min
sketch~\cite{xxx}, the cuckoo filter~\cite{xxx}, \todo{DC}{Add references and
add any other data structuers you feel should be named here.}
%
Next, while the Bloom filter was designed to represent
data collections~$\col$ that are sets, streaming data is
better modeled as a multiset.  Natural questions about multisets
extend beyond set-membership; for example, an important question in practice is
\emph{how many times does~$x$ appear in~$\col$?}  

Thus, our syntatic definition of data structures admits both
mutability and rich query spaces.
Formally, a data structure as a triple of algorithms $(\Rep, \Qry, \Up)$ denoting
the \emph{representation}, \emph{query-evaluation}, and \emph{Update} algorithms, respectively.
Associated to the data structure is a set of supported
queries~$\mathcal{Q}$, which are functions~$\qry$ over data objects,
and a set~$\mathcal{U}$ of allowed update functions.
For reasons we will elucidate in a moment, all three algorithms take a
key~$\ky$ as input, and both~$\Rep$ and~$\Qry$ may be randomized.


The combination of mutability and rich query spaces has significant
implications for security.  Consider the count-min
sketch structure~\cite{xxx}, which compactly (and approximately)
represents an updatable multiset~$\col$.  
Loosely, the representation of~$\col$ is a two-dimensional array
of counters. To add an instance of~$x$, the representation is
updated by hashing~$x$ to identify particular counters, and then
incrementing these.  An instance of~$x$ is removed
by hashing and decrementing the counters.  
The frequency of~$x$ is determined by taking
the minmum value~$v$ over the counters associated to it. (Counters are
typically floored at 0.)  Now, if one
restricts to set-membership queries ---~is $v>0$?~-- there is the
potential for both false-positive \emph{and} false-negative
responses.  In particular, if the representation is updated by
``removing'' an element~$y$ that does not appear in the
underlying~$\col$, one or more of the counters associated to~$x$ may
be decremented.  

Even though it is two-sided, the notion of error with respect to set-membership
queries is intuitive, even if one wants to associate different
fixed costs to false-positives and false negatives.  For frequency
queries, it is far less obvious.   One could define frequency-estimate
error in an absolute sense, i.e., the data structure errs if its
response is not exactly correct.  But even in the non-adaptive
setting, traditional analyses guarantee only that the response will be
\emph{close} to the exact frequency, with some probability that is close
to one.  Thus, our security definitions are parameterized by a
error-cost function~$\delta$: if the correct response to a query is~$a$ and
the data structure responds with~$a'$, the cost of the error is
$\delta(a,a') \geq 0$.  

They are also parameterized by an total-cost threshold, and
the adversary is considered to ``win'' if the total cost of the errors it induces
is greater than this value.  As we will see, even calculating the
total cost is not straightforward.  In particular, determining whether
or not the cost of a given error should be carried across (adaptive,
adversarial) updates to~$\col$ and its representation.

To summarize, our high-level contributions are: formal syntax for
mutable data structures, and two notions of adversarial 
correctness for these.  Our notions capture settings in which representations
are made public, or kept private, respectively.

We exercise our syntax and notions by
analyzing three important, real-world data structures: Bloom
filters~\cite{xxx}, counting filters~\cite{xxx}, and count-min
sketches~\cite{xxx}. 
In addition to the basic version of these, we
explore variations that allow for ``salted'' representations (i.e.,
per-representation randomness), secret keys (so that the mapping
from~$\col$ to representation is unpredictable, even if the
representation of one or more closely related~$\col'$ are known), or both.

\heading{Security findings.}
%\newcommand{\cellsize}{1cm}
\begin{figure}
\small
\centering
\begin{tabular}{|r | >{\centering}*{4}{m{\cellsize} |} | >{\centering}*{4}{m{\cellsize} |} }\hline
&\multicolumn{4}{c||}{Public Rep} & \multicolumn{4}{c|}{Private Rep} \\
  \cline{2-5}\cline{6-9}
&$\emptyset$ &salt &key &salt+key& $\emptyset$ &salt &key &salt+key\\ \hline
(Immutable) Bloom filter & & & & & & & & \\ \hline
Bloom filter & & & & & & & & \\ \hline
Counting filter  & & & & & & & & \\ \hline
Cuckoo filter & & & & & & & & \\ \hline
Count-min sketch & & & & & & & & \\ \hline
\end{tabular}
\caption{}
\label{fig:results-overview}
\end{figure}

\tsnote{Stuff to do, typed up quickly during our Tuesday afternoon meeting...}
\begin{itemize}
\item $n$-capped does not imply $\ell$-thresholded for any
  particular~$\ell$ except in the non-adaptive setting
\item \dctodo{Summary of Results!  Break out into items, here.}
\item \dctodo{A paragraph that summarizes the summary of results --
    the elevator pitch}
\item \dctodo{Add Section/sub-section summary blurbs to intro, to
    shape final intro story}
\item \anytodo{We need to tie our analytical settings to reality --
    what do salts imply in practice?  Keys?}
\item \dctodo{How do representation sizes compare to prior work?  What
    can we say about them in a concrete sense (not compared to prior
    work)?}
\item Make sure to hit the online vs. offline query issue
\item For parallel structure and completeness, there are a bunch of
  things we could consider, but don't.  Why?  Any reasons we can give
  for skipped cases (other than ``because it's hard'', and ``because
  it takes to much space'') will help convince reviewers that we
  aren't lazy
\item ...
\end{itemize}

\dcnote{Summaries follow:}
\begin{itemize}
  \item Overall pitch: We construct general definitions to capture the idea of a
  probabilistic data structure and define notions of what it means for these
  structures to be secure against an adversary attempting to force the
  structures to produce errors when queried. We show that three traditional data
  structures, the Bloom filter, the count-min sketch, and the counting filter,
  are insecure in their usual form. Any of these can be kept secure by keeping
  the representation private, adding a per-representation random salt to the
  hash functions, and introducing a simple `thresholding' procedure to ensure
  that the structures do not become too full. With Bloom filters, we can get
  away with not using thresholding, and even with only keeping a key
  for the hash functions secret rather than the entire filter, but the
  security bounds are weaker. If the filter is assumed to be immutable after
  construction, we can even get away with not keeping anything secret, though
  the security bound is \emph{much} weaker.
  \item Standard Bloom filters (4.1): With no salt or key there is no security
  in either sense.
  \item Salted Bloom filters (4.2): If the filters are immutable, we find a weak
  ($q_H$ features prominently) \errep\ bound. If the filters are mutable, we
  show an attack in the \errep\ setting and provide a decent bound for \erreps\
  security.
  \item Keyed Bloom filters (4.3): We get a decent bound for the \errep\
  setting.
  \item Salted, $\ell$-thresholded Bloom filters (4.4): We get a better
  \erreps\ bound than in the $n$-capped case.
  \item Count min-sketch (5.1-2): We show that \errep\ security is impossible
  regardless of which construction you use (key, threshold, etc.), but that
  \erreps\ security is achievable using a salt and $\ell$-thresholding.
  \item Counting filter (6.1-2): Same results as count min-sketch, with a
  slightly different security bound. This bound is given in terms of $\delta^+$
  and $\delta^-$, which describe the relative `badness' of false positives and
  false negatives. A large value of $\delta^-$ is more harmful than a large
  value of $\delta^+$, i.e. counting filters are better for avoiding false
  positives than false negatives. (Interestingly, the opposite is true in the
  non-adpative case.)
\end{itemize}
\tsnote{old stuff below here}

\ignore{ %possibly move elsewhere in the intro, or the opening to the
         %bloom filter section
Bloom filters are
ubiquitous in distributed computing, including web caches (e.g., Squid) and hash
tables (e.g., BigTable and Hadoop), resource and packet routing, and network
measurement. (We refer the reader to the
surveys~\cite{broder2004network,tarkoma2012theory} for a comprehensive list of
applications.) 
Bloom filters have also been modified and co-opted for security-critical
applications; perhaps unsurprisingly, things go wrong. Schnell
\etal~\cite{schnell2011novel} proposed using secretly-keyed Bloom filters in
order to enable privacy-preserving record linkage (PPRL) across data sets.  This
was deployed in medical-data applications in Australia, Brazil, Germany, and
Switzerland~\cite{niedermeyer2014cryptanalysis}. 
%As one exercise of our
%notions, we study their proposal in detail. % in Section~\ref{sec:bf-bigram}.
%
}


\heading{Data structures and their correctness.}
%
We formalize a data structure as a triple of algorithms $(\Rep, \Qry, \Up)$ denoting
the \emph{representation}, \emph{query-evaluation}, and \emph{Update} algorithms, respectively.
Associated to the data structure is a set of supported queries~$\mathcal{Q}$.
The representation algorithm is randomized, taking as input a
key~$\ky$ and a collection of data~$\col$, and returning a
representation~$\pub$ of~$\col$.  (To capture unkeyed data structures,
one sets $\ky=\varepsilon$.)
%
The deterministic query-evaluation algorithm~$\Qry$ uses~$\ky$ and $\pub$ in
order to respond to a requested query~$\qry \in \queries$ on~$\col$.
\textcolor{blue}{[[...]]}

For better efficiency, many data structures only approximately
represent the collection~$\col$. In this case, the query-evaluation
algorithm~$\Qry$ may err in its response to queries.  \oldstuff{Roughly
speaking,  our notion of adversarial correctness (\errep) captures how
difficult it is for an attacker (given $\pub$) to find~$r>0$ distinct queries on
which $\Qry$ returns an incorrect answer.}

For Bloom filters, the representation~$\pub$ includes a bit array~$M$ that
represents a set~$\col \subseteq \elts$ using hash functions
$h_1,\ldots,h_k$. The supported queries are the predicates
$\{\qry_x\}_{x\in\elts}$, where $\qry_x(\col)=1$ iff $x \in \col$. It is well
known that Bloom filters may have false positives, and their false-positive rate
for \emph{independently chosen} inputs and queries is well understood. (See
Appendix~\ref{sec:mitz}.) Our correctness notion quantitatively captures the
error rate even in the presence of an attacker that adaptively attempts to
induce errors. \textcolor{blue}{[[...]]}

We note that Naor and Yogev~\cite{naor2015bloom} were the first to formalize
adversarial correctness of Bloom filters and, indeed, their work
provided inspiration for this paper.  Our work significantly extends
theirs in several ways, as we will detail, shortly.  \textcolor{blue}{[[...]]}
% ss-rep
\if{0}{
  \anytodo{Several reviewers have made the same complaint : why these notions?
  In particular, are they interesting beyond an academic exercise?  We need to
  address this head-on.  One idea is to try to build something on top of these
  notions, but I really see that as a separate paper.  Unless we can build some
  \emph{well known} primitive... but I'm not sure what it would be, or how
  interesting.}
  %
  \cpnote{Alex Davidson's paper (ia.cr/2017/448) suggests that garbled Bloom
  filters (or some variation of them) can be used for private-set intersection. We
  could ask if privacy in our sense suffices for this application.
  But \ssrep is not the right notion since it requires a key, and \owrep is
  probably too weak. Davidson views GBFs as distributional virtual black-box
  obfuscators, which are stronger than \owrep-secure structures.}
  %
  \cpnote{To my thinking, these notions were originally devised from the
  perspective of what security properties do existing data structures admit. If
  our intention is to use these properties in order to achieve some higher-level
  goal, I don't think we have the right ones. Short of strengthening them, I think
  our best bet  is to \emph{own} our original perspective. To that end, the place
  we need the most motivation is \ssrep privacy of $\SKBF$, the PRF-based BF. See
  my comments in Section~\ref{sec:bf-prf} for two ways we've already thought of.}
}\fi

\heading{Constructions we analyze.}
%
We put our syntax and security notions to work in several case studies.
%
The brief description of Bloom filters given above was silent as to how the hash
functions $h_1, \ldots, h_k$ are chosen, and whether or not they are
public. In fact, these details have a significant effect on what notions of
security the resulting structure satisfies:
\begin{itemize}
  \item
    (Section~\ref{sec:bf}) If the hash functions are fixed and known to the
    attacker prior to the filter being constructed, the data structure offers
    neither correctness nor privacy for any practically interesting parameters.
    We show this by exhibiting explicit attacks and analyzing their performance.

  \item (Section~\ref{sec:bf-salt}) If \emph{salted} hash functions are used,
    and the adversary is given the salt only after the collection $\col$ is
    chosen, then %with modest changes to the parameters (i.e., the filter length and number of hashes), 
    the structure can achieve the same correctness guarantees in the adversarial setting as do Bloom filters in the traditional
    non-adversarial setting. 
    %(Our analysis here treats the hash functions as random oracles; the usual analysis treats them as ideal random functions.)
    We also show that this structure achieves our privacy notion of one-wayness.

  \item (Section~\ref{sec:bf-prf}) We explore a natural, keyed variant of a
    Bloom filter in which the hash functions are derived from a secretly keyed
    pseudorandom function. (This is similar to a construction proposed by Naor
    and Yogev~\cite{naor2015bloom}.) We show that this variant enjoys
    simulation-based privacy, as well as a tighter security bound for
    correctness than the salted Bloom filter.
\end{itemize}
%
\noindent
Our particular realization of the salted and secretly keyed Bloom filters
leverages results from Kirsch and Mitzenmacher~\cite{kirsch2006less} that allow
one to effectively implement $h_1,\ldots, h_k$ by making only two \emph{actual}
evaluations of an underlying hash function or PRF, respectively.
%
In addition to the comprehensive analysis of Bloom filters described above, we
also apply our definitions to:
\begin{itemize}
  \item (Section~\ref{sec:bf-bigram}) A keyed structure for privacy-preserving
    record linkage introduced by Schnell \etal~\cite{schnell2011novel}, and
    subsequently attacked by Niedermeyer
    \etal~\cite{niedermeyer2014cryptanalysis}. In our framework we are able to
    show precisely how their scheme breaks down.

  \item (Section~\ref{sec:dict}) A dictionary proposed by Charles
    and Chellapilla~\cite{charles2008bloomier2} that stores a set of~$n$
    key/value pairs, where the keys are arbitrary bitstrings and the values are
    of length at most~$m$, using just $O(mn)$ bits.
\end{itemize}

\heading{Future research directions.}
%
\ignore{
It would be interesting to extend our work to the case of \emph{mutable} data
structures. Specific examples to consider here are counting Bloom
filters~\cite{fan2000summary}, scalable Bloom
filters~\cite{almeida2007scalable}, count-min
sketches~\cite{cormode2005improved}, and hierarchical Bloom
filters~\cite{zhu2004hierarchical}, to name just a few in the extended Bloom
filter family.
}

% NOTE(all) Removed these citations: \cite{broder2004network,nojima2009cryptographically}
Our goal is to establish foundations for the security of  data
structures. But it would certainly be interesting to analyze high-level
protocols that use these data structures, e.g.
content-distribution networks~\cite{byers2002informed}, where many servers
propagate representations of their local cache to their neighbors. The Bloom
filter family alone has a wide range of practical applications, for example in
large database query processing~\cite{broder2004network}, routing algorithms for
peer-to-peer networks~\cite{reynolds2003efficient}, protocols for establishing
linkages between medical-record databases~\cite{schnell2011novel}, fair routing
of TCP packets~\cite{feng2001stochastic}, and Bitcoin wallet
synchronization~\cite{gervais2014privacy}.
%
Analyzing higher-level primitives or protocols will require establishing
appropriate syntax and security notions for those, too; hence we leave this for
future work.

%\heading{Related work.}
\paragraph{Comparison with Naor-Yogev}
As previously noted, Naor and Yogev~\cite{naor2015bloom} were the first to
formalize adversarial correctness of Bloom filters.  Our work extends theirs
significantly in several directions. First, we consider abstract data
structures, rather than only set-membership structures.  Even with respect to
the specific case of correctness for set-membership structures, our work offers
several advantages as compared to the Naor-Yogev treatment.
%
One, our syntax distinguishes between the (secret) key and the public portion of
a data structure, an important distinction that is missing in their work.
%
Two, the Naor-Yogev definition of correctness allows the adversary to make
several queries, some of which may produce incorrect results; the attacker then
succeeds if it outputs a \emph{fresh} query that causes an error. This
separation seems arbitrary, and we propose instead a parameterized definition in
which the attacker succeeds if it can cause a certain number of (distinct)
errors during its entire execution.
%
Three, Naor and Yogev analyze the correctness of a new Bloom filter variant of
their own design. In contrast, we are mainly interested in analyzing existing,
real-world constructions to understand their security.

\paragraph{Other related works}
There is a long tradition in computer science of designing structures that
concisely (but probabilistically) represent data so as to support some set of
queries, and each of these structures has its own interesting security
characterisitcs~\cite{chazelle2004bloomier,cormode2005improved,DP08a,DF03,fredman1984storing,mironov2011sketching}.

We have already mentioned the ubiquity of Bloom filters in support of efficient
network communication and computing protocols.  They also find use in
security-critical environments, including spam filters, (distributed)
denial-of-service attack detection, and deep packet
inspection~\cite{tarkoma2012theory}.  Recently, Bloom filters were proposed as a
means of efficient certificate-revocation list (CRL)
distribution~\cite{larisch2017crlite}, a crucial component of public-key
infrastructures.

% Correctness attacks
Correctness of data structures in adversarial settings is well-motivated in the
security literature and in practice.
%
Perhaps the earliest published attack on the correctness of a data structure was due to
Lipton and Naughton~\cite{lipton1993clocked} who showed that timing analysis of
record insertion in a hash table allows an adversary to adaptively choose
elements so as to increase look-up time, effectively degrading a service's
performance.
%
Crosby and Wallach~\cite{crosby2003denial} exploited hash collisions to increase
the average URL load time in Squid, a web proxy used for caching content in
order to reduce network bandwidth.
%
More recently, Gerbet \etal~\cite{gerbet2015power} described \emph{pollution
attacks} on Bloom filters, whereby an adversary inserts a number of
adaptively-chosen elements with the goal of forcing a high false-positive rate.
Although some of their attacks exploit weak (i.e., non-cryptographic) hash
functions (as do~\cite{crosby2003denial}), their methodology is effective even
for good choices of hash functions.
%
They suggest revised parameter choices for Bloom filters (i.e., filter length and
number of hashes) in order to cope with their attacks.


\ignore{
Finally, we note that the dictionary construction considered in
Section~\ref{sec:dict} bares resemblance (at least structurally) to
\emph{garbled Bloom filters}, a tool used recently for efficient private-set
intersection~\cite{dong2013when,rindal2017improved}.
}

% NOTE(all) Below are notes and references we considered adding to related work.
\if{0}{
  Correctness in adversarial settings has been considered for broader ranges of
  data structures.  Mironov, Naor, and Segev~\cite{mironov2011sketching} studied
  a setting in which non-colluding parties interact with a third-party
  \emph{referee} in order to compute a function of their data: For example,
  whether their sets are equal, or the approximate size of their intersection.
  The parties, which share a common reference string, but otherwise do not
  communicate, send a concise \emph{sketch} of their data to the referee, who
  performs the computation and publishes the result  The adversary is modeled as
  a malicious party attempting to skew the result.
  %
  \cpnote{It would be interesting to see if there's a connection between our
  notion of correctness and their setting.}
}\fi

\if{0}{
  \emph{Secure indexes}, proposed by Eu-Jin Goh~\cite{goh2003secure}, structure
  a document so that it can be searched by keyword if the querying party has a
  special \emph{trapdoor} for the keyword. The party issuing trapdoors has a
  secret key.  \jnote{I'm not sure the work of Goh is super relevant. Or, if it
  is, then so is any searchable encryption scheme.}
  %
  \cpnote{I agree ... I included it since it was cited in the
  survey~\cite{tarkoma2012theory} as an example of a ``secure'' Bloom filter
variant.}
}\fi

\if{0}{
  Other security notions for data structures, beyond correctness and privacy,
  have been considered.  For example, \emph{authenticated data
  structures}~\cite{tamassia2003authenticated} allow a trusted third party to
  certify the validity of a query on a data set maintained by an untrusted
  server.
}\fi

\if{0}{
  We recommend reading the Naor-Yogev paper for a survey of related work and a
  discussion of related papers. Here we mention a few additional practical
  works, but stress that this only scratches the surface.
  %
  \jnote{Rather random collection of papers using Bloom filters and variants. I
  removed it for now, since it's not clear that they have any particular
relevance to us. I kept only the refs that seemed directly relevant.}
  %
  As previously mentioned, Bloom filters and their relatives are some of the most
  widely used data structures supporting set-membership queries. As examples,
  Hbase, the open-source implementation of Google's BigTable storage
  system~\cite{chang2008bigtable}, a Hadoop-based NoSql database designed to
  handle large datasets, includes an implementation of Bloom filters and
  counting Bloom filters, and he Squid proxy~\cite{fan2000summary} uses a Bloom
  filter as a ``summary'' of the set of URLs in its cache in order to improve
  latency for web-object retrieval. Reynolds and
  Vahdat~\cite{reynolds2003efficient} proposed an efficient distributed search
  engine that can be used to search for files containing a particular keyword.
  Their search engine maps the keywords of each file into a Bloom filter; a
  look-up of the keyword in the Bloom filter tells whether the node has files
  containing that keyword or not. Stochastic Fair Blue~\cite{feng2001stochastic}
  uses counting Bloom filter to manage non-responsive TCP traffic.
}\fi

\if{0}{
  \cite{gao2006internet} is an application of BFs for detecting pollution
  attacks on web caches.
  %
  \heading{Related work: attacks}
  \tsnote{Brought these back into the text just to help Chris get up to speed.}
  \begin{itemize}
    \item Niedermeyer et al., ``Cryptanalysis of Basic Bloom Filters Used for
      Privacy-Preserving Record Linkage'', breaking privacy of
      secret-hash-function Bloom filters. \tsnote{Journal of Privacy and
      Confidentiality, 2014}

    \item Gerbet, Kumar and Lauradoux, ``The power of evil choices in bloom
      filters''. \tsnote{DSN'15: Looks like a real goldmine of related work!}

    \item Crosby and Wallach, ``Denial of Service via Algorithmic Complexity
      Attacks'' \tsnote{Gives attacks on Squid}

    \item Gao et al., ``Internet Cache Pollution Attacks and Countermeasures''
  \end{itemize}

  %\ignore{
  \heading{Related work: definitions(?)}
  \begin{itemize}
    \item Nojima and Kadobayashi, ``Cryptographically Secure Bloom Filters''.
      \tsnote{Gives some security definitions for privacy. Quick scan, not super
      clear what they achieve. The definition of client-privacy (Definition 1) for
      example, makes no sense to me.  Actually, likewise for server-privacy
      (Definition 2).  Both seem vague and thoroughly underspecified.}

    \item Naor and Yogev

    \item Eujin Goh, ``Secure Indexes'' \tsnote{A secure index can
      be used for set membership.  Builds a secret-key data
      structure (an Index) that allows searching for keyword~$w$
      if one holds the trapdoor $T_w$ for~$w$, where the trapdoor
      depends on the secret key.  Main construction uses
      traditional Bloom filters and a PRF.  Construction appears
      quite inefficient, needing a very long secret key, turning a
      keyword~$w$ into a bunch of PRF outputs, and then storing
      each of these PRF outputs in the BF.  Haven't read the full
      analysis; don't know if this was ever published. }
      \jnote{Never published. I think this work uses Bloom filters
      for encrypted search; I don't remember the paper having much
      to say about Bloom filters themselves.}
  \end{itemize}

  \heading{Related work: constructions}
  \begin{itemize}
    \item Bellovin and Cheswick, ``Privacy-Enhanced Searches Using Encrypted Bloom
    Filters''.

  \item Kerschbaum , ``Public-Key Encrypted Bloom Filters with
    Applications to Supply Chain Integrity''.

  \item S\"{a}rell\"{a} et al., ``BloomCasting: Security in Bloom Filter Based Multicast''.

  \item Dong, Chen,
      Wen, ``When Private Set Intersection Meets Big Data: An Efficient and
      Scaleable Protocol'' \tsnote{``garbled bloom filters'', which actually store
      the set element by storing~$k$ xor-shares, one at each of the~$k$ hash
      indices (with care for reusing shares if hash collisions occur); also
      and``oblivious bloom intersection''}\tsnote{If the filter and the hash
      functions are public, there is a naive attack that works for some
      interesting parameters.}

  \item Tarkoma, Rothenberg, Lagerspetz ``Theory and Practice of Bloom Filters in Distributed
      Systems''
      %
      \tsnote{Great high-level coverage.  Only found preprint version though.}
      \cpnote{{ieeexplore.ieee.org/iel5/9739/6151681/05751342.pdf}}

    \item Durham, Kantarcioglu, Xue, Kuzu, Malin ``Composite Bloom Filters for
      Secure Record Linkage'' \tsnote{Per-field BFs, sampled and composed into
      single BF that is then permuted by a secret random permutation.  No clear
      statement of the problem that is being solved.  Should pull full version and
      get details.}
  \end{itemize}

  \heading{Related work: tangential}
  \begin{itemize}
    \item Chang and Mitzenmacher ``Privacy Preserving Keyword Searches on Remote Encrypted Data''.

    \item Mitzenmacher and Vadhan. ``Why Simple Hash Functions Work: Exploiting
      the Entropy in a Data Stream''.

    \item Dodis et al. ``Fuzzy Extractors: How to Generate Strong Keys from
      Biometrics and Other Noisy Data'' \tsnote{Introduces ``secure sketches'',
      which is a representation of a single-element set that is information
      theoretically private (up to some function of the min-entropy of the
      element); only tangentially related to ``sketches'' as defined in the Bloom
      filter literature.}
  \end{itemize}
}\fi



\section{Syntax}
\label{sec:syntax}
We start with a universe $\mathcal{D}$ of data objects, a key space $\mathcal{K}$, a set $\mathcal{R}$ of responses equipped with a metric $d: \mathcal{R}^2 \to [0,\infty)$, a set $\mathcal{Q} = \{\mathsf{qry}: \mathcal{D} \to \mathcal{R}\}$ of queries, and a set $\mathcal{U} = \{\mathsf{up}: \mathcal{D} \to \mathcal{D}\}$ of possible updates. A {\em mutable data structure} is a tuple $\Pi = (\textsc{Rep},\textsc{Qry},\textsc{Up})$, where:

\begin{itemize}
  \item $\textsc{Rep}: \mathcal{K} \times \mathcal{D} \to \{0,1\}^*$ is a randomized {\em representation algorithm}, taking as input a key $K \in \mathcal{K}$ and data object $D \in \mathcal{D}$, and outputting the public representation $\mathsf{pub} \in \{0,1\}^*$ of $D$. We write this as $\mathsf{pub} \gets^\$ \textsc{Rep}_K(D)$.
  \item $\textsc{Qry}: \mathcal{K} \times \{0,1\}^* \times \mathcal{Q} \to \mathcal{R}$ is a deterministic {\em query-evaluation algorithm}, taking as input $K \in \mathcal{K}$, $\mathsf{pub} \in \{0,1\}^*$, and $\mathsf{qry} \in \mathcal{Q}$, and outputting an answer $a \in \mathcal{R}$. We write this as $a \gets \textsc{Qry}_K(\mathsf{pub},\mathsf{qry})$.
  \item $\textsc{Up}: \mathcal{K} \times \{0,1\}^* \times \mathcal{U} \to \{0,1\}^*$ is a randomized {\em update algorithm}, taking as input $K \in \mathcal{K}$, $\mathsf{pub} \in \{0,1\}^*$, and $\mathsf{up} \in \mathcal{U}$, and outputting an updated representation $\mathsf{pub}'$. We write this as $\mathsf{pub}' \gets^\$ \textsc{Up}_K(\mathsf{pub},\mathsf{up})$.
\end{itemize}

Unkeyed data structures are a special case where $\mathcal{K} = \{\epsilon\}$, and immutable data structures are a special case where the update algorithm deterministically returns $\textsc{Up}(\key,\mathsf{pub},\mathsf{up})=\mathsf{pub}$ for all inputs.  In the latter case, we will often drop mention of the update algorithm.


\section{Notions of Adversarial Correctness}
\label{sec:correctness}

\begin{figure}[t]
  \twoColsNoDivide{0.48}
  {
    \experimentv{$\Exp{\errep}_{\struct,r}(\advA)$}\\[2pt]
      $\setC \gets \emptyset$; $\ct,\err_0 \gets 0$;
      $\key \getsr \keys$\\
      $i \getsr \advA^{\REPO,\UPO,\QRYO}$\\
      return $(\err_i \geq r)$ \tsnote{Or something... possibly a function of $\vec\err_i=(\err_i[j])_{j\in[q]}$}
    \\[6pt]
    \oraclev{$\REPO(\col)$}\\[2pt]
      $\ct\gets\ct+1$;
      $\col_\ct \gets \col$\\
      $\pub_\ct \getsr \Rep_\key(\col)$\\
      return $\pub_\ct$
  }
  {
    \oraclev{$\UPO(i, \up)$}\\[2pt]
      $\col_i \gets \up(\col_i)$;
      $\pub_i \getsr \Up_\key(\pub_i, \up)$\\
      return $\pub_i$
    \\[9pt]
    \oraclev{$\QRYO(i, \qry)$}\\[2pt]
      if $(i,\qry) \in \setC$ then return $\bot$\\
      $\setC \gets \setC \union \{(i,\qry)\}$; $a \gets \Qry_K(\pub_i, \qry)$\\
      $\err_i \gets \err_i + d(a,\qry(\col_i))$\\
      return $a$
  }
  \caption{Adversarial correctness for a mutable data structure.}
  \vspace{6pt}\hrule
  \label{fig:security}
\end{figure}

An adversarial notion of correctness is given by the following experiment for a mutable data structure $\Pi$ and error capacity $r$. A key $K$ is sampled from the key space $\mathcal{K}$. An adversary $A$ is equipped with oracles $\mathbf{Rep}(D)$, $\mathbf{Qry}(i,\mathsf{qry})$, and $\mathbf{Up}(i,\mathsf{up})$. The adversary may use $\mathbf{Rep}$ arbitrarily many times to gain representations $\mathsf{pub}_i$ of data objects $D_i$, in each case initializing a corresponding $err_i$ to zero. The $\mathbf{Qry}$ oracle takes the index of the $\mathsf{pub}_i$ to query along with a query object, returning $\bot$ if the same query has been made previously and the result of $\textsc{Qry}_K(\mathsf{pub}_i,\mathsf{qry})$ otherwise. If this does not agree with $\mathsf{qry}(D_i)$, $err_i$ is incremented. Finally, the $\mathbf{Up}$ oracle updates $D_i$ to $\mathsf{up}(D_i)$ and $\mathsf{pub}_i$ to $\textsc{Up}_K(\mathsf{pub}_i,\mathsf{up})$, returning the latter result. If any of the $err_i$ exceeds $r$, the experiment is a success for the adversary.

\subsection{Example Data Structures}

In general, each probabilistic data structure has some bound on the error size per query, assuming the adversary is not fully adaptive. In each case we want to show that allowing the adversary full adaptivity does not significantly increase the error rate. Generally, we have a data object space $\mathcal{D} \subseteq 2^\mathcal{X}$, a collection of subsets of some universe $\mathcal{X}$, and a response space $\mathcal{R} = \{0,1\}$ with the usual metric of $d(m,n) = |m-n|$. Then the query space consists of indicator functions $\mathsf{qry_x}$ for $x \in \mathcal{X}$, so that $\mathsf{qry_x}(D)$ is 1 if and only if $x \in D$. The update space at least consists of insertions, and may also include deletions.

A typical case occurs with standard Bloom filters~\cite{bloomfilter}. Since there are no false negatives, the size of the error a non-adaptive adversary is expected to create per query is simply equal to the false positive rate, which is on the order of $(1-e^{-\frac{kn}{m}})^k$ for an $m$-bit array with $k$ hash functions storing up to $n$ values. Compressed Bloom filters~\cite{xxx} operate in the same way, with a false positive rate which must also take into account the degree of compression~\todo{David}{Please address}\tsnote{what is it?}. 

Counting Bloom filters~\cite{xxx} and cuckoo filters~\cite{xxx} are both extensions of this notion which increase $\mathcal{U}$ to include deletion operations, but fortunately the notions of correctness are still straightforward given that queries are simply testing for set membership. \todo{David}{Please address}\tsnote{Counting bloom filters are meant for frequency queries, not just set membership, so I don't know why you are lumping them in with cuckoo filters.}  

Bloomier filters~\cite{xxx} instead enlarge the response space $\mathcal{R}$, and alter the data object space to a set of the form $\mathcal{D} \subseteq \mathcal{R}^\mathcal{X}$. \todo{David}{Please address}\tsnote{Not following this. Please explain.} One of the response values is $\bot$ to indicate that value has not been associated with any element of $\mathcal{R}$, and the only type of error that can occur is that a value which should return $\bot$ instead returns some other element of $\mathcal{R}$. Again we have a situation where `false positives' are the only type of error which can occur.

Stable Bloom filters~\cite{xxx} are an example of a structure where typical choices of $\mathcal{D}$ and $\mathcal{U}$ will not work. Here, objects probabilistically decay from the filter over time, but our syntax requires that updates are functions rather than randomized algorithms. \todo{David}{Please address}\tsnote{I think there's some confusion here(?)  In the SBF, the update function is deterministic, but the algorithm that \emph{implements} the update, i.e. \textsc{Up}, is randomized.  The stream is the stream, it's only the representation of the stream that ``forgets'' that it saw stream elements (loosely speaking).}  One possibility is to have $\mathcal{D} \subseteq \{0,\ldots,t\}^X$ for some universe $X$ and natural number $t$, so that each element of the universe $X$ is associated with its remaining time to live. \todo{David}{Please address}\tsnote{I imagine that the max value is ``baked into'' \textsc{Up}.  It's not like the value changes; it's just a parameter of the DS, like the number of hash functions or the size of the array.}  In that case, each update function is of the form $\mathsf{up}_x$ for some $x \in X$, which decrements all nonzero values and then sets the value associated with $x$ to $t$. The query functions are then of the form $\mathsf{qry}_x$ for some $x \in X$ and return 0 if and only if the value associated with $x$ is 0. Note that when this data object is represented as a stable Bloom filter of $m$ bits with $p$ values decremented each update, the SBF's maximum time-to-live must be set to $\frac{pt}{m}$. Unfortunately, there are still some issues since it is not until stability is reached that the low error bound is guaranteed. If an adversary is allowed to generate the representations this stability is far from a guarantee.

Depending on the implementation, count-min sketch~\cite{xxx} may use a data object space $\mathcal{D} \subseteq \mathbb{N}^X$ or $\mathcal{D} \subseteq \mathbb{Z}^X$. In the former case we have two further sub-cases, the case where all updates increment the value associated with $x \in X$ and the case where updates may either increment or decrement the associated value (but not below 0). Additionally, count-min sketch supports multiple different types of queries, in each case yielding a response in $\mathbb{Z}$. For a point query, the difference between the query and the true value is bounded by $n\epsilon$ with probability $1-\delta$, where $n$ is the sum of the true frequencies of the stream elements. The maximum error of a single query is simply $n$, in the case that an element has never actually been added to the set but has incorrectly had all its counters incremented each of $n$ times another element has been added, so the expected non-adaptive error size is bounded above by $n\epsilon(1-\delta)+n\delta = n(\delta+\epsilon-\delta\epsilon)$.  Similarly, we find that the error size of an inner product query is bounded above by $n_1n_2\epsilon(1-\delta)+n_1n_2\delta = n_1n_2(\delta+\epsilon-\delta\epsilon)$. Range queries are tricky and I'm not quite sure on the precise bound in terms of $\delta$ and $\epsilon$.

\subsection{Correctness Proofs \& Attacks}

First, we note that any structure which is insecure in the immutable case is also insecure in the mutable case. Any adversary in the immutable case is identical to a corresponding adversary in the mutable case which simply never makes use of the $\UPO$ oracle. From this we know that standard (unsalted, unkeyed) Bloom filters cannot ensure correctness in the mutable case.

In the case that is no secret key, we have a result analogous to the immutable case showing that full-strength correctness is equivalent to correctness relative to an experiment where the adversary makes only a single query to the $\REPO$ oracle. \todo{David}{please typeset this into a nice theorem and proof.  Try to work out the concrete security statement, using the immutable-case draft as a guide for typesetting and style.} For any adversary $A$ in the ordinary experiment, we construct an adversary $B$ as follows. First, $B$ samples $q$ from $[q_R]$, where $q_R$ is the number of $\REPO$ queries $A$ makes, and then simulates $A$. When $A$ makes its $q$th $\REPO$ query, $B$ queries its $\REPO$ oracle for the resulting representation $\mathsf{pub}$, storing this value and returning it to $A$. For all preceding and following $\REPO$ queries, $B$ simply executes $\textsc{Rep}$ itself and returns the result. When $A$ makes a $\QRYO$ query to $\mathsf{pub}_q$, $B$ queries its $\QRYO$ oracle and returns the result to $A$. For all other $\QRYO$ queries, $B$ returns $\bot$ if $(i,\mathsf{qry})$ has been asked before and otherwise executes $\textsc{Qry}$ to determine the result of the query. Similarly, for any $\UPO$ query $A$ makes to $\mathsf{pub}_q$, $B$ queries its $\UPO$ oracle and returns the result, while for $\UPO$ queries to other $\mathsf{pub}_i$ we have $B$ execute $\textsc{Up}$ and return the result. Finally, any RO queries from $A$ are passed to the RO for $B$.

The advantage of $B$ is smaller than that of $A$ by at most a factor of $q_R$, since there is a $\frac{1}{q_R}$ chance that $B$ randomly selects the same $q$ that $A$ eventually chooses as its output.

Say that the update algorithm $\textsc{Up}_K$ is {\em invertible} if for every representation $\mathsf{pub}$ and update $\mathsf{up} \in \mathcal{U}$ there is $\mathsf{up}' \in \mathcal{U}$ such that $P(\textsc{Up}(\textsc{Up}(\mathsf{pub},\mathsf{up}),\mathsf{up}') = \mathsf{pub}) = 1$. In particular, the updates for both counting Bloom filters and count-min sketches are invertible. Each also has the additional feature of having a natural choice of {\em empty} structure such that every other structure can be constructed by starting with the empty structure and performing a corresponding sequence of update operations.
We will show that for any mutable data structure with invertible updates and an empty structure, salts can provide no additional guarantee of correctness over an unsalted structure. \todo{David}{Again, please typeset as a nice theorem with concrete advantage bounds.} Consider an adversary $A$ in the case of a non-salted data structure which makes $q_R$ queries to $\REPO$ and $q_T$ queries to $\QRYO$. We construct an adversary $B$ for the salted case which produces the same errors as follows. First, $B$ calls the $\REPO$ oracle on the empty set, receiving an empty representation $\mathsf{pub}$ together with the salt used to create the representation. Whenever $A$ makes a query of the form $\REPO(\mathcal{S})$, $B$ performs a sequence of $\textsc{Up}$ operations on $\mathsf{pub}$ to transform it into $\mathcal{S}$, returns the resulting representation, stores $\mathcal{S}$, and then performs the opposite updates in reverse order to return to the original representation $\mathsf{pub}$. Then if $A$ makes a query of the form $\QRYO(\mathcal{S},\mathsf{qry})$ [...]

In general this may use as many as $2n(q_R+q_T)$ update queries, where $n$ is the longest minimal sequence of update operations needed to generate a data structure in the space (which in a counting filter is equal to the maximum supported multiset size, and in count-min sketch is equal to the sum of the absolute values of the elements' frequences). However, in the case of an unkeyed structure it turns out that far fewer updates are required. By our earlier result, every adversary has a corresponding adversary which makes only a single $\REPO$ query and has an advantage differing by at most a factor of $q_R$. In this case, there is no need to `reset' the representation every time an oracle is queried. Because only one data structure is created at a time, the adversary $B$ can initialize an empty representation, append elements to form the representation of the single $\mathcal{S}$ given as input to the only $\REPO$ query. [...]

As a concrete example, consider the heuristic attack used against standard (unsalted, unkeyed) Bloom filters. We can perform the same attack against salted counting Bloom filters with the following tweaks. We begin with a set $T$ of $s$ test queries and a set $T'$ of $r$ target queries. We want to induce false positives, by which we mean inputs to $\textsc{Qry}$ which produce an output of 1 or more when the correct response is 0. We construct a tree whose nodes are the subsets of $T$ with no more than $n$ elements, ordered by the $\subseteq$ relation with the empty set as the root. Then we take $\REPO(\{\})$ and perform a depth-first search of the tree. Every time we descend a level to a new node, that node contains one element not contained in its parent node. First we call $\UPO$ to insert this element into the representation, and then we check to see whether all the elements of $T'$ are false positives (which can be done simply by examining the filter in the keyless case, or by querying $\QRYO$ in the keyed case). When we ascend a level, we call $\UPO$ to delete the element corresponding to that node. This search displays exactly the same behavior as the corresponding attack on standard Bloom filters, except that $\UPO$ queries take the place of $\REPO$ queries (and as such, the fact that $\REPO$ has a random salt is irrelevant).

Furthermore, both of the heuristics used in the original attack can be adapted to use here. [...]

\section{Bloom Filter Results}
\label{sec:bloom}
%\subsection{Bloom filter lower bounds}

%\begin{table}[thp]
\begin{center}
  \begin{tabular}{ | c | c |c | c | c | }
    \hline
    Salt & Key & \parbox{.75in}{Private\\ Representation} & \parbox{.75in}{Irreversible\\ Updates} & Result \\ \hline\hline
    0 & 0 & 0 & 0 & Attack 1 \\ \hline
    0 & 0 & 0 & 1 & Attack 1 \\ \hline
    0 & 0 & 1 & 0 & Attack 1 \\ \hline
    0 & 0 & 1 & 1 & Attack 1 \\ \hline
    0 & 1 & 0 & 0 & Attack 2 \\ \hline
    0 & 1 & 0 & 1 & Attack 2 \\ \hline
    0 & 1 & 1 & 0 & Attack 2 \\ \hline
    0 & 1 & 1 & 1 & Attack 2 \\ \hline
    1 & 0 & 0 & 0 & Attack 3 \\ \hline
    1 & 0 & 0 & 1 & Attack 3 \\ \hline
    1 & 0 & 1 & 0 & ? \\ \hline
    1 & 0 & 1 & 1 & Theorem~\ref{thm:bf-priv-salt-bound} \\ \hline
    1 & 1 & 0 & 0 & Attack 4 \\ \hline
    1 & 1 & 0 & 1 & Theorem~\ref{thm:bf-key-bound} \\ \hline
    1 & 1 & 1 & 0 & ? \\ \hline
    1 & 1 & 1 & 1 & Theorem~\ref{thm:bf-key-bound} \\
    \hline
  \end{tabular}
\end{center}
\tsnote{This table feels unintutive, somehow. ``Salt'', ``Key'', ``Irreversible updates'' are all syntactic properties; ``Private Representation'' is an artifact of the attack model.  We need a better way to organize and present these results.}
\caption{Summary table for Bloom filters. \textbf{Attack 1:}The adversary chooses a maximally large test set $\col$ and simulates $\Rep$ to produce a representation $\pub$. The adversary then simulates $\Rep$ for many arbitrarily chosen singleton sets disjoint from $\col$, checking each one to see if its element is a false positive for $\pub$. Once it has accumulated $r$ false positives, call $\REPO$ on $\col$, call $\QRYO$ for each false positive found, and return 1.  \textbf{Attack 2:} The adversary chooses a maximally large test set $\col$ and calls $\REPO$ to produce a representation $\pub$. The adversary then calls $\REPO$ for many arbitrarily chosen singleton sets disjoint from $\col$, using $\QRYO$ on each to determine if its element is a false positive for the representation constructed for $\col$. Once it has accumualated $r$ false positives, return 1. \textbf{Attack 3:}  The adversary chooses a maximally large test set $\col$ and calls $\REPO$ to receive a representation $\pub$ together with the salt $\salt$. Using the known salt, the adversary simulates $\Rep$ for many arbitrarily chosen singleton sets disjoint from $\col$, checking each one to see if its element is a false positive for $\pub$. Once it has accumulated $r$ false positives, call $\QRYO$ for each such false positive and return 1. \textbf{Attack 4:}The adversary chooses a test set $\col_0$ and a target set $\col_1$, performing a search on representable subsets of $\col_0$ represented as a tree ordered by $\subseteq$. Moving up or down in the tree is accomplished using $\UPO$, and at each node $\QRYO$ is called for each element of $\col_1$ to determine which are false positives. Once $r$ false positives are found, halt and return 1. }
\label{tab:main}
\end{table}
\todo{DC (lead)}{Explain away all cases except the one(s) for which we give
security (upperbound) theorems.} The standard Bloom filter shows a variety of
different behaviors depending on its exact implementation. If the hash functions
used are chosen beforehand and potentially known to the adversary, this public
information allows offline attacks to be mounted against the data structure
which can produce potentially damaging false positives. In the case of immutable
Bloom filters, making use of a per-representation salt is sufficient to prevent
these attacks, though depending on the use case the use of non-fixed
per-representation randomness may or may not be feasible. Furthermore, in the
case of mutable Bloom filters there are additional difficulties with offline
attacks due to adversarially-chosen updates. To guarantee correctness in this
case we must additionally guarantee that representations can be kept private
from the adversary.

A generic attack against unsalted, unkeyed data structures handling set
membership queries is for the adversary to choose a large set $\col$ of
potential filter elements and simulate $\REPO$ to produce representations for
$\{x\}$ for each $x \in \col$. Given these, it can perform offline computations
to determine disjoint $\setT$, $\setR \subseteq \col$ such that the elements of
$\setR$ all produce errors when queried for membership in $\REPO(\setT)$. After
a sufficiently large $\setR$ has been found, the adversary makes a single
$\REPO$ call on $\setT$ and makes one $\QRYO$ call per element of $\setR$,
resulting in guaranteed success for the adversary with only a minimal number of
queries performed.

In general, such an attack may not be computationally feasible in the real
world. However, because the attack is entirely offline, many structures with
non-negligible error probabilities are vulnerable to these attacks. For example,
consider a Bloom filter with false positive probability $10^{-5}$ with an
adversary wishing to construct $r = 10$ false positives. The adversary can fix
$\setT$ in advance and perform somewhere on the order of a million hash queries
to random elements in order to construct a $\setR$ of size 10, and then perform
only a single $\REPO$ calls and ten $\QRYO$ calls to the service hosting the
Bloom filter. This is likely to be much more feasible for the adversary than
performing on the order of a million $\QRYO$ calls in an online attack which
randomly guesses elements until it accumulates 10 false positives.

The use of a salt without a private key in the public representation setting is
insufficient to defeat this attack. In this setting, the adversary need only
make its $\REPO$ query for $\setT$ in advance, at which point it will receive
both the representation $\pub$ and the salt $\salt$ used to construct it. Using
this known salt, the adversary is still able to simulate $\REPO$ for arbitrary
singleton representations. The previous attack therefore still works with the
same (minimal) number of $\QRYO$ calls at the very end of the experiment, after
it has determined $\setR$ using offline computations.

The opposite of this, using a private key without a salt, does weaken the attack
somewhat. Even with public representations, the adversary cannot locally
simulate $\REPO$ without guessing the private key. However, they can still
outperform random $\QRYO$ calls by making $\REPO$ queries for singleton elements
without fixing any $\setT$ in advance. After selecting a random set $\col$ of
size $q_R-1$, the adversary performs offline computations to find $\setT$,
$\setR \subseteq \col$ such that the elements of $\setR$ are false positives for
the representation of $\setT$ (which can be computed from the representations of
the singleton subsets of $\setT$). The adversary wins if there is a partition
where $\setR$ produces at least $r$ errors on $\REPO(\setT)$.

% ...

Using a salted Bloom filter in the private representation setting, however, does
provide some security. At the time a representation is created, the structure
chooses a salt $\salt$ which it will use for all further queries and updates. In
order for maximum security to be guaranteed, we must ensure that the
representation, and in particular the salt, is kept secret from the adversary.
We define this structure $\SBF[H,k,m,n,\lambda]$ as the Bloom filter structure
that uses $H(s) = (h_1(s),\ldots,h_k(s))$ for hashing inputs to $k$ values in
$[m]$. Furthermore, each call of $\Rep$ first involves picking a salt $\salt$
from the salt space $\bits^\lambda$, and all hashes made to insert or query for
an element $x$ are determined using $H(x \Vert \salt)$. Finally, the parameter
$n$ means that any attempts to represent sets with more than $n$ elements fail.
\todo{DC lead}{Specify what exactly is being analyzed.  What are the updates?}

\begin{theorem}[Correctness Bound for Private-Representation Salted Bloom Filters]\label{thm:bf-priv-salt-bound}
Fix integers $k, m, n, \lambda, r\geq 0$, let $H \colon \bits^* \to [m]$ be a function, and let $\struct_s = \SBF[H,k,m,n,\lambda]$.
  For every $t, q_R, q_T, q_U, q_H \geq 0$, it holds that
  \begin{eqnarray*}
    \Adv{\erreps}_{\struct_\saltybloom,r}(t,&q_R,& q_T, q_U, q_H) \leq \\ && q_R \cdot
     \left[
      \frac{q_H}{2^\lambda} +
      {\dbinom{q_T+q_U}{r}} p(k, m, n+s)^r
    \right] \,,
\end{eqnarray*}
where $s$ is defined to be $\min(r,q_U)$, $H$ is modeled as a random oracle, and $p(k, m, n+s)$ is the standard, non-adaptive false-positive probability on a Bloom filter with the given parameters.
\end{theorem}

This proof first reduces to the single-representation case, which as shown in
lemma~\ref{lemma:errep} will reduce the adversary's advantage by at most a
factor of $q_R$. The main idea behind the proof is to remove the adversary's
adaptivity a step at a time. We isolate the possibility of the adversary
guessing the salt, which would allow it to mount its own offline attack on the
filter without relying on the $\QRYO$ oracle. If the adversary does not guess
the salt, the outputs of the $\REPO$, $\QRYO$, and $\UPO$ oracles are
unpredictable to the adversary, producing uniformly randomly distributed bits to
set (for $\REPO$ and $\UPO$) or to check (for $\QRYO$). Under the assumption
that the adversary does not predict the salt, queries made to distinct elements
are independent of each other. The only remaining issue is that the adversary
can potentially gain an advantage by testing whether some object $x$ is a false
positive for the filter, and then updating the filter to include $x$ only if the
test query returned `false'. An analysis shows that this is now (once imperfect
pseudorandom functions and salt collisions have been dealt with) the only way
for the adversary to gain an advantage over making queries to an immutable Bloom
filter. Because this adaptive strategy introduces tricky conditional
possibilities, we cannot compute an exact value for the adversary's advantage.
Instead, we move to an alternate scenario where each $\QRYO$ also produces a
free update and every $\UPO$ first performs a free query. This makes $\QRYO$ and
$\UPO$ calls indistinguishable, so that the adversary is effectively making a
series of independent random queries that each have a chance to increment the
error counter. Because the number of 1s in the filter can only increase, the
probability of a false positive from any one of these queries is bounded above
by the probability of a false positive on the final maximally-sized filter, a
probability which is given by the Kirsch and Mitzenmacher bound.

\begin{figure*}
  \boxThmBFSaltCorrect{0.48}
  {
    \underline{$\game_0(\advA)$}\\[2pt]
      $\col \getsr \advA^H$; $\setC \gets \emptyset$; $\err \gets 0$\\
      $\pub \getsr \Rep[H](\col)$\\
      $\bot \getsr \advA^{H,\QRYO,\UPO}$\\
      return $(\err \geq r)$
    \\[6pt]
    \oraclev{$\QRYO(\qry_x)$}\\[2pt]
      if $\qry_x \in \mathcal{C}$ then return $\bot$\\
      $\setC \gets \setC \union \{\qry_x\}$\\
      $a \gets \Qry[H](\pub, \qry_x)$\\
      if $a \neq \qry_x(\col)$ then $\err \gets \err + 1$\\
      return~$a$
    \\[6pt]
    \oraclev{$\UPO(\up_x)$}\\[2pt]
      $\setC \gets \emptyset$\\
      $a \gets \Qry[H](\pub, \qry_x)$\\
      if $\qry_x \in \setC$ and $a \neq \qry_x(\col)$ then\\
      \tab $\err \gets \err-1$\\
      $\col \gets \col \union \{x\}$\\
      $\pub \gets \Up[H](\pub,\up_x)$\\
      return~$\bot$
    \\[4pt]
    \hspace*{-4pt}\rule{1.043\textwidth}{.4pt}
    \\[5pt]
    \oraclev{$\HASHO_1(\salt,x)$} \hfill\diffplus{$\game_2$}\;{$\game_1$}\hspace*{3pt}\\
      $\hh \getsr [m]^2$; $\vv \gets \fff(\hh$)\\
      if $\salt = \salt^*$ then\\
      \tab $\bad_1 \gets 1$; \diffplus{return $\vv$}\\
      if $T[\salt,x]$ is defined then $\vv \gets T[\salt,x]$\\
      $T[\salt,x] \gets \vv$;
      return $\vv$
  }
  {
    \underline{$\game_1(\advB)$}\\[2pt]
      $\salt^* \getsr \bits^\lambda$;
      $\col \getsr \advB^{\HASHO_1}$\\
      $\pub \gets \Repx[\HASHO_2](\col, \salt^*)$\\
      $\setC \gets \emptyset$;
      $\err \gets 0$\\
      $\bot \getsr \advB^{\HASHO_1,\QRYO,\UPO}$\\
      return $(\err \geq r)$
    \\[6pt]
    \oraclev{$\QRYO(\qry_x)$}\\[2pt]
      if $\qry_x \in \mathcal{C}$ then return $\bot$\\
      $\setC \gets \setC \cup \{\qry_x\}$\\
      $a \gets \Qry[\HASHO_2](\pub, \qry_x)$\\
      if $a \neq \qry_x(\col)$ then $\err \gets \err + 1$\\
      return~$a$
    \\[6pt]
    \oraclev{$\UPO(\up_x)$}\\[2pt]
      $\setC \gets \emptyset$\\
      $a \gets \Qry[H](\pub, \qry_x)$\\
      if $a \neq \qry_x(\col)$ and $\qry_x \in \setC$ then\\
      \tab $\err \gets \err-1$\\
      $\col \gets \col \union \{x\}$\\
      $\pub \gets \Up[\HASHO_2](\pub,\up_x)$\\
      return~$\bot$
    \\[6pt]
    \oraclev{$\HASHO_2(\salt,x)$}\\[2pt]
      $\hh \getsr [m]^2$; $\vv \gets \fff(\hh$)\\
      if $T[\salt,x]$ is defined then\\
      \tab $\vv \gets T[\salt,x]$\\
      $T[\salt,x] \gets \vv$;
      return $\vv$
  }
  {
    \underline{$\game_3(\advB)$}\\[2pt]
    \oraclev{$\QRYO(\qry_x)$}\\[2pt]
      $a \gets \Qry[\HASHO_3](\pub, \qry_x)$\\
      if $a \neq \qry_x(\col)$ then $\err \gets \err + 1$\\
      $\col \gets \col \union \{x\}$
      $\pub \gets \Up[\HASHO_2](\pub,\up_x)$\\
      return~$a$
  }
  {
    \oraclev{$\UPO(\up_x)$}\\[2pt]
      $a \gets \Qry[\HASHO_3](\pub, \qry_x)$\\
      if $a \neq \qry_x(\col)$ then $\err \gets \err + 1$\\
      $\col \gets \col \union \{x\}$
      $\pub \gets \Up[\HASHO_2](\pub,\up_x)$\\
      return~$\bot$
    \\[6pt]
    \oraclev{$\HASHO_i(\salt,x)$}\\[2pt]
      $\hh \getsr [m]^2$; $\vv \gets \fff(\hh$)\\
      return $\vv$
  }
  \caption{Games 0--3 for proof of Theorem~\ref{thm:bf-priv-salt-bound}.}
  \label{fig:bf-priv-salt-bound}
\end{figure*}

\begin{proof}

We first reduce from the $\erreps$ case to the $\erreps1$ case, which by
lemma~\ref{lemma:errep} may scale the adversary's advantage only by a factor of
$q_R$. The game~$\game_0$ is exactly equivalent to the $\erreps1$ experiment, so
$\Adv{\errep1}_{\struct_s,r}(\advA) = \Prob{\game_0(\advA) = 1}$. In~$\game_1$
we split the hash oracle into three, giving the adversary access $\HASHO_1$ in
both stages of the game, while $\HASHO_2$ is reserved for oracular use by
$\Repx$, $\QRYO$, and $\UPO$. For any $\advA$ for~$\game_0$, there is $\advB$
for~$\game_1$ which produces the same advantage by simulating $\advA$. This
adversary first creates its own table $R$ with all values initially undefined.
When $\advA$ makes a query $w$ to $H$, $\advB$ returns $R[w]$ if that entry in
the table is defined. Otherwise, if there are $\salt \in \bits^\lambda$, $j \in
[k]$, and $x \in \bits^*$ such that $w = \langle\salt, j, x\rangle$, forward
$(\salt,x)$ to $\HASHO_1$. For each $j \in [k]$, set $R[\langle\salt, j,
x\rangle] = \vv_j$, where $\vv$ is the output of the $\HASHO_1$ oracle. If there
is no such triple $\langle\salt, j, x\rangle$, just sample $r$ from $[m]$
uniformly and set $R[w] = r$. In either case, return $R[w]$ to $\advA$. When
$\advA$ outputs its collection $\col$, $\advB$ outputs $\col$ as well. Any
queries by $\advA$ to $\QRYO$ or $\UPO$ are forwarded to $\advB$'s corresponding
oracle. The simulation is perfect because $\Rep[H](\col)$ and $\Up[H](\col,\up)$
are identically distributed to $\Rep[\HASHO_2](\col)$ and
$\Up[\HASHO_2](\col,\up)$. Because we have a perfect simulation,
$\Adv{\erreps1}_{\struct_s,r}(\advA) = \Prob{\game_1(\advA) = 1}$.

The game~$\game_2$ is the same as~$\game_1$ until $\bad_1$ is set, which occurs
exactly when $\advB$ sends $(\salt^*,x)$ to $\HASHO_1$ for some $x$. In the
first phase, there is again a $q_1/2^\lambda$ chance of the adversary guessing
the salt. In the second phase, the random sampling used by $\HASHO_i$ ensures
that each call the adversary makes to the $\HASHO_i$ oracle is independent of
all previous calls. We therefore have a $q_2/2^\lambda$ chance of the adversary
guessing the salt during this phase, for a total chance of $q_H/2^\lambda$
chance of the adversary guessing the salt at some point during the experiment.
Then $\Adv{\erreps_1}_{\struct_s,r}(\advA) \le \Prob{\game_2(\advB) = 1} +
q_H/2^\lambda$. Having taken this into account, we may now assume the adversary
never guesses the salt.

We want to show that alternating between sequences of queries and sequences of
updates is no better than making one long series of updates and then one long
sequence of queries. There are three types of updates the adversary can make:
updates to add elements that have been queried and found to be false positives;
updates to add elements that have been queried and found not to be false
positives; and updates to add elements that have not been queried yet. We may
assume without loss of generality that the adversary never makes the first type
of update, since doing so is never beneficial (it does not change the
representation at all and decreases the number of errors the adversary has
found).

Note that the choices of $\vv$ constructed by the $\HASHO_i$ oracles are
independent of all previous queries. Because of this, any update of type 3 is
equivalent to any other update of type 3; the probability of any bit being
flipped by one update is the same as the probability of the bit being flipped by
the other update. Similarly, any update of type 2 is equivalent to any other
update of type 2, but is not the same as type 3 since the probability is
conditioned on $\vv$ not being a false positive. We assume the worst case,
namely that all updates are type 2 (i.e. at least one bit is flipped by each
update).

Because the adversary never guesses the salt, $\HASHO_1$ simply functions as a
random oracle. Furthermore, we can assume the adversary never adds an element of
$\col$ to $\col$ and never makes a $\QRYO$ call for an element which is already
in $\col$, since neither of these provides any additional information and
neither affects the rest of the experiment in any way.

Now we move to the game~$\game_3$. Here each $\QRYO$ query also calls $\UPO$ to
add that element to $\col$. Additionally, the penalty for adding known false
positives is removed. To avoid penalizing the adversary by prematurely maxing
out the number of elements in $\col$ because of added false positives, we also
increase the maximum size of $\col$ from $n$ to $n+s$, where $s = \min(r,q_U)$.
Because the adversary (without loss of generality) stops after accumulating $r$
errors, only $\min(r,q_U)$ false positives will be added to $\col$ and so a
maximum size of $n+s$ is sufficient to produce no penalty for the adversary.
Furthermore, each $\UPO$ call is preceded by a $\QRYO$ call. Neither of these
changes can produce a worse result for the adversary, so $\Prob{\game_2(\advB) =
1} \le \Prob{\game_3(\advB) = 1}$. Now, however, there is no longer any
distinction between $\QRYO$ and $\UPO$ calls. All calls to either oracle are
independent of each other and produce the same effect, querying and then
updating $\col$. Each of these queries for false positives is at most as
successful as a query to a Bloom filter with $n+r$ elements, so the adversary's
probability of finding a false positive on any query is bounded above by the
standard success rate for a Bloom filter with those parameters. The adversary is
required to produce $r$ errors over the course of $q_T+q_U$ queries, which by
the binomial theorem gives an advantage bound of $\Prob{\game_3(\advB) = 1} \le
\binom{q_T+q_U}{r}p(k, m, n+s)^r$.

The full adversarial advantage is then
$$\Adv{\erreps_1}_{\struct_s,r}(\advA) \le q_R \cdot \left(\frac{q_H}{2^\lambda} + \binom{q_T+q_U}{r}p(k, m, n+s)^r\right).$$

\missingqed
\end{proof}

%Without `thresholding', where the addition to a filter fails if a certain proportion of bits are set to 1, the bound improves somewhat.

%Once we have reduced to the case of $\HASHO_1$ functioning as a random oracle and $\QRYO$ calls only being made to distinct elements not in $\col$, the adversary has only three ways of interacting with the filter: querying an element not in $\col$ which has not yet been queried, inserting a new element into $\col$ which has not yet been queried, and inserting a new element into $\col$ which has been queried and was found to not be a false positive. In the non-threshold case, the last of these is exploitable, since an element which is confirmed not to be a false positive will necessarily flip at least one bit when inserted into the representation. However, when a maximum proportion $p$ of the bits are allowed to be set to 1, adding known non-false-positives rather than unqueried elements only causes the threshold to be reached faster, without providing any additional benefit to the adversary. In any case the adversary can do no better than setting exactly $pm$ bits to 1, in which case the false positive rate will be exactly $p^k$ and the adversarial advantage will be

%$$\Adv{\erreps_1}_{\struct_s,r}(\advA) \le q_R \cdot \left(\frac{q_H}{2^\lambda} + \binom{q_T}{r}p^{kr}\right).$$

For the salted and keyed case with a built-in threshold, we use the notion of query independence to demonstrate that, if a thresholding assumption is used to prevent the filter from becoming too full, we can attain a good error bound even in the public-representation case.

%NOTE: n is the THRESHOLD here
\begin{theorem}[Correctness Bound for Salted and Keyed Bloom Filters]\label{thm:bf-key-bound}
Fix integers $k, m, n, \lambda, r\geq 0$, let $\Pi$ be the salted and keyed Bloom filter described above with PRF $F$, and let $d$ be the error function assigning a weight of 1 to a false positive and 0 to a correct response.
  For every $t, q_R, q_T, q_U, q_H \geq 0$ such that $q_T \geq r$, it holds that
  \begin{eqnarray*}
    \Adv{\errep}_{\struct,r,d}(t,&q_R,& q_T, q_U, q_H) \leq \\ && I_{(n/m)^k}(r, q_T-r+1) \,.
  \end{eqnarray*}
where $I$ is the regularized incomplete beta function.
\end{theorem}

\begin{proof}
In order to show that a salted, secretly-keyed Bloom filter is secure, we want
to show that the structure has query independence up to leakage that is defined
to be the number of bits in the filter which are set to 1. If the PRF $F$ is
good, and the key is unknown to the adversary, the adversary cannot use the
random oracle to predict what representations will look like ahead of time. In
particular, by a straightforward conditioning argument,
$\Adv{\game_0}_{\struct,r}(\advA) \le \Adv{\prf}(F) +
\Adv{\game_1}_{\struct,r}(\advA)$, where $\game_1$ is the same game but with the
pseudorandom function replaced by a lazily-evaluated random function. In this
game, the adversary cannot possibly guess queries ahead of time with better than
50/50 odds, since the output of the query depends on the output of a true random
function on an untested input. We can then immediately apply the independence
lemma to see that $\Adv{\game_1}_{\struct,r}(\advA) \le I_p(r, q_T-r+1)$, where
$p$ is the false positive probability. In a filter where $\textsf{lk}(\pub)$ out
of $m$ bits are set to 1, this probability is simply $(\textsf{lk}(\pub)/m)^k$.
The adversary can maximize this value by setting as many bits to 1 as possible.
With a threshold of $n$ bits allowed to be set to 1, the overall advantage is
then
$$\Adv{\errep}_{\struct,r,d}(\advA) \le \Adv{\prf}(F) + I_{(n/m)^k}(r, q_T-r+1)$$
\end{proof}

%The case of a Bloom filter whose representations are always public but which uses a private key in addition to a salt is almost identical. In practice the bound is actually somewhat stronger: the instance of $q_H$ in the bound is replaced with $q_R$, which means that the adversary which attempts to guess the salt must rely entirely on `online' queries as opposed to `offline' queries to $q_H$.

%These two theorems together show that in order to guarantee maximal correctness in the streaming setting, one must ensure that representations are kept private from potential adversaries, and possibly use a secret key in addition to a salt when implementing a Bloom filter. In either case, the hidden information is necessary to protect the filter from an adversary.

We can use this final and best bound to demonstrate some of the possible bounds
with various parameter settings. Figure ? shows the upper bound for a various
combination of parameters, starting from the default parameters of $k = 4$, $m =
1024$, $n = 100$, $r = 10$, $q_R = 1$, $q_T = 100$, and $q_U = 100$. The most
significant factor in the adversarial advantage is $q_R$, with even small
increases producing a drastic increase in the adversarial advantage. Since this
is the only part of the error bound that is not clearly associated with an
attack, it is possible that this term could be reduced or eliminated.

%\includegraphics[scale=0.75]{BF_Fig}


\section{Counting Filter Results}
\label{sec:counting}
\begin{figure}
  \twoColsNoDivide{0.22}
  {
    \underline{$\Rep^R_K(\col)$}\\[2pt]
      $\salt \getsr \bits^\lambda$\\
      $\pub \gets \langle \zeroes(m), \salt\rangle$\\
      for $x \in \col$ do \\
        $\tab \pub \gets \Up^R_K(\pub, \up_{x,1})$\\
        $\tab$if $\pub = \bot$ then return $\bot$\\
      return $\pub$
    \\[6pt]
      \underline{$\Qry^R_K(\langle \v.M, \salt\rangle,x)$}\\[2pt]
      $\v.X \gets R_K(\salt \cat x)$\\
      for $i \in \v.X$ do\\
        $\tab$if $\v.M[i] = 0$ then return 0\\
      return 1
  }
  {
    \underline{$\Up^R_K(\langle \v.M, \salt\rangle, \up_{x,b})$}\\[2pt]
      if $c \geq n$ then return $\bot$\\
      $\v.M' \gets \v.M$;
      $\v.X \gets R_K(\salt \cat x)$\\
      for $i$ in $\v.X$ do\\
      $\tab$ $a \gets \v.M'[i]$\\
      $\tab$ if $a = 0 \wedge b < 0$ then return $\bot$\\
      $\tab \v.M'[i] \gets \v.M'[i] + b$\\
      $\v.M \gets \v.M'$;
      return $\langle \v.M, \salt, c+b \rangle$
  }
  \caption{Keyed structure $\countbloom[R,\ell,\lambda]$ given by
  $(\Rep^R,\Qry^R,\Up^R)$ is used to define the $\ell$-thresholded version of a
  counting filter. The parameters are a function $R:
  \keys\by\bits^* \to [m]^k$ and integers $\ell, \lambda \geq0$. A concrete scheme
  is given by a particular choice of parameters. The function $\hw'$, used to
  determine if the filter is full, is defined in Section~\ref{sec:prelims}.}
  \label{fig:cbf-def}
\end{figure}

Counting filters are a modified version of Bloom filters which are designed to,
like a count min-sketch, allow for deletion as well as insertion. Unlike CMSs,
however, counting filters are designed to handle set membership queries rather
than frequency queries. Despite this, the two structures are closely related in
terms of security properties. We show that \errep\ security is similarly
impossible, but that with the $\ell$-thresholding assumption built into the
definition presented in Figure~\ref{fig:cbf-def}, a bound for \errep\ can be
achieved which is very close to the bound for a count min-sketch.

\heading{Error function for frequency queries}
%
Unlike with a Bloom filter or count min-sketch, counting filters must account
for two different types of error: false positives and false negatives. To be as
general as possible, we define a parametrized error function~$\delta$ for
positive $\delta^+, \delta^- \in \R$ as
\begin{equation}
  \delta(x, y) =
  \begin{cases}
    0 & \text{if}\ x = y \\
    \delta^+ & \text{if}\ x = 1, y = 0 \\
    \delta^- & \text{if}\ x = 0, y = 1
  \end{cases}
\end{equation}
This means that false positives are given a weight of $\delta^+$ while false
negatives are given a weight of $\delta^-$, and correct responses are given a
weight of 0.

\subsection{Insecurity of Public Counting Filters}
Any counting filter construction necessarily fails to satisfy \errep\
correctness for the same reasons as in the case of count min-sketch. In
particular, the adversary can call $\REPO(\emptyset)$ to receive an empty
representation, insert an element $x$, observe which counters are incremented by
this insertion, and then delete $x$. By doing this repeatedly, the adversary can
gain information about which elements overlap with which combinations of other
elements, and can therefore mount the same target-set coverage attack as in the
count min-sketch case. In fact, the attack is slightly easier to pull off, since
once a good test set is found the adversary need only insert each element of the
set once in order to cause the elements of the target set to become false
positives.

\subsection{Private Thresholded Counting Filters}

\begin{theorem}
For integers $q_R, q_T, q_H, r, t \geq 0$, let $p_\ell = ((\ell+1)/m)^k$ and
$r' = \lfloor r/\max(\delta^+,k\delta^-) \rfloor$. If $r' > p_\ell q_T$, it holds that
  \begin{equation*}
  \begin{aligned}
   \Adv{\erreps}_{\Pi,\delta,r}(t, q_R,q_T,q_U,q_H,q_V) &\leq\\
     & q_R \cdot \left[\frac{q_H}{2^\lambda} + e^{r'-p_\ell q_T}\left(\frac{p_\ell q_T}{r'}\right)^{r'}\right],
  \end{aligned}
\end{equation*}
where $H$ is modeled as a random oracle.
\end{theorem}
The proof is somewhat similar to that of Theorem~\ref{thm:scms-erreps-tex},
but a major difference comes from the difference in $\Qry$. In particular,
a string which is an element of the underlying data object cannot
produce an error through being overestimated, but it can produce an error
through being \emph{underestimated} if the adversary causes false negatives.
\begin{proof}
  \begin{figure*}
\twoCols{0.47}
{
  \vspace{-7pt}
  \experimentv{$\game_{0}(\advA)$}\hfill\diffplus{$\game_1$}\\[2pt]
    $M^* \gets \bot$;
    $\setS \gets \emptyset$;
    $\salt^* \getsr \bits^\lambda$\\
    $\advB^{\REPO,\QRYO,\UPO,\HASHO_1}$;
    return $\big[\sum_x \err[x] \geq r\big]$
  \\[6pt]
  \oraclev{$\HASHO_c(\salt \cat x)$}\\[2pt]
    $\vv \getsr [m]^k$\\
    if $\salt=\salt^*$ and $c = 1$ then \com{Caller is~$\advB$}\\
    \tab $\bad_1 \gets 1$; \diffplus{return $\vv$}\\
    if $T[Z,x] = \bot$ then $\vv \gets T[Z,x]$\\
    $T[Z,x] \gets \vv$; return $\vv$
  \\[6pt]
  \oraclev{$\QRYO(\qry_x)$}\\[2pt]
    $X \gets \HASHO_3(\salt^* \cat x)$;
    $a \gets \infty$;
    $\setS \gets \setS \cup \{x\}$\\
    for $i$ in $[1..k]$ do\\
      $\tab a \gets \min(a, M[i][X[i]])$\\
    if $\err[x] < \delta(a,\qry_x(\col^*))$ then
          $\err[x] \gets \delta(a,\qry_x(\col^*))\diffplus{+k}$\\
    return $a$
  \\[6pt]
  \oraclev{$\REPO(\col)$}\\[2pt]
    for $i$ in $[1..k]$ do\\
      $\tab M^*[i] \gets 0^m$\\
    $\setS^* \gets \col$\\
    for $x \in \col$ do\\
    $\tab\UPO(\up_x)$\\
    return $\top$
}
{
  \vspace{-7pt}
  \oraclev{$\UPO(\up_{x,b})$}\\[2pt]
    if $w'(M^*) > \ell$ then return $\top$\\
    $M' \gets M^*$\\
    for $i$ in $[1..k]$ do\\
      $\tab$ if $M'[i][X[i]] = 0$ and $b < 0$ then return $\top$\\
      $\tab M'[i][X[i]] \gets M'[i][X[i]] + b$\\
    $M^* \gets M'$\\
    if $\err[x] \neq \bot$ then\\
      $\tab a \gets \QRYO(\qry_x)$\\
      $\tab\err[x] \gets \min(\delta(a,\qry_x(\col^*)),err[x])$\\
    $\setS^* \gets \up_{x,b}(\setS^*)$;
    return $\top$
  \vspace{6pt}\hrule\vspace{3pt}
  \oraclev{$\REPO(\col)$}\hfill\diffplus{$\game_3$}\\[2pt]
    for $i$ in $[1..k]$ do\\
      $\tab M^*[i] \gets 0^m$\\
    $\setS^* \gets \col$\\
    for $x \in \col$ do\\
    $\tab\UPO(\up_x)$\\
    \diffplusbox{for $i$ in $[1..k]$ do\\
      $\tab$while $w(M[i]) < \ell+1$ do\\
        $\tab\tab j \getsr [m]$;
        $M[i][j] \gets 1$}
    return $\top$
}
\caption{Games 0--3 for proof of Theorem~\ref{thm:scms-erreps-th}.}
\label{fig:sbf-erreps/games}
\end{figure*}

As with the proof of Theorem~\ref{thm:sbf-errep-immutable}, we derive a bound in
the \erreps1 case and then use Lemma~\ref{thm:lemma1} to move from \erreps1 to
the more general \erreps case. Because we are in the \erreps1 case, we may
assume without loss of generality that the adversary does not call $\REVO$,
since revealing the only representation automatically prevents the adversary
from winning.

We begin with a game~$\game_0$ which has identical behavior to the \erreps1
experiment for a counting filter. As in the proof of
Theorem~\ref{thm:sbf-errep-immutable}, we have a
$\bad_1$ flag that gets set if the adversary ever calls $\HASHO_1$ with the
actual salt used by the representation. By a very similar argument, we can
move to~$\game_1$, where the behavior is different only when the $\bad_1$ flag
is set, with a bound of
\begin{equation}
  \Prob{\game_0(\advA)=1} \leq
    q_H/2^\lambda + \Prob{\game_1(\advA)=1} \,.
\end{equation}

Unlike in the case of a count min-sketch, it is entirely possible for deletions
to benefit the adversary in this game. In particular, if $x$ is found to be a
false positive, deleting $x$ may cause up to $k$ elements of $\col$ to become
false negatives. We therefore move to a game~$\game_2$ where the adversary gets
credit for either a single false positive or for $k$ false negatives whenever it
finds a false positive, but where the adversary cannot delete any false
positives that it finds. We let $r' = \lfloor r/\max(\delta^+,k\delta^-)\rfloor$
represent the number of false positives the adversary has to find in~$\game_2$
in order to win. In order to prevent the adversary from being penalized by the
filter becoming full too early, we also raise the thresold from $\ell$ to
$\ell+r'$ in~$\game_2$. Now for any $\advA$ for~$\game_1$, we can construct
$\advB$ for~$\game_2$ that simulates $\advA$, keeping track of all query
responses and forwarding all oracle queries in the natural way, except that
calls to delete false positives are ignored. Since $\UPO$ never fails for
$\advB$ due to the increased threshold, and since $\advB$ gets automatic credit
for any false negatives that might have been caused by deleting false positives,
$\advB$ succeeds whenever $\advA$ does, i.e.
$\Prob{\game_1(\advA)=1} \le \Prob{\game_2(\advB) = 1}$.

Since the remaining deletions do not cause errors, we can use the same argument
as in the proof of Theorem~\ref{thm:scms-erreps-th} to reduce from $\advB$ to an
adversary $\advC$ which does not make deletions at all. In~$\game_3$, we further
reduce from a counting filter to a normal Bloom filter by capping each of the
counters in the filter at 1. Since no deletions are performed, a counter
in~$\game_3(\advC)$ is nonzero if and only if the same counter
in~$\game_2(\advC)$ is nonzero. So $\QRYO$ behaves the same in~$\game_3$ as it
did in~$\game_2$, and $\Prob{\game_2(\advC)=1} \le \Prob{\game_3(\advC) = 1}$.

Note that~$\game_3$ is actually simulating an ordinary Bloom filter, since all
`counters' in the filter are restricted to the range $\bits$, there are no
deletions, and any insertions just set the corresponding bits to 1. In fact,
this game is identical to~$\game_2$ in the proof of
Theorem~\ref{thm:sbf-erreps-th} except that the adversary need only accumulate
$r'$ errors instead of $r$ errors and the threshold is $\ell+r'$ instead of
$\ell$. An identical argument allows us to reach the binomial bound of
\begin{equation}
   \Prob{\game_3(\advC)=1} \le
     \sum_{i=r'}^{q_T} \binom{q_T}{i}p_\ell^i(1-p_\ell)^{q_T-i} \,,
\end{equation}
where $p_\ell$ is now defined to be $((\ell+k+r')/m)^k$. Then the standard
Chernoff bound, along with Lemma~\ref{thm:lemma1}, yields the final bound of
\begin{equation}
   \Adv{\erreps}_{\Pi,\delta,r}(\advA) \leq
     q_R \cdot \left[\frac{q_H}{2^\lambda} + e^{r'-p_\ell q_T}\left(\frac{p_\ell q_T}{r'}\right)^{r'}\right]s.
\end{equation}
\end{proof}

\section{Cuckoo Filter Results}
\label{cuckoo}
Cuckoo filters provide a way of filtering elements with lower space overhead
than counting filters while still allowing for deletion. While they are not as
closely related to Bloom filters as counting filters are, and so we do not have
the precise lower bound that \textit{any} attack against a Bloom filter also
works against a cuckoo filter, it is true that since cuckoo filters are
set-membership structures the specific bounds given in the Bloom filter section
apply here. To establish acceptable upper bounds, we consider the case of a
salted cuckoo filter with private representations. By `salted', we mean that
both the hashing and the fingerprinting makes use of a salt chosen on a
per-represntation basis. Using this definition, we can make use of the lemmas
that allow us to talk about the structure in the single-represntation case with
true random functions. A bound for the false positive rate in cuckoo filters is
given by $\binom{n}{2k+1}\left(\frac{2}{2^f \cdot m}\right)^{2k}$ for a set of
$n$ elements in a filter with $m$ buckets holding up to $k$ entries each, with
fingerprint size $f$.

\begin{theorem}[Correctness Bound for Cuckoo Filters]\label{thm:cuckoo-salt-bound}
Fix integers $n, b, k, \lambda, r\geq 0$, where $m \geq \lambda$.
  For every $t, q_R, q_T, q_U, q_H \geq 0$ such that $q_T \geq r$, it holds that
  $$\Adv{\erreps}_{\struct,r}(t, q_R, q_T, q_U, q_H) \leq q_R \cdot
     \left[
      \frac{q_H}{2^\lambda} +
      {\dbinom{q_T+q_U}{r}} P(n,m,b,k+r)^r
    \right]$$
where $P(n,m,b,k)$ is the standard, non-adaptive false-positive probability on a
cuckoo filter with those parameters.
\end{theorem}

\begin{figure*}
  \boxThmBFSaltCorrect{0.48}
  {
    \underline{$\game_0(\advA)$}\\[2pt]
      $\col \getsr \advA^H$; $\setC \gets \emptyset$; $\err \gets 0$\\
      $\pub \getsr \Rep[\HASHO](\col)$\\
      $\bot \getsr \advA^{\HASHO,\QRYO,\UPO}$\\
      return $(\err \geq r)$
    \\[6pt]
    \oraclev{$\QRYO(x)$}\\[2pt]
      if $x \in \setC$ then return $\bot$\\
      $\setC \gets \setC \union \{x\}$\\
      $a \gets \Qry[\HASHO](\pub, \qry_x)$\\
      if $a \neq [x \in \col]$ then $\err \gets \err + 1$\\
      return~$a$
    \\[6pt]
    \oraclev{$\UPO(x,b)$}\\[2pt]
      $\setC \gets \emptyset$\\
      $a \gets \Qry[\HASHO](\pub, \qry_x)$\\
      if $x \in \setC$ and $a \neq [x \in \col]$ then\\
      \tab $\err \gets \err-1$\\
      if $b = 1$ then\\
      \tab $\col \gets \col \cup \{x\}$\\
      else\\
      \tab $\col \gets \col \setminus \{x\}$\\
      $\pub \gets \Up[\HASHO](\pub,\up_{x,b})$\\
      return~$\bot$
    \\[6pt]
    \oraclev{$\HASHO(x)$}\\
      $\hh \getsr [m]^2$; $\vv \gets \fff(\hh$)\\
      if $T[x]$ is defined then $\vv \gets T[x]$\\
      $T[x] \gets \vv$;
      return $\vv$
  }
  {
    \underline{$\game_1(\advA)$}\\[2pt]
    \oraclev{$\QRYO(x)$}\\[2pt]
      $\setC \gets \emptyset$\\
      $a \gets \Qry[\HASHO](\pub, \qry_x)$\\
      if $a \neq [x \in \col]$ then\\
      \tab $\err \gets \err+d(a,[x \in \col])$\\
      if $b = 1$ then\\
      \tab $\col \gets \col \cup \{x\}$\\
      else\\
      \tab $\col \gets \col \setminus \{x\}$\\
      $\pub \gets \Up[\HASHO](\pub,\up_{x,b})$\\
      return~$\bot$
    \\[6pt]
    \oraclev{$\UPO(x,b)$}\\[2pt]
      $\setC \gets \emptyset$\\
      $a \gets \Qry[\HASHO](\pub, \qry_x)$\\
      if $a \neq [x \in \col]$ then\\
      \tab $\err \gets \err+d(a,[x \in \col])$\\
      if $b = 1$ then\\
      \tab $\col \gets \col \cup \{x\}$\\
      else\\
      \tab $\col \gets \col \setminus \{x\}$\\
      $\pub \gets \Up[\HASHO](\pub,\up_{x,b})$\\
      return~$\bot$
  }
  {
  }
  {
  }
  \caption{Games 0--3 for proof of Theorem~\ref{thm:cuckoo-salt-bound}.}
  \label{fig:cuckoo-salt-bound}
\end{figure*}

\begin{proof}
Again we can use lemma~\ref{lemma:errep} and lemma~\ref{lemma:salttorand} to
reduce to the case of $\erreps1$ with true random functions in place of salted
hash functions. We use this as the first game $\game_0$, so that
$\Adv{\erreps}_{\struct_s,r}(\advA) \le q_R \cdot \left(\frac{q_H}{2^\lambda} +
\Prob{\game_0(\advA) = 1}\right).$.

Because membership tests rely on objects having the same fingerprint, and a
single bucket may contain multiple copies of the same fingerprint, deleting $n$
false positives can only produce up to $n$ false negatives, one per fingerprint
removed. Unless the filter becomes full, it is therefore never helpful for an
adversary to delete a false positive, since it cannot increase the number of
errors. We may then assume, subject to the condition that the adversary never
has to worry about the filter becoming full, that the adversary only makes
insertion updates. Furthermore, we may assume the adversary never inserts
elements which are already in the set, or elements which have already been
queried and found to be false positives, since these also cannot increase the
number of errors.

We want to show that alternating between sequences of queries and updates is no
better than making updates first and queries afterwards. We have shown that the
adversary only makes two types of updates: inserting unqueried elements which
are not in the set and inserting known true negatives.

We move to a second game where each $\QRYO$ call also inserts that element.
Additionally, the penalty for adding known false positives is removed. To avoid
penalizing the adversary by producing an overly-full filter which rejects
further insertions, we also increase the size of each bucket by $r$. Because we
may assume without loss of generality that the adversary stops after finding $r$
errors, only $r$ false positives will be inserted and so these extra $r$ slots
in each bucket are sufficient to ensure that if the buckets in the original game
did not fill then the buckets in this game will not fill. Furthermore, we modify
$\UPO$ to query for the element before inserting it. This cannot produce a worse
result for the adversary, and so $\Prob{\game_0(A) = 1} \le \Prob{\game_1(A) =
1}$. Since $\QRYO$ and $\UPO$ calls are identical and each call is independent
of all previous calls, the adversary's probability of success is exactly the
probability of accumulating $r$ random errors from queries. Each query is a
standard random query to a cuckoo filter, so by the binomial theorem we can get
a bound of $\binom{q_T+q_U}{r}P(b,k+r)^r$. This gives the result:

$$\Adv{\erreps}_{\struct_s,r}(\advA) \le q_R \cdot \left(\frac{q_H}{2^\lambda} +
\binom{q_U+q_T}{r}P(n,m,b,k+r)^r\right).$$

%\missingqed
\end{proof}

As with the other data structures, the graphs show how each of the parameters
affects the adversarial advantage. The default parameters are $k = 4$, $m =
256$, $n = 100$, $f = 4$, $r = 10$, $q_R = 1$, and $q_H = q_T = q_U = 100$.

%\includegraphics[scale=0.75]{CuF_Fig}


\section{Count-Min Sketch Results}
\label{cms}
\begin{figure}
  \twoColsNoDivide{0.22}
  {
    \underline{$\Rep^R_K(\col)$}\\[2pt]
      $\salt \getsr \bits^\lambda$\\
      for $i$ in $[1..k]$ do\\
        $\tab \v.M[i] \gets \zeroes(m)$\\
      $\pub \gets \langle \v.M, \salt\rangle$\\
      for $x \in \col$ do \\
        $\tab \pub \gets \Up^R_K(\pub, \up_{x,1})$\\
        $\tab$if $\pub = \bot$ then return $\bot$\\
      return $\pub$
    \\[6pt]
    \underline{$\Qry^R_K(\langle \v.M, \salt\rangle,\qry_x)$}\\[2pt]
      $\v.X \gets R_K(\salt \cat x)$;
      $a \gets \infty$\\
      for $i$ in $[1..k]$ do\\
      $\tab a \gets \min(a, \v.M[i][\v.X[i]])$\\
      return $a$
  }
  {
    \underline{$\Up^R_K(\langle \v.M, \salt\rangle,\up_{x,b})$}\\[2pt]
      $\mathit{full} \gets \bigvee_{i\in[1..k]} [\hw'(M[i]) > \ell]$\\
      if $\mathit{full}$ then return $\bot$\\
      $\v.M' \gets \v.M$;
      $\v.X \gets R_K(\salt \cat x)$\\
      for $i$ in $[1..k]$ do\\
      $\tab a \gets \v.M'[i][\v.X[i]]$\\
      $\tab$ if $a = 0 \wedge b < 0$ then return $\bot$\\
      $\tab \v.M'[i][\v.X[i]] \gets a + b$\\
      $\v.M \gets \v.M'$\\
      return $\langle \v.M, \salt \rangle$
  }
  \caption{Keyed structure $\sketch[R,\ell,\lambda]$ given by
  $(\Rep^R,\Qry^R,\Up^R)$ is used to define count min-sketch variants.
  The parameters are a function $R: \keys\by\bits^* \to [m]^k$ and integers
  $\ell, \lambda \geq0$. A concrete scheme is given by a particular choice of
  parameters. The function $\hw'$, used to determine if the sketch is full, is
  defined in Section~\ref{sec:prelims}.}
  \label{fig:cms-def}
\end{figure}

The count min-sketch (CMS) data structure is designed to concisely estimate the
number of times a datum has occurred in a data stream. In other words, it is
designed to estimate the frequency of each element of a multiset.  The data
structure is similar to a Bloom filter, but instead of a length-$m$ array of
bits it uses a $k$-by-$m$ array of counters. It is designed to deal with streams
of data in the non-negative turnstile model~\cite{cormode2005improved}, which
means the sketch accomodates both insertions and deletions but does not allow
any entries to have a negative frequency. Our construction
$\sketch[R,\ell,\lambda]$ defined in Figure~\ref{fig:bf-def} involves an
$\ell$-thresholded variation of this structure. As in the case of Bloom filters,
this does not significantly change the operation of the sketch in a
non-adversarial setting, since in general~$\ell$ can be closely approximated
given knowledge of~$n$.
%
\cpnote{This isn't clear. What's~$n$? How do I approximate~$\ell$ given ``knowledge'' of~$n$?}
%
In the presence of an adversary, however, we expect this variation to provide
better security bounds.
%
\cpnote{Better than what? If you had capped instead of thresholded? This section
and the previous one should be readable on their own. If you want to bring up
points you made in the last section, that's totally fine, but you need to point
the reader to where they can read more.}

We will show that, in the \errep\ setting, the count min-sketch structure is
insecure regardless of whether $\ell$-thresholding is used or not, and
regardless of the details of the behavior of the function $R$. On the other
hand, we show that \erreps\ security is achievable even under the assumption
that a salted but unkeyed hash function is used, i.e. $\lambda > 0$ but $\keys =
\{\emptyset\}$.

\heading{Non-adaptive error bound}
%
The CMS is designed to minimize the number of elements whose
frequencies are overestimated, while still allowing for reasonably low memory
usage. For a function $\rho$ and integer $\lambda\ge0$, let
$\sketch[\id^\rho,n,\lambda] = (\Rep^\rho,\Qry^\rho,\Up^\rho)$ as defined
previously. If $\col$ is a multiset containing a total of $n$ elements
(counting duplicates as separate elements), i.e. $\col \in \Func(\bits^*,\N)$ in
our syntax, and $x \in \bits^*$ is any string, possibly but not necessarily a
member of $\col$, we define the error probability as
\begin{equation}\label{eq:bf-fp}
  \begin{aligned}
    P_{k,m}(n) =
      \Pr\big[&\rho \getsr \Func(\bits^*,[m]^k);
              \pub \getsr \Rep^\rho(\setS): \\
              &\Qry^\rho(\pub, \qry_x) > \qry_x(\setS)+\frac{en}{m} \given \pub \ne \bot
      \big] \,.
  \end{aligned}
\end{equation}
%
(Here as above, $e$ denotes the base of the natural logarithm.)
%
Informally, $P_{k,m}(n)$ is the probability that some~$x$ is overestimated by a
non-negligible amount in the representation of some $\setS$ containing a total
of $n$ elements, when a random function is used for hashing. Cormode and
Muthukrishnan~\cite{cormode2005improved} show that this probability is bounded
above by $e^{-k}$. This structure does not provide a bound for underestimation
of frequencies, since it is designed for use cases where overestimates are
considered harmful but underestimates are not.

\heading{Error function for frequency queries}
%
The count min-sketch is designed for settings where overestimation in particular
is undesirable, and so we aim to provide tight bounds on the size of
overestimates but makes no guarantees about underestimates. To make the bounds
simpler while staying conservative in our assumptions, we will use an error
function that counts \emph{any} overestimate as an error, not just overestimates
larger than some lower bound of significance. In particular, we define~$\delta$
as
%
\begin{equation}
  \delta(x, y) =
  \begin{cases}
    1 & \text{if}\ x > y \\
    0 & \text{otherwise.}
  \end{cases}
\end{equation}

\subsection{Insecurity of public sketches}\label{sec:pub-sketch-bad}

Unlike in the Bloom filter case, good security bounds cannot be achieved for a
count-min sketch in the \errep\ setting even if salts and/or private keys are
used. The insecurity is due to the relative power of the $\UPO$ oracle compared
to the Bloom filter. Not only does it allow for deletion as well as
insertion, but since updates are \emph{added} to the representation rather than
being combined with bitwise-OR, the adversary gains more information from
seeing updates occur. Because of these differences, the adversary can mount
an attack similar to the target-set coverage attack for a Bloom filter even if a
PRF is used for
hashing. First, the adversary calls $\REPO(\emptyset)$ to get an empty
representation. The adversary can then call $\UPO$ to insert an element into the
set, see exactly what the outputs of each of the hash functions are, and then
call $\UPO$ again to delete the element. By doing this repeatedly, an adversary
can determine the outputs of the PRF for $u$ different inputs using $2u$ calls
to $\UPO$. Once a sufficiently large number of PRF outputs has been determined,
the adversary can construct the test and target set used for the target-set
coverage attack (Section~\ref{sec:bad-bfs}). The adversary then calls $\UPO$ several more times to insert
each element of the test set into the sketch, and then
each element of the target set will be overestimated.

In actual use, this specific attack may not be feasible for the adversary.
However, as long as the sketch is public, the adversary can easily determine the
exact results of inserting or deleting any element just by seeing which counters
are incremented or decremented. For this reason it is not enough that the
function used to perform queries and updates is impossible for the adversary to
simulate, since the adversary can build a lookup table just by watching the
sketch as it is updated. Instead, we must require that the sketch itself be kept
secret from the adversary.

\subsection{Private, $\ell$-thresholded sketches}

Given the success of $\ell$-thresholding in the case of Bloom filters, we
continue using this tweak in the case of count min-sketches. Between
thresholding and the use of a per-representation random salt, we are able to
establish an upper bound on the number of overestimates in a count min-sketch.
However, the bound is not quite as good as in the case of a salted and
thresholded Bloom filter, which is unsurprising given the increased flexibility
provided by the update algorithm coupled with the additional
information returned by the query evaluation algorithm.
%
Formally, we consider the structure given by $\Pi = \sketch[H,\ell,\lambda]$ for
a hash function $H: \bits^* \to [m]^k$, which we will model as a random oracle.

\begin{theorem}[\erreps\ security of thresholded CMS]\label{thm:scms-erreps-th}
Let $p_\ell = ((\ell+1)/m)^k$. For all $q_R, q_T, q_H, r, t \geq 0$
it holds that
  \begin{equation*}
  \begin{aligned}
    \Adv{\erreps}_{\Pi,\delta,r}(O(t),\,&q_R,q_T,q_U,q_H,q_V) \leq \\
     & q_R \cdot \left[\frac{q_H}{2^\lambda} + e^{r'-p_\ell q_T}\left(\frac{p_\ell q_T}{r'}\right)^r\right],
  \end{aligned}
\end{equation*}
where $H$ is modeled as a random oracle, $r' = \lfloor r/(k+1) \rfloor$, and $r'
> p_\ell q_T$.
\end{theorem}

The theorem uses several reductions to gradually whittle away at the flexibility
the adversary has in performing repeated insertions, deletions, and queries to
the same elements. The $r'$ in place of $r$ in the bound comes from the fact
that, if the adversary finds that some $x$ is overestimated, it may be able to
produce as many as $k$ additional overestimates by inserting $x$. We take this
into account by automatically giving the adversary credit for all $k$ additional
overestimates as soon as it discovers the false positive. After taking this into
account, we can reduce to the standard binomial argument in which the adversary
seeks to find $r'$ overestimates by making arbitrary queries.
%
We defer the proof to Appendix~\ref{sec:proof/scms-erreps-th}.

%\begin{proof}
%  \begin{figure*}
\twoCols{0.47}
{
  \vspace{-7pt}
  \experimentv{$\game_{0}(\advA)$}\hfill\diffminus{$\game_1$}\diffplus{$\game_2$}\\[2pt]
    $M^* \gets \bot$;
    $\setS \gets \emptyset$;
    $\salt^* \getsr \bits^\lambda$\\
    $\advB^{\REPO,\QRYO,\UPO,\HASHO_1}$;
    return $\big[\sum_x \err[x] \geq r\big]$
  \\[6pt]
  \oraclev{$\HASHO_c(\salt \cat x)$}\\[2pt]
    $\vv \getsr [m]^k$\\
    if $\salt=\salt^*$ and $c = 1$ then \com{Caller is~$\advB$}\\
    \tab $\bad_1 \gets 1$; \diffplus{\diffminus{return $\vv$}}\\
    if $T[Z,x] = \bot$ then $\vv \gets T[Z,x]$\\
    $T[Z,x] \gets \vv$; return $\vv$
  \\[6pt]
  \oraclev{$\QRYO(\qry_x)$}\\[2pt]
    $X \gets \HASHO_3(\salt^* \cat x)$;
    $a \gets \infty$;
    $\setS \gets \setS \cup \{x\}$\\
    for $i$ in $[1..k]$ do\\
      $\tab a \gets \min(a, M[i][X[i]])$\\
    if $\err[x] < \delta(a,\qry_x(\col^*))$ then
          $\err[x] \gets \delta(a,\qry_x(\col^*))\diffplus{+k}$\\
    return $a$
  \\[6pt]
  \oraclev{$\REPO(\col)$}\\[2pt]
    for $i$ in $[1..k]$ do\\
      $\tab M^*[i] \gets 0^m$\\
    $\setS^* \gets \col$\\
    for $x \in \col$ do\\
    $\tab\UPO(\up_x)$\\
    return $\top$
}
{
  \vspace{-7pt}
  \oraclev{$\UPO(\up_{x,b})$}\\[2pt]
    if $w'(M^*) > \ell$ then return $\top$\\
    $M' \gets M^*$\\
    for $i$ in $[1..k]$ do\\
      $\tab$ if $M'[i][X[i]] = 0$ and $b < 0$ then return $\top$\\
      $\tab M'[i][X[i]] \gets M'[i][X[i]] + b$\\
    $M^* \gets M'$\\
    if $\err[x] \neq \bot$ then\\
      $\tab a \gets \QRYO(\qry_x)$\\
      $\tab\err[x] \gets \min(\delta(a,\qry_x(\col^*)),err[x])$\\
    $\setS^* \gets \up_{x,b}(\setS^*)$;
    return $\top$
  \vspace{6pt}\hrule\vspace{3pt}
  \oraclev{$\REPO(\col)$}\hfill\diffplus{$\game_3$}\\[2pt]
    for $i$ in $[1..k]$ do\\
      $\tab M^*[i] \gets 0^m$\\
    $\setS^* \gets \col$\\
    for $x \in \col$ do\\
    $\tab\UPO(\up_x)$\\
    \diffplusbox{for $i$ in $[1..k]$ do\\
      $\tab$while $w(M[i]) < \ell+1$ do\\
        $\tab\tab j \getsr [m]$;
        $M[i][j] \gets 1$}
    return $\top$
}
\caption{Games 0--3 for proof of Theorem~\ref{thm:scms-erreps-th}.}
\label{fig:sbf-erreps/games}
\end{figure*}

As with the proof of Theorem~\ref{thm:sbf-errep-immutable}, we derive a bound in
the \erreps1 case and then use Lemma~\ref{thm:lemma1} to move from \erreps1 to
the more general \erreps case. Because we are in the \erreps1 case, we may
assume without loss of generality that the adversary does not call $\REVO$,
since revealing the only representation automatically prevents the adversary
from winning.

We begin with a game~$\game_0$ which has identical behavior to the \erreps1
experiment for a salted CMS. As in the proofs of
Theorems~\ref{thm:sbf-errep-immutable} and~\ref{thm:sbf-erreps}, we have a
$\bad_1$ flag that gets set if the adversary ever calls $\HASHO_1$ with the
actual salt used by the representation. By an almost identical argument, we can
move to~$\game_1$, where the behavior is different only when the $\bad_1$ flag
is set, with a bound of
\begin{equation}
  \Prob{\game_0(\advA)=1} \leq
    q_H/2^\lambda + \Prob{\game_1(\advA)=1} \,.
\end{equation}

The key differences between this proof and the Bloom filter proof are the more
complex response space of $\QRYO$ ($\N$ rather than $\bits$) and the fact that
both elements of $\col$ and nonelements of $\col$ may produce errors.

%
\ignore{
It is difficult to bound the probability that, if $\QRYO(\qry_x)$ finds that $x$
is overestimated by 1, the adversary can quickly determine which elements to
re-insert into the sketch in order to increase this error beyond the $n\epsilon$
threshold of `significance'. We therefore move to~$\game_2$, where the adversary
gets credit for any response which overestimates the true value, regardless of
the magnitude of the error. Since this can only increase the probability of a
query producing an error, we have
$\Prob{\game_1(\advA) = 1} \le \Prob{\game_2(\advA) = 1}$.}

As a first step in dealing with the $\UPO$ oracle, we want to show that deletion
is never helpful for the adversary. So, for any $\advA$, we construct an
adversary $\advB$ that simulates $\advA$, forwarding all oracle queries in the
natural way, except that it ignores any $\UPO(\up_{x,-1})$ calls, i.e. any
deletions. Because deleting $x$ does not change whether $x$ is overestimated or
not, ignoring deletions does not affect whether later calls of the form
$\QRYO(\qry_x)$ will produce an error. Furthermore, if $y \neq x$, then the
probability of $\QRYO(\qry_y)$ causing an error can only increase if $x$'s
deletion is ignored, since the deletion of $x$ decreases counter values without
decreasing the true frequency of $y$. Therefore
$\Prob{\game_1(\advA) = 1} \le \Prob{\game_1(\advB) = 1}$, and we have reduced
to the case of an adversary whose $\UPO$ calls only consist of insertions.

%%

Next, we move from $\advB$ to an $\advC$ that never inserts an element more than
once. Similarly to the previous step, $\advC$ simulates $\advB$, tracking the
elements of $\col$ and forwarding $\advB$'s oracle queries in
the natural way, except that any $\UPO$ queries to insert an element already
present in $\col$ are ignored. First, inserting $x$ does not
change whether $x$ is overestimated or not, so $\advC$ ignoring the re-insertion
does not affect whether later $\QRYO(\qry_x)$ calls will produce an error. For
$y \neq x$, the fact that $\advB$ makes no deletions is key. The value of the
counters associated with $y$ by the hash functions must be at least equal to the
true frequency of $y$, and $\QRYO(\qry_y)$ will find an overestimate if these
counters are all strictly greater than the true frequency. Since updates are
deterministic, re-inserting $x$ can only increment the same counters that were
incremented by the original insertion of $x$, and so this re-insertion cannot
cause $y$ to become overestimated if it was not already. So all $\QRYO$ calls
are just as likely to produce an error for $\advC$ as they are for $\advB$, and
$\Prob{\game_1(\advB) = 1} = \Prob{\game_1(\advC) = 1}$.

As a third step, we move from~$\game_1$ to a~$\game_2$ where the adversary gains
$k+1$ `points' for finding an query which produces an
overestimate, but which prevents the adversary from querying elements of $\col$.
These extra points are necessary because, unlike in the case of a Bloom
filter, inserting an overestimated element $x$ can cause other elements of
$\col$ to become overestimated. In particular, if one of the counters
incremented by the insertion of $x$ is shared with an element of $\col$ that is
not overestimated, that element may become overestimated. However, if that
counter is shared with multiple elements of $\col$, that counter is already an
overestimate for all of the elements associated with it, and so no more than one
overestimate can be caused per counter incremented by the insertion of $x$.
Since inserting $x$ increments $k$ counters, at most $k$ errors can be caused in
this way. For any adversary $\advC$ for~$\game_2$, we can construct $D$
for~$\game_3$ that simulates $\advC$ perfectly except that it ignores any oracle
calls that would insert these elements. Since $D$ already gets credit equal to
the maximum number of errors these insertions could cause in addition to the
credit for the original overestimate, $D$ accumulates at least as many errors as
$\advC$ does, and so $\Prob{\game_1(\advC) = 1} \le \Prob{\game_2(D) = 1}$.

Analagously to the proof of~\ref{thm:sbf-erreps-th}, we now move to a
game~$\game_3$ where $\REPO$ randomly fills the sketch to capacity after
inserting the elements of $\col$, so that each row has $\ell+1$ nonzero
counters. For any $D$ for~$\game_3$ we construct $E$ for~$\game_4$ that
simulates $D$, forwarding $\REPO$, $\QRYO$, and $\HASHO_1$ calls but ignoring
$\UPO$ calls. By a very similar argument, $E$ achieves at least the same
advantage as $D$ by having a maximally full sketch as soon as $\REPO$ is called,
and so $\Prob{\game_2(D) = 1} \le \Prob{\game_3{E} = 1}$.

The probability of $E$ winning can now be given by another binomial bound. Since
each row of the sketch is a uniformly random bitmap with $\ell+1$ out of $m$
bits set to 1, the probability of any particular $\QRYO$ call causing a
collision within a single row $i$ is $(\ell+1)/m$, and the probability of a
collision in every row (i.e. an error) is $((\ell+1)/m)^k$. The adversary has a
total of $q_T$ attempts, and wins if it accumulates $\lfloor r/(k+1) \rfloor$
successes. So, letting $p_\ell = ((\ell+1)/m)^k$ and
$r' = \lfloor r/(k+1) \rfloor$, we have
\begin{equation}
   \Prob{\game_3(E)=1} \le
     \sum_{i=r'}^{q_T} \binom{q_T}{i}p_\ell^i(1-p_\ell)^{q_T-i} \,.
\end{equation}
Applying the usual Chernoff bound and applying Lemma~\ref{thm:lemma1} turns this
into the final bound of
\begin{equation}
   \Adv{\erreps}_{\Pi,\delta,r}(\advA) \leq
     q_R \cdot \left[\frac{q_H}{2^\lambda} + e^{r'-p_\ell q_T}\left(\frac{p_\ell q_T}{r'}\right)^r\right].
\end{equation}
%\end{proof}

\subsection{Discussion}

Unlike Bloom filters, there is no simple tweak that can be performed
to a count min-sketch to provide good \errep\ security bounds. In particular, it
does not achieve security even in the immutable setting, and adding a secret key
does not help.
%
However, the bound above shows that in settings where sketches can be assumed
secret it is possible to prove an upper bound on the number of overestimates an
adversary can cause. In particular, we recommend the combination of random
per-representation salts and $\ell$-thresholding in order to mitigate possible
attacks in the \erreps\ setting.

The bound we achieve is based on the same binomial bound as in the case of Bloom
filters, but has a notable difference in the form of $r'$ replacing $r$. This
negatively impacts the amount of space the filter must take up in order to
provide low error bounds, but because the scaling factor between $r$ and $r'$ is
only $k+1$, the difference should not be unacceptably extreme given reasonable
parameter choices. We also note that it is possible this bound can be improved
to reduce the impact on sketch size, since the initial factor of $q_R$ does not
have an obvious attack associated with it which would make this bound tight.
%
(The same is true, of course, of Bloom filters.)


\section{Concrete Data Structures}
\dctodo{This text needs to be broken up and massaged in to appropriate
  places in the document, e.g., in the sections on specific
  structures.  Some of it I will lift to the intro for future work,
  limitations of our formalisms.}
\label{sec:structures}
In general, each probabilistic data structure has some bound on the error size per query, assuming the adversary is not fully adaptive. In each case we want to show that allowing the adversary full adaptivity does not significantly increase the error rate. Generally, but not always, we have a response space $\mathcal{R}$ and a data object space $\mathcal{D} \subseteq Func(\mathcal{X},\mathcal{R})$ for some universe $\mathcal{X}$, so that each object in the universe is associated with a single correct response. The usual query space consists of indicator functions $\qry_x$ for $x \in \mathcal{X}$, so that $\qry_x(\col) = \col(x)$ for all $\col \in \mathcal{D}$. The update space at least consists of insertions to add a single element, but may also include deletions of single elements.

A typical case occurs with standard Bloom filters~\cite{bloomfilter}. In this case we have a response space $\mathcal{R} = \bits$ and an object space $\mathcal{D} = [\mathcal{X}]^{\le n}$ consisting of all subsets of some universe $\mathcal{X}$ that have cardinality no greater than a constant $n$. The queries are standard membership queries for each element of $\mathcal{X}$, and the update space consists of insertions. An update to insert the element $x$ is denoted by $\up_x$. These data structures are constructed using a set of hash functions, each of which maps an object to a position in an array of bits. When an object is added to the filter, either during the filter's creation or during a later update, each of the bits the object is mapped to by each of the hash functions is set to 1. To make a membership query, one can simply hash the queried object and check whether all of the associated bits are 1. Updates may only add more elements to the set: with a traditional Bloom filter structure, deletion is impossible. Because of this, Bloom filters have no false negatives, and we may assume without loss of generality that the error function is simply $d(1,0) = 1$. The size of the error a non-adaptive adversary is expected to create per query is simply equal to the false positive rate, which is on the order of $(1-e^{-\frac{kn}{m}})^k$ for an $m$-bit array with $k$ hash functions storing up to $n$ values.

Compressed Bloom filters~\cite{xxx} operate in the same way, with a false positive rate which must also take into account the degree of compression. The maximum amount we can compress a Bloom filter is determined by the probability that a given bit in the filter is 0, which is $\left(1-\frac{1}{m}\right)^{kn}$. This is closely approximated by $p = e^{-\frac{kn}{m}}$. For a given $p$, an optimal compressor will reduce an $m$-bit filter to $mH(p)$ bits, where $H$ is the entropy function $H(p) = -p\log p - (1-p)\log(1-p)$. Because of this, in order to compress the original $m$-bit filter down to $z$ bits we must have $z \ge mH(p)$. Note that an ordinary Bloom filter has false positive rate $(1-p)^k$; in the compressed case we instead have a false positive rate of $(1-p)^{-\frac{z \ln p}{nH(p)}}$.

Counting bloom filters~\cite{xxx} allow deletion by broadening the data object space to $\mathcal{D} \subseteq Func(\mathcal{X},\Z)$. With insertion and deletion both possible, we denote an insertion update for $x$ by $\up_{x,1}$ and a deletion update for $x$ by $\up_{x,0}$. Under ideal conditions the range of the data objects is not just in $\Z$ but in $\N$ specifically, but negative multiplicities may be introduced by the deletion of objects which are not actually in the set. The queries for a counting Bloom filter are still binary, operating by projecting the underlying multiset down to an ordinary set which contains a given element if and only if the multiset contains that element with multiplicity at least 1. The introduction of a deletion operation means that false positives and false negatives are both possible. Depending on the application, either of these may be worse than the other, and an error function $d$ must be chosen accordingly.

Cuckoo filters~\cite{xxx} are use the same extension of the data object space and update space, still allowing for only set membership queries along with insertion and deletion updates. Though the structure itself is different, the syntax is the same as for counting filters..

Depending on the implementation, count-min sketch~\cite{xxx} may similarly use a data object space $\mathcal{D} \subseteq Func(\mathcal{X},\Z)$ or may use a smaller space $\mathcal{D} \subseteq Func(\mathcal{X},\N)$. In the former case we have two further sub-cases, the case where all updates increment the value associated with $x \in X$ and the case where updates may either increment or decrement the associated value (but not below 0). Additionally, count-min sketch supports multiple different types of queries, in each case yielding a response in $\mathbb{Z}$. For a point query, the difference between the query and the true value is bounded by $n\epsilon$ with probability $1-\delta$, where $n$ is the sum of the true frequencies of the stream elements. The maximum error of a single query is simply $n$, in the case that an element has never actually been added to the set but has incorrectly had all its counters incremented each of $n$ times another element has been added, so the expected non-adaptive error size is bounded above by $n\epsilon(1-\delta)+n\delta = n(\delta+\epsilon-\delta\epsilon)$.  Similarly, we find that the error size of an inner product query is bounded above by $n_1n_2\epsilon(1-\delta)+n_1n_2\delta = n_1n_2(\delta+\epsilon-\delta\epsilon)$. Finally, range queries [...]

Bloomier filters~\cite{xxx} are designed to represent arbitrary functions instead of set membership or multiplicity, providing a more general response space $\mathcal{R}$ than the binary $\{0,1\}$ used with Bloom filters and altering the data object space to a set of the form $\mathcal{D} \subseteq Func(\mathcal{X},\mathcal{R})$. One of the response values is $\bot$ to indicate that value has not been associated with any element of $\mathcal{R}$, and the only type of error that can occur is that a value which should return $\bot$ instead returns some other element of $\mathcal{R}$. Again we have a situation where `false positives' are the only type of error which can occur. Update queries allow for changes in the element of $\mathcal{R}$ associated with each element of $\mathcal{X}$, though changing an element's associated value to or from $\bot$ is not permitted: neither insertion nor deletion is possible with this structure.

Stable Bloom filters~\cite{xxx} are an example of a structure in the Bloom filter family which our notions cannot work with easily. One way in which they are unique is in featuring a probabilistic update algorithm, causing objects in the filter to probabilistically decay over time. While our syntax accounts for that, the more significant difference is that a stable Bloom filter only has good accuracy guarantees once it has seen enough updates to `stabilize'. Our current security notions do not encompass this type of conditional accuracy guarantee.

Many of these structures follow the same design philosophy in order to reduce false positives. Specifically, the value associated with an element (such as a single bit in the case of a set represented by a Bloom filter, or an integer in the case of a multiset represented by count-min sketch) is stored several times. When the data is to be retrieved, we check each of the locations the data has been stored and take the minimal value across all the locations. For example, a Bloom filter returns 0 if any of the hash locations has a 0 bit, while a point query for count-min sketch returns the minimum value across all stored locations. This is as strict an approach as possible when attempting to avoid overestimating the number of times an element has been added to the underlying (multi)set, but can cause problems in an adversarial environment. The tendency to underestimate the number of times an element is represented in the (multi)set means that an adversary can cause errors by decrementing only a single one of the several values associated with the element.

Consider the case of a counting Bloom filter representing a multiset $\col$. When the filter is constructed, each of the $x \in \col$ causes several counters in the filter to be incremented. When a query is made for an element $x'$, we hash $x'$ several times and check whether the value associated with each of the hash values is positive, returning 1 if so and 0 otherwise. If $x' \not\in \col$ but $x'$ does return 1 when queried, we can update the filter to decrement the multiplicity of $x'$, decrementing each of the hash values as well. For every 1 which is decremented to 0 by this operation, a false negative is created: an object which is contained in $\col$ but which will return 0 when queried. This potentially introduces as many false negatives as there are locations associated with an object. In the case of Bloom filters this is an optimal choice, since false negatives are impossible and we can simply optimize for the lowest possible chance of a false positive.

We can consider the possibility of a more symmetric data structure which balances between resistance to false positives and resistance to false negatives. For example, consider a `balanced' counting Bloom filter which answers queries by checking whether the \emph{majority} of associated values are positive, rather than checking whether all associated values are positive. For count-min sketch, we can consider the case of a `count-average sketch' or `count-median-sketch' which returns the mean or median of the associated values rather than the minimum.

More generally, for any structure of this type with object space $\mathcal{D} \subset Func(\mathcal{U},\mathcal{V})$ for some universe $\mathcal{U}$ of objects with associated values in $\mathcal{V}$, we can take a function $f: \mathcal{V}^* \to \mathcal{R}$ and use it to define $\qry: \mathcal{U} \times Func(\mathcal{U},\mathcal{V}) \to \mathcal{R}$. We evaluate the function by evaluating $f(v_0,\ldots)$ where the $v_i$ are the several values associated with the element being queried. Each of the above structures uses $f = \min$, but many other functions could be used.

%\section{Privacy}
%\label{sec:privacy}
%
The semantic security of the immutable case cannot be easily extended to the mutable case. If the same experiment is used, but with the adversary additionally given access to an $\UPO$ oracle, the adversary can often easily learn the composition of the original set using this oracle. For example, with a standard Bloom filter, the adversary can attempt to insert an element of its choice. The representation will remain the same if and only if that element was already in the filter.

There is also the question of whether there is a natural analog to one-wayness for mutable data structures. Even if the adversary is not allowed to choose the underlying data object and must attempt to guess its contents from its public representation, in the mutable case we must assume the adversary can see the representation change over time as updates occur. We assume there is no way to know in advance which updates will be carried out, and hence no known distribution over $\mathcal{U}$. Therefore, to be cautious, we should let the adversary choose the updates. If the adversary can choose and apply any update it likes, the security notion would of course be impossible to achieve (the adversary will know what elements were added by the updates it chose to make). An alternative is to have the two experiments. In each, the adversary chooses two updates with identical leakage, and the oracle either consistently applies the first (in experiment 0) or consistently applies the second (in experiment 1). But this again leads to problems where the adversary can observe whether the representation has changed in order to determine which elements have previously been added to it.

In short, it is extremely unclear how to extend privacy notions to the mutable case without making the adversary so powerful that they can easily discern the contents of the data structures in question.



\newpage

\begin{appendix}
  %\section{Comparison to Naor-Yogev}
  %\label{sec:compare-defs}
\label{sec:errepone}
\newcommand{\ny}{\notionfont{NY}}

%\fixme{Uses $(t,q,\epsilon)$-style notion and theorem statements.  May
%be okay because this is a self contained appendix, but better if
%consistent with the rest of the document.}
In this appendix we compare the Naor-Yogev definition of
correctness~\cite{naor2015bloom} and ours. (We focus only on data structures
with what they call \emph{steady representations}, which is the only kind of
data structure we study in this work.)
%
Although their definitions is specific to the case of Bloom filters, and do not
incorporate keys, we generalize it in the natural way.

\begin{figure}[t]
  \twoCols{0.48}
  {
    \experimentv{$\Exp{\ny}_{\struct}(\advA)$}\\[2pt]
      $\setC \gets \emptyset$;
      $\ky \getsr \keys$;
      $\col \getsr \advA$\\
      $\pub \getsr \Rep_\ky(\col)$\\
      $z \getsr A^{\QRYO}(\pub)$;
      $a \gets \Qry_\ky(\pub,\qry_z)$\\
      return $\left( a = 1 \AND z \not\in \col \union \setC \right)$
  \\[6pt]
    \oraclev{$\QRYO(\qry_x)$}\\[2pt]
      $\setC \gets \setC \union \{x\}$\\
      return $\Qry_\ky(\pub, \qry_x)$
  }
  {
    \experimentv{$\Exp{\errepone}_{\struct,r}(\advA)$}\\[2pt]
      $\setC \gets \emptyset$;
      $\err \gets 0$;
      $\ky \gets \keys$;
      $\col \getsr \advA$\\
      $\pub \getsr \Rep_\ky(\col)$\\
      $\bot \getsr \advA^{\QRYO}(\pub)$\\
      return $(\err \ge r)$
    \\[6pt]
    \oraclev{$\QRYO(\qry)$}\\[2pt]
      if $\qry \in \setC$ then return $\bot$\\
      $\setC \gets \setC \cup \{\qry\}$;
      $a \gets \Qry_\ky(\pub,\qry)$\\
      if $a \neq \qry(\col)$ then $\err\gets\err+1$\\
      return~$a$
  }
  \caption{\textbf{Left:} The Naor-Yogev (\ny) definition of correctness for set-membership structure
  $\struct = (\Rep, \Qry)$.
  %
  \textbf{Right:} The \errepone notion for (set-membership) structure $\struct =
  (\Rep, \Qry)$. (Equivalent to \errep with $q_R=1$.)}
  \label{fig:ny-correct}
  \vspace{6pt}\hrule
\end{figure}
%
Let $\struct = (\Rep, \Qry)$ be a set-membership structure
for~$\elts$ with key space~$\keys$.
%i
Consider the \ny experiment defined in the left panel of
Figure~\ref{fig:ny-correct}, associated with~$\struct$ and an adversary~$\advA$.
%
First, ~$\advA$  outputs a set~$\col$ of size~$n$.
%
Next, a key~$\ky$ is chosen and the representation algorithm
is executed on~$\ky$ and $\col$, resulting in~$\pub$.
%
Then $\advA$ is executed with input~$\pub$ and with access to
an oracle~$\QRYO$ as in our definition of correctness.
%
Finally, $\advA$ outputs a value $z \in \elts$; it succeeds if
$z \not\in \col$, it never previously queried $\qry_z$
to $\QRYO$, and $\Qry_\ky(\pub, \qry_z)=1$.
%
We define the advantage of~$\advA$ in attacking~$\struct$ as
\[
  \Adv{\ny}_{\struct}(\advA) \bydef
  \Prob{\Exp{\ny}_{\struct}(\advA)=1}\,,
\]
%
and let $\Adv{\ny}_{\struct}(t, q)$ denote the maximum of
this value, taken over all $\advA$ running for at most $t$
steps and making at most $q$ oracle queries.

We remark that there are two important differences between this
definition and~\cite[Definition~2.4]{naor2015bloom}.
%
First, the adversary in Figure~\ref{fig:ny-correct} is given
the representation~$\pub$, but not the key, in the second
stage of its attack; in contrast, Naor-Yogev (implicitly)
assume the entire data structure is private.
%In fact, this is necessary in their setting, since
%they do not syntactically distinguish between the public
%representation and a \emph{key} for structure.  For example,
%they consider a construction from a pseudorandom permutation
%(PRP) for which the representation algorithm encodes a secret
%key. If the adversary were given the representation, then it
%would be impossible to appeal to the security of the underlying
%PRP.
%
Second, we allow the attacker in Figure~\ref{fig:ny-correct} to
choose the set~$\col$, whereas Naor-Yogev treat it as a
parameter of the experiment.

With these modifications in place, we have a basis for
comparing the \ny definition to our own \errep. For the sake of exposition, we
will restrict ourselves to the case of~$q_R=1$. In the right-hand panel of
Figure~\ref{fig:ny-correct}, we define a game $\errepone$, which is equivalent
to \errep when the adversary is restricted to just one~$\REPO$ query. (That is,
for any structure~$\struct$ and integers $t,q,r\geq0$, it holds that
$\Adv{\errep}_{\struct,r}(t,1,q) = \Adv{\errepone}_{\struct,r}(t,q)$.)
%
%
%Let $\Adv{\ny}_{\struct}(t,q) = \max_\advA \Adv{\ny}_{\struct}(\advA)$,
%where~$\advA$ is an adversary running in at most~$t$ steps (relative to some
%model of computation), and making at most~$q$ queries to its~$\QRYO$.
\begin{theorem}
  Let $\struct$ be a set-membership data structure.
  %
  If $\Adv{\ny}_{\struct}(t,q) \leq \epsilon$, then for any $r \geq 1$
  and $q' \leq q+1$ it holds that $\Adv{\errepone}_{\struct,r}(t,q') \leq
  q'\epsilon/r$.
\end{theorem}
\begin{proof}
  Assume that for some $q' \leq q+1$ and $r \geq 1$ there is an adversary $A$
  running in time~$t$ and making $q'$ oracle queries such that
  $\Adv{\errepone}_{\struct,r}(A) > q'\epsilon/r$.
  %
  (Note we may assume that $A$ always makes exactly $q'$ queries without loss of
  generality.)
  %
  This means that with probability at least $q'\epsilon/r$ in an execution of
  $\Exp{\errepone}_{\struct,r}(A)$, we have that $A$ makes at least $r$ distinct queries
  to $\QRYO$ for which an incorrect answer is returned.
  %
  Let $A'$ be the algorithm that simply runs~$A$, but chooses uniformly one of
  the $q'$ queries of $A$ to its $\QRYO$ oracle and outputs that query as
  its final output.
  %
  Then with probability at least $r/q' \cdot (q'\epsilon/r)=\epsilon$ the query
  chosen by $A'$ leads to an incorrect answer, and was not previously asked to
  the $\QRYO$ oracle.  Since the running time of $A'$ is at most $t$, and
  it makes at most $q'-1 \leq q$ queries to its oracle, this is a contradiction.
  \hfill\qed
\end{proof}

We remark that the above is tight, at least for $r=1$. Specifically, consider a
scheme in which every query is independently answered incorrectly with
probability~$\epsilon$. Such a scheme satisfies $\Adv{\ny}_{\struct}(t,q)
\leq \epsilon$ for any $t, q$, however an adversary making $q=1/\epsilon$
queries has constant advantage with respect to our correctness definition (for
$r=1$).

In the other direction, we show that correctness for $r=1$
easily implies correctness with respect to the Naor-Yogev
definition.
\begin{theorem}
  Let $\struct$ be a set-membership structure.
  %
  If $\Adv{\errepone}_{\struct,1}(t,q) \leq \epsilon$, then
  $\Adv{\ny}_{\struct}(t,q-1) \leq \epsilon$.
\end{theorem}
\begin{proof}
  Assume there is an adversary $A$ running in time~$t$ and making at most $q-1$
  oracle queries such that $\Adv{\ny}_{\Pi}(A) > \epsilon$.
  %
  Let $A'$ be the algorithm that simply runs~$A$, forwarding the oracle queries
  of~$A$ to its own oracle, until~$A$ terminates with output~$z$; then, $A'$
  sends~$\qry_z$ to $\QRYO$.
  %
  It is immediate that $A'$ makes at most $q$ oracle queries, and
  $\Adv{\errepone}_{\struct,1}(A') \geq \Adv{\ny}_{\struct}(A)$, a contradiction.
  \hfill\qed
\end{proof}
%
For $r > 1$, however, we have the following separation:
\begin{theorem}
  For every integer $r \geq 1$ and set~$\elts$, there is a set-membership structure
  $\struct$ for~$\elts$ for which
  %
  $\Adv{\errepone}_{\struct,r+1}(t,q)=0$ for all integers $t, q \geq 0$, but
  $\Adv{\ny}_{\struct}(O(1),0) = 1$.
\end{theorem}
\begin{proof}
  Fix an integer $r \geq 1$, a set~$\elts$, and distinct values $x_1, \ldots,
  x_r \in \elts$.
  %
  Define $\struct = (\Rep, \Qry)$ so that $\Rep(\col)$ outputs
  $\col$ and $\Qry(\col, \qry_y)$ outputs $\qry_y(\col)$ if
  $y \notin \{ x_1, \ldots, x_r \}$, but outputs $1-\qry_y(\col)$ otherwise.
  %
  This scheme always answers incorrectly for~$r$ fixed queries, and answers
  correctly for every other query.
  %
  The claim follows.
  \hfill\qed
\end{proof}

%\jnote{Just noticed something (else) odd about the NY definition: we cannot assume w.l.o.g.\
%that $A$ makes exactly $q$ queries, and in fact it is possible to have cases where increasing
%the number of queries the attacker makes can decrease its advantage! This seems like
%another drawback of the definition.}\tsnote{Really?  That seems worth
%pointing out, as part of our list of complaints.}


  % TODO(cjpatton) Add this back in.
  %\section{The standard Bloom filter bound}
  %% mitz.tex
%
% summary of the Mitzenmacher result for BFs.
\label{sec:mitz}
\newcommand{\hashfam}{\setfont{H}}
\newcommand{\sizes}{\setfont{N}}
\newcommand{\length}{m}
\newcommand{\hashes}{k}
\newcommand{\FP}{\procfont{fp}}
\def\vin(#1,#2){#1 sub}

\begin{figure}[t]
  \centering
  \hspace*{-8pt}
  \includegraphics[page=1,scale=0.625]{fig/bf-viz}
  \vspace{-24pt}
  \caption{
    Function $f(k,m,n) = (1-e^{-kn/m})^k$ for various values of~$k$,
    $m$, and $n$.
    %
    In both plots, darker lines are for larger set sizes ($n$); values
    range from $n=100$ to~$200$.
    \textbf{Left:} varying filter length ($m$) and $k=4$ (the number
    of hash functions used in Squid).
    %
    \textbf{Right:} varying number of hashes ($k$) and $m=1024$.
    %
  }
  \vspace{6pt}
  \hrule
  \label{fig:bf-viz}
\end{figure}

This appendix summarizes the results of Kirsch and
Mitzenmacher~\cite{kirsch2006less} for the false-positive probability of Bloom
filters using double hashing.
%
Our presentation is less general than theirs, but suffices for understanding the
results in our paper.

\newcommand{\mset}{\procfont{mset}}
\newcommand{\hashscheme}{\capgreekfont{\Gamma}}
\heading{Additional notation.}
%
If~$\vv$ is a vector, then let~$\mset(\vv)$ denote the multiset comprised of its
elements.
%
Following~\cite{kirsch2006less}, if~$\setM$ is a multiset, we write $i,i
\in \setM$ to mean that~$i$ appears in~$\setM$ at least twice.

\heading{Hashing schemes.}
%
Let~$\elts$ be a set. A \emph{hashing scheme}~$\hashscheme$ for~$\elts$ is a
quadruple $(\hash, \length, \hashes, \sizes)$.
%
The last element is an infinite set $\sizes \subseteq \N$ denoting the permitted
\emph{set sizes}. (We do not require that $\sizes = \N$.)
%
Functions $\length$ and~$\hashes$ specify the filter length~$\length(n)$ and
number of hashes~$\hashes(n)$ respectively for set size~$n$.
%gq
The first element is a function~$\hash\colon\N\by\elts\to\N^*$ mapping a
parameter~$n\in\sizes$ and $x\in\elts$ to a $\hashes(n)$-vector of natural
numbers, which represent bit locations in a filter of length~$\length(n)$.
%
Formally, for every $n \in \sizes$, $\hash(n,\cdot)$ is a random function
specifying a joint distribution on a collection of random variables $\{
\hash(n,x)\colon x\in\elts\}$.
%
We associate to~$\hashscheme$, set~$\col\subseteq\elts$, and
$z\in\elts\setminus\col$ a \emph{false positive event}, denoted
$\FP(\col,z)$, which occurs if for every $j \in [\hashes(n)]$ there
exists some $x\in\col$ such that $\vv_j \in \mset(\hash(n,x))$, where $\vv =
\hash(n,z)$ and $n = \setlen{\col}$.

Fix a set~$\elts$ and a hashing scheme~$\hashscheme = (\hash, \length, \hashes,
\sizes)$ for~$\elts$.  Fix $z\in\elts$ and a collection of subsets $\{ \col_n
\}_{n\in\sizes}$ of $\elts$, where $z\not\in\col_n$ and $\setlen{\col_n} = n$
for each $n \in \sizes$.
%
The following says that, if~$\hashscheme$ satisfies certain conditions, the
false positive probability converges to the approximate bound
of~\cite{broder2004network} as~$n$ increases.
%
\begin{theorem}[Lemma 4.1 of~\cite{kirsch2006less}]
  \label{thm:mitz1}
  Suppose there exist $\lambda, k\in\N$ and a function $\gamma(n) \in o(1/n)$ such
  that for every $n \in \sizes$, it holds that
  \begin{enumerate}
    \item $\hashes(n) = k$;
    \item $\length(n) = O(n)$;
    \item $\{\hash(n, x)\colon x\in\elts\}$ are independent and
      identically distributed;

    \item for every $x \in \elts$, it holds that
      \[
        \max_{i\in\length(n)} \left|
          \Prob{ i \in \mset(\hash(n, x)) } - \lambda/kn
        \right| = O(\gamma(n)) \,;
      \]

    \item and for every $x \in \elts$, it holds that
      \[
        \max_{i,j\in\length(n)}
          \Prob{ i,j \in \mset(\hash(n, x)) } = O(\gamma(n)) \,.
      \]
  \end{enumerate}
  %
  Then $\lim_{n\goesto\infty} \Prob{\FP(\col_n,z)=1} = \left(1 -
  e^{-\lambda/k}\right)^k$.
\end{theorem}
%
Kirsch and Mitzenmacher prove that the \emph{double hashing scheme} defined by
\[
  \hash(n,x)_j = 1 + (h_1(n,x) + j\cdot h_2(n,x) \mod m(n)) \,,
\]
for each $j\in[k]$, where $k(n)=k$ and $m(n)=cn$ for some constants~$k$ and $c$, and
$h_1(n,\cdot)$ and~$h_2(n,\cdot)$ are independent and identically distributed
random functions with range $[m(n)]$, satisfies the conditions of
Theorem~\ref{thm:mitz1} for $\lambda = k^2/c$ and $\gamma(n) = 1/n^2$. (See
\cite[Thorem 5.2]{kirsch2006less}.) Thus, the false-positive probability
converges to $(1-e^{-k/c})^k$ as~$n\goesto\infty$.
%
But how close to this limit is the false positive probability for some
fixed~$n$? To address this question, Kirsch and Mitzenmacher provide an analysis
of the rate of convergence.
%
\begin{theorem}[Theorem 6.1 of~\cite{kirsch2006less}]\label{thm:mitz2}
  Suppose~$\hashscheme$ satisfies the conditions of Theorem~\ref{thm:mitz1}
  for~$\lambda$, $k$, and~$\gamma(n)$. For every~$n\in\sizes$, it holds that
  \[
    \left|
      \Prob{\FP(\col_n,z)=1} - \left(1 - e^{-\lambda/k}\right)^k
    \right| = O(n\gamma(n) + 1/n) \,.
  \]
\end{theorem}
%
For the double hashing scheme in particular, we have that the false positive
probability is at most
$
  (1 - e^{-k/c})^k + O(1/n)
$
for any $n\in\sizes$. (See~\cite[Theorem 6.2]{kirsch2006less}.)

Fix integers $k,m,n,\lambda \geq 0$ and let $H \colon \bits^* \to [m]^k$ and $F
\colon \bits^\lambda\by\bits^*\to[m]^k$ be functions.
%
Let $\struct_\saltybloom = \SBF[\hashbf[H],k,m,n,\lambda]$ and $\struct_\prfbloom =
\SKBF[\hashlin[F],k,m,n,\lambda]$ as defined in Figure~\ref{fig:bf-prf}.
%
If~$H$ and~$F$ are random functions, then $\hashbf[H]$ and $\hashlin[F]$ are
both realizations of the double hashing scheme for a particular choice of~$n$.
%
From Thoerem~\ref{thm:mitz2} it follows that the false positive probability
for~$\struct_\saltybloom$ and~$\struct_\prfbloom$ is at most
$
  (1 - e^{-kn/m})^k + O(1/n).
$
(See Figure~\ref{fig:bf-viz} for a visualization of this bound.)
%
In the proof of security for~$\struct_\saltybloom$
(Theorem~\ref{thm:bf-salt-correct}), we model~$H$ as a random oracle;
%
for~$\struct_\prfbloom$ (Theorem~\ref{thm:bf-prf-correct}), we can treat~$F$ as
a random function assuming it is a good PRF.


\end{appendix}

%\begin{thebibliography}{9}
%\bibitem{bloomfilter} 
%Burton Bloom.
%\textit{Space/time trade-offs in hash coding with allowable errors.}
%Commun. ACM 13, 7 (July 1970), 422-426.
%\end{thebibliography}

%\section{Old-stuff}
%\input{toms-mutable-filters}


\end{document}

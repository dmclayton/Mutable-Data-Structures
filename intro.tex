Data structures are fundamental to essentially all areas of computer science.
The traditional approach to analyzing the correctness of a data structure is to
assume that all inputs, and all queries, are independent of any internal
randomness used to construct it.  But as highlighted by Naor and
Yogev (CRYPTO '15~\cite{naor2015bloom}), there are important use-cases in which the inputs
and queries may be chosen \emph{adversarially} and \emph{adaptively}, based on
partial information and prior observations about the data structure. Attacks of
this sort can be used to disrupt or reduce the availability of real systems
\cite{crosby2003denial,gerbet2015power,lipton1993clocked}.

Naor and Yogev (NY) formalized a notion of adversarial correctness for
Bloom-filter-like structures. A Bloom filter provides a compact representation,
which we denote by $\pub$, of a set~$\col$. The representation is a length-$m$
bit-array (initally all zeros), and elements $x \in \col$ are added to it by
computing hash values $h_1(x),h_2(x),\ldots,h_k(x)\in [m]$, then setting the
indicated array positions to~$1$.  A Bloom filter supports set membership
queries, i.e., ``is $x\in\col$?'', by hashing~$x$ and responding positively iff
all of the indicated positions hold a 1-bit.  When $h_1,\ldots,h_k$ are modeled
as random functions, and~$\col$ is independent of these, classical results
relate $|\col|,m,k$ to the probability of false-positive query
responses~\cite{broder2004network,kirsch2006less}.
%
NY revisited these results from a security perspective, by
formalizing an attack model in which the adversary specifies a
(fixed) set~$\col$ that may
depend on the hash functions, and is then allowed to adaptively query the
(immutable) representation~$\pub$ in an effort to induce errors.

This work expands upon NY in several, practically relevant ways.  To begin, our
attack model allows the adversary to adaptively \emph{update} the
collection~$\col$, thereby capturing settings in which the target data may
change over time, e.g., streaming data applications. Many data structures are
designed for this setting and natively support \emph{mutable} representations,
such as the counting filter~\cite{fan2000summary}, count-min
sketch~\cite{cormode2005improved}, cuckoo filter~\cite{fan2014cuckoo}, and
stable Bloom filter~\cite{deng2006approximately}, and our syntax is designed to
capture these data structures in addition to those structures which are
immutable once constructed.

What all of these have in common is that they are designed to \emph{compactly}
represent the data so that certain types of mutations and queries are supported,
but a small amount of error is permitted.
%
While the Bloom filter was designed to represent data collections~$\col$
that are sets, streaming data is better modeled as a multiset.  Natural
questions about multisets extend beyond set-membership; for example, an
important question in practice is \emph{how many times does~$x$ appear
in~$\col$?} As with Bloom filters, the challenge is to answer this question with
as little space consumption as possible, at the cost of admitting a reaosonable
amount of error.

Thus, our syntatic definition of data structures admits both mutability and rich
query spaces.  Formally, a data structure is a triple of algorithms $(\Rep,
\Qry, \Up)$ denoting the \emph{representation}, \emph{query-evaluation}, and
\emph{Update} algorithms, respectively. Associated to the data structure is a
set of supported query \emph{functions}~$\mathcal{Q}$, and a set~$\mathcal{U}$
of allowed update functions.  For reasons we will elucidate in a moment, all
three algorithms take a key~$\ky$ as input, and both~$\Rep$ and~$\Up$ may be
randomized.


The combination of mutability and rich query spaces has significant implications
for security. Consider the counting filter structure~\cite{fan2000summary},
which compactly (and approximately) represents an updatable multiset~$\col$.
Instead of bit array, a counting filter represeents a set~$\col$ as an array
of~$m$ integers, which we will call \emph{counters}. To add~$x$ to the set, the
representation is updated by hashing~$x$ to get $h_1(x), \ldots, h_k(x)\in[m]$,
just as we do for Bloom filters. We increment the corresponding counter in the
array for eachin of these. Unlike Bloom filters, counting filters also support
\emph{deletions}. To delete $x$, we hash~$x$ just as before, but decrement the
counters instaed of increemnting them.
%
Thus, whether~$x$ is a member of the
multiset is determined by checking that all of the counters associated with
it are non-zero.%
%
(Counters are typically floored at 0.)
%
This structure admits both false-positive \emph{and} false-negative responses.
In particular, if the representation is updated by ``removing'' an element~$y$
that does not appear in the underlying~$\col$, one or more of the counters
associated to~$x$ may be decremented, potentially causing~$x$ to become a false
negative.

Because the structures we consider may err in many ways, it is necessary to
account for errors in a general way. Thus, we parameterize our experiments by an
\emph{error-cost} function~$\delta$ such that, if the correct response to a
query is~$a$ and the data structure responds with~$a'$, the cost of the error is
$\delta(a,a') \geq 0$.
%
Security games are additionally parameterized by an total-cost threshold, and
the adversary is considered to ``win'' if the total cost of the errors it induces
is greater than this value.  As we will see, even calculating the
total cost is not straightforward.  In particular, we must determine whether
or not the cost of a given error should be carried across (adaptive,
adversarial) updates to~$\col$ and its representation.

\todo{TS}{Name and summarize the security notions. (Im)mutability being captured in
the atatck, not the syntax. Explain that \errep\ is stronger \erreps, and
therefore the more desirable target; but we find that certain structures are
only secure in the \erreps\ setting. Argue that this setting is of practical
interest.}

To summarize, our high-level contributions are: formal syntax for
mutable data structures, and two notions of adversarial
correctness for these.  Our notions capture settings in which representations
are made public, or kept private, respectively.

We exercise our syntax and notions by analyzing three important, real-world data
structures: Bloom filters~\cite{bloom1970space} (Section~\ref{sec:bloom}), count
min-sketches~\cite{cormode2005improved} (Section~\ref{sec:sketch}), and counting
filters~\cite{fan2000summary} (Section~\ref{sec:count}), summarized in
Figure~\ref{fig:tab-structures}. Each of these supports different queries and
update operations, and taken together, these structures exhibit the full range
adversarial settings our notions encounter. It may initially come as a surprise
that \emph{none of these structures meets security in our setting}, at least as
they are usually deployed. In particular, if the data being represented,
the updates, and the queries all may depend on the choice of hash function, then each of
these structures is susceptible to a class of attacks we call \emph{target-set
coverage attacks} (described in Section~\ref{sec:bad-bfs}). (These are closely
related to \emph{pollution attacks} against standard Bloom
filters~\cite{gerbet2015power}, which we will discuss in some detail.)
%
However, depending on the security setting (whether the representatino is
public, and whether updates are permitted), these structures can be refined in
simple, intuitive ways that allow us to achieve security in our setting.

\heading{Bloom filters}
%
It is well-known that standard Bloom filters do not perform well in adversarial
settings~\cite{naor2015bloom,gerbet2015power}; we first corroborate these
findings via an explicit, \erreps\ attack (Section~\ref{sec:bad-bfs}).
%
We then consider the security of several variants of the basic Bloom
filter for which we can derive correctness bounds.
%
The first idea is to generate a short, random \emph{salt}, which we prepend to
the input of the hash. Thus, instead of computing $h_i(x)$ for each $1\leq i
\leq k$ we compute $h_i(Z \cat x)$, where~$Z$ is a short (say, 128-bit) string
chosen by the representation algorithm.
%
Our first positive result is for this \emph{salted} Bloom filter in the
immutable/public-representation setting (Theorem~\ref{thm:sbf-errep-immutable}).
%
In the analysis we model the hashing algorithm as a random oracle
(ROM)~\cite{BR93}. In fact, this is closely aligned with the usual analysis,
which models the hash as a random function~\cite{broder2004network}. Because of
modeling choice, however, our security argument must account for the
precomputation performed by the adversary via the random oracle. This leads to
fairly weak bounds, which means that larger filters must be used to achieve a
reasoanble, correctness upper bound (Figure~\ref{fig:bf-bound}). On the other
hand we show that we can do much better if the representation is kept private
(Theorem~\ref{thm:sbf-erreps}). This result is also in the mutable setting.
%
We derive a similar bound for \emph{keyed} Bloom filters, which in addition to a
salt, use a pseudorandom function (PRF) instead of a hash function. This result is
in the mutable \emph{and} public-representation setting
(Theorem~\ref{thm:kbf-errep}).

Normally, Bloom filters are considered to be ``full'' when some pre-determined
\emph{capacity} is reached; indeed, Bloom filter parameters are generally chosen
as a function of this maximum capacity~\cite{kirsch2006less}.
%
We also consider a differnt notion of fullness whereby the filter is deemed full
once the Hamming weight of the filter (i.e., the number of 1s) crosses a
pre-determined \emph{threshold}. We show that this approach has substantial
analytical value: with Theorem~\ref{thm:sbf-erreps-th}, we re-consider the
security of salted BFs in the mutable/private setting and show that defining
fulllness this way allos us to exhbit substantially tighter bounds.


\heading{Count min-sketches}
Following the deep dive into Bloom filters in Section~\ref{sec:bloom}, we then
consider count min-sketches (CMSes), which provide a compact representation of a
multiset, allowing additions and deletions, and yielding approximate queires for
approximate frequency of an element in the multiset. While a count-min sketch
hashes in much the same way as a Bloom filter, it uses a 2D array of nonnegative
integer counters rather than a linear array of bits, allowing the structure to
keep track of how many times each counter is incremented.
%
Despite its similarity to Bloom filters, CMSes are not secure in the
public-representation setting, even if we use a salt or use a PRF in place of
the hash function. The fact that the adversary can see exactly which filters are
incremented or decremented each update, along with the fact that updates can be
trivially reversed (deletion undoes insertion and vice versa) allows the
adversary to mount attacks by trial and error even if it lacks the ability to
predict in advance where an element will be sent by the hash functions.
%
However, we are able to derive a good correctness in the mutable/private setting
(Theorem~\ref{thm:scms-erreps-th}), using a per-representation salt and a a
notion of ``fullness'' similar to threshold Bloom filters.

\heading{Counting filters}
We conclude in Section~\ref{sec:count} with counting filters (already alluded
to above).
%
Like CMSes, these also support both additions and deletions, but count filters
only support set-membership queries (and not multiset frequency).
%
In spite of this semantic difference, CMSes and counting filters are
so structurally related that they exhibit similar security properties
(Theorem~\ref{thm:counting-erreps}).

\heading{Conclusion}
\todo{CP}{...}

\begin{figure*}[tp]
\begin{center}
\small
  \begin{tabular}{ |p{1.75cm} | p{2.5cm} | p{2.95cm} | p{4cm} | p{3.7cm}|}
    \hline
    {\bf Structure} & {\bf Data Objects} & {\bf Supported Queries} & {\bf Supported Updates} & {\bf Parameters} \\ \hline
    \parbox[c]{1.5cm}{Bloom\\ filter (Fig.~\ref{fig:bf-def})}
          & \parbox[c][6ex]{2cm}{Sets\\$\col\subseteq \bits^*$} %, or\\ $\col \in \Func(\bits^*,\{0,1\})$}
          & $\qry_x(\col) = [x \in \col]$
          &  $\up_x(\col) = \col \cup \{x\}$
          & \parbox[c]{4cm}{$n$, max $|\col|$\\$k$, \# hash functions\\$m$, array size (bits)}
          \\\hline
     \parbox[c]{2cm}{$\ell$-thresholded\\ Bloom filter\\ (Fig.~\ref{fig:bft-def})}
          & \parbox[c]{2.5cm}{Sets\\ $\col \subseteq \bits^*$}
          & $\qry_x(\col) = [x \in \col]$
          & \parbox[c][10ex]{4cm}{$\up_x(\col) = \col \cup \{x\}$}
          & \parbox[c]{3.75cm}{$\ell$, max no. 1s in array\\$k$, \# hash functions\\$m$, array size (bits)}
          \\ \hline
     \parbox[c]{2cm}{Count-min\\ sketch (Fig.~\ref{fig:cms-def})}
          & \parbox[c]{2.5cm}{Multisets\\ $\col \in \Func(\bits^*,\N)$}
          & $\qry_x(\col) = \col(x)$
          & \parbox[c][10ex]{4cm}{$\up_{x,0}(\col)(x) = \col(x)+1$ \\ $\up_{x,1}(\col)(x) = \col(x)-1$ \\ $\up_{x,b}(\col)(y) = \col(y)$ for $x \neq y$}
          & \parbox[c]{3.75cm}{$\ell$, max no. nonzero counters\\$k$, \# hash functions and arrays\\$m$, array size (counters)}
          \\ \hline
    \parbox[c]{1.5cm}{Counting\\ filter (Fig.~\ref{fig:cbf-def})}
          & \parbox[c]{2.5cm}{Multisets\\ $\col \in \Func(\bits^*,\N)$}
          & $\qry_x(\col) = [\col(x) > 0]$
          & \parbox[c][10ex]{4cm}{$\up_{x,0}(\col)(x) = \col(x)+1$ \\ $\up_{x,1}(\col)(x) = \col(x)-1$ \\ $\up_{x,b}(\col)(y) = \col(y)$ for $x \neq y$}
          & \parbox[c]{3.5cm}{$\ell$, max no. non-zero counters\\$k$, \# hash functions\\$m$, array size (counters)}
         \\ \hline
    \ignore{\parbox[c]{1.5cm}{Cuckoo\\ filter}
          & \parbox[c]{2.5cm}{Multisets\\ $\col \in \Func(\bits^*,\N)$}
          & $\qry_x(\col) = [\col(x) > 0]$
          & \parbox[c][10ex]{4cm}{$\up_{x,0}(\col)(x) = \col(x)+1$ \\ $\up_{x,1}(\col)(x) = \col(x)-1$ \\ $\up_{x,b}(\col)(y) = \col(y)$ for $x \neq y$}
          & \parbox[c]{3.5cm}{$n$, max $|\col|$\\$m$, \# buckets\\$b$, bucket size (entries)\\$f$, fingerprint size (bits)}
          \\ \hline}
  \end{tabular}
\caption{The data structrues that we consider. Each data structure yields a
space-efficient representation of its input data object and, in the presense of
non-adaptive attacks, provides approximately correct responses to the supported
queries.  For counting filters and count-min sketches, typical
implementations prevent updates that would cause $\col(x)-1 < 0$.}
  \label{fig:structures-summary}
  \label{fig:tab-structures}
\end{center}
\end{figure*}


\heading{Future work}
\todo{TS/CP}{Organize this pile of cruft.}
%
\ignore{
It would be interesting to extend our work to the case of \emph{mutable} data
structures. Specific examples to consider here are counting Bloom
filters~\cite{fan2000summary}, scalable Bloom
filters~\cite{almeida2007scalable}, count-min
sketches~\cite{cormode2005improved}, and hierarchical Bloom
filters~\cite{zhu2004hierarchical}, to name just a few in the extended Bloom
filter family.
}
%
% NOTE(all) Removed these citations: \cite{broder2004network,nojima2009cryptographically}
Our goal is to establish foundations for the security of  data
structures. But it would certainly be interesting to analyze high-level
protocols that use these data structures, e.g.
content-distribution networks~\cite{byers2002informed}, where many servers
propagate representations of their local cache to their neighbors. The Bloom
filter family alone has a wide range of practical applications, for example in
large database query processing~\cite{broder2004network}, routing algorithms for
peer-to-peer networks~\cite{reynolds2003efficient}, protocols for establishing
linkages between medical-record databases~\cite{schnell2011novel}, fair routing
of TCP packets~\cite{feng2001stochastic}, and Bitcoin wallet
synchronization~\cite{gervais2014privacy}.
%
Analyzing higher-level primitives or protocols will require establishing
appropriate syntax and security notions for those, too; hence we leave this for
future work.

To our knowledge, ours is the first work to propose general
notions of \emph{privacy} for abstract data structures.  However, a variety of
data structures with interesting privacy properties have been proposed.  For
example, variants of Bloom filters that ensure privacy of the \emph{query} have
been studied~\cite{bellovin2004privacy,nojima2009cryptographically}.

\subsection{Related work}
\paragraph{Comparison with Naor-Yogev}
As previously noted, Naor and Yogev~\cite{naor2015bloom} were the first to
formalize adversarial correctness of Bloom filters.  Our work extends theirs
significantly in several directions. First, we consider abstract data
structures, rather than only set-membership structures.  Even with respect to
the specific case of correctness for set-membership structures, our work offers
several advantages as compared to the Naor-Yogev treatment.
%
One, our syntax distinguishes between the (secret) key and the public portion of
a data structure, an important distinction that is missing in their work.
%
Two, the Naor-Yogev definition of correctness allows the adversary to make
several queries, some of which may produce incorrect results; the attacker then
succeeds if it outputs a \emph{fresh} query that causes an error. This
separation seems arbitrary, and we propose instead a parameterized definition in
which the attacker succeeds if it can cause a certain number of (distinct)
errors during its entire execution.
%
Three, Naor and Yogev analyze the correctness of a new Bloom filter variant of
their own design. In contrast, we are mainly interested in analyzing existing,
real-world constructions to understand their security.

\paragraph{Other related works}
There is a long tradition in computer science of designing structures that
concisely (but probabilistically) represent data so as to support some set of
queries, and each of these structures has its own interesting security
characterisitcs~\cite{chazelle2004bloomier,cormode2005improved,DP08a,DF03,fredman1984storing,mironov2011sketching}.

We have already mentioned the ubiquity of Bloom filters in support of efficient
network communication and computing protocols.  They also find use in
security-critical environments, including spam filters, (distributed)
denial-of-service attack detection, and deep packet
inspection~\cite{tarkoma2012theory}.  Recently, Bloom filters were proposed as a
means of efficient certificate-revocation list (CRL)
distribution~\cite{larisch2017crlite}, a crucial component of public-key
infrastructures.

% Correctness attacks
Correctness of data structures in adversarial settings is well-motivated in the
security literature and in practice.
%
Perhaps the earliest published attack on the correctness of a data structure was due to
Lipton and Naughton~\cite{lipton1993clocked} who showed that timing analysis of
record insertion in a hash table allows an adversary to adaptively choose
elements so as to increase look-up time, effectively degrading a service's
performance.
%
Crosby and Wallach~\cite{crosby2003denial} exploited hash collisions to increase
the average URL load time in Squid, a web proxy used for caching content in
order to reduce network bandwidth.
%
More recently, Gerbet \etal~\cite{gerbet2015power} described \emph{pollution
attacks} on Bloom filters, whereby an adversary inserts a number of
adaptively-chosen elements with the goal of forcing a high false-positive rate.
Although some of their attacks exploit weak (i.e., non-cryptographic) hash
functions (as do~\cite{crosby2003denial}), their methodology is effective even
for good choices of hash functions.
%
They suggest revised parameter choices for Bloom filters (i.e., filter length and
number of hashes) in order to cope with their attacks.


\ignore{
Finally, we note that the dictionary construction considered in
Section~\ref{sec:dict} bares resemblance (at least structurally) to
\emph{garbled Bloom filters}, a tool used recently for efficient private-set
intersection~\cite{dong2013when,rindal2017improved}.
}

% NOTE(all) Below are notes and references we considered adding to related work.
\if{0}{
  Correctness in adversarial settings has been considered for broader ranges of
  data structures.  Mironov, Naor, and Segev~\cite{mironov2011sketching} studied
  a setting in which non-colluding parties interact with a third-party
  \emph{referee} in order to compute a function of their data: For example,
  whether their sets are equal, or the approximate size of their intersection.
  The parties, which share a common reference string, but otherwise do not
  communicate, send a concise \emph{sketch} of their data to the referee, who
  performs the computation and publishes the result  The adversary is modeled as
  a malicious party attempting to skew the result.
  %
  \cpnote{It would be interesting to see if there's a connection between our
  notion of correctness and their setting.}
}\fi

\if{0}{
  \emph{Secure indexes}, proposed by Eu-Jin Goh~\cite{goh2003secure}, structure
  a document so that it can be searched by keyword if the querying party has a
  special \emph{trapdoor} for the keyword. The party issuing trapdoors has a
  secret key.  \jnote{I'm not sure the work of Goh is super relevant. Or, if it
  is, then so is any searchable encryption scheme.}
  %
  \cpnote{I agree ... I included it since it was cited in the
  survey~\cite{tarkoma2012theory} as an example of a ``secure'' Bloom filter
variant.}
}\fi

\if{0}{
  Other security notions for data structures, beyond correctness and privacy,
  have been considered.  For example, \emph{authenticated data
  structures}~\cite{tamassia2003authenticated} allow a trusted third party to
  certify the validity of a query on a data set maintained by an untrusted
  server.
}\fi

\if{0}{
  We recommend reading the Naor-Yogev paper for a survey of related work and a
  discussion of related papers. Here we mention a few additional practical
  works, but stress that this only scratches the surface.
  %
  \jnote{Rather random collection of papers using Bloom filters and variants. I
  removed it for now, since it's not clear that they have any particular
relevance to us. I kept only the refs that seemed directly relevant.}
  %
  As previously mentioned, Bloom filters and their relatives are some of the most
  widely used data structures supporting set-membership queries. As examples,
  Hbase, the open-source implementation of Google's BigTable storage
  system~\cite{chang2008bigtable}, a Hadoop-based NoSql database designed to
  handle large datasets, includes an implementation of Bloom filters and
  counting Bloom filters, and he Squid proxy~\cite{fan2000summary} uses a Bloom
  filter as a ``summary'' of the set of URLs in its cache in order to improve
  latency for web-object retrieval. Reynolds and
  Vahdat~\cite{reynolds2003efficient} proposed an efficient distributed search
  engine that can be used to search for files containing a particular keyword.
  Their search engine maps the keywords of each file into a Bloom filter; a
  look-up of the keyword in the Bloom filter tells whether the node has files
  containing that keyword or not. Stochastic Fair Blue~\cite{feng2001stochastic}
  uses counting Bloom filter to manage non-responsive TCP traffic.
}\fi

\if{0}{
  \cite{gao2006internet} is an application of BFs for detecting pollution
  attacks on web caches.
  %
  \heading{Related work: attacks}
  \tsnote{Brought these back into the text just to help Chris get up to speed.}
  \begin{itemize}
    \item Niedermeyer et al., ``Cryptanalysis of Basic Bloom Filters Used for
      Privacy-Preserving Record Linkage'', breaking privacy of
      secret-hash-function Bloom filters. \tsnote{Journal of Privacy and
      Confidentiality, 2014}

    \item Gerbet, Kumar and Lauradoux, ``The power of evil choices in bloom
      filters''. \tsnote{DSN'15: Looks like a real goldmine of related work!}

    \item Crosby and Wallach, ``Denial of Service via Algorithmic Complexity
      Attacks'' \tsnote{Gives attacks on Squid}

    \item Gao et al., ``Internet Cache Pollution Attacks and Countermeasures''
  \end{itemize}

  %\ignore{
  \heading{Related work: definitions(?)}
  \begin{itemize}
    \item Nojima and Kadobayashi, ``Cryptographically Secure Bloom Filters''.
      \tsnote{Gives some security definitions for privacy. Quick scan, not super
      clear what they achieve. The definition of client-privacy (Definition 1) for
      example, makes no sense to me.  Actually, likewise for server-privacy
      (Definition 2).  Both seem vague and thoroughly underspecified.}

    \item Naor and Yogev

    \item Eujin Goh, ``Secure Indexes'' \tsnote{A secure index can
      be used for set membership.  Builds a secret-key data
      structure (an Index) that allows searching for keyword~$w$
      if one holds the trapdoor $T_w$ for~$w$, where the trapdoor
      depends on the secret key.  Main construction uses
      traditional Bloom filters and a PRF.  Construction appears
      quite inefficient, needing a very long secret key, turning a
      keyword~$w$ into a bunch of PRF outputs, and then storing
      each of these PRF outputs in the BF.  Haven't read the full
      analysis; don't know if this was ever published. }
      \jnote{Never published. I think this work uses Bloom filters
      for encrypted search; I don't remember the paper having much
      to say about Bloom filters themselves.}
  \end{itemize}

  \heading{Related work: constructions}
  \begin{itemize}
    \item Bellovin and Cheswick, ``Privacy-Enhanced Searches Using Encrypted Bloom
    Filters''.

  \item Kerschbaum , ``Public-Key Encrypted Bloom Filters with
    Applications to Supply Chain Integrity''.

  \item S\"{a}rell\"{a} et al., ``BloomCasting: Security in Bloom Filter Based Multicast''.

  \item Dong, Chen,
      Wen, ``When Private Set Intersection Meets Big Data: An Efficient and
      Scaleable Protocol'' \tsnote{``garbled bloom filters'', which actually store
      the set element by storing~$k$ xor-shares, one at each of the~$k$ hash
      indices (with care for reusing shares if hash collisions occur); also
      and``oblivious bloom intersection''}\tsnote{If the filter and the hash
      functions are public, there is a naive attack that works for some
      interesting parameters.}

  \item Tarkoma, Rothenberg, Lagerspetz ``Theory and Practice of Bloom Filters in Distributed
      Systems''
      %
      \tsnote{Great high-level coverage.  Only found preprint version though.}
      \cpnote{{ieeexplore.ieee.org/iel5/9739/6151681/05751342.pdf}}

    \item Durham, Kantarcioglu, Xue, Kuzu, Malin ``Composite Bloom Filters for
      Secure Record Linkage'' \tsnote{Per-field BFs, sampled and composed into
      single BF that is then permuted by a secret random permutation.  No clear
      statement of the problem that is being solved.  Should pull full version and
      get details.}
  \end{itemize}

  \heading{Related work: tangential}
  \begin{itemize}
    \item Chang and Mitzenmacher ``Privacy Preserving Keyword Searches on Remote Encrypted Data''.

    \item Mitzenmacher and Vadhan. ``Why Simple Hash Functions Work: Exploiting
      the Entropy in a Data Stream''.

    \item Dodis et al. ``Fuzzy Extractors: How to Generate Strong Keys from
      Biometrics and Other Noisy Data'' \tsnote{Introduces ``secure sketches'',
      which is a representation of a single-element set that is information
      theoretically private (up to some function of the min-entropy of the
      element); only tangentially related to ``sketches'' as defined in the Bloom
      filter literature.}
  \end{itemize}
}\fi




\ignore{
\tsnote{old stuff below here}

\ignore{ %possibly move elsewhere in the intro, or the opening to the
         %bloom filter section
Bloom filters are
ubiquitous in distributed computing, including web caches (e.g., Squid) and hash
tables (e.g., BigTable and Hadoop), resource and packet routing, and network
measurement. (We refer the reader to the
surveys~\cite{broder2004network,tarkoma2012theory} for a comprehensive list of
applications.) 
Bloom filters have also been modified and co-opted for security-critical
applications; perhaps unsurprisingly, things go wrong. Schnell
\etal~\cite{schnell2011novel} proposed using secretly-keyed Bloom filters in
order to enable privacy-preserving record linkage (PPRL) across data sets.  This
was deployed in medical-data applications in Australia, Brazil, Germany, and
Switzerland~\cite{niedermeyer2014cryptanalysis}. 
%As one exercise of our
%notions, we study their proposal in detail. % in Section~\ref{sec:bf-bigram}.
%
}


\heading{Data structures and their correctness.}
%
We formalize a data structure as a triple of algorithms $(\Rep, \Qry, \Up)$ denoting
the \emph{representation}, \emph{query-evaluation}, and \emph{Update} algorithms, respectively.
Associated to the data structure is a set of supported queries~$\mathcal{Q}$.
The representation algorithm is randomized, taking as input a
key~$\ky$ and a collection of data~$\col$, and returning a
representation~$\pub$ of~$\col$.  (To capture unkeyed data structures,
one sets $\ky=\varepsilon$.)
%
The deterministic query-evaluation algorithm~$\Qry$ uses~$\ky$ and $\pub$ in
order to respond to a requested query~$\qry \in \queries$ on~$\col$.
\textcolor{blue}{[[...]]}

For better efficiency, many data structures only approximately
represent the collection~$\col$. In this case, the query-evaluation
algorithm~$\Qry$ may err in its response to queries.  \oldstuff{Roughly
speaking,  our notion of adversarial correctness (\errep) captures how
difficult it is for an attacker (given $\pub$) to find~$r>0$ distinct queries on
which $\Qry$ returns an incorrect answer.}

For Bloom filters, the representation~$\pub$ includes a bit array~$M$ that
represents a set~$\col \subseteq \elts$ using hash functions
$h_1,\ldots,h_k$. The supported queries are the predicates
$\{\qry_x\}_{x\in\elts}$, where $\qry_x(\col)=1$ iff $x \in \col$. It is well
known that Bloom filters may have false positives, and their false-positive rate
for \emph{independently chosen} inputs and queries is well understood. (See
Appendix~\ref{sec:mitz}.) Our correctness notion quantitatively captures the
error rate even in the presence of an attacker that adaptively attempts to
induce errors. \textcolor{blue}{[[...]]}

We note that Naor and Yogev~\cite{naor2015bloom} were the first to formalize
adversarial correctness of Bloom filters and, indeed, their work
provided inspiration for this paper.  Our work significantly extends
theirs in several ways, as we will detail, shortly.  \textcolor{blue}{[[...]]}
% ss-rep
\if{0}{
  \anytodo{Several reviewers have made the same complaint : why these notions?
  In particular, are they interesting beyond an academic exercise?  We need to
  address this head-on.  One idea is to try to build something on top of these
  notions, but I really see that as a separate paper.  Unless we can build some
  \emph{well known} primitive... but I'm not sure what it would be, or how
  interesting.}
  %
  \cpnote{Alex Davidson's paper (ia.cr/2017/448) suggests that garbled Bloom
  filters (or some variation of them) can be used for private-set intersection. We
  could ask if privacy in our sense suffices for this application.
  But \ssrep is not the right notion since it requires a key, and \owrep is
  probably too weak. Davidson views GBFs as distributional virtual black-box
  obfuscators, which are stronger than \owrep-secure structures.}
  %
  \cpnote{To my thinking, these notions were originally devised from the
  perspective of what security properties do existing data structures admit. If
  our intention is to use these properties in order to achieve some higher-level
  goal, I don't think we have the right ones. Short of strengthening them, I think
  our best bet  is to \emph{own} our original perspective. To that end, the place
  we need the most motivation is \ssrep privacy of $\SKBF$, the PRF-based BF. See
  my comments in Section~\ref{sec:bf-prf} for two ways we've already thought of.}
}\fi

\heading{Constructions we analyze.}
%
We put our syntax and security notions to work in several case studies.
%
The brief description of Bloom filters given above was silent as to how the hash
functions $h_1, \ldots, h_k$ are chosen, and whether or not they are
public. In fact, these details have a significant effect on what notions of
security the resulting structure satisfies:
\begin{itemize}
  \item
    (Section~\ref{sec:bf}) If the hash functions are fixed and known to the
    attacker prior to the filter being constructed, the data structure offers
    neither correctness nor privacy for any practically interesting parameters.
    We show this by exhibiting explicit attacks and analyzing their performance.

  \item (Section~\ref{sec:bf-salt}) If \emph{salted} hash functions are used,
    and the adversary is given the salt only after the collection $\col$ is
    chosen, then %with modest changes to the parameters (i.e., the filter length and number of hashes), 
    the structure can achieve the same correctness guarantees in the adversarial setting as do Bloom filters in the traditional
    non-adversarial setting. 
    %(Our analysis here treats the hash functions as random oracles; the usual analysis treats them as ideal random functions.)
    We also show that this structure achieves our privacy notion of one-wayness.

  \item (Section~\ref{sec:bf-prf}) We explore a natural, keyed variant of a
    Bloom filter in which the hash functions are derived from a secretly keyed
    pseudorandom function. (This is similar to a construction proposed by Naor
    and Yogev~\cite{naor2015bloom}.) We show that this variant enjoys
    simulation-based privacy, as well as a tighter security bound for
    correctness than the salted Bloom filter.
\end{itemize}
%
\noindent
Our particular realization of the salted and secretly keyed Bloom filters
leverages results from Kirsch and Mitzenmacher~\cite{kirsch2006less} that allow
one to effectively implement $h_1,\ldots, h_k$ by making only two \emph{actual}
evaluations of an underlying hash function or PRF, respectively.
%
In addition to the comprehensive analysis of Bloom filters described above, we
also apply our definitions to:
\begin{itemize}
  \item (Section~\ref{sec:bf-bigram}) A keyed structure for privacy-preserving
    record linkage introduced by Schnell \etal~\cite{schnell2011novel}, and
    subsequently attacked by Niedermeyer
    \etal~\cite{niedermeyer2014cryptanalysis}. In our framework we are able to
    show precisely how their scheme breaks down.

  \item (Section~\ref{sec:dict}) A dictionary proposed by Charles
    and Chellapilla~\cite{charles2008bloomier2} that stores a set of~$n$
    key/value pairs, where the keys are arbitrary bitstrings and the values are
    of length at most~$m$, using just $O(mn)$ bits.
\end{itemize}
}

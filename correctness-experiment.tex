\begin{figure*}[t]
\fourColsNoDivide{0.22}{.18}{.275}{.25}
{
  \raisebox{-5pt}{\experimentv{$\Exp{\errep}_{\struct,\delta,r}(\advA)$
          \diffminus{$\Exp{\erreps}_{\struct,\delta,r}(\advA)$}}}\\[2pt]
    $\setP \gets \emptyset$; $\ct \gets 0$; $\key \getsr \keys$\\
    $i \getsr \advA^{\REPO,\UPO,\QRYO,\highlighto{\REVO}}$\\
    \diffminus{if $i \in \setP$ then return 0} \\[2.0pt]
    return $[\sum_\qry \err_i[\qry] \geq r]$
  \\[8pt]
  \diffplusbox{
  \oraclev{$\HASHO(X)$}\\[2pt]
    if $X \not\in \setX$ then return $\bot$\\
    if $T[X]=\bot$ then $T[X] \getsr \setY$\\
    return $T[X]$
  }\vspace{-4pt}
}
{
  \oraclev{$\REPO(\col)$}\\[2pt]
    $\pub \getsr \Rep_\key(\col)$\\
    if $\pub = \bot$ return $\bot$\\
    $\ct\gets\ct+1$ \\
    $\pub_\ct \gets \pub$\\
    %$\setC_\ct \gets \emptyset$\\
    $\col_\ct \gets \col$\\
    $\mathsf{rv} \gets \pub_\ct$; \diffminus{$\mathsf{rv} \gets \top$}\\
    return $\mathsf{rv}$
}
{
  \oraclev{$\UPO(i, \up)$}\\[2pt]
    %$X \gets \col^{v[i]}_i
    %$v[i] \gets v[i]+1$\\
    $\pub \getsr \Up_\key(\pub_i, \up)$\\
    if $\pub = \bot$ return $\bot$\\
    $\col_i \gets \up(\col_i)$\\
    $\pub_i \gets \pub$\\
    %$\setC_i \gets \emptyset$\\
    for $\qry$ in $\err_i$ do\\
    \tab $a \gets \Qry_K(\pub_i, \qry)$\\
    \tab if $\err_i[\qry] > \delta(a,\qry(\col_i))$ then\\
    \tab\tab$\err_i[\qry] \gets \delta(a,\qry(\col_i))$\\
    $\mathsf{rv} \gets \pub_i$; \diffminus{$\mathsf{rv} \gets \top$}\\
    return $\mathsf{rv}$
}
{
  \oraclev{$\QRYO(i, \qry)$}\\[2pt]
    \cpnote{$\setC_i$ undefined and not used anywhere}\\
    if $\qry \in \setC_i$ then return $\bot$\\
    %$\setC_i \gets \setC \union \{\qry\}$\\
    $a \gets \Qry_K(\pub_i, \qry)$\\
    if $\err_i[\qry] < \delta(a,\qry(\col_i))$ then\\
    \tab$\err_i[\qry] \gets \delta(a,\qry(\col_i))$\\
    return $a$
  \\[6pt]
  \oraclev{$\REVO(i)$}\\[2pt]
    $\setP \gets \setP \cup \{i\}$ \\
    return $\pub_i$
}
  \caption{Two notions of adversarial correctness. The \errep\ notion captures
  correctness when the representation is always known to the adversary, while
  the \erreps\ notion captures correctness when the representation is secret.
  When modeling a function $H:\setX\to\setY$ as a random oracle, the $\HASHO$
  oracle is given to $\advA$, $\Rep$, $\Up$ and $\Qry$.}
  \vspace{6pt}\hrule
  \label{fig:security}
\end{figure*}


\newcommand{\oneColCCS}[2]{
  \makebox[.3\textwidth]{
    \begin{tabular}{|@{\gamespadleft}l@{\gamespad}|}
    \hline
    \rule{0pt}{1\normalbaselineskip}
    \begin{minipage}[t]{#1\textwidth}\gamesfontsize
      #2 \vspace{6pt}
    \end{minipage} \\
    \hline
  \end{tabular}
  }
}

\ignore{
\begin{figure}[t]
\oneColCCS{0.25}
{
\oraclev{$H(X)$}\\[2pt]
$V \getsr \mathcal{V}$\\
if $\mathrm{T}[X] \neq \undefn$ then $V \gets \mathrm{T}[X]$\\
$\mathrm{T}[X]\gets V$\\
return~$V$
}
\caption{Pseudocode for a Random Oracle~$H$ with outputs in set $\mathcal{V}$}
\label{fig:ROM}
\end{figure}
}

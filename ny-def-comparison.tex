\label{sec:compare-defs}
\label{sec:errepone}
\newcommand{\ny}{\notionfont{NY}}

%\fixme{Uses $(t,q,\epsilon)$-style notion and theorem statements.  May
%be okay because this is a self contained appendix, but better if
%consistent with the rest of the document.}
In this appendix we compare the Naor-Yogev definition of
correctness~\cite{naor2015bloom} and ours. (We focus only on data structures
with what they call \emph{steady representations}, which is the only kind of
data structure we study in this work.)
%
Although their definitions is specific to the case of Bloom filters, and do not
incorporate keys, we generalize it in the natural way.

\begin{figure}[t]
  \twoCols{0.48}
  {
    \experimentv{$\Exp{\ny}_{\struct}(\advA)$}\\[2pt]
      $\setC \gets \emptyset$;
      $\ky \getsr \keys$;
      $\col \getsr \advA$\\
      $\pub \getsr \Rep_\ky(\col)$\\
      $z \getsr A^{\QRYO}(\pub)$;
      $a \gets \Qry_\ky(\pub,\qry_z)$\\
      return $\left( a = 1 \AND z \not\in \col \union \setC \right)$
  \\[6pt]
    \oraclev{$\QRYO(\qry_x)$}\\[2pt]
      $\setC \gets \setC \union \{x\}$\\
      return $\Qry_\ky(\pub, \qry_x)$
  }
  {
    \experimentv{$\Exp{\errepone}_{\struct,r}(\advA)$}\\[2pt]
      $\setC \gets \emptyset$;
      $\err \gets 0$;
      $\ky \gets \keys$;
      $\col \getsr \advA$\\
      $\pub \getsr \Rep_\ky(\col)$\\
      $\bot \getsr \advA^{\QRYO}(\pub)$\\
      return $(\err \ge r)$
    \\[6pt]
    \oraclev{$\QRYO(\qry)$}\\[2pt]
      if $\qry \in \setC$ then return $\bot$\\
      $\setC \gets \setC \cup \{\qry\}$;
      $a \gets \Qry_\ky(\pub,\qry)$\\
      if $a \neq \qry(\col)$ then $\err\gets\err+1$\\
      return~$a$
  }
  \caption{\textbf{Left:} The Naor-Yogev (\ny) definition of correctness for set-membership structure
  $\struct = (\Rep, \Qry)$.
  %
  \textbf{Right:} The \errepone notion for (set-membership) structure $\struct =
  (\Rep, \Qry)$. (Equivalent to \errep with $q_R=1$.)}
  \label{fig:ny-correct}
  \vspace{6pt}\hrule
\end{figure}
%
Let $\struct = (\Rep, \Qry)$ be a set-membership structure
for~$\elts$ with key space~$\keys$.
%i
Consider the \ny experiment defined in the left panel of
Figure~\ref{fig:ny-correct}, associated with~$\struct$ and an adversary~$\advA$.
%
First, ~$\advA$  outputs a set~$\col$ of size~$n$.
%
Next, a key~$\ky$ is chosen and the representation algorithm
is executed on~$\ky$ and $\col$, resulting in~$\pub$.
%
Then $\advA$ is executed with input~$\pub$ and with access to
an oracle~$\QRYO$ as in our definition of correctness.
%
Finally, $\advA$ outputs a value $z \in \elts$; it succeeds if
$z \not\in \col$, it never previously queried $\qry_z$
to $\QRYO$, and $\Qry_\ky(\pub, \qry_z)=1$.
%
We define the advantage of~$\advA$ in attacking~$\struct$ as
\[
  \Adv{\ny}_{\struct}(\advA) \bydef
  \Prob{\Exp{\ny}_{\struct}(\advA)=1}\,,
\]
%
and let $\Adv{\ny}_{\struct}(t, q)$ denote the maximum of
this value, taken over all $\advA$ running for at most $t$
steps and making at most $q$ oracle queries.

We remark that there are two important differences between this
definition and~\cite[Definition~2.4]{naor2015bloom}.
%
First, the adversary in Figure~\ref{fig:ny-correct} is given
the representation~$\pub$, but not the key, in the second
stage of its attack; in contrast, Naor-Yogev (implicitly)
assume the entire data structure is private.
%In fact, this is necessary in their setting, since
%they do not syntactically distinguish between the public
%representation and a \emph{key} for structure.  For example,
%they consider a construction from a pseudorandom permutation
%(PRP) for which the representation algorithm encodes a secret
%key. If the adversary were given the representation, then it
%would be impossible to appeal to the security of the underlying
%PRP.
%
Second, we allow the attacker in Figure~\ref{fig:ny-correct} to
choose the set~$\col$, whereas Naor-Yogev treat it as a
parameter of the experiment.

With these modifications in place, we have a basis for
comparing the \ny definition to our own \errep. For the sake of exposition, we
will restrict ourselves to the case of~$q_R=1$. In the right-hand panel of
Figure~\ref{fig:ny-correct}, we define a game $\errepone$, which is equivalent
to \errep when the adversary is restricted to just one~$\REPO$ query. (That is,
for any structure~$\struct$ and integers $t,q,r\geq0$, it holds that
$\Adv{\errep}_{\struct,r}(t,1,q) = \Adv{\errepone}_{\struct,r}(t,q)$.)
%
%
%Let $\Adv{\ny}_{\struct}(t,q) = \max_\advA \Adv{\ny}_{\struct}(\advA)$,
%where~$\advA$ is an adversary running in at most~$t$ steps (relative to some
%model of computation), and making at most~$q$ queries to its~$\QRYO$.
\begin{theorem}
  Let $\struct$ be a set-membership data structure.
  %
  If $\Adv{\ny}_{\struct}(t,q) \leq \epsilon$, then for any $r \geq 1$
  and $q' \leq q+1$ it holds that $\Adv{\errepone}_{\struct,r}(t,q') \leq
  q'\epsilon/r$.
\end{theorem}
\begin{proof}
  Assume that for some $q' \leq q+1$ and $r \geq 1$ there is an adversary $A$
  running in time~$t$ and making $q'$ oracle queries such that
  $\Adv{\errepone}_{\struct,r}(A) > q'\epsilon/r$.
  %
  (Note we may assume that $A$ always makes exactly $q'$ queries without loss of
  generality.)
  %
  This means that with probability at least $q'\epsilon/r$ in an execution of
  $\Exp{\errepone}_{\struct,r}(A)$, we have that $A$ makes at least $r$ distinct queries
  to $\QRYO$ for which an incorrect answer is returned.
  %
  Let $A'$ be the algorithm that simply runs~$A$, but chooses uniformly one of
  the $q'$ queries of $A$ to its $\QRYO$ oracle and outputs that query as
  its final output.
  %
  Then with probability at least $r/q' \cdot (q'\epsilon/r)=\epsilon$ the query
  chosen by $A'$ leads to an incorrect answer, and was not previously asked to
  the $\QRYO$ oracle.  Since the running time of $A'$ is at most $t$, and
  it makes at most $q'-1 \leq q$ queries to its oracle, this is a contradiction.
  \hfill\qed
\end{proof}

We remark that the above is tight, at least for $r=1$. Specifically, consider a
scheme in which every query is independently answered incorrectly with
probability~$\epsilon$. Such a scheme satisfies $\Adv{\ny}_{\struct}(t,q)
\leq \epsilon$ for any $t, q$, however an adversary making $q=1/\epsilon$
queries has constant advantage with respect to our correctness definition (for
$r=1$).

In the other direction, we show that correctness for $r=1$
easily implies correctness with respect to the Naor-Yogev
definition.
\begin{theorem}
  Let $\struct$ be a set-membership structure.
  %
  If $\Adv{\errepone}_{\struct,1}(t,q) \leq \epsilon$, then
  $\Adv{\ny}_{\struct}(t,q-1) \leq \epsilon$.
\end{theorem}
\begin{proof}
  Assume there is an adversary $A$ running in time~$t$ and making at most $q-1$
  oracle queries such that $\Adv{\ny}_{\Pi}(A) > \epsilon$.
  %
  Let $A'$ be the algorithm that simply runs~$A$, forwarding the oracle queries
  of~$A$ to its own oracle, until~$A$ terminates with output~$z$; then, $A'$
  sends~$\qry_z$ to $\QRYO$.
  %
  It is immediate that $A'$ makes at most $q$ oracle queries, and
  $\Adv{\errepone}_{\struct,1}(A') \geq \Adv{\ny}_{\struct}(A)$, a contradiction.
  \hfill\qed
\end{proof}
%
For $r > 1$, however, we have the following separation:
\begin{theorem}
  For every integer $r \geq 1$ and set~$\elts$, there is a set-membership structure
  $\struct$ for~$\elts$ for which
  %
  $\Adv{\errepone}_{\struct,r+1}(t,q)=0$ for all integers $t, q \geq 0$, but
  $\Adv{\ny}_{\struct}(O(1),0) = 1$.
\end{theorem}
\begin{proof}
  Fix an integer $r \geq 1$, a set~$\elts$, and distinct values $x_1, \ldots,
  x_r \in \elts$.
  %
  Define $\struct = (\Rep, \Qry)$ so that $\Rep(\col)$ outputs
  $\col$ and $\Qry(\col, \qry_y)$ outputs $\qry_y(\col)$ if
  $y \notin \{ x_1, \ldots, x_r \}$, but outputs $1-\qry_y(\col)$ otherwise.
  %
  This scheme always answers incorrectly for~$r$ fixed queries, and answers
  correctly for every other query.
  %
  The claim follows.
  \hfill\qed
\end{proof}

%\jnote{Just noticed something (else) odd about the NY definition: we cannot assume w.l.o.g.\
%that $A$ makes exactly $q$ queries, and in fact it is possible to have cases where increasing
%the number of queries the attacker makes can decrease its advantage! This seems like
%another drawback of the definition.}\tsnote{Really?  That seems worth
%pointing out, as part of our list of complaints.}

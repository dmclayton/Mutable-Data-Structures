\begin{figure}
  \twoColsNoDivide{0.22}
  {
    \underline{$\Rep^R_K(\col)$}\\[2pt]
      $\salt \getsr \bits^\lambda$\\
      for $i$ in $[1..k]$ do\\
        $\tab \v.M[i] \gets \zeroes(m)$\\
      $\pub \gets \langle \v.M, \salt\rangle$\\
      for $x \in \col$ do \\
        $\tab \pub \gets \Up^R_K(\pub, \up_{x,1})$\\
        $\tab$if $\pub = \bot$ then return $\bot$\\
      return $\pub$
    \\[6pt]
    \underline{$\Qry^R_K(\langle \v.M, \salt\rangle,\qry_x)$}\\[2pt]
      $\v.X \gets R_K(\salt \cat x)$;
      $a \gets \infty$\\
      for $i$ in $[1..k]$ do\\
      $\tab a \gets \min(a, \v.M[i][\v.X[i]])$\\
      return $a$
  }
  {
    \underline{$\Up^R_K(\langle \v.M, \salt\rangle,\up_{x,b})$}\\[2pt]
      $\mathit{full} \gets \bigvee_{i\in[1..k]} [\hw'(M[i]) > \ell]$\\
      if $\mathit{full}$ then return $\bot$\\
      $\v.M' \gets \v.M$;
      $\v.X \gets R_K(\salt \cat x)$\\
      for $i$ in $[1..k]$ do\\
      $\tab a \gets \v.M'[i][\v.X[i]]$\\
      $\tab$ if $a = 0 \wedge b < 0$ then return $\bot$\\
      $\tab \v.M'[i][\v.X[i]] \gets a + b$\\
      $\v.M \gets \v.M'$;
      return $\langle \v.M, \salt, c+b \rangle$
  }
  \caption{Keyed structure $\sketch[R,\ell,\lambda]$ given by
  $(\Rep^R,\Qry^R,\Up^R)$ is used to define count min-sketch variants used to
  rerpresent streams that never fill the rows of the sketch to more than $\ell$.
  The parameters are a function $R: \keys\by\bits^* \to [m]^k$ and integers
  $\ell, \lambda \geq0$. A concrete scheme is given by a particular choice of
  parameters. The function $\hw'$, used to determine if the sketch is full, is
  defined in Section~\ref{sec:prelims}.}
  \label{fig:cms-def}
\end{figure}

The count min-sketch data structure is somewhat similar to a Bloom filter in
construction, but instead of a length-$m$ array of bits it uses a $k$-by-$m$
array of counters. It is designed to deal with streams of data in the
non-negative turnstile model~\cite{cormode2005improved}, which means the sketch
accomodates both insertions and deletions but does not allow any entries to have
a negative frequency. Our construction $\sketch[R,\ell,\lambda]$ described in
Figure~\ref{fig:bf-def} involves an $\ell$-thresholded variation of this
structure. As in the case of Bloom filters, this does not significantly change
the operation of the sketch in a non-adversarial setting, since in general
$\ell$ can be closely approximated given knowledge of $n$. In the presence of an
adversary, however, we expect this variation to provide better security bounds.

In the \errep\ setting, the count min-sketch structure is insecure regardless of
whether $\ell$-thresholding is used or not, and regardless of the details of the
behavior of the function $R$. On the other hand, we show that \erreps\ security
is achievable even under the assumption that a salted but unkeyed hash function
is used, i.e. $\lambda > 0$ but $\keys = \{\emptyset\}$.

\heading{Non-adaptive error bound}
%
The count min-sketch is designed to minimize the number of elements whose
frequencies are overestimated, while still allowing for reasonably low memory
usage. For a function $\rho$ and integer $\lambda\ge0$, let
$\sketch[\id^\rho,n,\lambda] = (\Rep^\rho,\Qry^\rho,\Up^\rho)$ as defined
previously. Then for $\col \in \Func(\bits^*,\N)$ \cpnote{What's this? Do you
mean $\rho$?} a multiset containing a total
of $n$ elements (counting duplicates as separate elements), and for any
$x \in \bits^*$, we define the error probability for a count-min sketch as
\begin{equation}\label{eq:bf-fp}
  \begin{aligned}
    P_{k,m}(n) =
      \Pr\big[&\rho \getsr \Func(\bits^*,[m]^k);
              \pub \getsr \Rep^\rho(\setS): \\
              &\Qry^\rho(\pub, \qry_x) > \qry_x(\setS)+\frac{en}{m} \given \pub \ne \bot
      \big] \,.
  \end{aligned}
\end{equation}

\cpnote{What's $\col$? What if $x$ is not in the set?}

Informally, $P_{k,m}(n)$ is the probability that some~$x$ is overestimated by a
non-negligible amount in the representation of some $\setS$ containing a total
of $n$ elements, when a random function is used for hashing. Cormode and
Muthukrishnan~\cite{cormode2005improved} show that this probability is bounded
above by $e^{-k}$. CMS does not provide a bound for underestimation of
frequencies, since it is designed for use cases where overestimates are
considered harmful but underestimates are not.

\heading{Error function for frequency queries}
%
\cpnote{This isn't used anywhere, so why are we defining it?}
%
The most natural error function to correspond with the non-adaptive error bound
is the~$\delta$ defined as
\begin{equation}
  \delta(x, y) =
  \begin{cases}
    1 & \text{if}\ x - y > \frac{en}{m} \\
    0 & \text{otherwise.}
  \end{cases}
\end{equation}
%
\todo{DC}{Write $\delta$ as $\delta_{m,n}$ here and below, since it's a value
that depends on parameters of the structure.}
%
This means that a non-negligible \emph{overestimate} is considered an error,
while underestimates are not. This is in line with the definition of the CMS
data structure, which provides a tight bound on the number of accidental
overestimates which are likely to occur (in the absence of an adaptive
adversary) but does not place any bounds on underestimation. In the case of
$\ell$-thresholded sketches defined later, we will actually move to a stricter
error function than this.

\subsection{Insecurity of public sketches}

Unlike in the Bloom filter case, good security bounds cannot be achieved for a
count-min sketch in the \errep\ setting even if salts and/or private keys are
used. Given a stronger $\UPO$ oracle \cpnote{Do you mean ``given that the $\UPO$
oracle is more poweerful''? If so, then say why it is.} the adversary can mount
an attack similar to the target-set coverage attack for a Bloom filter even if a
PRF is used for
hashing. First, the adversary calls $\REPO(\emptyset)$ to get an empty
representation. The adversary can then call $\UPO$ to insert an element into the
set, see exactly what the outputs of each of the hash functions are, and then
call $\UPO$ again to delete the element. By doing this repeatedly, an adversary
can determine the outputs of the PRF for $i$ different inputs using $2i$ calls
to $\UPO$. Once a sufficiently large number of PRF outputs has been determined,
the adversary can construct the test and target set used for the target-set
coverage attack. The adversary then calls $\UPO$ several more times to insert
each element of the test set into the sketch at least $(en)/m$ times, and then
each element of the target set will be overestimated by an unacceptably large
amount.

In actual use, this specific attack may not be feasible for the adversary.
However, as long as the sketch is public, the adversary can easily determine the
exact results of inserting or deleting any element just by seeing which counters
are incremented or decremented. For this reason it is not enough that the
function used to perform queries and updates is impossible for the adversary to
simulate, since the adversary can build a lookup table just by watching the
sketch as it is updated. Instead, we must require that the sketch itself is kept
secret from the saversary.

\subsection{Private, $\ell$-thresholded sketches}

As in the case of Bloom filters, it is possible to tweak the count min-sketch
definition by placing a bound on how full the filter itself can get. In this
case, we were only able to establish an error bound for these $\ell$-thresholded
sketches. Even then, the bound is clearly weaker than for Bloom filters. We
emphasize that data structures which allow for both insertions and deletions of
arbitrary strings give potential adversaries a great deal more flexibility for
use in attacks.

Since $n$ is no longer a parameter of the structure, the error function
described above is not well-defined for a thresholded sketch. To be as
conservative as possible, and to make the proof simpler, we use an alternate
error function given by
\begin{equation}
  \delta(x, y) =
  \begin{cases}
    1 & \text{if}\ x - y > 0 \\
    0 & \text{otherwise.}
  \end{cases}
\end{equation}
In other words, the adversary gets credit for producing any degree of
overestimation, regardless of how significant.

\begin{theorem}[\erreps\ security of thresholded BFs]\label{thm:scms-erreps-th}
Let $p_\ell = ((\ell+1)/m)^k$ and $r' = \lfloor r/(k+1) \rfloor$. For integers $q_R, q_T, q_H, r, t \geq 0$ such
that $r' > p_\ell q_T$, it holds that
  \begin{equation*}
  \begin{aligned}
    \Adv{\errep}_{\Pi,\delta,r}(t, q_R,q_T,q_U,q_H,q_V) &\leq \\
     & q_R \cdot \left[\frac{q_H}{2^\lambda} + e^{r'-p_\ell q_T}\left(\frac{p_\ell q_T}{r'}\right)^r\right],
  \end{aligned}
\end{equation*}
where $H$ is modeled as a random oracle.
\end{theorem}

\begin{proof}
  \ignore{\begin{figure*}
\twoColsNoDivide{0.47}
{
  \vspace{-7pt}
  \experimentv{$\game_{0}(\advB)$}\hfill \diffminus{$\game_1$}\diffplus{$\game_2$}\\[2pt]
    $M^* \gets \bot$;
    $\salt^* \getsr \bits^\lambda$\\
    $\advB^{\REPO,\QRYO,\UPO,\HASHO_1}$;
    return $\big[\sum_x \err[x] \geq r\big]$
  \\[6pt]
  \oraclev{$\HASHO_c(\salt \cat x)$}\\[2pt]
    $\vv \getsr [m]^k$\\
    if $\salt=\salt^*$ and $c = 1$ then \com{Caller is~$\advB$}\\
    \tab $\bad_1 \gets 1$; \diffplus{\diffminus{return $\vv$}}\\
    if $T[Z,x] = \bot$ then $\vv \gets T[Z,z]$\\
    $T[Z,x] \gets \vv$; return $\vv$
}
{
  \oraclev{$\QRYO(\qry_x)$}\\[2pt]
    $X \gets \bmap_m(\HASHO_3(\salt^* \cat x))$;
    $a \gets X = M^* \AND X$\\
    if $\err[x] < \delta(a,\qry_x(\col^*))$ then
          $\err[x] \gets \delta(a,\qry_x(\col^*))$\\
    \diffplus{$\UPO(\up_x)$}\\
    return $a$
  \\[6pt]
  \oraclev{$\REPO(\col)$}\\[2pt]
    $M^* \gets \bigvee_{x \in \col} \bmap_m(\HASHO_2(\salt^* \cat x))$;
    $\setS^* \gets \col$;
    return $\top$
  \\[6pt]
  \oraclev{$\UPO(\up_x)$}\\[2pt]
    if $w(M) > \ell$ then return $\top$\\
    if $\QRYO(\qry_x) = 1$ then $\err[x] \gets 0$\\
    $M^* \gets M^* \vee \bmap_m(\HASHO_2(\salt^* \cat x))$;
    $\setS^* \gets \up_x(\setS)$;
    return $\top$
}
\caption{Games 0, 1, and 2 for proof of Theorem~\ref{thm:sbf-erreps}.}
\label{fig:sbf-erreps/games}
\end{figure*}}

As with the proof of Theorem~\ref{thm:sbf-errep-immutable}, we derive a bound in
the \erreps1 case and then use Lemma~\ref{thm:lemma1} to move from \erreps1 to
the more general \erreps case. Because we are in the \erreps1 case, we may
assume without loss of generality that the adversary does not call $\REVO$,
since revealing the only representation automatically prevents the adversary
from winning.

We begin with a game~$\game_0$ which has identical behavior to the \erreps1
experiment for a salted CMS. As in the proofs of
Theorems~\ref{thm:sbf-errep-immutable} and~\ref{thm:sbf-erreps}, we have a
$\bad_1$ flag that gets set if the adversary ever calls $\HASHO_1$ with the
actual salt used by the representation. By an almost identical argument, we can
move to~$\game_1$, where the behavior is different only when the $\bad_1$ flag
is set, with a bound of
\begin{equation}
  \Prob{\game_0(\advA)=1} \leq
    q_H/2^\lambda + \Prob{\game_1(\advA)=1} \,.
\end{equation}

The key differences between this proof and the Bloom filter proof are the more
complex response space of $\QRYO$ ($\N$ rather than $\bits$) and the possibility
of achieving an error through either overestimation or underestimation of the
actual frequency.

%

It is difficult to bound the probability that, if $\QRYO(\qry_x)$ finds that $x$
is overestimated by 1, the adversary can quickly determine which elements to
re-insert into the sketch in order to increase this error beyond the $n\epsilon$
threshold of `significance'. We therefore move to~$\game_2$, where the adversary
gets credit for any response which overestimates the true value, regardless of
the magnitude of the error. Since this can only increase the probability of a
query producing an error, we have
$\Prob{\game_1(\advA) = 1} \le \Prob{\game_2(\advA) = 1}$.

As a first step in dealing with the $\UPO$ oracle, we want to show that deletion
is never helpful for the adversary. So, for any $\advA$, we construct an
adversary $\advB$ that simulates $\advA$, forwarding all oracle queries in the
natural way, except that it ignores any $\UPO(\up_{x,-1})$ calls, i.e. any
deletions. Because deleting $x$ does not change whether $x$ is overestimated or
not, ignoring deletions does not affect whether later calls of the form
$\QRYO(\qry_x)$ will produce an error. Furthermore, if $y \neq x$, then the
probability of $\QRYO(\qry_y)$ causing an error can only increase if $x$'s
deletion is ignored, since the deletion of $x$ decreases counter values without
decreasing the true frequency of $y$. Therefore
$\Prob{\game_2(\advA) = 1} \le \Prob{\game_2(\advB) = 1}$, and we have reduced
to the case of an adversary whose $\UPO$ calls only consist of insertions.

%%

Next, we move from $\advB$ to an $\advC$ that never inserts an element more than
once. Similarly to the previous step, $\advC$ simulates $\advB$, tracking the
elements of $\col$ and forwarding $\advB$'s oracle queries in
the natural way, except that any $\UPO$ queries to insert an element already
present in $\col$ are ignored. First, inserting $x$ does not
change whether $x$ is overestimated or not, so $\advC$ ignoring the re-insertion
does not affect whether later $\QRYO(\qry_x)$ calls will produce an error. For
$y \neq x$, the fact that $\advB$ makes no deletions is key. The value of the
counters associated with $y$ by the hash functions must be at least equal to the
true frequency of $y$, and $\QRYO(\qry_y)$ will find an overestimate if these
counters are all strictly greater than the true frequency. Since updates are
deterministic, re-inserting $x$ can only increment the same counters that were
incremented by the original insertion of $x$, and so this re-insertion cannot
cause $y$ to become overestimated if it was not already. So all $\QRYO$ calls
are just as likely to produce an error for $\advC$ as they are for $\advB$, and
$\Prob{\game_2(\advB) = 1} = \Prob{\game_2(\advC) = 1}$.

As a third step, we move from~$\game_2$ to a~$\game_3$ where the adversary gains
$k+1$ `points' worth of errors for finding an query which produces an
overestimate, but which prevents the adversary from querying elements of $\col$.
These extra points are necessary because, unlike in the case of a Bloom
filter, inserting an overestimated element $x$ can cause other elements of
$\col$ to become overestimated. In particular, if one of the counters
incremented by the insertion of $x$ is shared with an element of $\col$ that is
not overestimated, that element may become overestimated. However, if that
counter is shared with multiple elements of $\col$, that counter is already an
overestimate for all of the elements associated with it, and so no more than one
overestimate can be caused per counter incremented by the insertion of $x$.
Since inserting $x$ increments $k$ counters, at most $k$ errors can be caused in
this way. For any adversary $\advC$ for~$\game_2$, we can construct $D$
for~$\game_3$ that simulates $\advC$ perfectly except that it ignores any oracle
calls that would insert these elements. Since $D$ already gets credit equal to
the maximum number of errors these insertions could cause in addition to the
credit for the original overestimate, $D$ accumulates at least as many errors as
$\advC$ does, and so $\Prob{\game_2(\advC) = 1} \le \Prob{\game_3(D) = 1}$.

Analagously to the proof of~\ref{thm:sbf-erreps-th}, we now move to a
game~$\game_4$ where $\REPO$ randomly fills the sketch to capacity after
inserting the elements of $\col$, so that each row has $\ell+1$ nonzero
counters. For any $D$ for~$\game_3$ we construct $E$ for~$\game_4$ that
simulates $D$, forwarding $\REPO$, $\QRYO$, and $\HASHO_1$ calls but ignoring
$\UPO$ calls. By a very similar argument, $E$ achieves at least the same
advantage as $D$ by having a maximally full filter as soon as $\REPO$ is called,
and so $\Prob{\game_3(D) = 1} \le \Prob{\game_4{E} = 1}$.

The probability of $E$ winning can now be given by another binomial bound. Since
each row of the sketch is a uniformly random bitmap with $\ell+1$ out of $m$
bits set to 1, the probability of any particular $\QRYO$ call causing a
collision within a single row $i$ is $(\ell+1)/m$, and the probability of a
collision in every row (i.e. an error) is $((\ell+1)/m)^k$. The adversary has a
total of $q_T$ attempts, and wins if it accumulates $\lfloor r/(k+1) \rfloor$
successes. So, letting $p_\ell = ((\ell+1)/m)^k$ and
$r' = \lfloor r/(k+1) \rfloor$, we have
\begin{equation}
   \Prob{\game_4(D)=1} \le
     \sum_{i=r'}^{q_T} \binom{q_T}{i}p_\ell^i(1-p_\ell)^{q_T-i} \,.
\end{equation}
Applying the usual Chernoff bound and applying Lemma~\ref{thm:lemma1} turns this
into the final bound of
\begin{equation}
   \Adv{\erreps}_{\Pi,\delta,r}(\advA) \leq
     q_R \cdot \left[\frac{q_H}{2^\lambda} + e^{r'-p_\ell q_T}\left(\frac{p_\ell q_T}{r'}\right)^r\right].
\end{equation}
\end{proof}
\begin{figure}
  \twoColsNoDivide{0.22}
  {
    \underline{$\Rep^R_K(\col)$}\\[2pt]
      $\salt \getsr \bits^\lambda$\\
      for $i$ in $[1..k]$ do\\
        $\tab \v.M[i] \gets \zeroes(m)$\\
      $\pub \gets \langle \v.M, \salt\rangle$\\
      for $x \in \col$ do \\
        $\tab \pub \gets \Up^R_K(\pub, \up_{x,1})$\\
        $\tab$if $\pub = \bot$ then return $\bot$\\
      return $\pub$
    \\[6pt]
    \underline{$\Qry^R_K(\langle \v.M, \salt\rangle,\qry_x)$}\\[2pt]
      $\v.X \gets R_K(\salt \cat x)$;
      $a \gets \infty$\\
      for $i$ in $[1..k]$ do\\
      $\tab a \gets \min(a, \v.M[i][\v.X[i]])$\\
      return $a$
  }
  {
    \underline{$\Up^R_K(\langle \v.M, \salt\rangle,\up_{x,b})$}\\[2pt]
      $\mathit{full} \gets \bigvee_{i\in[1..k]} [\hw'(M[i]) > \ell]$\\
      if $\mathit{full}$ then return $\bot$\\
      $\v.M' \gets \v.M$;
      $\v.X \gets R_K(\salt \cat x)$\\
      for $i$ in $[1..k]$ do\\
      $\tab a \gets \v.M'[i][\v.X[i]]$\\
      $\tab$ if $a = 0 \wedge b < 0$ then return $\bot$\\
      $\tab \v.M'[i][\v.X[i]] \gets a + b$\\
      $\v.M \gets \v.M'$\\
      return $\langle \v.M, \salt \rangle$
  }
  \caption{Keyed structure $\sketch[R,\ell,\lambda]$ given by
  $(\Rep^R,\Qry^R,\Up^R)$ is used to define count min-sketch variants.
  The parameters are a function $R: \keys\by\bits^* \to [m]^k$ and integers
  $\ell, \lambda \geq0$. A concrete scheme is given by a particular choice of
  parameters. The function $\hw'$, used to determine if the sketch is full, is
  defined in Section~\ref{sec:prelims}.}
  \label{fig:cms-def}
\end{figure}

The count min-sketch (CMS) data structure is designed to concisely estimate the
number of times a datum has occurred in a data stream. In other words, it is
designed to estimate the frequency of each element of a multiset.  The data
structure is similar to a Bloom filter, but instead of a length-$m$ array of
bits it uses a $k$-by-$m$ array of counters. It is designed to deal with streams
of data in the non-negative turnstile model~\cite{cormode2005improved}, which
means the sketch accomodates both insertions and deletions but does not allow
any entries to have a negative frequency. Our construction
$\sketch[R,\ell,\lambda]$ defined in Figure~\ref{fig:bf-def} involves an
$\ell$-thresholded variation of this structure. As in the case of Bloom filters,
this does not significantly change the operation of the sketch in a
non-adversarial setting, since in general~$\ell$ can be closely approximated
given knowledge of~$n$.
%
\cpnote{This isn't clear. What's~$n$? How do I approximate~$\ell$ given ``knowledge'' of~$n$?}
%
In the presence of an adversary, however, we expect this variation to provide
better security bounds.
%
\cpnote{Better than what? If you had capped instead of thresholded? This section
and the previous one should be readable on their own. If you want to bring up
points you made in the last section, that's totally fine, but you need to point
the reader to where they can read more.}

We will show that, in the \errep\ setting, the count min-sketch structure is
insecure regardless of whether $\ell$-thresholding is used or not, and
regardless of the details of the behavior of the function $R$. On the other
hand, we show that \erreps\ security is achievable even under the assumption
that a salted but unkeyed hash function is used, i.e. $\lambda > 0$ but $\keys =
\{\emptyset\}$.

\heading{Non-adaptive error bound}
%
The CMS is designed to minimize the number of elements whose
frequencies are overestimated, while still allowing for reasonably low memory
usage. For a function $\rho$ and integer $\lambda\ge0$, let
$\sketch[\id^\rho,n,\lambda] = (\Rep^\rho,\Qry^\rho,\Up^\rho)$ as defined
previously. If $\col$ is a multiset containing a total of $n$ elements
(counting duplicates as separate elements), i.e. $\col \in \Func(\bits^*,\N)$ in
our syntax, and $x \in \bits^*$ is any string, possibly but not necessarily a
member of $\col$, we define the error probability as
\begin{equation}\label{eq:bf-fp}
  \begin{aligned}
    P_{k,m}(n) =
      \Pr\big[&\rho \getsr \Func(\bits^*,[m]^k);
              \pub \getsr \Rep^\rho(\setS): \\
              &\Qry^\rho(\pub, \qry_x) > \qry_x(\setS)+\frac{en}{m} \given \pub \ne \bot
      \big] \,.
  \end{aligned}
\end{equation}
%
(Here as above, $e$ denotes the base of the natural logarithm.)
%
Informally, $P_{k,m}(n)$ is the probability that some~$x$ is overestimated by a
non-negligible amount in the representation of some $\setS$ containing a total
of $n$ elements, when a random function is used for hashing. Cormode and
Muthukrishnan~\cite{cormode2005improved} show that this probability is bounded
above by $e^{-k}$. This structure does not provide a bound for underestimation
of frequencies, since it is designed for use cases where overestimates are
considered harmful but underestimates are not.

\heading{Error function for frequency queries}
%
The count min-sketch is designed for settings where overestimation in particular
is undesirable, and so we aim to provide tight bounds on the size of
overestimates but makes no guarantees about underestimates. To make the bounds
simpler while staying conservative in our assumptions, we will use an error
function that counts \emph{any} overestimate as an error, not just overestimates
larger than some lower bound of significance. In particular, we define~$\delta$
as
%
\begin{equation}
  \delta(x, y) =
  \begin{cases}
    1 & \text{if}\ x > y \\
    0 & \text{otherwise.}
  \end{cases}
\end{equation}

\subsection{Insecurity of public sketches}\label{sec:pub-sketch-bad}

Unlike in the Bloom filter case, good security bounds cannot be achieved for a
count-min sketch in the \errep\ setting even if salts and/or private keys are
used. The insecurity is due to the relative power of the $\UPO$ oracle compared
to the Bloom filter. Not only does it allow for deletion as well as
insertion, but since updates are \emph{added} to the representation rather than
being combined with bitwise-OR, the adversary gains more information from
seeing updates occur. Because of these differences, the adversary can mount
an attack similar to the target-set coverage attack for a Bloom filter even if a
PRF is used for
hashing. First, the adversary calls $\REPO(\emptyset)$ to get an empty
representation. The adversary can then call $\UPO$ to insert an element into the
set, see exactly what the outputs of each of the hash functions are, and then
call $\UPO$ again to delete the element. By doing this repeatedly, an adversary
can determine the outputs of the PRF for $u$ different inputs using $2u$ calls
to $\UPO$. Once a sufficiently large number of PRF outputs has been determined,
the adversary can construct the test and target set used for the target-set
coverage attack (Section~\ref{sec:bad-bfs}). The adversary then calls $\UPO$ several more times to insert
each element of the test set into the sketch, and then
each element of the target set will be overestimated.

In actual use, this specific attack may not be feasible for the adversary.
However, as long as the sketch is public, the adversary can easily determine the
exact results of inserting or deleting any element just by seeing which counters
are incremented or decremented. For this reason it is not enough that the
function used to perform queries and updates is impossible for the adversary to
simulate, since the adversary can build a lookup table just by watching the
sketch as it is updated. Instead, we must require that the sketch itself be kept
secret from the adversary.

\subsection{Private, $\ell$-thresholded sketches}

Given the success of $\ell$-thresholding in the case of Bloom filters, we
continue using this tweak in the case of count min-sketches. Between
thresholding and the use of a per-representation random salt, we are able to
establish an upper bound on the number of overestimates in a count min-sketch.
However, the bound is not quite as good as in the case of a salted and
thresholded Bloom filter, which is unsurprising given the increased flexibility
provided by the update algorithm coupled with the additional
information returned by the query evaluation algorithm.
%
Formally, we consider the structure given by $\Pi = \sketch[H,\ell,\lambda]$ for
a hash function $H: \bits^* \to [m]^k$, which we will model as a random oracle.

\begin{theorem}[\erreps\ security of thresholded CMS]\label{thm:scms-erreps-th}
Let $p_\ell = ((\ell+1)/m)^k$. For all $q_R, q_T, q_H, r, t \geq 0$
it holds that
  \begin{equation*}
  \begin{aligned}
    \Adv{\erreps}_{\Pi,\delta,r}(O(t),\,&q_R,q_T,q_U,q_H,q_V) \leq \\
     & q_R \cdot \left[\frac{q_H}{2^\lambda} + e^{r'-p_\ell q_T}\left(\frac{p_\ell q_T}{r'}\right)^r\right],
  \end{aligned}
\end{equation*}
where $H$ is modeled as a random oracle, $r' = \lfloor r/(k+1) \rfloor$, and $r'
> p_\ell q_T$.
\end{theorem}

The theorem uses several reductions to gradually whittle away at the flexibility
the adversary has in performing repeated insertions, deletions, and queries to
the same elements. The $r'$ in place of $r$ in the bound comes from the fact
that, if the adversary finds that some $x$ is overestimated, it may be able to
produce as many as $k$ additional overestimates by inserting $x$. We take this
into account by automatically giving the adversary credit for all $k$ additional
overestimates as soon as it discovers the false positive. After taking this into
account, we can reduce to the standard binomial argument in which the adversary
seeks to find $r'$ overestimates by making arbitrary queries.
%
We defer the proof to Appendix~\ref{sec:proof/scms-erreps-th}.

%\begin{proof}
%  \begin{figure*}
\twoCols{0.47}
{
  \vspace{-7pt}
  \experimentv{$\game_{0}(\advA)$}\hfill\diffminus{$\game_1$}\diffplus{$\game_2$}\\[2pt]
    $M^* \gets \bot$;
    $\setS \gets \emptyset$;
    $\salt^* \getsr \bits^\lambda$\\
    $\advB^{\REPO,\QRYO,\UPO,\HASHO_1}$;
    return $\big[\sum_x \err[x] \geq r\big]$
  \\[6pt]
  \oraclev{$\HASHO_c(\salt \cat x)$}\\[2pt]
    $\vv \getsr [m]^k$\\
    if $\salt=\salt^*$ and $c = 1$ then \com{Caller is~$\advB$}\\
    \tab $\bad_1 \gets 1$; \diffplus{\diffminus{return $\vv$}}\\
    if $T[Z,x] = \bot$ then $\vv \gets T[Z,x]$\\
    $T[Z,x] \gets \vv$; return $\vv$
  \\[6pt]
  \oraclev{$\QRYO(\qry_x)$}\\[2pt]
    $X \gets \HASHO_3(\salt^* \cat x)$;
    $a \gets \infty$;
    $\setS \gets \setS \cup \{x\}$\\
    for $i$ in $[1..k]$ do\\
      $\tab a \gets \min(a, M[i][X[i]])$\\
    if $\err[x] < \delta(a,\qry_x(\col^*))$ then
          $\err[x] \gets \delta(a,\qry_x(\col^*))\diffplus{+k}$\\
    return $a$
  \\[6pt]
  \oraclev{$\REPO(\col)$}\\[2pt]
    for $i$ in $[1..k]$ do\\
      $\tab M^*[i] \gets 0^m$\\
    $\setS^* \gets \col$\\
    for $x \in \col$ do\\
    $\tab\UPO(\up_x)$\\
    return $\top$
}
{
  \vspace{-7pt}
  \oraclev{$\UPO(\up_{x,b})$}\\[2pt]
    if $w'(M^*) > \ell$ then return $\top$\\
    $M' \gets M^*$\\
    for $i$ in $[1..k]$ do\\
      $\tab$ if $M'[i][X[i]] = 0$ and $b < 0$ then return $\top$\\
      $\tab M'[i][X[i]] \gets M'[i][X[i]] + b$\\
    $M^* \gets M'$\\
    if $\err[x] \neq \bot$ then\\
      $\tab a \gets \QRYO(\qry_x)$\\
      $\tab\err[x] \gets \min(\delta(a,\qry_x(\col^*)),err[x])$\\
    $\setS^* \gets \up_{x,b}(\setS^*)$;
    return $\top$
  \vspace{6pt}\hrule\vspace{3pt}
  \oraclev{$\REPO(\col)$}\hfill\diffplus{$\game_3$}\\[2pt]
    for $i$ in $[1..k]$ do\\
      $\tab M^*[i] \gets 0^m$\\
    $\setS^* \gets \col$\\
    for $x \in \col$ do\\
    $\tab\UPO(\up_x)$\\
    \diffplusbox{for $i$ in $[1..k]$ do\\
      $\tab$while $w(M[i]) < \ell+1$ do\\
        $\tab\tab j \getsr [m]$;
        $M[i][j] \gets 1$}
    return $\top$
}
\caption{Games 0--3 for proof of Theorem~\ref{thm:scms-erreps-th}.}
\label{fig:sbf-erreps/games}
\end{figure*}

As with the proof of Theorem~\ref{thm:sbf-errep-immutable}, we derive a bound in
the \erreps1 case and then use Lemma~\ref{thm:lemma1} to move from \erreps1 to
the more general \erreps case. Because we are in the \erreps1 case, we may
assume without loss of generality that the adversary does not call $\REVO$,
since revealing the only representation automatically prevents the adversary
from winning.

We begin with a game~$\game_0$ which has identical behavior to the \erreps1
experiment for a salted CMS. As in the proofs of
Theorems~\ref{thm:sbf-errep-immutable} and~\ref{thm:sbf-erreps}, we have a
$\bad_1$ flag that gets set if the adversary ever calls $\HASHO_1$ with the
actual salt used by the representation. By an almost identical argument, we can
move to~$\game_1$, where the behavior is different only when the $\bad_1$ flag
is set, with a bound of
\begin{equation}
  \Prob{\game_0(\advA)=1} \leq
    q_H/2^\lambda + \Prob{\game_1(\advA)=1} \,.
\end{equation}

The key differences between this proof and the Bloom filter proof are the more
complex response space of $\QRYO$ ($\N$ rather than $\bits$) and the fact that
both elements of $\col$ and nonelements of $\col$ may produce errors.

%
\ignore{
It is difficult to bound the probability that, if $\QRYO(\qry_x)$ finds that $x$
is overestimated by 1, the adversary can quickly determine which elements to
re-insert into the sketch in order to increase this error beyond the $n\epsilon$
threshold of `significance'. We therefore move to~$\game_2$, where the adversary
gets credit for any response which overestimates the true value, regardless of
the magnitude of the error. Since this can only increase the probability of a
query producing an error, we have
$\Prob{\game_1(\advA) = 1} \le \Prob{\game_2(\advA) = 1}$.}

As a first step in dealing with the $\UPO$ oracle, we want to show that deletion
is never helpful for the adversary. So, for any $\advA$, we construct an
adversary $\advB$ that simulates $\advA$, forwarding all oracle queries in the
natural way, except that it ignores any $\UPO(\up_{x,-1})$ calls, i.e. any
deletions. Because deleting $x$ does not change whether $x$ is overestimated or
not, ignoring deletions does not affect whether later calls of the form
$\QRYO(\qry_x)$ will produce an error. Furthermore, if $y \neq x$, then the
probability of $\QRYO(\qry_y)$ causing an error can only increase if $x$'s
deletion is ignored, since the deletion of $x$ decreases counter values without
decreasing the true frequency of $y$. Therefore
$\Prob{\game_1(\advA) = 1} \le \Prob{\game_1(\advB) = 1}$, and we have reduced
to the case of an adversary whose $\UPO$ calls only consist of insertions.

%%

Next, we move from $\advB$ to an $\advC$ that never inserts an element more than
once. Similarly to the previous step, $\advC$ simulates $\advB$, tracking the
elements of $\col$ and forwarding $\advB$'s oracle queries in
the natural way, except that any $\UPO$ queries to insert an element already
present in $\col$ are ignored. First, inserting $x$ does not
change whether $x$ is overestimated or not, so $\advC$ ignoring the re-insertion
does not affect whether later $\QRYO(\qry_x)$ calls will produce an error. For
$y \neq x$, the fact that $\advB$ makes no deletions is key. The value of the
counters associated with $y$ by the hash functions must be at least equal to the
true frequency of $y$, and $\QRYO(\qry_y)$ will find an overestimate if these
counters are all strictly greater than the true frequency. Since updates are
deterministic, re-inserting $x$ can only increment the same counters that were
incremented by the original insertion of $x$, and so this re-insertion cannot
cause $y$ to become overestimated if it was not already. So all $\QRYO$ calls
are just as likely to produce an error for $\advC$ as they are for $\advB$, and
$\Prob{\game_1(\advB) = 1} = \Prob{\game_1(\advC) = 1}$.

As a third step, we move from~$\game_1$ to a~$\game_2$ where the adversary gains
$k+1$ `points' for finding an query which produces an
overestimate, but which prevents the adversary from querying elements of $\col$.
These extra points are necessary because, unlike in the case of a Bloom
filter, inserting an overestimated element $x$ can cause other elements of
$\col$ to become overestimated. In particular, if one of the counters
incremented by the insertion of $x$ is shared with an element of $\col$ that is
not overestimated, that element may become overestimated. However, if that
counter is shared with multiple elements of $\col$, that counter is already an
overestimate for all of the elements associated with it, and so no more than one
overestimate can be caused per counter incremented by the insertion of $x$.
Since inserting $x$ increments $k$ counters, at most $k$ errors can be caused in
this way. For any adversary $\advC$ for~$\game_2$, we can construct $D$
for~$\game_3$ that simulates $\advC$ perfectly except that it ignores any oracle
calls that would insert these elements. Since $D$ already gets credit equal to
the maximum number of errors these insertions could cause in addition to the
credit for the original overestimate, $D$ accumulates at least as many errors as
$\advC$ does, and so $\Prob{\game_1(\advC) = 1} \le \Prob{\game_2(D) = 1}$.

Analagously to the proof of~\ref{thm:sbf-erreps-th}, we now move to a
game~$\game_3$ where $\REPO$ randomly fills the sketch to capacity after
inserting the elements of $\col$, so that each row has $\ell+1$ nonzero
counters. For any $D$ for~$\game_3$ we construct $E$ for~$\game_4$ that
simulates $D$, forwarding $\REPO$, $\QRYO$, and $\HASHO_1$ calls but ignoring
$\UPO$ calls. By a very similar argument, $E$ achieves at least the same
advantage as $D$ by having a maximally full sketch as soon as $\REPO$ is called,
and so $\Prob{\game_2(D) = 1} \le \Prob{\game_3{E} = 1}$.

The probability of $E$ winning can now be given by another binomial bound. Since
each row of the sketch is a uniformly random bitmap with $\ell+1$ out of $m$
bits set to 1, the probability of any particular $\QRYO$ call causing a
collision within a single row $i$ is $(\ell+1)/m$, and the probability of a
collision in every row (i.e. an error) is $((\ell+1)/m)^k$. The adversary has a
total of $q_T$ attempts, and wins if it accumulates $\lfloor r/(k+1) \rfloor$
successes. So, letting $p_\ell = ((\ell+1)/m)^k$ and
$r' = \lfloor r/(k+1) \rfloor$, we have
\begin{equation}
   \Prob{\game_3(E)=1} \le
     \sum_{i=r'}^{q_T} \binom{q_T}{i}p_\ell^i(1-p_\ell)^{q_T-i} \,.
\end{equation}
Applying the usual Chernoff bound and applying Lemma~\ref{thm:lemma1} turns this
into the final bound of
\begin{equation}
   \Adv{\erreps}_{\Pi,\delta,r}(\advA) \leq
     q_R \cdot \left[\frac{q_H}{2^\lambda} + e^{r'-p_\ell q_T}\left(\frac{p_\ell q_T}{r'}\right)^r\right].
\end{equation}
%\end{proof}

\subsection{Discussion}

Unlike Bloom filters, there is no simple tweak that can be performed
to a count min-sketch to provide good \errep\ security bounds. In particular, it
does not achieve security even in the immutable setting, and adding a secret key
does not help.
%
However, the bound above shows that in settings where sketches can be assumed
secret it is possible to prove an upper bound on the number of overestimates an
adversary can cause. In particular, we recommend the combination of random
per-representation salts and $\ell$-thresholding in order to mitigate possible
attacks in the \erreps\ setting.

The bound we achieve is based on the same binomial bound as in the case of Bloom
filters, but has a notable difference in the form of $r'$ replacing $r$. This
negatively impacts the amount of space the filter must take up in order to
provide low error bounds, but because the scaling factor between $r$ and $r'$ is
only $k+1$, the difference should not be unacceptably extreme given reasonable
parameter choices. We also note that it is possible this bound can be improved
to reduce the impact on sketch size, since the initial factor of $q_R$ does not
have an obvious attack associated with it which would make this bound tight.
%
(The same is true, of course, of Bloom filters.)

\label{sec:sketch}
\begin{figure}
  \twoColsNoDivide{0.33}
  {
    \underline{$\Rep^R_K(\col)$}\\[2pt]
      $\salt \getsr \bits^\lambda$ \com{Choose a salt $\salt$}\\
      for $i$ in $[1..k]$ do\\
        $\tab \v.M[i] \gets \zeroes(m)$\\
      $\pub \gets \langle \v.M, \salt\rangle$\\
      for $x \in \col$ do \\
        $\tab \pub \gets \Up^R_K(\pub, \up_{x,1})$\\
        $\tab$if $\pub = \bot$ then return $\bot$\\
      return $\pub$
    \\[6pt]
    \underline{$\Qry^R_K(\langle \v.M, \salt\rangle,\qry_x)$}\\[2pt]
      $\v.X \gets R_K(\salt \cat x)$;
      $a \gets \infty$\\
      for $i$ in $[1..k]$ do\\
      $\tab a \gets \min(a, \v.M[i][\v.X[i]])$\\
      return $a$
  }
  {
    \underline{$\Up^R_K(\langle \v.M, \salt\rangle,\up_{x,b})$}\\[2pt]
      $\mathit{full} \gets \bigvee_{i\in[1..k]} [\hw'(M[i]) > \ell]$\\
      if $\mathit{full}$ then return $\bot$\\
      $\v.M' \gets \v.M$;
      $\v.X \gets R_K(\salt \cat x)$\\
      for $i$ in $[1..k]$ do\\
      $\tab a \gets \v.M'[i][\v.X[i]]$\\
      $\tab$ if $a = 0 \wedge b < 0$ then return $\bot$\\
      $\tab \v.M'[i][\v.X[i]] \gets a + b$\\
      $\v.M \gets \v.M'$\\
      return $\langle \v.M, \salt \rangle$
  }
  \caption{Keyed structure $\sketch[R,\ell,\lambda]$ given by
  $(\Rep^R,\Qry^R,\Up^R)$ is used to define count min-sketch variants.
  The parameters are a function $R: \keys\by\bits^* \to [m]^k$ and integers
  $\ell, \lambda \geq0$. A concrete scheme is given by a particular choice of
  parameters. The function $\hw'$, used to determine if the sketch is full, is
  defined in Section~\ref{sec:prelims}.}
  \label{fig:cms-def}
\end{figure}

The count min-sketch (CMS) data structure is designed to concisely estimate the
number of times a datum has occurred in a data stream. In other words, it is
designed to estimate the frequency of each element of a multiset. The data
structure is similar to a Bloom filter, but instead of a length-$m$ array of
bits it uses a $k$-by-$m$ array of counters. It is designed to deal with streams
of data in the non-negative turnstile model~\cite{cormode2005improved}, which
means the sketch accommodates both insertions and deletions but does not allow
any entries to have a negative frequency.
Despite this, we will see that the two structures are closely related in
terms of security properties. We show that \errep\ security is similarly
impossible, but employing $\ell$-thresholding allows for \erreps\ security with a
bound that is close to count min-sketch.

\heading{Non-adaptive error bound}
%
The CMS is designed to minimize the number of elements whose
frequencies are overestimated, while still allowing for reasonably low memory
usage. For a function $\rho$ and integer $\lambda\ge0$, let
$\sketch[\id^\rho,n,\lambda] = (\Rep^\rho,\Qry^\rho,\Up^\rho)$ as defined in
Figure~\ref{fig:cms-def}. If $\col$ is a multiset containing a total of $n$ elements
(counting duplicates as separate elements), i.e. $\col \in \Func(\bits^*,\N)$ in
our syntax, and $x \in \bits^*$ is any string, possibly but not necessarily a
member of $\col$, we define the error probability as
\begin{equation}\label{eq:bf-fp}
  \begin{aligned}
    P'_{k,m}(n) =
      \Pr\big[&\rho \getsr \Func(\bits^*,[m]^k);
              \pub \getsr \Rep^\rho(\setS): \\
              &\Qry^\rho(\pub, \qry_x) > \qry_x(\setS)+\frac{en}{m} \given \pub \ne \bot
      \big] \,.
  \end{aligned}
\end{equation}
%
(Here as above, $e$ denotes the base of the natural logarithm.)
%
Informally, $P'_{k,m}(n)$ is the probability that some~$x$ is overestimated by a
non-negligible amount in the representation of some $\setS$ containing a total
of $n$ elements, when a random function is used for hashing. Cormode and
Muthukrishnan~\cite{cormode2005improved} show that this probability is bounded
above by $e^{-k}$. This structure does not provide a bound for underestimation
of frequencies, since it is designed for use cases where overestimates are
considered harmful but underestimates are not.

\heading{Error function for frequency queries}
%
The count min-sketch is designed for settings where overestimation in particular
is undesirable, and so we aim to provide tight bounds on the size of
overestimates but makes no guarantees about underestimates. To make the bounds
simpler while staying conservative in our assumptions, we will use an error
function that counts \emph{any} overestimate as an error, not just overestimates
larger than some lower bound of significance. In particular, we define~$\delta$
as
%
\begin{equation}
  \delta(x, y) =
  \begin{cases}
    1 & \text{if}\ x > y \\
    0 & \text{otherwise.}
  \end{cases}
\end{equation}

Note that other $\delta(x,y)$ may be preferable in some applications. For
example, if the degree of error is significant, it may be desirable to use a $\delta$
which only counts as an error if $x$ and $y$ differ by more than some threshold value,
or to use a function such as $\delta(x,y) = |x-y|$. In this paper we use this
error bound because it is in some sense the most conservative, counting any
overestimate as an error.

\subsection{Insecurity of public sketches}

The count min-sketch structure necessarily fails to satisfy \errep\
correctness for the same reasons as in the case of a counting filter. In
particular, the adversary can call $\REPO(\emptyset)$ to receive an empty
representation, insert an element $x$, observe which counters are incremented by
this insertion, and then delete $x$. (Again, this is possible only because the
sketch is public). By doing this repeatedly, the adversary can
gain information about which elements overlap with which combinations of other
elements, and can therefore mount the same attack described in
Section~\ref{sec:pub-count-bad}.

\subsection{Private, $\ell$-thresholded sketches}
\label{sec:sketch-proof}

Given the success of $\ell$-thresholding in the case of Bloom filters, we
continue using this tweak in the case of count min-sketches. Between
thresholding and the use of a per-representation random salt, we are able to
establish an upper bound on the number of overestimates in a count min-sketch.
However, the bound is not quite as good as in the case of a salted and
thresholded Bloom filter, which is unsurprising given the increased flexibility
provided by the update algorithm coupled with the additional
information returned by the query evaluation algorithm.
%
Formally, we consider the structure given by $\Pi = \sketch[H,\ell,\lambda]$ for
a hash function $H: \bits^* \to [m]^k$, which we will model as a random oracle.

\begin{theorem}[\erreps\ security of thresholded CMS]\label{thm:scms-erreps-th}
Let $p_\ell = ((\ell+1)/m)^k$. For all $q_R, q_T, q_U, q_H, q_V, r, t \geq 0$
it holds that
  \begin{equation*}
  \begin{aligned}
    \Adv{\erreps}_{\Pi,\delta,r}(O(t),\,&q_R,q_T,q_U,q_H,q_V) \leq q_R \cdot \left[\frac{q_H}{2^\lambda} + e^{r'-p_\ell q_T}\left(\frac{p_\ell q_T}{r'}\right)^r\right],
  \end{aligned}
\end{equation*}
where $H$ is modeled as a random oracle, $r' = \lfloor r/(k+1) \rfloor$, and $r'
> p_\ell q_T$.
\end{theorem}

The proof uses a game-playing argument in which we gradually whittle away at the flexibility
the adversary has in performing repeated insertions, deletions, and queries to
the same elements. The $r'$ in place of $r$ in the bound comes from the fact
that, if the adversary finds that some $x$ is overestimated, it may be able to
produce as many as $k$ additional overestimates by inserting $x$. We take this
into account by automatically giving the adversary credit for all $k$ additional
overestimates as soon as it discovers the false positive. After taking this into
account, we can reduce to the standard binomial argument in which the adversary
seeks to find $r'$ overestimates by making arbitrary queries.

\proc{
  \begin{proof}[Proof Sketch of Theorem~\ref{thm:scms-erreps-th}]
  \ignore{\begin{figure*}
\twoColsNoDivide{0.47}
{
  \vspace{-7pt}
  \experimentv{$\game_{0}(\advB)$}\hfill \diffminus{$\game_1$}\diffplus{$\game_2$}\\[2pt]
    $M^* \gets \bot$;
    $\salt^* \getsr \bits^\lambda$\\
    $\advB^{\REPO,\QRYO,\UPO,\HASHO_1}$;
    return $\big[\sum_x \err[x] \geq r\big]$
  \\[6pt]
  \oraclev{$\HASHO_c(\salt \cat x)$}\\[2pt]
    $\vv \getsr [m]^k$\\
    if $\salt=\salt^*$ and $c = 1$ then \com{Caller is~$\advB$}\\
    \tab $\bad_1 \gets 1$; \diffplus{\diffminus{return $\vv$}}\\
    if $T[Z,x] = \bot$ then $\vv \gets T[Z,z]$\\
    $T[Z,x] \gets \vv$; return $\vv$
}
{
  \oraclev{$\QRYO(\qry_x)$}\\[2pt]
    $X \gets \bmap_m(\HASHO_3(\salt^* \cat x))$;
    $a \gets X = M^* \AND X$\\
    if $\err[x] < \delta(a,\qry_x(\col^*))$ then
          $\err[x] \gets \delta(a,\qry_x(\col^*))$\\
    \diffplus{$\UPO(\up_x)$}\\
    return $a$
  \\[6pt]
  \oraclev{$\REPO(\col)$}\\[2pt]
    $M^* \gets \bigvee_{x \in \col} \bmap_m(\HASHO_2(\salt^* \cat x))$;
    $\setS^* \gets \col$;
    return $\top$
  \\[6pt]
  \oraclev{$\UPO(\up_x)$}\\[2pt]
    if $w(M) > \ell$ then return $\top$\\
    if $\QRYO(\qry_x) = 1$ then $\err[x] \gets 0$\\
    $M^* \gets M^* \vee \bmap_m(\HASHO_2(\salt^* \cat x))$;
    $\setS^* \gets \up_x(\setS)$;
    return $\top$
}
\caption{Games 0, 1, and 2 for proof of Theorem~\ref{thm:sbf-erreps}.}
\label{fig:sbf-erreps/games}
\end{figure*}}

As with the proof of Theorem~\ref{thm:sbf-errep-immutable}, we derive a bound in
the \erreps1 case and then use Lemma~\ref{thm:lemma1} to move from \erreps1 to
the more general \erreps case. Because we are in the \erreps1 case, we may
assume without loss of generality that the adversary does not call $\REVO$,
since revealing the only representation automatically prevents the adversary
from winning.

We begin with a game~$\game_0$ which has identical behavior to the \erreps1
experiment for a salted CMS. As in the proofs of
Theorems~\ref{thm:sbf-errep-immutable} and~\ref{thm:sbf-erreps}, we have a
$\bad_1$ flag that gets set if the adversary ever calls $\HASHO_1$ with the
actual salt used by the representation. By an almost identical argument, we can
move to~$\game_1$, where the behavior is different only when the $\bad_1$ flag
is set, with a bound of
\begin{equation}
  \Prob{\game_0(\advA)=1} \leq
    q_H/2^\lambda + \Prob{\game_1(\advA)=1} \,.
\end{equation}

The key differences between this proof and the Bloom filter proof are the more
complex response space of $\QRYO$ ($\N$ rather than $\bits$) and the possibility
of achieving an error through either overestimation or underestimation of the
actual frequency.

%

It is difficult to bound the probability that, if $\QRYO(\qry_x)$ finds that $x$
is overestimated by 1, the adversary can quickly determine which elements to
re-insert into the sketch in order to increase this error beyond the $n\epsilon$
threshold of `significance'. We therefore move to~$\game_2$, where the adversary
gets credit for any response which overestimates the true value, regardless of
the magnitude of the error. Since this can only increase the probability of a
query producing an error, we have
$\Prob{\game_1(\advA) = 1} \le \Prob{\game_2(\advA) = 1}$.

As a first step in dealing with the $\UPO$ oracle, we want to show that deletion
is never helpful for the adversary. So, for any $\advA$, we construct an
adversary $\advB$ that simulates $\advA$, forwarding all oracle queries in the
natural way, except that it ignores any $\UPO(\up_{x,-1})$ calls, i.e. any
deletions. Because deleting $x$ does not change whether $x$ is overestimated or
not, ignoring deletions does not affect whether later calls of the form
$\QRYO(\qry_x)$ will produce an error. Furthermore, if $y \neq x$, then the
probability of $\QRYO(\qry_y)$ causing an error can only increase if $x$'s
deletion is ignored, since the deletion of $x$ decreases counter values without
decreasing the true frequency of $y$. Therefore
$\Prob{\game_2(\advA) = 1} \le \Prob{\game_2(\advB) = 1}$, and we have reduced
to the case of an adversary whose $\UPO$ calls only consist of insertions.

%%

Next, we move from $\advB$ to an $\advC$ that never inserts an element more than
once. Similarly to the previous step, $\advC$ simulates $\advB$, tracking the
elements of $\col$ and forwarding $\advB$'s oracle queries in
the natural way, except that any $\UPO$ queries to insert an element already
present in $\col$ are ignored. First, inserting $x$ does not
change whether $x$ is overestimated or not, so $\advC$ ignoring the re-insertion
does not affect whether later $\QRYO(\qry_x)$ calls will produce an error. For
$y \neq x$, the fact that $\advB$ makes no deletions is key. The value of the
counters associated with $y$ by the hash functions must be at least equal to the
true frequency of $y$, and $\QRYO(\qry_y)$ will find an overestimate if these
counters are all strictly greater than the true frequency. Since updates are
deterministic, re-inserting $x$ can only increment the same counters that were
incremented by the original insertion of $x$, and so this re-insertion cannot
cause $y$ to become overestimated if it was not already. So all $\QRYO$ calls
are just as likely to produce an error for $\advC$ as they are for $\advB$, and
$\Prob{\game_2(\advB) = 1} = \Prob{\game_2(\advC) = 1}$.

As a third step, we move from~$\game_2$ to a~$\game_3$ where the adversary gains
$k+1$ `points' worth of errors for finding an query which produces an
overestimate, but which prevents the adversary from querying elements of $\col$.
These extra points are necessary because, unlike in the case of a Bloom
filter, inserting an overestimated element $x$ can cause other elements of
$\col$ to become overestimated. In particular, if one of the counters
incremented by the insertion of $x$ is shared with an element of $\col$ that is
not overestimated, that element may become overestimated. However, if that
counter is shared with multiple elements of $\col$, that counter is already an
overestimate for all of the elements associated with it, and so no more than one
overestimate can be caused per counter incremented by the insertion of $x$.
Since inserting $x$ increments $k$ counters, at most $k$ errors can be caused in
this way. For any adversary $\advC$ for~$\game_2$, we can construct $D$
for~$\game_3$ that simulates $\advC$ perfectly except that it ignores any oracle
calls that would insert these elements. Since $D$ already gets credit equal to
the maximum number of errors these insertions could cause in addition to the
credit for the original overestimate, $D$ accumulates at least as many errors as
$\advC$ does, and so $\Prob{\game_2(\advC) = 1} \le \Prob{\game_3(D) = 1}$.

Analagously to the proof of~\ref{thm:sbf-erreps-th}, we now move to a
game~$\game_4$ where $\REPO$ randomly fills the sketch to capacity after
inserting the elements of $\col$, so that each row has $\ell+1$ nonzero
counters. For any $D$ for~$\game_3$ we construct $E$ for~$\game_4$ that
simulates $D$, forwarding $\REPO$, $\QRYO$, and $\HASHO_1$ calls but ignoring
$\UPO$ calls. By a very similar argument, $E$ achieves at least the same
advantage as $D$ by having a maximally full filter as soon as $\REPO$ is called,
and so $\Prob{\game_3(D) = 1} \le \Prob{\game_4{E} = 1}$.

The probability of $E$ winning can now be given by another binomial bound. Since
each row of the sketch is a uniformly random bitmap with $\ell+1$ out of $m$
bits set to 1, the probability of any particular $\QRYO$ call causing a
collision within a single row $i$ is $(\ell+1)/m$, and the probability of a
collision in every row (i.e. an error) is $((\ell+1)/m)^k$. The adversary has a
total of $q_T$ attempts, and wins if it accumulates $\lfloor r/(k+1) \rfloor$
successes. So, letting $p_\ell = ((\ell+1)/m)^k$ and
$r' = \lfloor r/(k+1) \rfloor$, we have
\begin{equation}
   \Prob{\game_4(D)=1} \le
     \sum_{i=r'}^{q_T} \binom{q_T}{i}p_\ell^i(1-p_\ell)^{q_T-i} \,.
\end{equation}
Applying the usual Chernoff bound and applying Lemma~\ref{thm:lemma1} turns this
into the final bound of
\begin{equation}
   \Adv{\erreps}_{\Pi,\delta,r}(\advA) \leq
     q_R \cdot \left[\frac{q_H}{2^\lambda} + e^{r'-p_\ell q_T}\left(\frac{p_\ell q_T}{r'}\right)^r\right].
\end{equation}
\end{proof}
}

\full{
  \begin{proof}[Proof of Theorem~\ref{thm:scms-erreps-th}]
  \ignore{\begin{figure*}
\twoColsNoDivide{0.47}
{
  \vspace{-7pt}
  \experimentv{$\game_{0}(\advB)$}\hfill \diffminus{$\game_1$}\diffplus{$\game_2$}\\[2pt]
    $M^* \gets \bot$;
    $\salt^* \getsr \bits^\lambda$\\
    $\advB^{\REPO,\QRYO,\UPO,\HASHO_1}$;
    return $\big[\sum_x \err[x] \geq r\big]$
  \\[6pt]
  \oraclev{$\HASHO_c(\salt \cat x)$}\\[2pt]
    $\vv \getsr [m]^k$\\
    if $\salt=\salt^*$ and $c = 1$ then \com{Caller is~$\advB$}\\
    \tab $\bad_1 \gets 1$; \diffplus{\diffminus{return $\vv$}}\\
    if $T[Z,x] = \bot$ then $\vv \gets T[Z,z]$\\
    $T[Z,x] \gets \vv$; return $\vv$
}
{
  \oraclev{$\QRYO(\qry_x)$}\\[2pt]
    $X \gets \bmap_m(\HASHO_3(\salt^* \cat x))$;
    $a \gets X = M^* \AND X$\\
    if $\err[x] < \delta(a,\qry_x(\col^*))$ then
          $\err[x] \gets \delta(a,\qry_x(\col^*))$\\
    \diffplus{$\UPO(\up_x)$}\\
    return $a$
  \\[6pt]
  \oraclev{$\REPO(\col)$}\\[2pt]
    $M^* \gets \bigvee_{x \in \col} \bmap_m(\HASHO_2(\salt^* \cat x))$;
    $\setS^* \gets \col$;
    return $\top$
  \\[6pt]
  \oraclev{$\UPO(\up_x)$}\\[2pt]
    if $w(M) > \ell$ then return $\top$\\
    if $\QRYO(\qry_x) = 1$ then $\err[x] \gets 0$\\
    $M^* \gets M^* \vee \bmap_m(\HASHO_2(\salt^* \cat x))$;
    $\setS^* \gets \up_x(\setS)$;
    return $\top$
}
\caption{Games 0, 1, and 2 for proof of Theorem~\ref{thm:sbf-erreps}.}
\label{fig:sbf-erreps/games}
\end{figure*}}

As with the proof of Theorem~\ref{thm:sbf-errep-immutable}, we derive a bound in
the \erreps1 case and then use Lemma~\ref{thm:lemma1} to move from \erreps1 to
the more general \erreps case. Because we are in the \erreps1 case, we may
assume without loss of generality that the adversary does not call $\REVO$,
since revealing the only representation automatically prevents the adversary
from winning.

We begin with a game~$\game_0$ which has identical behavior to the \erreps1
experiment for a salted CMS. As in the proofs of
Theorems~\ref{thm:sbf-errep-immutable} and~\ref{thm:sbf-erreps}, we have a
$\bad_1$ flag that gets set if the adversary ever calls $\HASHO_1$ with the
actual salt used by the representation. By an almost identical argument, we can
move to~$\game_1$, where the behavior is different only when the $\bad_1$ flag
is set, with a bound of
\begin{equation}
  \Prob{\game_0(\advA)=1} \leq
    q_H/2^\lambda + \Prob{\game_1(\advA)=1} \,.
\end{equation}

The key differences between this proof and the Bloom filter proof are the more
complex response space of $\QRYO$ ($\N$ rather than $\bits$) and the possibility
of achieving an error through either overestimation or underestimation of the
actual frequency.

%

It is difficult to bound the probability that, if $\QRYO(\qry_x)$ finds that $x$
is overestimated by 1, the adversary can quickly determine which elements to
re-insert into the sketch in order to increase this error beyond the $n\epsilon$
threshold of `significance'. We therefore move to~$\game_2$, where the adversary
gets credit for any response which overestimates the true value, regardless of
the magnitude of the error. Since this can only increase the probability of a
query producing an error, we have
$\Prob{\game_1(\advA) = 1} \le \Prob{\game_2(\advA) = 1}$.

As a first step in dealing with the $\UPO$ oracle, we want to show that deletion
is never helpful for the adversary. So, for any $\advA$, we construct an
adversary $\advB$ that simulates $\advA$, forwarding all oracle queries in the
natural way, except that it ignores any $\UPO(\up_{x,-1})$ calls, i.e. any
deletions. Because deleting $x$ does not change whether $x$ is overestimated or
not, ignoring deletions does not affect whether later calls of the form
$\QRYO(\qry_x)$ will produce an error. Furthermore, if $y \neq x$, then the
probability of $\QRYO(\qry_y)$ causing an error can only increase if $x$'s
deletion is ignored, since the deletion of $x$ decreases counter values without
decreasing the true frequency of $y$. Therefore
$\Prob{\game_2(\advA) = 1} \le \Prob{\game_2(\advB) = 1}$, and we have reduced
to the case of an adversary whose $\UPO$ calls only consist of insertions.

%%

Next, we move from $\advB$ to an $\advC$ that never inserts an element more than
once. Similarly to the previous step, $\advC$ simulates $\advB$, tracking the
elements of $\col$ and forwarding $\advB$'s oracle queries in
the natural way, except that any $\UPO$ queries to insert an element already
present in $\col$ are ignored. First, inserting $x$ does not
change whether $x$ is overestimated or not, so $\advC$ ignoring the re-insertion
does not affect whether later $\QRYO(\qry_x)$ calls will produce an error. For
$y \neq x$, the fact that $\advB$ makes no deletions is key. The value of the
counters associated with $y$ by the hash functions must be at least equal to the
true frequency of $y$, and $\QRYO(\qry_y)$ will find an overestimate if these
counters are all strictly greater than the true frequency. Since updates are
deterministic, re-inserting $x$ can only increment the same counters that were
incremented by the original insertion of $x$, and so this re-insertion cannot
cause $y$ to become overestimated if it was not already. So all $\QRYO$ calls
are just as likely to produce an error for $\advC$ as they are for $\advB$, and
$\Prob{\game_2(\advB) = 1} = \Prob{\game_2(\advC) = 1}$.

As a third step, we move from~$\game_2$ to a~$\game_3$ where the adversary gains
$k+1$ `points' worth of errors for finding an query which produces an
overestimate, but which prevents the adversary from querying elements of $\col$.
These extra points are necessary because, unlike in the case of a Bloom
filter, inserting an overestimated element $x$ can cause other elements of
$\col$ to become overestimated. In particular, if one of the counters
incremented by the insertion of $x$ is shared with an element of $\col$ that is
not overestimated, that element may become overestimated. However, if that
counter is shared with multiple elements of $\col$, that counter is already an
overestimate for all of the elements associated with it, and so no more than one
overestimate can be caused per counter incremented by the insertion of $x$.
Since inserting $x$ increments $k$ counters, at most $k$ errors can be caused in
this way. For any adversary $\advC$ for~$\game_2$, we can construct $D$
for~$\game_3$ that simulates $\advC$ perfectly except that it ignores any oracle
calls that would insert these elements. Since $D$ already gets credit equal to
the maximum number of errors these insertions could cause in addition to the
credit for the original overestimate, $D$ accumulates at least as many errors as
$\advC$ does, and so $\Prob{\game_2(\advC) = 1} \le \Prob{\game_3(D) = 1}$.

Analagously to the proof of~\ref{thm:sbf-erreps-th}, we now move to a
game~$\game_4$ where $\REPO$ randomly fills the sketch to capacity after
inserting the elements of $\col$, so that each row has $\ell+1$ nonzero
counters. For any $D$ for~$\game_3$ we construct $E$ for~$\game_4$ that
simulates $D$, forwarding $\REPO$, $\QRYO$, and $\HASHO_1$ calls but ignoring
$\UPO$ calls. By a very similar argument, $E$ achieves at least the same
advantage as $D$ by having a maximally full filter as soon as $\REPO$ is called,
and so $\Prob{\game_3(D) = 1} \le \Prob{\game_4{E} = 1}$.

The probability of $E$ winning can now be given by another binomial bound. Since
each row of the sketch is a uniformly random bitmap with $\ell+1$ out of $m$
bits set to 1, the probability of any particular $\QRYO$ call causing a
collision within a single row $i$ is $(\ell+1)/m$, and the probability of a
collision in every row (i.e. an error) is $((\ell+1)/m)^k$. The adversary has a
total of $q_T$ attempts, and wins if it accumulates $\lfloor r/(k+1) \rfloor$
successes. So, letting $p_\ell = ((\ell+1)/m)^k$ and
$r' = \lfloor r/(k+1) \rfloor$, we have
\begin{equation}
   \Prob{\game_4(D)=1} \le
     \sum_{i=r'}^{q_T} \binom{q_T}{i}p_\ell^i(1-p_\ell)^{q_T-i} \,.
\end{equation}
Applying the usual Chernoff bound and applying Lemma~\ref{thm:lemma1} turns this
into the final bound of
\begin{equation}
   \Adv{\erreps}_{\Pi,\delta,r}(\advA) \leq
     q_R \cdot \left[\frac{q_H}{2^\lambda} + e^{r'-p_\ell q_T}\left(\frac{p_\ell q_T}{r'}\right)^r\right].
\end{equation}
\end{proof}
}

Now that we have proven Theorem~\ref{thm:scms-erreps-th}, we now return to the case
of the counting filter for the proof of Theorem~\ref{thm:counting-erreps}. While the two structures
appear rather different due to the two-dimensional nature of sketches and their
support for frequency queries rather than simple membership queries, the security
proofs turn out to be quite similar. In particular, due to our choice of error
function, the proofs coincide as soon as we have managed to reduce to the case
where the adversary makes no deletions. In this scenario, false positives arise
in a counting filter in precisely the same way that overestimates arise in a count
min-sketch. Once we have reduced to the no-deletion scenario, the proofs become
almost identical, an interesting result which might extend to other data structures
as well.

\proc{
  \begin{proof}[Proof Sketch of Theorem~\ref{thm:counting-erreps}]
  \begin{figure*}
\twoCols{0.47}
{
  \vspace{-7pt}
  \experimentv{$\game_{0}(\advA)$}\hfill\diffplus{$\game_1$}\\[2pt]
    $\v.M^* \gets \bot$;
    $\setS \gets \emptyset$;
    $\salt^* \getsr \bits^\lambda$\\
    $\advB^{\REPO,\QRYO,\UPO,\HASHO_1}$;
    return $\big[\sum_x \err[x] \geq r\big]$
  \\[6pt]
  \oraclev{$\HASHO_c(\salt \cat x)$}\\[2pt]
    $\vv \getsr [m]^k$\\
    if $\salt=\salt^*$ and $c = 1$ then \com{Caller is~$\advB$}\\
    \tab $\bad_1 \gets 1$; \diffplus{return $\vv$}\\
    if $T[Z,x] = \bot$ then $\vv \gets T[Z,x]$\\
    $T[Z,x] \gets \vv$; return $\vv$
  \\[6pt]
  \oraclev{$\QRYO(\qry_x)$}\\[2pt]
    $\v.X \gets \HASHO_3(\salt^* \cat x)$;
    $\setS \gets \setS \cup \{x\}$;
    $a = 1$\\
    for $i$ in $\v.X$ do\\
      $\tab$if $\v.M[i] = 0$ then $a = 0$\\
    if $\err[x] < \delta(a,\qry_x(\col^*))$ then
          $\err[x] \gets \delta(a,\qry_x(\col^*))$\\
    return $a$
  \\[6pt]
  \oraclev{$\REPO(\col)$}\\[2pt]
    $\v.M^* \gets 0^m$\\
    $\setS^* \gets \col$\\
    for $x \in \col$ do\\
      $\tab\UPO(\up_x)$\\
    return $\top$
  \\[6pt]
  \oraclev{$\UPO(\up_{x,b})$}\\[2pt]
    if $w'(\v.M^*) > \ell$ then return $\top$\\
    $\v.X \gets \HASHO_3(\salt^* \cat x)$;
    $\v.M' \gets \v.M^*$\\
    for $i$ in $\v.X$ do\\
      $\tab$ if $\v.M'[i] = 0$ and $b < 0$ then return $\top$\\
      $\tab \v.M'[i] \gets \v.M'[i] + b$\\
    if $b > 0$ and $\QRYO(\qry_x) = 1$ then $\err_i[x] \gets 0$\\
    if $b < 0$ and $\QRYO(\qry_x) = 0$ then $\err_i[x] \gets 0$\\
    $\v.M^* \gets \v.M'$;
    $\setS^* \gets \up_{x,b}(\setS^*)$;
    return $\top$
}
{
  \vspace{-7pt}
  \experimentv{$\game_2(\advA)$}\hfill\diffplus{$\game_2$}\\[2pt]
    $\v.M^* \gets \bot$;
    $\setS \gets \emptyset$;
    \diffplus{$\setR \gets \emptyset$; $r' \gets \lfloor r/\max(\delta^+,k\delta^-)\rfloor$}\\
    $\salt^* \getsr \bits^\lambda$\\
    $\advB^{\REPO,\QRYO,\UPO,\HASHO_1}$;
    return $\big[\sum_x \err[x] \geq r\big]$
  \\[6pt]
  \oraclev{$\QRYO(\qry_x)$}\\[2pt]
    $\v.X \gets \HASHO_3(\salt^* \cat x)$;
    $\setS \gets \setS \cup \{x\}$;
    $a = 1$\\
    for $i$ in $\v.X$ do\\
      $\tab$if $\v.M[i] = 0$ then $a = 0$\\
    if $\err[x] < \delta(a,\qry_x(\col^*))$ then
          $\err[x] \gets \delta(a,\qry_x(\col^*))$\\
    if $\err[x] > 0$ then $\setR \gets \setR \cup \{x\}$\\
    return $a$
  \\[6pt]
  \oraclev{$\UPO(\up_{x,b})$}\\[2pt]
    if $w'(\v.M^*) > \ell$\diffplus{$+r'$} then return $\top$\\
    \diffplus{if $x \in \setR$ and $b < 0$ then return $\top$}\\
    $\v.X \gets \HASHO_3(\salt^* \cat x)$;
    $\v.M' \gets \v.M^*$\\
    for $i$ in $\v.X$ do\\
      $\tab$ if $\v.M'[i] = 0$ and $b < 0$ then return $\top$\\
      $\tab \v.M'[i] \gets \v.M'[i] + b$\\
    if $b > 0$ and $\QRYO(\qry_x) = 1$ then $\err_i[x] \gets 0$\\
    if $b < 0$ and $\QRYO(\qry_x) = 0$ then $\err_i[x] \gets 0$\\
    $\v.M^* \gets \v.M'$;
    $\setS^* \gets \up_{x,b}(\setS^*)$;
    return $\top$
  \vspace{6pt}\hrule\vspace{3pt}
  \oraclev{$\UPO(\up_{x,b})$}\hfill\diffminus{$\game_2$}\diffplus{$\game_3$}\\[2pt]
    if $w'(\v.M^*) > \ell+r'$ then return $\top$\\
    if $x \in \setR$ and $b < 0$ then return $\top$\\
    $\v.X \gets \HASHO_3(\salt^* \cat x)$;
    $\v.M' \gets \v.M^*$\\
    for $i$ in $\v.X$ do\\
      $\tab$ if $\v.M'[i] = 0$ and $b < 0$ then return $\top$\\
      \diffminus{$\tab \v.M'[i] \gets \v.M'[i] + b$}\\
      \diffplus{$\tab \v.M'[i] \gets \min(\v.M'[i] + b, 1)$}\\
    if $b > 0$ and $\QRYO(\qry_x) = 1$ then $\err_i[x] \gets 0$\\
    if $b < 0$ and $\QRYO(\qry_x) = 0$ then $\err_i[x] \gets 0$\\
    $\v.M^* \gets \v.M'$;
    $\setS^* \gets \up_{x,b}(\setS^*)$;
    return $\top$
}
\caption{Games 0--3 for proof of Theorem~\ref{thm:scbf-erreps-th}.}
\label{fig:sbf-erreps/games}
\end{figure*}

As with the proof of Theorem~\ref{thm:sbf-errep-immutable}, we derive a bound in
the \erreps1 case and then use Lemma~\ref{thm:lemma1} to move from \erreps1 to
the more general \erreps case. Because we are in the \erreps1 case, we may
assume without loss of generality that the adversary does not call $\REVO$,
since revealing the only representation automatically prevents the adversary
from winning.

We begin with a game~$\game_0$ which has identical behavior to the \erreps1
experiment for a counting filter. As in the proof of
Theorem~\ref{thm:sbf-errep-immutable}, we have a
$\bad_1$ flag that gets set if the adversary ever calls $\HASHO_1$ with the
actual salt used by the representation. By a very similar argument, we can
move to~$\game_1$, where the behavior is different only when the $\bad_1$ flag
is set, with a bound of
\begin{equation}
  \Prob{\game_0(\advA)=1} \leq
    q_H/2^\lambda + \Prob{\game_1(\advA)=1} \,.
\end{equation}

Unlike in the case of a count min-sketch, it is entirely possible for deletions
to benefit the adversary in this game. In particular, if $x$ is found to be a
false positive, deleting $x$ may cause up to $k$ elements of $\col$ to become
false negatives. We therefore move to a game~$\game_2$ where the adversary gets
credit for either a single false positive or for $k$ false negatives whenever it
finds a false positive, but where the adversary cannot delete any false
positives that it finds. We let $r' = \lfloor r/\max(\delta^+,k\delta^-)\rfloor$
represent the number of false positives the adversary has to find in~$\game_2$
in order to win. In order to prevent the adversary from being penalized by the
filter becoming full too early, we also raise the thresold from $\ell$ to
$\ell+r'$ in~$\game_2$. Now for any $\advA$ for~$\game_1$, we can construct
$\advB$ for~$\game_2$ that simulates $\advA$, keeping track of all query
responses and forwarding all oracle queries in the natural way, except that
calls to delete false positives are ignored. Since $\UPO$ never fails for
$\advB$ due to the increased threshold, and since $\advB$ gets automatic credit
for any false negatives that might have been caused by deleting false positives,
$\advB$ succeeds whenever $\advA$ does, i.e.
$\Prob{\game_1(\advA)=1} \le \Prob{\game_2(\advB) = 1}$.

Since the remaining deletions do not cause errors, we can use the same argument
as in the proof of Theorem~\ref{thm:scms-erreps-th} to reduce from $\advB$ to an
adversary $\advC$ which does not make deletions at all. In~$\game_3$, we further
reduce from a counting filter to a normal Bloom filter by capping each of the
counters in the filter at 1. Since no deletions are performed, a counter
in~$\game_3(\advC)$ is nonzero if and only if the same counter
in~$\game_2(\advC)$ is nonzero. So $\QRYO$ behaves the same in~$\game_3$ as it
did in~$\game_2$, and $\Prob{\game_2(\advC)=1} \le \Prob{\game_3(\advC) = 1}$.

Note that~$\game_3$ is actually simulating an ordinary Bloom filter, since all
`counters' in the filter are restricted to the range $\bits$, there are no
deletions, and any insertions just set the corresponding bits to 1. In fact,
this game is identical to~$\game_2$ in the proof of
Theorem~\ref{thm:sbf-erreps-th} except that the adversary need only accumulate
$r'$ errors instead of $r$ errors and the threshold is $\ell+r'$ instead of
$\ell$. An identical argument allows us to reach the binomial bound of
\begin{equation}
   \Prob{\game_3(\advC)=1} \le
     \sum_{i=r'}^{q_T} \binom{q_T}{i}p_\ell^i(1-p_\ell)^{q_T-i} \,,
\end{equation}
where $p_\ell$ is now defined to be $((\ell+k+r')/m)^k$. Then the standard
Chernoff bound, along with Lemma~\ref{thm:lemma1}, yields the final bound of
\begin{equation}
   \Adv{\erreps}_{\Pi,\delta,r}(\advA) \leq
     q_R \cdot \left[\frac{q_H}{2^\lambda} + e^{r'-p_\ell q_T}\left(\frac{p_\ell q_T}{r'}\right)^{r'}\right]s.
\end{equation}
\end{proof}
}

\full{
  \begin{proof}[Proof of Theorem~\ref{thm:counting-erreps}]
  As with the proof of Theorem~\ref{thm:sbf-errep-immutable}, we derive a bound in
the \erreps1 case and then use Lemma~\ref{thm:lemma1} to move from \erreps1 to
the more general case of \erreps. Because we are in the \erreps1 case, we may
assume without loss of generality that the adversary does not call $\REVO$,
since revealing the only representation automatically prevents the adversary
from winning.

As in the proof of
Theorem~\ref{thm:sbf-errep-immutable}, we first add a
$\bad_1$ flag that gets set if the adversary ever hashes with the
actual salt used by the representation. By a very similar argument, we can
move to~$\game_1$, where the behavior is different only when the $\bad_1$ flag
is set, which happens with probability at most $q_H/2^\lambda$.

Unlike in the case of a count min-sketch, it is entirely possible for deletions
to benefit the adversary in this game. In particular, if $x$ is found to be a
false positive, deleting $x$ may cause up to $k$ elements of $\col$ to become
false negatives. We therefore move to a game~$\game_2$ where the adversary gets
credit for either a single false positive or for $k$ false negatives whenever it
finds a false positive, but where the adversary cannot delete any false
positives that it finds. We let $r' = \lfloor r/\max(\delta^+,k\delta^-)\rfloor$
represent the number of false positives the adversary has to find in~$\game_2$
in order to win. In order to prevent the adversary from being penalized by the
filter becoming full too early, we also raise the threshold from $\ell$ to
$\ell+r'$ in~$\game_2$. Now for any $\advA$ for~$\game_1$, we can construct
$\advB$ for~$\game_2$ that simulates $\advA$, keeping track of all query
responses and forwarding all oracle queries in the natural way, except that
calls to delete false positives are ignored. Since $\UPO$ never fails for
$\advB$ due to the increased threshold, and since $\advB$ gets automatic credit
for any false negatives that might have been caused by deleting false positives,
$\advB$ succeeds whenever $\advA$ does, i.e.
$\Prob{\game_1(\advA)=1} \le \Prob{\game_2(\advB) = 1}$.

Since the remaining deletions do not cause errors, we can use the same argument
as in the proof of Theorem~\ref{thm:scms-erreps-th} to reduce from $\advB$ to an
adversary $\advC$ which does not make deletions at all. In~$\game_3$, we further
reduce from a counting filter to a normal Bloom filter by capping each of the
counters in the filter at 1. Since no deletions are performed, a counter
in~$\game_3(\advC)$ is nonzero if and only if the same counter
in~$\game_2(\advC)$ is nonzero. So $\QRYO$ behaves the same in~$\game_3$ as it
did in~$\game_2$, and $\Prob{\game_2(\advC)=1} \le \Prob{\game_3(\advC) = 1}$.

Note that~$\game_3$ is actually simulating an ordinary Bloom filter, since all
`counters' in the filter are restricted to the range $\bits$, there are no
deletions, and any insertions just set the corresponding bits to 1. In fact,
this game is identical to~$\game_2$ in the proof of
Theorem~\ref{thm:sbf-erreps-th} except that the adversary need only accumulate
$r'$ errors instead of $r$ errors and the threshold is $\ell+r'$ instead of
$\ell$. We can therefore use the same argument to produce the final bound of
\begin{equation}
   \Adv{\erreps}_{\Pi,\delta,r}(\advA) \leq
     q_R \cdot \left[\frac{q_H}{2^\lambda} + e^{r'-p_\ell q_T}\left(\frac{p_\ell
     q_T}{r'}\right)^{r'}\right] \,.
\end{equation}

\end{proof}
}

\subsection{Discussion}
The results for count min-sketches are similar to the results for counting
filters, as might be expected given the similarities in terms of both the
supported updates and the structure of the representations themselves (any
count min-sketch can be transformed into a counting filter by adding all the
rows together element-wise.) In particular, we see that count min-sketches
which are publicly visible cannot provide good security guarantees. This means
that sketches intended for a security-sensitive setting should be kept
hidden from potential adversaries. Furthermore, our bound relies on
per-representation random salts and $\ell$-thresholding, so these changes should
also be taken into account when constructing secure count min-sketches. The size
increase of the sketches is comparable to the size increase of counting filters,
but does not need to take into account multiple types of errors

The bound we achieve is based on the same binomial bound as in the case of Bloom
filters, but has a notable difference in the form of $r'$ replacing $r$. This
negatively impacts the amount of space the filter must take up in order to
provide low error bounds, but because the scaling factor between $r$ and $r'$ is
only $k+1$, the difference should not be unacceptably extreme given reasonable
parameter choices. We also note that it is possible this bound can be improved
to reduce the impact on sketch size, since the initial factor of $q_R$ does not
have an obvious attack associated with it which would make this bound tight.
%
(The same is true, of course, of Bloom and counting filters.)

\begin{figure}
  \twoColsNoDivide{0.22}
  {
    \underline{$\Rep^R_K(\col)$}\\[2pt]
      $\salt \getsr \bits^\lambda$\\
      for $i$ in $[1..k]$ do\\
        $\tab M[i] \gets 0^m$\\
      $\pub \gets \langle M, \salt, 0 \rangle$
      for $x \in \col$ do \\
        $\tab \pub \gets \Up^R_K(\pub, \qry_x)$\\
        $\tab$if $\pub = \bot$ then return $\bot$\\
      return $\pub$
    \\[6pt]
    \underline{$\Qry^R_K(\langle M, \salt, c \rangle,\qry_x)$}\\[2pt]
      $X \gets R_K(\salt \cat x)$;
      $m \gets \infty$\\
      for $i$ in $[1..k]$ do\\
      $\tab m \gets \min(m, M[i][X[i]])$\\
      return $m$
  }
  {
    \underline{$\Up^R_K(\langle M, \salt, c \rangle,\up_{x,b})$}\\[2pt]
      if $c \geq n$ then return $\bot$\\
      $M' \gets M$;
      $X \gets R_K(\salt \cat x)$\\
      for $i$ in $[1..k]$ do\\
      $\tab$ if $M'[i][X[i]] = 0$ and $b < 0$ then return $\bot$\\
      $\tab M'[i][X[i]] \gets M'[i][X[i]] + b$\\
      return $\langle M', \salt, c+b \rangle$
  }
  \caption{Keyed structure $\sketch[R,n,\lambda]$ given by
  $(\Rep^R,\Qry^R,\Up^R)$ is used to define count min-sketch variants used to
  rerpresent streams of size at most~$n$. Parameters are a function $R:
  \keys\by\bits^* \to [m]^k$ and integers $n, \lambda \geq0$. A concrete scheme
  is given by a particular choice of parameters.
  %
  }
  \label{fig:cms-def}
\end{figure}

The count min-sketch data structure is somewhat similar to a Bloom filter in
construction, but instead of a length-$m$ array of bits it uses a $k$-by-$m$
array of counters. Our construction $\sketch[R,n,\lambda]$ described
in~\ref{fig:bf-def} captures the count-min sketch in what is known as the
non-negative turnstile model. This means that the stream of data handled by the
sketch can accomodate both insertions and deletions, but the counters themselves
are constrained to always be nonnegative. While the data structure supports a
variety of possible queries, we focus here on \emph{frequency} queries, which
determine how many times an element is present in the sketch.

\heading{Error function for frequency queries}
%
Throughout this section we will use a general error function~$\delta$ defined as
\begin{equation}
  \delta(x, y) =
  \begin{cases}
    1 & \text{if}\ x - y > n\epsilon \\
    0 & \text{otherwise.}
  \end{cases}
\end{equation}

This means that a non-negligible \emph{overestimate} is considered an error,
while underestimates are not. This is in line with the definition of the CMS
data structure, which provides a tight bound on the number of accidental
overestimates which are likely to occur (in the absence of an adaptive
adversary) but does not place any bounds on underestimation.

\subsection{Insecurity of Public Sketches}

Unlike in the Bloom filter case, good security bounds cannot be achieved for a
count-min sketch in the \errep\ setting even if salts and/or private keys are
used. Given a stronger $\UPO$ oracle, the adversary can mount an attack similar
to the target-set coverage attack for a Bloom filter even if a PRF is used for
hashing. First, the adversary calls $\REPO(\emptyset)$ to get an empty
representation. The adversary can then call $\UPO$ to insert an element into the
set, see exactly what the outputs of each of the hash functions are, and then
call $\UPO$ again to delete the element. By doing this repeatedly, an adversary
can determine the outputs of the PRF for $i$ different inputs using $2i$ calls
to $\UPO$. Once a sufficiently large number of PRF outputs has been determined,
the adversary can construct the test and target set used for the target-set
coverage attack.

\subsection{Private Thresholded Sketches}

\begin{figure}
  \twoColsNoDivide{0.22}
  {
    \underline{$\Rep^R_K(\col)$}\\[2pt]
      $\salt \getsr \bits^\lambda$\\
      for $i$ in $[1..k]$ do\\
        $\tab M[i] \gets 0^m$\\
      $\pub \gets \langle M, \salt \rangle$
      for $x \in \col$ do \\
        $\tab \pub \gets \Up^R_K(\pub, \qry_x)$\\
        $\tab$if $\pub = \bot$ then return $\bot$\\
      return $\pub$
    \\[6pt]
    \underline{$\Qry^R_K(\langle M, \salt, c \rangle,\qry_x)$}\\[2pt]
      $X \gets R_K(\salt \cat x)$;
      $m \gets \infty$\\
      for $i$ in $[1..k]$ do\\
      $\tab m \gets \min(m, M[i][X[i]])$\\
      return $m$
  }
  {
    \underline{$\Up^R_K(\langle M, \salt, c \rangle,\up_{x,b})$}\\[2pt]
      if $hw'(M) > \ell$ then return $\bot$\\
      $M' \gets M$;
      $X \gets R_K(\salt \cat x)$\\
      for $i$ in $[1..k]$ do\\
      $\tab$ if $M'[i][X[i]] = 0$ and $b < 0$ then return $\bot$\\
      $\tab M'[i][X[i]] \gets M'[i][X[i]] + b$\\
      return $\langle M', \salt, c+b \rangle$
  }
  \caption{A slightly modified structure, $\sketch_\mathrm{st}[R,\ell,\lambda]$ given by
  $(\Rep^R,\Qry^R,\Up^R)$ which uses the number of nonzero counters ($\hw'$, as
  defined in Section~\ref{sec:prelims}) to decide if the filter is full.
  %
  }
  \label{fig:cmst-def}
\end{figure}

As in the case of Bloom filters, it is possible to tweak the count min-sketch
definition by placing a bound on how full the filter itself can get. In this
case, we were only able to establish an error bound for these $\ell$-thresholded
sketches. Even then, the bound is clearly weaker than for Bloom filters. We
emphasize that data structures which allow for both insertions and deletions of
arbitrary strings give potential adversaries a great deal more flexibility for
use in attacks.

\begin{theorem}[Correctness Bound for Count-Min Sketch]\label{thm:count-ms-bound}
Let $p_\ell = ((\ell+1)/m)^k$ and $r' = \lfloor r/(k+1) \rfloor$. For integers $q_R, q_T, q_H, r, t \geq 0$ such
that $r' > p_\ell q_T$, it holds that
  \begin{equation*}
  \begin{aligned}
    \Adv{\errep}_{\Pi,\delta,r}(t, q_R,q_T,q_U,q_H) &\leq \\
     & q_R \cdot \left[\frac{q_H}{2^\lambda} + e^{r'-p_\ell q_T}\left(\frac{p_\ell q_T}{r'}\right)^r\right].
  \end{aligned}
\end{equation*}
\end{theorem}

\begin{proof}
  \ignore{\begin{figure*}
\twoColsNoDivide{0.47}
{
  \vspace{-7pt}
  \experimentv{$\game_{0}(\advB)$}\hfill \diffminus{$\game_1$}\diffplus{$\game_2$}\\[2pt]
    $M^* \gets \bot$;
    $\salt^* \getsr \bits^\lambda$\\
    $\advB^{\REPO,\QRYO,\UPO,\HASHO_1}$;
    return $\big[\sum_x \err[x] \geq r\big]$
  \\[6pt]
  \oraclev{$\HASHO_c(\salt \cat x)$}\\[2pt]
    $\vv \getsr [m]^k$\\
    if $\salt=\salt^*$ and $c = 1$ then \com{Caller is~$\advB$}\\
    \tab $\bad_1 \gets 1$; \diffplus{\diffminus{return $\vv$}}\\
    if $T[Z,x] = \bot$ then $\vv \gets T[Z,z]$\\
    $T[Z,x] \gets \vv$; return $\vv$
}
{
  \oraclev{$\QRYO(\qry_x)$}\\[2pt]
    $X \gets \bmap_m(\HASHO_3(\salt^* \cat x))$;
    $a \gets X = M^* \AND X$\\
    if $\err[x] < \delta(a,\qry_x(\col^*))$ then
          $\err[x] \gets \delta(a,\qry_x(\col^*))$\\
    \diffplus{$\UPO(\up_x)$}\\
    return $a$
  \\[6pt]
  \oraclev{$\REPO(\col)$}\\[2pt]
    $M^* \gets \bigvee_{x \in \col} \bmap_m(\HASHO_2(\salt^* \cat x))$;
    $\setS^* \gets \col$;
    return $\top$
  \\[6pt]
  \oraclev{$\UPO(\up_x)$}\\[2pt]
    if $w(M) > \ell$ then return $\top$\\
    if $\QRYO(\qry_x) = 1$ then $\err[x] \gets 0$\\
    $M^* \gets M^* \vee \bmap_m(\HASHO_2(\salt^* \cat x))$;
    $\setS^* \gets \up_x(\setS)$;
    return $\top$
}
\caption{Games 0, 1, and 2 for proof of Theorem~\ref{thm:sbf-erreps}.}
\label{fig:sbf-erreps/games}
\end{figure*}}

As with the proof of Theorem~\ref{thm:sbf-errep-immutable}, we derive a bound in
the \erreps1 case and then use Lemma~\ref{thm:lemma1} to move from \erreps1 to
the more general \erreps case. Because we are in the \erreps1 case, we may
assume without loss of generality that the adversary does not call $\REVO$,
since revealing the only representation automatically prevents the adversary
from winning.

We begin with a game~$\game_0$ which has identical behavior to the \erreps1
experiment for a salted CMS. As in the proofs of
Theorems~\ref{thm:sbf-errep-immutable} and~\ref{thm:sbf-erreps}, we have a
$\bad_1$ flag that gets set if the adversary ever calls $\HASHO_1$ with the
actual salt used by the representation. By an almost identical argument, we can
move to~$\game_1$, where the behavior is different only when the $\bad_1$ flag
is set, with a bound of
\begin{equation}
  \Prob{\game_0(\advA)=1} \leq
    q_H/2^\lambda + \Prob{\game_1(\advA)=1} \,.
\end{equation}

The key differences between this proof and the Bloom filter proof are the more
complex response space of $\QRYO$ ($\N$ rather than $\bits$) and the possibility
of achieving an error through either overestimation or underestimation of the
actual frequency.

%

It is difficult to bound the probability that, if $\QRYO(\qry_x)$ finds that $x$
is overestimated by 1, the adversary can quickly determine which elements to
re-insert into the sketch in order to increase this error beyond the $n\epsilon$
threshold of `significance'. We therefore move to~$\game_2$, where the adversary
gets credit for any response which overestimates the true value, regardless of
the magnitude of the error. Since this can only increase the probability of a
query producing an error, we have
$\Prob{\game_1(\advA) = 1} \le \Prob{\game_2(\advA) = 1}$.

As a first step in dealing with the $\UPO$ oracle, we want to show that deletion
is never helpful for the adversary. So, for any $\advA$, we construct an
adversary $\advB$ that simulates $\advA$, forwarding all oracle queries in the
natural way, except that it ignores any $\UPO(\up_{x,-1})$ calls, i.e. any
deletions. Because deleting $x$ does not change whether $x$ is overestimated or
not, ignoring deletions does not affect whether later calls of the form
$\QRYO(\qry_x)$ will produce an error. Furthermore, if $y \neq x$, then the
probability of $\QRYO(\qry_y)$ causing an error can only increase if $x$'s
deletion is ignored, since the deletion of $x$ decreases counter values without
decreasing the true frequency of $y$. Therefore
$\Prob{\game_2(\advA) = 1} \le \Prob{\game_2(\advB) = 1}$, and we have reduced
to the case of an adversary whose $\UPO$ calls only consist of insertions.

%%

Next, we move from $\advB$ to an $\advC$ that never inserts an element more than
once. Similarly to the previous step, $\advC$ simulates $\advB$, tracking the
elements of $\col$ and forwarding $\advB$'s oracle queries in
the natural way, except that any $\UPO$ queries to insert an element already
present in $\col$ are ignored. First, inserting $x$ does not
change whether $x$ is overestimated or not, so $\advC$ ignoring the re-insertion
does not affect whether later $\QRYO(\qry_x)$ calls will produce an error. For
$y \neq x$, the fact that $\advB$ makes no deletions is key. The value of the
counters associated with $y$ by the hash functions must be at least equal to the
true frequency of $y$, and $\QRYO(\qry_y)$ will find an overestimate if these
counters are all strictly greater than the true frequency. Since updates are
deterministic, re-inserting $x$ can only increment the same counters that were
incremented by the original insertion of $x$, and so this re-insertion cannot
cause $y$ to become overestimated if it was not already. So all $\QRYO$ calls
are just as likely to produce an error for $\advC$ as they are for $\advB$, and
$\Prob{\game_2(\advB) = 1} = \Prob{\game_2(\advC) = 1}$.

As a third step, we move from~$\game_2$ to a~$\game_3$ where the adversary gains
$k+1$ `points' worth of errors for finding an query which produces an
overestimate, but which prevents the adversary from querying elements of $\col$.
These extra points are necessary because, unlike in the case of a Bloom
filter, inserting an overestimated element $x$ can cause other elements of
$\col$ to become overestimated. In particular, if one of the counters
incremented by the insertion of $x$ is shared with an element of $\col$ that is
not overestimated, that element may become overestimated. However, if that
counter is shared with multiple elements of $\col$, that counter is already an
overestimate for all of the elements associated with it, and so no more than one
overestimate can be caused per counter incremented by the insertion of $x$.
Since inserting $x$ increments $k$ counters, at most $k$ errors can be caused in
this way. For any adversary $\advC$ for~$\game_2$, we can construct $D$
for~$\game_3$ that simulates $\advC$ perfectly except that it ignores any oracle
calls that would insert these elements. Since $D$ already gets credit equal to
the maximum number of errors these insertions could cause in addition to the
credit for the original overestimate, $D$ accumulates at least as many errors as
$\advC$ does, and so $\Prob{\game_2(\advC) = 1} \le \Prob{\game_3(D) = 1}$.

Analagously to the proof of~\ref{thm:sbf-erreps-th}, we now move to a
game~$\game_4$ where $\REPO$ randomly fills the sketch to capacity after
inserting the elements of $\col$, so that each row has $\ell+1$ nonzero
counters. For any $D$ for~$\game_3$ we construct $E$ for~$\game_4$ that
simulates $D$, forwarding $\REPO$, $\QRYO$, and $\HASHO_1$ calls but ignoring
$\UPO$ calls. By a very similar argument, $E$ achieves at least the same
advantage as $D$ by having a maximally full filter as soon as $\REPO$ is called,
and so $\Prob{\game_3(D) = 1} \le \Prob{\game_4{E} = 1}$.

The probability of $E$ winning can now be given by another binomial bound. Since
each row of the sketch is a uniformly random bitmap with $\ell+1$ out of $m$
bits set to 1, the probability of any particular $\QRYO$ call causing a
collision within a single row $i$ is $(\ell+1)/m$, and the probability of a
collision in every row (i.e. an error) is $((\ell+1)/m)^k$. The adversary has a
total of $q_T$ attempts, and wins if it accumulates $\lfloor r/(k+1) \rfloor$
successes. So, letting $p_\ell = ((\ell+1)/m)^k$ and
$r' = \lfloor r/(k+1) \rfloor$, we have
\begin{equation}
   \Prob{\game_4(D)=1} \le
     \sum_{i=r'}^{q_T} \binom{q_T}{i}p_\ell^i(1-p_\ell)^{q_T-i} \,.
\end{equation}
Applying the usual Chernoff bound and applying Lemma~\ref{thm:lemma1} turns this
into the final bound of
\begin{equation}
   \Adv{\erreps}_{\Pi,\delta,r}(\advA) \leq
     q_R \cdot \left[\frac{q_H}{2^\lambda} + e^{r'-p_\ell q_T}\left(\frac{p_\ell q_T}{r'}\right)^r\right].
\end{equation}
\end{proof}
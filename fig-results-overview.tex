\newcommand{\cellsize}{1.2cm}
\newcommand{\atk}{\xmark}
\newcommand{\noatk}{\cmark}
\newcommand{\secres}[1]{Thm #1}
\newcommand{\dontknow}{?}
\begin{figure}
\small
\centering
%\begin{tabular}{|r | *{4}{>{\centering} m{\cellsize} |} 
%                            | *{4}{>{\centering} m{\cellsize} |} }\hline
\begin{tabular}{| r | 
  >{\centering} m{\cellsize} | 
  >{\centering} m{\cellsize} | 
  >{\centering} m{\cellsize} | 
  >{\centering} m{\cellsize} || 
  >{\centering} m{\cellsize} | 
  >{\centering} m{\cellsize} | 
  >{\centering} m{\cellsize} | 
  >{\centering\arraybackslash} m{\cellsize} | 
} \hline
&\multicolumn{4}{c||}{\bf Public Rep} & \multicolumn{4}{c|}{\bf Private Rep} \\
  \cline{2-5}\cline{6-9}
&$\emptyset$ &{salt} &{key} &{salt+key}
&$\emptyset$ &{salt} &{key} &{salt+key} \\ \hline
%%%
(static) Bloom filter 
& \atk %pub, empty
&  %pub, salt
&  %pub, key
&  %pub, salt and key
&  %priv, empty
&  %priv, salt
&  %priv, key
&  %priv, salt and key 
\\ \hline
Bloom filter 
& \atk %pub, empty
& \atk %pub, salt
& \atk %pub, key
&  ? %pub, salt and key
& \atk %priv, empty
& \secres{1} %priv, salt
& \atk %priv, key
& \secres{2} %priv, salt and key 
\\ \hline
Counting filter 
& \atk %pub, empty
&  %pub, salt
&  %pub, key
&  %pub, salt and key
&  %priv, empty
&  %priv, salt
&  %priv, key
&  %priv, salt and key 
\\ \hline
Cuckoo filter 
& \atk %pub, empty
&  %pub, salt
&  %pub, key
&  %pub, salt and key
&  %priv, empty
&  %priv, salt
&  %priv, key
&  %priv, salt and key 
\\ \hline
Count-min sketch 
& \atk %pub, empty
&  %pub, salt
&  %pub, key
&  %pub, salt and key
&  %priv, empty
&  %priv, salt
&  %priv, key
&  %priv, salt and key 
\\ \hline
\end{tabular}
\caption{Summary of results.  An entry of `\xmark' means there is a
  (query efficient) attack, which we discuss in the body.  
  An entry of `\secres{$n$}' means that we
  explicitly prove a security bound for the structure in Theorem $n$. 
  An entry of `\cmark'  means that the structure is secure, but we do
  not give an explicit result in this submission. An entry of
  `\dontknow' means we do not address this case.}
\label{fig:results-overview}
\end{figure}
\begin{figure*}
\twoCols{0.47}
{
  \vspace{-7pt}
  \experimentv{$\game_{0}(\advA)$}\hfill\diffminus{$\game_1$}\diffplus{$\game_2$}\\[2pt]
    $M^* \gets \bot$;
    $\setS \gets \emptyset$;
    $\salt^* \getsr \bits^\lambda$\\
    $\advB^{\REPO,\QRYO,\UPO,\HASHO_1}$;
    return $\big[\sum_x \err[x] \geq r\big]$
  \\[6pt]
  \oraclev{$\HASHO_c(\salt \cat x)$}\\[2pt]
    $\vv \getsr [m]^k$\\
    if $\salt=\salt^*$ and $c = 1$ then \com{Caller is~$\advB$}\\
    \tab $\bad_1 \gets 1$; \diffplus{\diffminus{return $\vv$}}\\
    if $T[Z,x] = \bot$ then $\vv \gets T[Z,x]$\\
    $T[Z,x] \gets \vv$; return $\vv$
  \\[6pt]
  \oraclev{$\QRYO(\qry_x)$}\\[2pt]
    $X \gets \HASHO_3(\salt^* \cat x)$;
    $a \gets \infty$;
    $\setS \gets \setS \cup \{x\}$\\
    for $i$ in $[1..k]$ do\\
      $\tab a \gets \min(a, M[i][X[i]])$\\
    if $\err[x] < \delta(a,\qry_x(\col^*))$ then
          $\err[x] \gets \delta(a,\qry_x(\col^*))\diffplus{+k}$\\
    return $a$
  \\[6pt]
  \oraclev{$\REPO(\col)$}\\[2pt]
    for $i$ in $[1..k]$ do\\
      $\tab M^*[i] \gets 0^m$\\
    $\setS^* \gets \col$\\
    for $x \in \col$ do\\
    $\tab\UPO(\up_x)$\\
    return $\top$
}
{
  \vspace{-7pt}
  \oraclev{$\UPO(\up_{x,b})$}\\[2pt]
    if $w'(M^*) > \ell$ then return $\top$\\
    $M' \gets M^*$\\
    for $i$ in $[1..k]$ do\\
      $\tab$ if $M'[i][X[i]] = 0$ and $b < 0$ then return $\top$\\
      $\tab M'[i][X[i]] \gets M'[i][X[i]] + b$\\
    $M^* \gets M'$\\
    if $\err[x] \neq \bot$ then\\
      $\tab a \gets \QRYO(\qry_x)$\\
      $\tab\err[x] \gets \min(\delta(a,\qry_x(\col^*)),err[x])$\\
    $\setS^* \gets \up_{x,b}(\setS^*)$;
    return $\top$
  \vspace{6pt}\hrule\vspace{3pt}
  \oraclev{$\REPO(\col)$}\hfill\diffplus{$\game_3$}\\[2pt]
    for $i$ in $[1..k]$ do\\
      $\tab M^*[i] \gets 0^m$\\
    $\setS^* \gets \col$\\
    for $x \in \col$ do\\
    $\tab\UPO(\up_x)$\\
    \diffplusbox{for $i$ in $[1..k]$ do\\
      $\tab$while $w(M[i]) < \ell+1$ do\\
        $\tab\tab j \getsr [m]$;
        $M[i][j] \gets 1$}
    return $\top$
}
\caption{Games 0--3 for proof of Theorem~\ref{thm:scms-erreps-th}.}
\label{fig:sbf-erreps/games}
\end{figure*}

As with the proof of Theorem~\ref{thm:sbf-errep-immutable}, we derive a bound in
the \erreps1 case and then use Lemma~\ref{thm:lemma1} to move from \erreps1 to
the more general \erreps case. Because we are in the \erreps1 case, we may
assume without loss of generality that the adversary does not call $\REVO$,
since revealing the only representation automatically prevents the adversary
from winning.

We begin with a game~$\game_0$ which has identical behavior to the \erreps1
experiment for a salted CMS. As in the proofs of
Theorems~\ref{thm:sbf-errep-immutable} and~\ref{thm:sbf-erreps}, we have a
$\bad_1$ flag that gets set if the adversary ever calls $\HASHO_1$ with the
actual salt used by the representation. By an almost identical argument, we can
move to~$\game_1$, where the behavior is different only when the $\bad_1$ flag
is set, with a bound of
\begin{equation}
  \Prob{\game_0(\advA)=1} \leq
    q_H/2^\lambda + \Prob{\game_1(\advA)=1} \,.
\end{equation}

The key differences between this proof and the Bloom filter proof are the more
complex response space of $\QRYO$ ($\N$ rather than $\bits$) and the fact that
both elements of $\col$ and nonelements of $\col$ may produce errors.

As a first step in dealing with the $\UPO$ oracle, we want to show that deletion
is never helpful for the adversary. So, for any $\advA$, we construct an
adversary $\advB$ that simulates $\advA$, forwarding all oracle queries in the
natural way, except that it ignores any $\UPO(\up_{x,-1})$ calls, i.e. any
deletions. Because deleting $x$ does not change whether $x$ is overestimated or
not, ignoring deletions does not affect whether later calls of the form
$\QRYO(\qry_x)$ will produce an error. Furthermore, if $y \neq x$, then the
probability of $\QRYO(\qry_y)$ causing an error can only increase if $x$'s
deletion is ignored, since the deletion of $x$ decreases counter values without
decreasing the true frequency of $y$. Therefore
$\Prob{\game_1(\advA) = 1} \le \Prob{\game_1(\advB) = 1}$, and we have reduced
to the case of an adversary whose $\UPO$ calls only consist of insertions.

%%

Next, we move from $\advB$ to an $\advC$ that never inserts an element more than
once. Similarly to the previous step, $\advC$ simulates $\advB$, tracking the
elements of $\col$ and forwarding $\advB$'s oracle queries in
the natural way, except that any $\UPO$ queries to insert an element already
present in $\col$ are ignored. First, inserting $x$ does not
change whether $x$ is overestimated or not, so $\advC$ ignoring the re-insertion
does not affect whether later $\QRYO(\qry_x)$ calls will produce an error. For
$y \neq x$, the fact that $\advB$ makes no deletions is key. The value of the
counters associated with $y$ by the hash functions must be at least equal to the
true frequency of $y$, and $\QRYO(\qry_y)$ will find an overestimate if these
counters are all strictly greater than the true frequency. Since updates are
deterministic, re-inserting $x$ can only increment the same counters that were
incremented by the original insertion of $x$, and so this re-insertion cannot
cause $y$ to become overestimated if it was not already. So all $\QRYO$ calls
are just as likely to produce an error for $\advC$ as they are for $\advB$, and
$\Prob{\game_1(\advB) = 1} = \Prob{\game_1(\advC) = 1}$.

As a third step, we move from~$\game_1$ to a~$\game_2$ where the adversary gains
$k+1$ `points' for finding an query which produces an
overestimate, but which prevents the adversary from querying elements of $\col$.
These extra points are necessary because, unlike in the case of a Bloom
filter, inserting an overestimated element $x$ can cause other elements of
$\col$ to become overestimated. In particular, if one of the counters
incremented by the insertion of $x$ is shared with an element of $\col$ that is
not overestimated, that element may become overestimated. However, if that
counter is shared with multiple elements of $\col$, that counter is already an
overestimate for all of the elements associated with it, and so no more than one
overestimate can be caused per counter incremented by the insertion of $x$.
Since inserting $x$ increments $k$ counters, at most $k$ errors can be caused in
this way. For any adversary $\advC$ for~$\game_2$, we can construct $D$
for~$\game_3$ that simulates $\advC$ perfectly except that it ignores any oracle
calls that would insert these elements. Since $D$ already gets credit equal to
the maximum number of errors these insertions could cause in addition to the
credit for the original overestimate, $D$ accumulates at least as many errors as
$\advC$ does, and so $\Prob{\game_1(\advC) = 1} \le \Prob{\game_2(D) = 1}$.

Analagously to the proof of~\ref{thm:sbf-erreps-th}, we now move to a
game~$\game_3$ where $\REPO$ randomly fills the sketch to capacity after
inserting the elements of $\col$, so that each row has $\ell+1$ nonzero
counters. For any $D$ for~$\game_3$ we construct $E$ for~$\game_4$ that
simulates $D$, forwarding $\REPO$, $\QRYO$, and $\HASHO_1$ calls but ignoring
$\UPO$ calls. By a very similar argument, $E$ achieves at least the same
advantage as $D$ by having a maximally full sketch as soon as $\REPO$ is called,
and so $\Prob{\game_2(D) = 1} \le \Prob{\game_3{E} = 1}$.

The probability of $E$ winning can now be given by another binomial bound. Since
each row of the sketch is a uniformly random bitmap with $\ell+1$ out of $m$
bits set to 1, the probability of any particular $\QRYO$ call causing a
collision within a single row $i$ is $(\ell+1)/m$, and the probability of a
collision in every row (i.e. an error) is $((\ell+1)/m)^k$. The adversary has a
total of $q_T$ attempts, and wins if it accumulates $\lfloor r/(k+1) \rfloor$
successes. So, letting $p_\ell = ((\ell+1)/m)^k$ and
$r' = \lfloor r/(k+1) \rfloor$, we have
\begin{equation}
   \Prob{\game_3(E)=1} \le
     \sum_{i=r'}^{q_T} \binom{q_T}{i}p_\ell^i(1-p_\ell)^{q_T-i} \,.
\end{equation}
Applying the usual Chernoff bound and applying Lemma~\ref{thm:lemma1} turns this
into the final bound of
\begin{equation}
   \Adv{\erreps}_{\Pi,\delta,r}(\advA) \leq
     q_R \cdot \left[\frac{q_H}{2^\lambda} + e^{r'-p_\ell q_T}\left(\frac{p_\ell q_T}{r'}\right)^r\right].
\end{equation}
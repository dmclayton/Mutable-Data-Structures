\begin{figure*}
\cpnote{It should be clear from this figure which boxed statments are included
  in which games. Without the context of the text, the way someone would read
  this is ``gamke 0 includes all unboxed statements; game 1 includes all unboxed
  statments and all boxed statements; and game 2 includes all unboxed and
  gray-boxed statements.'' Please clarify this figure so that it's clear what
  statments are included in what games.}
\twoColsNoDivide{0.47}
{
  \vspace{-7pt}
  \experimentv{$\game_{0}(\advB)$}\\[2pt]
    $M^* \gets \bot$;
    $\salt^* \getsr \bits^\lambda$\\
    $\advB^{\REPO,\QRYO,\UPO,\HASHO_1}$;
    return $\big[\sum_x \err[x] \geq r\big]$
  \\[6pt]
  \oraclev{$\HASHO_c(\salt \cat x)$}\hfill \diffminus{$\game_1$}\\[2pt]
    $\vv \getsr [m]^k$\\
    if $\salt=\salt^*$ and $c = 1$ then \com{Caller is~$\advB$}\\
    \tab $\bad_1 \gets 1$; \diffminus{return $\vv$}\\
    if $T[Z,x] = \bot$ then $\vv \gets T[Z,z]$\\
    $T[Z,x] \gets \vv$; return $\vv$
}
{
  \oraclev{$\QRYO(\qry_x)$}\hfill \diffplus{$\game_2$}\\[2pt]
    $X \gets \bmap_m(\HASHO_3(\salt^* \cat x))$;
    $a \gets X = M^* \AND X$\\
    if $\err[x] < \delta(a,\qry_x(\col^*))$ then
          $\err[x] \gets \delta(a,\qry_x(\col^*))$\\
    \diffplus{$\UPO(\up_x)$}\\
    return $a$
  \\[6pt]
  \oraclev{$\REPO(\col)$}\\[2pt]
    \cpnote{I believe this is incorrect, since $\Rep$ is now thresholding.}
    $M^* \gets \bigvee_{x \in \col} \bmap_m(\HASHO_2(\salt^* \cat x))$;
    $\setS^* \gets \col$;
    return $\top$
  \\[6pt]
  \oraclev{$\UPO(\up_x)$}\\[2pt]
    if $w(M) > \ell$ then return $\top$\\
    if $\QRYO(\qry_x) = 1$ then $\err[x] \gets 0$\\
    $M^* \gets M^* \vee \bmap_m(\HASHO_2(\salt^* \cat x))$;
    $\setS^* \gets \up_x(\setS)$;
    return $\top$
}
\caption{Games 0, 1, and 2 for proof of Theorem~\ref{thm:sbf-erreps}.}
\label{fig:sbf-erreps/games}
\end{figure*}

This proof follows a similar structure to the previous one.
%
\cpnote{Don't refer to thoerems this way; say
Theorem~\ref{thm:sbf-errep-immutable}. That way this reference isn't dangling if
we restructure the paper.}
%
The primary distinction is that in the final game, unless the adversary `gets
lucky' and guesses the salt, they should only be able to produce errors with
$q_T$ queries, as opposed to both $q_T$ and $q_H$ queries.
%
Just as in the proof of Theorem~\ref{thm:sbf-errep-immutable}, we will assume
the adversary just makes a single call to $\REPO$ and use Lemma~\ref{thm:lemma1}
to complete the bound. Let $\advA$ be an \erreps\ adversary making exactly 1
call to $\REPO$, $q_T$ calls to $\QRYO$, $q_U$ calls to $\UPO$, and $q_H$ calls
to $\HASHO$. Because $\advA$ creates only a single representation, it will
necessarily lose if it calls $\REVO$ on that representation. We may therefore
assume without loss of generality that $\advA$ makes no calls to $\REVO$, and
because of this we omit $\REVO$ from each of the games.
%
\cpnote{Good. Somehwere in the body we'll need to justify why we include~$\REVO$
in the experiment, since at this point the reader has no reason to believe that
$\REVO$ captures something useful.}

In addition to the assumptions of the previous theorem \todo{DC}{Fix ref}, we
assume without loss of generality that the adversary never uses $\UPO$ to insert
an element into $\col$ which is already present in the set \todo{DC}{Double
check this assumption isn't made above}, and never uses $\UPO$ to insert an
element $x$ where $\QRYO(\qry_x)$ has already been called and has returned a
positive result. Since these insertions do not change the filter and in the
latter case may actually reduce the error count \cpnote{WHAT?? I don't think
that's true. Remember that the error function is \textbf{fixed} at this point.},
the adversary would gain no advantage from performing these updates.
Furthermore, we assume without loss of generality that an adversary halts as
soon as it determines it has accumulated enough errors to win the experiment.

We begin with a game~$\game_0(\advB)$
(Figure~\ref{fig:sbf-errep-immutable/games}) similar to the previous
proof\todo{DC}{fix ref}, except that it also defines an~$\UPO$ oracle. Again, we
observe that for every~$\advA$ there exists a~$\advB$  such that
\begin{equation}
  \Adv{\errep}_{\Pi,\delta,r}(\advA) \leq \Prob{\game_0(\advB) = 1}
\end{equation}
and~$\advB$ has the same query resources as~$\advA$.

Since we are seeking a stronger bound, we now wish to isolate the possibility
that the adversary \emph{ever} guesses the salt, as opposed to just guessing the
salt before calling $\REPO$. This is no longer a trivial task for the adversary
because the representations are private, and so $\REPO$ does not directly reveal
the salt. We therefore set the~$\bad_1$ flag whenever the adversary manages to
guess the salt, without the requirement that $M^* = \bot$. However, since the
adversary is still limited to a total of $q_H$ $\HASHO$ queries, regardless of
when the queries are made, we can follow nearly the same argument as in the
previous proof to get the bound
%
\begin{eqnarray}
  \Prob{\game_0(\advB)=1} &\leq&
    \Prob{\game_1(\advB)=1} + \Prob{\game_1(\advB) \sets \bad_1}\\
  &\leq&
    \Prob{\game_1(\advB)=1} + q_H/2^\lambda \,.
\end{eqnarray}
%
In~$\game_1$, we may now assume that the adversary never guesses the salt in its $\HASHO_1$ queries. This means that none of the inputs to $\HASHO_1$ is ever equal to any input to $\HASHO_2$ or $\HASHO_3$, both of which always use the salt $\salt^*$. Since each $\HASHO$ is modeled as a random oracle, the outputs of $\HASHO_1$ are therefore independent of the outputs of $\HASHO_2$ and $\HASHO_3$.
%
We still cannot move to the binomial distribution for non-adaptive queries,
however, since $\HASHO_2$ and $\HASHO_3$ queries are not necessarily independent
of each other. By one of our starting assumptions, the same input is never
provided twice to $\HASHO_2$ because the adversary never tries to insert an
element which is already in $\col$. We also argue \cpnote{Do you mean assume?} that (without loss of
generality) the same input is never provided twice to $\HASHO_3$, i.e. that the
same element is never queried twice.

\todo{DC}{Replace $\QRYO(x)$ with $\QRYO(\qry_x)$, here and everywhere else.}
On one hand, if a $\QRYO(x)$ call shows that $x$ is already a false positive for
$\pub$, further $\QRYO(x)$ calls cannot increase the adversary's error score. On
the other hand, if a $\QRYO(x)$ call shows that $x$ is not a false positive, it
is still possible for $x$ to become a false positive later on due to $\UPO$
calls. However, the adversary would obtain at least as large a chance of finding
a false positive by calling $\QRYO(y)$ for some previously unqueried $y
\not\in\col$, since the sequence of updates that could make $x$ a false positive
are just as likely to make $y$ a false positive by the independence of
$\HASHO_c$ calls on different inputs. We can therefore assume that the adversary
does not send repeated queries to $\QRYO$.
%
We have now reduced to a case where all hash queries are independent except for
$\QRYO$ and $\UPO$ calls to the same element. By our starting assumptions, an
adversary never calls $\QRYO$ on an element which has already been inserted, so
we need only consider the case of an element being queried before it is
inserted. In fact, doing this can be beneficial to a pollution attack, since
determining with $\QRYO$ that an element is not already a false positive informs
the adversary that inserting that element must necessarily set at least one new
bit in the filter to 1. Since all other possibilities have been eliminated, we
need only consider two types of update the adversary may make:
%
\begin{enumerate}
  \item Inserting an element which is not already in $\col$ and has previously been tested with $\QRYO$, returning a negative result.
  \item Inserting an element which is not already in $\col$ and has not previously been tested with $\QRYO$.
\end{enumerate}
Since calls to $\HASHO_3$ with different choices of $x$ are independent of each
other, and since $\HASHO_3$ uses random sampling, the effects of type 1 updates
on the representation are identically distributed. Similarly, since calls to
$\HASHO_2$ produce independent random results, the effects of type 2 updates on
the representation are also identically distributed. However, the effects of the
two types of update are \emph{not} identically distributed compared to each
other. In particular, making a type 1 update ensures that at least one new bit
in the filter will be set to 1, since the distribution of
$\bmap_m(\HASHO_2(\salt^* \cat x))$ is conditioned on not producing a false
positive. On the other hand, making a type 2 update provides no guarantee about
how many bits in the filter might be set to 1. Type 1 updates are therefore
always preferable for an adversary attempting to produce false positives.

\cpnote{Got here}
In fact, we can make an even stronger statement: it is always optimal for the
adversary to insert $x$ as soon as $\QRYO(x)$ reveals that $x$ is not a false
positive. Since this is a type 1 update, there is no `better' update which could
benefit the adversary more, so there is no reason for the adversary's next
$\UPO$ call to be anything other than $\UPO(\up_x)$. There is also no reason to
make additional $\QRYO$ calls before calling $\UPO(\up_x)$, since calling
$\UPO(\up_x)$ can only increase the probability that further $\QRYO$ calls
produce an error. We therefore assume without loss of generality that $\advB$
inserts $x$ after $\QRYO(x)$ returns a negative result, provided it still has at
least one $\UPO$ call remaining and this insertion would not increase the size
of the underlying set over the maximum number of $n$ elements.

For~$\game_2(C)$, then, we enforce this behavior, changing $\QRYO(x)$ to automatically insert $x$ into $\col$ after computing the correct response to the query. For any $\advB$ for~$\game_1$ we can construct $C$ for~$\game_2$ that simulates $\advB$ to attain the same advantage, forwarding oracle queries in the natural way except that any call of the form $\UPO(x)$ are ignored if $\QRYO(x)$ has been called previously. Ignoring these $\UPO$ calls does not negatively affect the adversary because in~$\game_2$ the original $\QRYO(x)$ call has already inserted $x$ into the set. Furthermore, the `enhanced' $\QRYO$ available to $C$ does not negatively affect $C$ at any point because inserting additional elements into the filter can only increase the probability that later queries return false positives. Then $C$ wins whenever $\advB$ does, and $\Prob{\game_1(\advB) = 1} \le \Prob{\game_2(C) = 1}$.

However, the parameters of the games played by $C$ and $\advB$ are slightly different. In particular, $\advB$ (and, by extension, $C$) may find up to $r$ false positives before halting. When $\advB$ finds these false positives they are by assumption not ever inserted into $\col$, while in the case of $C$ the $\QRYO$ oracle automatically inserts them into the set as soon as they are found. While inserting a false positive does not affect the filter itself in any way, it does increment the number of elements in the underlying set. Therefore if adversaries $\advB$ in~$\game_1$ are limited to representing a set of size $n$, we restrict adversaries $C$ in~$\game_2$ to representing sets of up to size $n+r$.

In~$\game_2$, $\UPO$ queries are actually superfluous. Since every element queried is automatically inserted into the set and the adversary never inserts an element more than once, $\UPO$ calls are now all of type 2. Since these insertions can only increase the chance of each following $\QRYO$ call being a false positive, it is optimal for the adversary to make all $\UPO$ calls at the beginning of the experiment, and then to make all $\QRYO$ calls. But this means we can assume without loss of generality that the adversary makes no $\UPO$ calls at all, since any elements added through $\UPO$ before any $\QRYO$ calls are made could just as easily have been included in the original call to $\REPO$ without affecting the adversary's advantage.

We now therefore only consider the case of a $\REPO$ call followed by the~$\game_2$ version of $\QRYO$ calls. Let $\setX$ be the set of all queries $\qry_x$ which are sent to $\QRYO$ over the course of the experiment. We necessarily have $|\setX| \le q_T$, and each $\qry_x \in \setX$ has some probability of causing an error. Since $\col^*$ never grows to contain more than $n+r$ elements, the false positive probability for each such $\qry_x$ is bounded above by $p_*$, the false-positive probability of a Bloom filter containing $n+r$ elements. So we have, analagously to the previous theorem,
\begin{equation}
   \Prob{\game_2(\advB)=1} \le
     \sum_{i=r}^{q_T} \binom{q_T}{i}p_*^i(1-p_*)^{q_T-i} \,,
\end{equation}
where $q_T$ replaces $q$ and the larger $p_*$ replaces $p$. Applying the same Chernoff bound reduces this to
\begin{equation}
   \Prob{\game_2(\advB)=1} \le
     e^{r-p_*q_T}\left(\frac{p_*q_T}{r}\right)^r.
\end{equation}

Again we apply Lemma~\ref{thm:lemma1} to get a final bound of
\begin{equation}
  \Adv{\erreps}_{\Pi,\delta,r}(\advA) \leq
    q_R \cdot \left[
      \frac{q_H}{2^\lambda} +
      \left(\frac{p_*q_T}{r}\right)^re^{r-p_*q_T}
    \right] \,.
\end{equation}

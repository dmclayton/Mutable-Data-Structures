\begin{figure*}
\threeColsOneDivideUnbalanced{0.40}{0.27}{0.27}
{
  \vspace{-7pt}
  \experimentv{$\game_{0}(\advB)$}
      \hfill \diffplus{$\game_1$}\\[2pt]
    $M^* \gets \bot$;
    $\salt^* \getsr \bits^\lambda$\\
    $\advB^{\REPO,\QRYO,\UPO,\HASHO_1}$;
    return $\big[\sum_x \err[x] \geq r\big]$
  \\[6pt]
  \oraclev{$\REPO(\col)$}\\[2pt]
    $M^* \gets \bigvee_{x \in \col} \bmap_m(\HASHO_2(\salt^* \cat x))$;
    $\setS^* \gets \col$;
    return $\top$
  \\[6pt]
  \oraclev{$\QRYO(\qry_x)$}\\[2pt]
    $X \gets \bmap_m(\HASHO_3(\salt^* \cat x))$;
    $a \gets X = M^* \AND X$\\
    if $\err[x] < \delta(a,\qry_x(\col^*))$ then
          $\err[x] \gets \delta(a,\qry_x(\col^*))$\\
    return $a$
  \\[6pt]
  \oraclev{$\UPO(\up_x)$}\\[2pt]
    $M^* \gets M^* \vee \bmap_m(\HASHO_2(\salt^* \cat x))$;
    $\setS^* \gets \up_x(\setS)$;
    return $\top$
  \\[6pt]
  \oraclev{$\HASHO_c(\salt \cat x)$}\\[2pt]
    $\vv \getsr [m]^k$\\
    if $\salt=\salt^*$ and $c = 1$ then \com{Caller is~$\advB$}\\
    \tab $\bad_1 \gets 1$; \diffplus{return $\vv$}\\
    if $T[Z,x] = \bot$ then $\vv \gets T[Z,z]$\\
    $T[Z,x] \gets \vv$; return $\vv$
}
{
  \vspace{-2pt}
  \oraclev{$\HASHO_c(\salt \cat x)$}\\[2pt]
    $\vv \getsr [m]^k$\\
    if $M^*=\bot$ and $\salt=\salt^*$ and $c=1$ then\\
    \tab $\bad_1 \gets 1$; return $\vv$\\
    if $T[Z,x] = \bot$ then $\vv \gets T[Z,z]$\\
    $T[Z,x] \gets \vv$\\[2pt]
    \diffplusbox{
    \com{Caller is~$\advB$ or $\QRYO$}\\
    if $c = 1$ or $c = 3$ then\\
    \tab if $\salt \ne \salt^*$  then return $\vv$\\
    \tab $\Ans[x] \gets \bmap_m(\vv) = M^* \AND \bmap_m(\vv)$\\
    \tab if $\err[x] < \delta(\Ans[x],\qry_x(\col^*))$ then
    \tab\tab $\err[x] \gets \delta(\Ans[x],\qry_x(\col^*))$
    }
    return $\vv$
}
{
  \vspace{-7pt}
  \oraclev{$\QRYO(\qry_x)$}\
      \hfill \diffminus{$\game_1$} \diffplus{$\game_2$}\\[2pt]
    \diffminusbox{%
      $X \gets \bmap_m(\HASHO_3(\salt^* \cat x))$\\
      $a \gets X = M^* \AND X$\\
      if $\err[x] < \delta(a,\qry_x(\col^*))$ then\\
      \tab $\err[x] \gets \delta(a,\qry_x(\col^*))$
    }\\[2pt]
    \diffplusbox{
      $\HASHO_3(Z^* \cat x)$\\
      $a \gets \Ans[x]$
    }
    return $a$
}
\caption{Games 0, 1, and 2 for proof of Theorem~\ref{thm:sbf-errep-priv}.}
\label{fig:sbf-errep-immutable/games}
\end{figure*}

This proof follows a similar structure to the previous one. The primary distinction is that in the final game, unless the adversary `gets lucky' and guesses the salt, they should only be able to produce errors with $q_T$ queries, as opposed to both $q_T$ and $q_H$ queries.
%
Just as in the proof of Theorem~\ref{thm:sbf-errep-immutable}, we will assume the adversary just makes a single call to $\REPO$ and use Lemma~\ref{thm:lemma1} to complete the bound. Let $\advA$ be an \erreps\ adversary making exactly 1 call to $\REPO$, $q_T$ calls to $\QRYO$, $q_U$ calls to $\UPO$, and $q_H$ calls to $\HASHO$. Because $\advA$ creates only a single representation, it will necessarily lose if it calls $\REVO$ on that representation. We may therefore assume without loss of generality that $\advA$ makes no calls to $\REVO$, and because of this we omit $\REVO$ from each of the games.

We begin with a game~$\game_0(\advB)$ similar to the previous proof, as shown in Figure~\ref{fig:sbf-errep-immutable/games}. Again we observe that for every~$\advA$ there exists a~$\advB$  such that
\begin{equation}
  \Adv{\errep}_{\Pi,\delta,r}(\advA) \leq \Prob{\game_0(\advB) = 1}
\end{equation}
and~$\advB$ has the same query resources as~$\advA$.

Since we are seeking a stronger bound, we now wish to isolate the possibility that the adversary \emph{ever} guesses the salt, as opposed to just guessing the salt before calling $\REPO$. This is no longer a trivial task for the adversary because the representations are private, and so $\REPO$ does not directly reveal the salt. We therefore set the~$\bad_1$ flag whenever the adversary manages to guess the salt, without the requirement that $M^* = \bot$. However, since the adversary is still limited to a total of $q_H$ $\HASHO$ queries, regardless of when the queries are made, we can follow nearly the same argument as in the previous proof to get the bound
%
\begin{eqnarray}
  \Prob{\game_0(\advB)=1} &\leq&
    \Prob{\game_1(\advB)=1} + \Prob{\game_1(\advB) \sets \bad_1}\\
  &\leq&
    \Prob{\game_1(\advB)=1} + q_H/2^\lambda \,.
\end{eqnarray}
%



\ignore{
\begin{figure*}
  \cpnote{I suggest re-doing this from scratch. The security experiment
  (\erreps) has changed and so has the construction. $\Repx$ and $\fff$ are not
  defined. Where's the $\REVO$ oracle?}
  \boxThmBFSaltCorrect{0.48}
  {
    \underline{$\game_0(\advA)$}\\[2pt]
      $\col \getsr \advA^H$; $\setC \gets \emptyset$; $\err \gets 0$\\
      $\pub \getsr \Rep[H](\col)$\\
      $\bot \getsr \advA^{H,\QRYO,\UPO}$\\
      return $(\err \geq r)$
    \\[6pt]
    \oraclev{$\QRYO(\qry_x)$}\\[2pt]
      if $\qry_x \in \mathcal{C}$ then return $\bot$\\
      $\setC \gets \setC \union \{\qry_x\}$\\
      $a \gets \Qry[H](\pub, \qry_x)$\\
      if $a \neq \qry_x(\col)$ then $\err \gets \err + 1$\\
      return~$a$
    \\[6pt]
    \oraclev{$\UPO(\up_x)$}\\[2pt]
      $\setC \gets \emptyset$\\
      $a \gets \Qry[H](\pub, \qry_x)$\\
      if $\qry_x \in \setC$ and $a \neq \qry_x(\col)$ then\\
      \tab $\err \gets \err-1$\\
      $\col \gets \col \union \{x\}$\\
      $\pub \gets \Up[H](\pub,\up_x)$\\
      return~$\bot$
    \\[4pt]
    \hspace*{-4pt}\rule{1.043\textwidth}{.4pt}
    \\[5pt]
    \oraclev{$\HASHO_1(\salt,x)$} \hfill\diffplus{$\game_2$}\;{$\game_1$}\hspace*{3pt}\\
      $\hh \getsr [m]^2$; $\vv \gets \fff(\hh)$\\
      if $\salt = \salt^*$ then\\
      \tab $\bad_1 \gets 1$; \diffplus{return $\vv$}\\
      if $T[\salt,x]$ is defined then $\vv \gets T[\salt,x]$\\
      $T[\salt,x] \gets \vv$;
      return $\vv$
  }
  {
    \underline{$\game_1(\advB)$}\\[2pt]
      $\salt^* \getsr \bits^\lambda$;
      $\col \getsr \advB^{\HASHO_1}$\\
      $\pub \gets \Repx[\HASHO_2](\col, \salt^*)$\\
      $\setC \gets \emptyset$;
      $\err \gets 0$\\
      $\bot \getsr \advB^{\HASHO_1,\QRYO,\UPO}$\\
      return $(\err \geq r)$
    \\[6pt]
    \oraclev{$\QRYO(\qry_x)$}\\[2pt]
      if $\qry_x \in \mathcal{C}$ then return $\bot$\\
      $\setC \gets \setC \cup \{\qry_x\}$\\
      $a \gets \Qry[\HASHO_2](\pub, \qry_x)$\\
      if $a \neq \qry_x(\col)$ then $\err \gets \err + 1$\\
      return~$a$
    \\[6pt]
    \oraclev{$\UPO(\up_x)$}\\[2pt]
      $\setC \gets \emptyset$\\
      $a \gets \Qry[H](\pub, \qry_x)$\\
      if $a \neq \qry_x(\col)$ and $\qry_x \in \setC$ then\\
      \tab $\err \gets \err-1$\\
      $\col \gets \col \union \{x\}$\\
      $\pub \gets \Up[\HASHO_2](\pub,\up_x)$\\
      return~$\bot$
    \\[6pt]
    \oraclev{$\HASHO_2(\salt,x)$}\\[2pt]
      $\hh \getsr [m]^2$; $\vv \gets \fff(\hh)$\\
      if $T[\salt,x]$ is defined then\\
      \tab $\vv \gets T[\salt,x]$\\
      $T[\salt,x] \gets \vv$;
      return $\vv$
  }
  {
    \underline{$\game_3(\advB)$}\\[2pt]
    \oraclev{$\QRYO(\qry_x)$}\\[2pt]
      $a \gets \Qry[\HASHO_3](\pub, \qry_x)$\\
      if $a \neq \qry_x(\col)$ then $\err \gets \err + 1$\\
      $\col \gets \col \union \{x\}$
      $\pub \gets \Up[\HASHO_2](\pub,\up_x)$\\
      return~$a$
  }
  {
    \oraclev{$\UPO(\up_x)$}\\[2pt]
      $a \gets \Qry[\HASHO_3](\pub, \qry_x)$\\
      if $a \neq \qry_x(\col)$ then $\err \gets \err + 1$\\
      $\col \gets \col \union \{x\}$
      $\pub \gets \Up[\HASHO_2](\pub,\up_x)$\\
      return~$\bot$
    \\[6pt]
    \oraclev{$\HASHO_i(\salt,x)$}\\[2pt]
      $\hh \getsr [m]^2$; $\vv \gets \fff(\hh)$\\
      return $\vv$
  }
  \caption{Games 0--3 for proof of Theorem~\ref{thm:sbf-erreps}.}
  \label{fig:sbf-erreps/games}
\end{figure*}

\cpnote{I understand the crux of the argument of how you deal with interleaved
updates/queries. It's a clever idea and I think it's believable. That said,
there's a lot of details that are omitted addressed.
%
Right now the biggest problem with this argument is that the games have
virtually nothing to do with the security notions and nothing to do with the
scheme being analyzed. They seem to be a carry-over from the old paper, but
things have change significantly. They will need to be rewritten. Try using the
games in Figure~\ref{fig:sbf-errep1/games} as a reference.}

\cpnote{I'm not clear on how update thresholding is used in the argument. From
my read it seems you're assuming some maximum represented set size, but we're
not maintaining a counter in the construction. Here's a hint: if thresholding is
necessary for security, then I'd expect $\ell$ to come up in the bound; if it
doesn't, then there'd better be a good reason.}

\cpnote{The argument silently assumes that there's no $\REVO$ oracle.}

\cpnote{Try starting this way:}
%
Just as in the proof of Theorem~\ref{thm:sbf-errep1} we will assume the
advwersary just makes a single query to~$\QRYO$ and use Lemma~\ref{thm:lemma1}
to complete the bound.
%
Let $\advA$ be an \erreps adversary making exactly~$1$ query to~$\REPO$, $q_T$
queries to~$\QRYO$, $q_U$ queries to~$\UPO$, and $q_H$ queries to~$\HASHO$.


\cpnote{Tip: If an claim follows easily from an argument made earlier than the
proof, then feel free to move quickly through it and refer the reader to the
argument for detials. The best you can do is say something like ``Equation (X)
follows nearly the same argument as used to deerive Equation (Y) ...''}

\cpnote{It'll be easier to apply Lemma~\ref{thm:lemma1} at the end.}
We first reduce from the \erreps case to the \erreps1 case, which by
lemma~\ref{lemma:errep} may scale the adversary's advantage only by a factor of
$q_R$. The game~$\game_0$ is exactly equivalent to the $\erreps1$ experiment, so
$\Adv{\errep1}_{\struct_s,r}(\advA) = \Prob{\game_0(\advA) = 1}$.%

In~$\game_1$ we split the hash oracle into three, giving the adversary access
$\HASHO_1$ in both stages of the game, while $\HASHO_2$ is reserved for oracular
use by $\Repx$ \cpnote{Undefined!}, $\QRYO$, and $\UPO$. For any $\advA$ for~$\game_0$, there is
$\advB$ for~$\game_1$ which produces the same advantage by simulating $\advA$.
This adversary first creates its own table $R$ with all values initially
undefined.  When $\advA$ makes a query $w$ to $H$, $\advB$ returns $R[w]$ if
that entry in the table is defined. Otherwise, if there are $\salt \in
\bits^\lambda$, $j \in [k]$, and $x \in \bits^*$ such that $w = \langle\salt, j,
x\rangle$, forward $(\salt,x)$ to $\HASHO_1$. For each $j \in [k]$, set
$R[\langle\salt, j, x\rangle] = \vv_j$, where $\vv$ is the output of the
$\HASHO_1$ oracle. If there is no such triple $\langle\salt, j, x\rangle$, just
sample $r$ from $[m]$ uniformly and set $R[w] = r$.
%
\cpnote{All of this is out date. This is about the linear hashing scheme of
Kirsch-Mitzenmacher, which we're no longer analyzing.}
%
In either case, return
$R[w]$ to $\advA$. When $\advA$ outputs its collection $\col$, $\advB$ outputs
$\col$ as well. Any queries by $\advA$ to $\QRYO$ or $\UPO$ are forwarded to
$\advB$'s corresponding oracle. The simulation is perfect because
$\Rep[H](\col)$ and $\Up[H](\col,\up)$ are identically distributed to
$\Rep[\HASHO_2](\col)$ and
$\Up[\HASHO_2](\col,\up)$. Because we have a perfect simulation,
$\Adv{\erreps1}_{\struct_s,r}(\advA) = \Prob{\game_1(\advA) = 1}$.

The game~$\game_2$ is the same as~$\game_1$ until $\bad_1$ is set, which occurs
exactly when $\advB$ sends $(\salt^*,x)$ to $\HASHO_1$ for some $x$. In the
first phase, there is again a $q_1/2^\lambda$ chance of the adversary guessing
the salt. In the second phase, the random sampling used by $\HASHO_i$ ensures
that each call the adversary makes to the $\HASHO_i$ oracle is independent of
all previous calls. We therefore have a $q_2/2^\lambda$ chance of the adversary
guessing the salt during this phase, for a total chance of $q_H/2^\lambda$
chance of the adversary guessing the salt at some point during the experiment.
Then $\Adv{\erreps}_{\struct_s,r}(\advA) \le \Prob{\game_2(\advB) = 1} +
q_H/2^\lambda$. Having taken this into account, we may now assume the adversary
never guesses the salt.
%
\cpnote{I would move quickly through this point. Just refer back to the step in
Theorem~\ref{thm:sbf-errep1}.}


\cpnote{The rest of this seems like the meat of the argument.}

We want to show that alternating between sequences of queries and sequences of
updates is no better than making one long series of updates and then one long
sequence of queries. There are three types of updates the adversary can make:
updates to add elements that have been queried and found to be false positives;
updates to add elements that have been queried and found not to be false
positives; and updates to add elements that have not been queried yet. We may
assume without loss of generality that the adversary never makes the first type
of update, since doing so is never beneficial (it does not change the
representation at all and decreases the number of errors the adversary has
found). \cpnote{Explicitly name these type-1, type-2, and type-3 updates, since
you refer to them below. IN fact, it might be beneficial to put them in a
bultted list.}

Note that the choices of $\vv$ constructed by the $\HASHO_i$ oracles are
independent of all previous queries. Because of this, any update of type-3 is
equivalent to any other update of type-3%
%
\cpnote{Careful with the word ``equivalent'': I think you mean ``have the same
distribution'' or something?}%
%
; the probability of any bit being flipped by one update is the same as the
probability of the bit being flipped by the other update. Similarly, any update
of type-2 is equivalent to any other update of type-2, but is not the same as
type-3 since the probability is conditioned on $\vv$ not being a false positive.
We assume the worst case, namely that all updates are type-2 (i.e. at least one
bit is flipped by each update).

Because the adversary never guesses the salt, $\HASHO_1$ simply functions as a
random oracle.
%
\cpnote{This isn't quite true. Go back to Thereom~\ref{fig:sbf-errep1} and think
about the semantics of~$\HASHO_1$ in games~0 and~1.}
%
Furthermore, we can assume the adversary never adds an element of
$\col$ to $\col$%
%
\cpnote{Word this differently. You mean update the data structure with an
element that is already in it?}
%
and never makes a $\QRYO$ call for an element which is already
in $\col$, since neither of these provides any additional information and
neither affects the rest of the experiment in any way.

Now we move to the game~$\game_3$. Here each $\QRYO$ query also calls $\UPO$ to
add that element to $\col$. \cpnote{Clever.}  Additionally, the penalty for adding known false
positives is removed. To avoid penalizing the adversary by prematurely maxing
out the number of elements in $\col$ because of added false positives, we also
increase the maximum size of $\col$ from $n$ to $n+s$, where $s = \min(r,q_U)$.
Because the adversary (without loss of generality) stops after accumulating $r$
errors, only $\min(r,q_U)$ false positives will be added to $\col$ and so a
maximum size of $n+s$ is sufficient to produce no penalty for the adversary.
Furthermore, each $\UPO$ call is preceded by a $\QRYO$ call. Neither of these
changes can produce a worse result for the adversary, so $\Prob{\game_2(\advB) =
1} \le \Prob{\game_3(\advB) = 1}$. Now, however, there is no longer any
distinction between $\QRYO$ and $\UPO$ calls. All calls to either oracle are
independent of each other and produce the same effect, querying and then
updating $\col$. Each of these queries for false positives is at most as
successful as a query to a Bloom filter with $n+r$ elements, so the adversary's
probability of finding a false positive on any query is bounded above by the
standard success rate for a Bloom filter with those parameters. The adversary is
required to produce $r$ errors over the course of $q_T+q_U$ queries, which by
the binomial theorem gives an advantage bound of $\Prob{\game_3(\advB) = 1} \leq
...$.
%
\cpnote{What about~$q_H$?}

\todo{DC (lead)}{Finish the bound by applying Lemma~\ref{thm:lemma1}.}}
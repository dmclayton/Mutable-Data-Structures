\begin{figure*}
\twoColsNoDivide{0.47}
{
  \vspace{-7pt}
  \experimentv{$\game_{0}(\advB)$}\hfill \diffminus{$\game_1$}\diffplus{$\game_2$}\\[2pt]
    $M^* \gets \bot$;
    $\salt^* \getsr \bits^\lambda$\\
    $\advB^{\REPO,\QRYO,\UPO,\HASHO_1}$;
    return $\big[\sum_x \err[x] \geq r\big]$
  \\[6pt]
  \oraclev{$\HASHO_c(\salt \cat x)$}\\[2pt]
    $\vv \getsr [m]^k$\\
    if $\salt=\salt^*$ and $c = 1$ then \com{Caller is~$\advB$}\\
    \tab $\bad_1 \gets 1$; \diffplus{\diffminus{return $\vv$}}\\
    if $T[Z,x] = \bot$ then $\vv \gets T[Z,x]$\\
    $T[Z,x] \gets \vv$; return $\vv$
}
{
  \oraclev{$\QRYO(\qry_x)$}\\[2pt]
    $X \gets \bmap_m(\HASHO_3(\salt^* \cat x))$;
    $a \gets X = M^* \AND X$\\
    if $\err[x] < \delta(a,\qry_x(\col^*))$ then
          $\err[x] \gets \delta(a,\qry_x(\col^*))$\\
    \diffplus{$\UPO(\up_x)$}\\
    return $a$
  \\[6pt]
  \oraclev{$\REPO(\col)$}\\[2pt]
    $M^* \gets \bigvee_{x \in \col} \bmap_m(\HASHO_2(\salt^* \cat x))$;
    $\setS^* \gets \col$;
    return $\top$
  \\[6pt]
  \oraclev{$\UPO(\up_x)$}\\[2pt]
    if $w(M) > \ell$ then return $\top$\\
    if $\QRYO(\qry_x) = 1$ then $\err[x] \gets 0$\\
    $M^* \gets M^* \vee \bmap_m(\HASHO_2(\salt^* \cat x))$;
    $\setS^* \gets \up_x(\setS^*)$;
    return $\top$
}
\caption{Games 0, 1, and 2 for proof of Theorem~\ref{thm:sbf-erreps}.}
\label{fig:sbf-erreps/games}
\end{figure*}

%This proof follows a similar structure to the proof of
%Theorem~\ref{thm:sbf-errep-immutable}.
%
%The primary distinction is that in the final game, unless the adversary ``gets
%lucky'' and guesses the salt, they should only be able to produce errors with
%$q_T$ queries, as opposed to both $q_T$ and $q_H$ queries.
%
Just as in the proof of Theorem~\ref{thm:sbf-errep-immutable}, we will assume
the adversary just makes a single call to $\REPO$ and use Lemma~\ref{thm:lemma1}
to complete the bound. Let $\advA$ be an \erreps\ adversary making exactly 1
call to $\REPO$, $q_T$ calls to $\QRYO$, $q_U$ calls to $\UPO$, and $q_H$ calls
to $\HASHO$. Because $\advA$ creates only a single representation, it will
necessarily lose if it calls $\REVO$ on that representation. We may therefore
assume without loss of generality that $\advA$ makes no calls to $\REVO$, and
because of this we omit $\REVO$ from each of the games.
%
%\cpnote{Good. Somehwere in the body we'll need to justify why we include~$\REVO$
%in the experiment, since at this point the reader has no reason to believe that
%$\REVO$ captures something useful.}

In addition to the assumptions of Theorem~\ref{thm:sbf-errep-immutable}, we
assume without loss of generality that the adversary never uses $\UPO$ to insert
an element into $\col$ which is already present in the set, and never uses
$\UPO$ to insert an element $x$ where $\QRYO(\qry_x)$ has already been called
and has returned a positive result. Since these insertions do not change the
filter, the adversary would gain no advantage from performing these updates.
Furthermore, we assume without loss of generality that an adversary halts as
soon as it determines it has accumulated enough errors to win the experiment.

We begin with a game~$\game_0(\advB)$
(Figure~\ref{fig:sbf-errep-immutable/games}) similar to the first game in the
proof of Theorem~\ref{thm:sbf-errep-immutable}, except that it also defines
an~$\UPO$ oracle. Again, we observe that for every~$\advA$ there exists
a~$\advB$ such that
\begin{equation}
  \Adv{\errep}_{\Pi,\delta,r}(\advA) \leq \Prob{\game_0(\advB) = 1}
\end{equation}
and~$\advB$ has the same query resources as~$\advA$.

Since we are seeking a stronger bound, we now wish to isolate the possibility
that the adversary \emph{ever} guesses the salt, as opposed to just guessing the
salt before calling $\REPO$. This is no longer a trivial task for the adversary
because the representations are private, and so $\REPO$ does not directly reveal
the salt. We therefore set the~$\bad_1$ flag whenever the adversary manages to
guess the salt, without the requirement that $M^* = \bot$. However, since the
adversary is still limited to a total of $q_H$ $\HASHO$ queries, regardless of
when the queries are made, we can follow nearly the same argument as in the
previous proof to get the bound
%
\begin{eqnarray}
  \Prob{\game_0(\advB)=1} \leq
    \Prob{\game_1(\advB)=1} + q_H/2^\lambda \,.
\end{eqnarray}
%
In~$\game_1$, $\HASHO_1$ queries are always independent of $\HASHO_2$ and
$\HASHO_3$ queries. In particular, is irrelevant whether the adversary guesses
the salt.
%
We still cannot move to the binomial distribution for non-adaptive queries,
however, since $\HASHO_2$ and $\HASHO_3$ queries are not necessarily independent
of each other. By one of our starting assumptions, the same input is never
provided twice to $\HASHO_2$ because the adversary never tries to insert an
element which is already in $\col$. We also want to show that an adversary never
queries the same element twice. To do this, let $\advB$ be an adversary
for~$\game_1$. We construct an adversary $\advC$ that achieves at least the same
advantage without making repeated queries. This $\advC$ simulates $\advB$,
maintaining a list of queries that have been made so far during the
game. Any $\REPO$, $\UPO$, or $\HASHO_1$ queries from $\advB$ are forwarded to $\advC$'s
oracles without performing additional computations. When $\advB$ makes a
$\QRYO(\qry_x)$ call, $\advC$ checks whether $x$ has already been queried. If
so, $\advC$ selects as $y$ the lexicographically first string such that
$y \not\in \col$ and such that $y$ has not been previously
queried, and calls $\QRYO(\qry_y)$ instead of $\QRYO(\qry_x)$.

Recall that $\advB$ makes no queries for elements of $\col$, so any repeated
queries must have returned either false positives or true negatives the first
time they were queried. If $\advB$ makes a $\QRYO(\qry_x)$ where $x$ was
previously found to be a false positive, the total number of errors cannot
possibly increase since $\advB$ has already gotten credit for this error. On the
other hand, if $\advB$ calls $\QRYO(\qry_x)$ for an $x$ that was previously
found to be a true negative, it is possible that $x$ has since become a false
positive due to $\UPO$ calls that have occurred since. However, since $\HASHO_3$
calls are independent of $\HASHO_2$ calls with different inputs, it is just as
likely that those intervening updates have made $y$ a false positive. Therefore,
regardless of what type of queries $\advB$ makes, $\advC$ makes queries that are
at least as likely to produce false positives and is therefore at least as
likely to win, i.e. $\Prob{\game_1(\advB) = 1} \le \Prob{\game_1(\advC) = 1}$.
Since $\advC$ only changes the inputs to some of $\advB$'s oracle queries, but
does not change whether or not a query is made, $\advB$ and $\advC$ have
identical query resources.

By the reduction from $\advB$ to $\advC$, we are now dealing with an adversary
where the hash queries are all independent except for $\QRYO$ and $\UPO$ calls
to the same element. We will now further reduce from~$\advC$ to $D$, where $D$
immediately follows any $\QRYO(\qry_x)$ that finds a true negative with a
call to $\UPO(\up_x)$ to insert that element. We have $D$ simulate $\advC$ while
maintaining a count of updates that have been performed so far. Any $\REPO$ and
$\HASHO_1$ calls from $\advC$ are forwarded to $D$'s oracles without
performing any additional computation. Any $\QRYO(\qry_x)$ call is also
forwarded, but if the oracle reveals the element is a true negative then $D$
immediately calls $\UPO(\up_x)$ unless $q_U$ updates have already been
performed. Finally, if $\advC$ makes an $\UPO$ call then $D$ forwards the call
unless it has already made $q_U$ updates, in which case it just returns $\top$
to $\advC$.

By our earlier assumptions, there are only two types of update $\advC$ may make:
%
\begin{enumerate}
  \item Inserting an element which is not already in $\col$ and has previously been tested with $\QRYO$, returning a negative result.
  \item Inserting an element which is not already in $\col$ and has not previously been tested with $\QRYO$.
\end{enumerate}
Since calls to $\HASHO_3$ with different choices of $x$ are independent of each
other, and since $\HASHO_3$ uses random sampling, the effects of type 1 updates
on the representation are identically distributed. Similarly, since calls to
$\HASHO_2$ produce independent random results, the effects of type 2 updates on
the representation are also identically distributed. However, the effects of the
two types of update are \emph{not} identically distributed compared to each
other. In particular, making a type 1 update ensures that at least one new bit
in the filter will be set to 1, since the distribution of
$\bmap_m(\HASHO_2(\salt^* \cat x))$ is conditioned on not producing a false
positive. On the other hand, making a type 2 update provides no guarantee about
how many bits in the filter might be set to 1. Type 1 updates are therefore
always preferable for an adversary attempting to produce false positives.

Note that, at any point during the experiment, $D$ has always made at least as
many updates as $\advC$ has. Furthermore, all of the updates made by $D$ but not
by $\advC$ are type 1 updates, which are maximally effective at increasing the
error rate. Therefore any $\QRYO$ calls made by $\advC$ have at least the same
probability of causing an error when made by $D$, and so
$\Prob{\game_1(\advC) = 1} \le \Prob{\game_1(D) = 1}$. Since $D$ is
capped at making $q_U$ total updates and its handling of other oracles is
identical, its query resources are still the same as $\advC$.

For~$\game_2$, then, we enforce this update-after-query behavior, changing
$\QRYO(\qry_x)$ to
automatically insert $x$ into $\col$ after computing the correct response to the
query. For any $D$ for~$\game_1$ we can construct $E$ for~$\game_2$ that
simulates $D$ to attain the same advantage, forwarding oracle queries in the
natural way except that any call of the form $\UPO(\up_x)$ are ignored if $\QRYO(\qry_x)$
has been called previously. Ignoring these $\UPO$ calls does not negatively
affect the adversary because in~$\game_2$ the original $\QRYO(\qry_x)$ call
already automatically inserts $x$ into the set. Then $E$ wins whenever $D$ does, and
$\Prob{\game_1(D) = 1} \le \Prob{\game_2(E) = 1}$. Since $E$ performs at most as
many oracle queries as $D$, its query resources are the same.

However, the parameters of the games played by $E$ and $D$ are slightly
different. In particular, $D$ (and, by extension, $E$) may find up to $r$
false positives before halting. When $D$ finds these false positives they
are by assumption not ever inserted into $\col$, while in the case of $E$ the
$\QRYO$ oracle automatically inserts them into the set as soon as they are
found. While inserting a false positive does not affect the filter itself in any
way, it does increment the number of elements in the underlying set. Therefore
if adversaries $D$ in~$\game_1$ are limited to representing a set of size
$n$, we restrict adversaries $E$ in~$\game_2$ to representing sets of up to size
$n+r$.

\ignore{In~$\game_2$, $\UPO$ queries are actually superfluous. Since every element
queried is automatically inserted into the set and the adversary never inserts
an element more than once, $\UPO$ calls are now all of type 2. Since these
insertions can only increase the chance of each following $\QRYO$ call being a
false positive, it is optimal for the adversary to make all $\UPO$ calls at the
beginning of the experiment, and then to make all $\QRYO$ calls. But this means
we can assume without loss of generality that the adversary makes no $\UPO$
calls at all\todo{DC}{That's true for~$E$, but not he adversary you started
with~$A$. The final bound should account for~$A$'s as well as the overhead of
the sequence of reductions.}, since any elements added through $\UPO$ before any $\QRYO$ calls
are made could just as easily have been included in the original call to $\REPO$
without affecting the adversary's advantage.}

We now therefore only consider the case of a $\REPO$ call followed by
the~$\game_2$ version of $\QRYO$ calls. Let $\setX$ be the set of all queries
$\qry_x$ which are sent to $\QRYO$ over the course of the experiment. We
necessarily have $|\setX| \le q_T$, and each $\qry_x \in \setX$ has some
probability of causing an error. Since $\col^*$ never grows to contain more than
$n+r$ elements regardless of what $\REPO$ or $\UPO$ calls are made, and because
the results of $\QRYO$ calls are independent of any prior oracle queries, the
false positive probability for each such $\qry_x$ is bounded
above by $p'$, the false-positive probability of a Bloom filter containing $n+r$
elements. So we have, analagously to the proof of
Theorem~\ref{thm:sbf-errep-immutable},
\begin{equation}
   \Prob{\game_2(E)=1} \le
     \sum_{i=r}^{q_T} \binom{q_T}{i}p'^i(1-p')^{q_T-i} \,,
\end{equation}
where $q_T$ replaces $q$ and the larger $p'$ replaces $p$. Applying the same Chernoff bound reduces this to
\begin{equation}
   \Prob{\game_2(E)=1} \le
     e^{r-p'q_T}\left(\frac{p'q_T}{r}\right)^r.
\end{equation}
%
Again we apply Lemma~\ref{thm:lemma1} to get a final bound of
\begin{equation}
  \Adv{\erreps}_{\Pi,\delta,r}(\advA) \leq
    q_R \cdot \left[
      \frac{q_H}{2^\lambda} +
      \left(\frac{p'q_T}{r}\right)^re^{r-p'q_T}
    \right] \,.
\end{equation}
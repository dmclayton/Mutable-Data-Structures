\begin{figure*}
\twoCols{0.47}
{
  \vspace{-7pt}
  \experimentv{$\game_{0}(\advA)$}\hfill\diffplus{$\game_1$}\\[2pt]
    $\v.M^* \gets \bot$;
    $\setS \gets \emptyset$;
    $\salt^* \getsr \bits^\lambda$\\
    $\advB^{\REPO,\QRYO,\UPO,\HASHO_1}$;
    return $\big[\sum_x \err[x] \geq r\big]$
  \\[6pt]
  \oraclev{$\HASHO_c(\salt \cat x)$}\\[2pt]
    $\vv \getsr [m]^k$\\
    if $\salt=\salt^*$ and $c = 1$ then \com{Caller is~$\advB$}\\
    \tab $\bad_1 \gets 1$; \diffplus{return $\vv$}\\
    if $T[Z,x] = \bot$ then $\vv \gets T[Z,x]$\\
    $T[Z,x] \gets \vv$; return $\vv$
  \\[6pt]
  \oraclev{$\QRYO(\qry_x)$}\\[2pt]
    $\v.X \gets \HASHO_3(\salt^* \cat x)$;
    $\setS \gets \setS \cup \{x\}$;
    $a = 1$\\
    for $i$ in $\v.X$ do\\
      $\tab$if $\v.M[i] = 0$ then $a = 0$\\
    if $\err[x] < \delta(a,\qry_x(\col^*))$ then
          $\err[x] \gets \delta(a,\qry_x(\col^*))$\\
    return $a$
  \\[6pt]
  \oraclev{$\REPO(\col)$}\\[2pt]
    $\v.M^* \gets 0^m$\\
    $\setS^* \gets \col$\\
    for $x \in \col$ do\\
      $\tab\UPO(\up_x)$\\
    return $\top$
  \\[6pt]
  \oraclev{$\UPO(\up_{x,b})$}\\[2pt]
    if $w'(\v.M^*) > \ell$ then return $\top$\\
    $\v.X \gets \HASHO_3(\salt^* \cat x)$;
    $\v.M' \gets \v.M^*$\\
    for $i$ in $\v.X$ do\\
      $\tab$ if $\v.M'[i] = 0$ and $b < 0$ then return $\top$\\
      $\tab \v.M'[i] \gets \v.M'[i] + b$\\
    if $b > 0$ and $\QRYO(\qry_x) = 1$ then $\err_i[x] \gets 0$\\
    if $b < 0$ and $\QRYO(\qry_x) = 0$ then $\err_i[x] \gets 0$\\
    $\v.M^* \gets \v.M'$;
    $\setS^* \gets \up_{x,b}(\setS^*)$;
    return $\top$
}
{
  \vspace{-7pt}
  \experimentv{$\game_2(\advA)$}\hfill\diffplus{$\game_2$}\\[2pt]
    $\v.M^* \gets \bot$;
    $\setS \gets \emptyset$;
    \diffplus{$\setR \gets \emptyset$; $r' \gets \lfloor r/\max(\delta^+,k\delta^-)\rfloor$}\\
    $\salt^* \getsr \bits^\lambda$\\
    $\advB^{\REPO,\QRYO,\UPO,\HASHO_1}$;
    return $\big[\sum_x \err[x] \geq r\big]$
  \\[6pt]
  \oraclev{$\QRYO(\qry_x)$}\\[2pt]
    $\v.X \gets \HASHO_3(\salt^* \cat x)$;
    $\setS \gets \setS \cup \{x\}$;
    $a = 1$\\
    for $i$ in $\v.X$ do\\
      $\tab$if $\v.M[i] = 0$ then $a = 0$\\
    if $\err[x] < \delta(a,\qry_x(\col^*))$ then
          $\err[x] \gets \delta(a,\qry_x(\col^*))$\\
    if $\err[x] > 0$ then $\setR \gets \setR \cup \{x\}$\\
    return $a$
  \\[6pt]
  \oraclev{$\UPO(\up_{x,b})$}\\[2pt]
    if $w'(\v.M^*) > \ell$\diffplus{$+r'$} then return $\top$\\
    \diffplus{if $x \in \setR$ and $b < 0$ then return $\top$}\\
    $\v.X \gets \HASHO_3(\salt^* \cat x)$;
    $\v.M' \gets \v.M^*$\\
    for $i$ in $\v.X$ do\\
      $\tab$ if $\v.M'[i] = 0$ and $b < 0$ then return $\top$\\
      $\tab \v.M'[i] \gets \v.M'[i] + b$\\
    if $b > 0$ and $\QRYO(\qry_x) = 1$ then $\err_i[x] \gets 0$\\
    if $b < 0$ and $\QRYO(\qry_x) = 0$ then $\err_i[x] \gets 0$\\
    $\v.M^* \gets \v.M'$;
    $\setS^* \gets \up_{x,b}(\setS^*)$;
    return $\top$
  \vspace{6pt}\hrule\vspace{3pt}
  \oraclev{$\UPO(\up_{x,b})$}\hfill\diffminus{$\game_2$}\diffplus{$\game_3$}\\[2pt]
    if $w'(\v.M^*) > \ell+r'$ then return $\top$\\
    if $x \in \setR$ and $b < 0$ then return $\top$\\
    $\v.X \gets \HASHO_3(\salt^* \cat x)$;
    $\v.M' \gets \v.M^*$\\
    for $i$ in $\v.X$ do\\
      $\tab$ if $\v.M'[i] = 0$ and $b < 0$ then return $\top$\\
      \diffminus{$\tab \v.M'[i] \gets \v.M'[i] + b$}\\
      \diffplus{$\tab \v.M'[i] \gets \min(\v.M'[i] + b, 1)$}\\
    if $b > 0$ and $\QRYO(\qry_x) = 1$ then $\err_i[x] \gets 0$\\
    if $b < 0$ and $\QRYO(\qry_x) = 0$ then $\err_i[x] \gets 0$\\
    $\v.M^* \gets \v.M'$;
    $\setS^* \gets \up_{x,b}(\setS^*)$;
    return $\top$
}
\caption{Games 0--3 for proof of Theorem~\ref{thm:counting-erreps}.}
\label{fig:cbf-erreps/games}
\end{figure*}

As with the proof of Theorem~\ref{thm:sbf-errep-immutable}, we derive a bound in
the \erreps1 case and then use Lemma~\ref{thm:lemma1} to move from \erreps1 to
the more general \erreps case. Because we are in the \erreps1 case, we may
assume without loss of generality that the adversary does not call $\REVO$,
since revealing the only representation automatically prevents the adversary
from winning.

We begin with a game~$\game_0$, shown in Figure~\ref{fig:cbf-erreps/games}, which has identical behavior to the \erreps1
experiment for a counting filter. As in the proof of
Theorem~\ref{thm:sbf-errep-immutable}, we have a
$\bad_1$ flag that gets set if the adversary ever calls $\HASHO_1$ with the
actual salt used by the representation. By a very similar argument, we can
move to~$\game_1$, where the behavior is different only when the $\bad_1$ flag
is set, with a bound of
\begin{equation}
  \Prob{\game_0(\advA)=1} \leq
    q_H/2^\lambda + \Prob{\game_1(\advA)=1} \,.
\end{equation}

Unlike in the case of a count min-sketch, it is entirely possible for deletions
to benefit the adversary in this game. In particular, if $x$ is found to be a
false positive, deleting $x$ may cause up to $k$ elements of $\col$ to become
false negatives. We therefore move to a game~$\game_2$ where the adversary gets
credit for either a single false positive or for $k$ false negatives whenever it
finds a false positive, but where the adversary cannot delete any false
positives that it finds. We let $r' = \lfloor r/\max(\delta^+,k\delta^-)\rfloor$
represent the number of false positives the adversary has to find in~$\game_2$
in order to win. In order to prevent the adversary from being penalized by the
filter becoming full too early, we also raise the threshold from $\ell$ to
$\ell+r'$ in~$\game_2$. Now for any $\advA$ for~$\game_1$, we can construct
$\advB$ for~$\game_2$ that simulates $\advA$, keeping track of all query
responses and forwarding all oracle queries in the natural way, except that
calls to delete false positives are ignored. Since $\UPO$ never fails for
$\advB$ due to the increased threshold, and since $\advB$ gets automatic credit
for any false negatives that might have been caused by deleting false positives,
$\advB$ succeeds whenever $\advA$ does, i.e.
$\Prob{\game_1(\advA)=1} \le \Prob{\game_2(\advB) = 1}$.

Since the remaining deletions do not cause errors, we can use the same argument
as in the proof of Theorem~\ref{thm:scms-erreps-th} to reduce from $\advB$ to an
adversary $\advC$ which does not make deletions at all. In~$\game_3$, we further
reduce from a counting filter to a normal Bloom filter by capping each of the
counters in the filter at 1. Since no deletions are performed, a counter
in~$\game_3(\advC)$ is nonzero if and only if the same counter
in~$\game_2(\advC)$ is nonzero. So $\QRYO$ behaves the same in~$\game_3$ as it
did in~$\game_2$, and $\Prob{\game_2(\advC)=1} \le \Prob{\game_3(\advC) = 1}$.

Note that~$\game_3$ is actually simulating an ordinary Bloom filter, since all
`counters' in the filter are restricted to the range $\bits$, there are no
deletions, and any insertions just set the corresponding bits to 1. In fact,
this game is identical to~$\game_2$ in the proof of
Theorem~\ref{thm:sbf-erreps-th} except that the adversary need only accumulate
$r'$ errors instead of $r$ errors and the threshold is $\ell+r'$ instead of
$\ell$. An identical argument allows us to reach the binomial bound of
\begin{equation}
   \Prob{\game_3(\advC)=1} \le
     \sum_{i=r'}^{q_T} \binom{q_T}{i}p_\ell^i(1-p_\ell)^{q_T-i} \,,
\end{equation}
where $p_\ell$ is now defined to be $((\ell+k+r')/m)^k$. Then the standard
Chernoff bound, along with Lemma~\ref{thm:lemma1}, yields the final bound of
\begin{equation}
   \Adv{\erreps}_{\Pi,\delta,r}(\advA) \leq
     q_R \cdot \left[\frac{q_H}{2^\lambda} + e^{r'-p_\ell q_T}\left(\frac{p_\ell
     q_T}{r'}\right)^{r'}\right] \,.
\end{equation}

\begin{figure*}
  \cpnote{As above, these games need to be revised. There's no $\HASHO$ oracle
  here, since we're in the standard model!}
  \boxThmBFSaltCorrect{0.48}
  {
    \underline{$\game_0(\advA)$}\\[2pt]
      $\key \getsr \keys$; $i \gets 0$\\
      $\bot \getsr \advA^{\HASHO,\REPO,\QRYO,\UPO}$\\
      return $[\sum \err \geq r]$
    \\[6pt]
    \oraclev{$\REPO(\col)$}\\[2pt]
      $i \gets i + 1$\\
      $\col_i \gets \col$\\
      $\pub_i \gets \Rep_K[\HASHO](\col)$\\
      return $\pub_i$
    \\[6pt]
    \oraclev{$\QRYO(i,\qry_x)$}\\[2pt]
      $a \gets \Qry[\HASHO](\pub_i, \qry_x)$\\
      if $a \neq \qry_x(\col)$ then $\err_i[x] \gets 1$\\
      return~$a$
    \\[6pt]
    \oraclev{$\UPO(i,\up_x)$}\\[2pt]
      $a \gets \Qry[\HASHO](\pub_i, \qry_x)$\\
      if $\err_i[x] = 1$ then $\err_i \gets 0$\\
      $\col_i \gets \col_i \union \{x\}$\\
      $\pub_i \gets \Up[\HASHO](\pub_i,\up_x)$\\
      return~$\pub_i$
    \\[4pt]
    \oraclev{$\HASHO(x)$}\\[2pt]
      return $H(x)$
    %\hspace*{-4pt}\rule{1.043\textwidth}{.4pt}
  }
  {
  \underline{$\game_1(\advB)$}\\[2pt]\\
    \oraclev{$\HASHO(i,x)$}\\
      if $T[i, x]$ is $\undefn$ then $T[i, x] \getsr [m]^k$\\
      return $T[i, x]$
    \\[6pt]
    \underline{$\game_2(\advB)$}\\[2pt]\\
      $\key \getsr \keys$; $i \gets 0$; $\setC \gets \emptyset$\\
      $\bot \getsr \advA^{\HASHO,\REPO,\QRYO,\UPO}$\\
      return $[\sum \err \geq r]$
    \\[6pt]
    \oraclev{$\QRYO(\qry_x)$}\\[2pt]
      do\\
      $\tab y \getsr \mathcal{X}$\\
      while $y \in \col \cup \setC$\\
      $a \gets \Qry[\HASHO](\pub_i, \qry_y)$\\
      if $a \neq \qry_y(\col)$ then $\err_i[y] \gets 1$\\
      return~$a$
  }
  {
  }
  {
  }
  \caption{Games 0--2 for proof of Theorem~\ref{thm:bf-priv-key-bound}.}
  \label{fig:bf-priv-salt-bound}
\end{figure*}

\cpnote{In general, the same notes as above apply here. I can sort of see how
the argument works, but the lack of sufficient detail makes it difficult for me
to decide whether you're right.}

\cpnote{Tip: When beginning a proof, be very, VERY, \underline{\textbf{VERY!}}
clear about the initial game ($\game_0$). In this case, it should be essentiall
\errep\ with the given experiment paremeters $(\delta, r)$, the given adversary,
and the given scheme. Then you can start simplifyingi $\game_0$ in preparation
for the next step in the proof. As a rule of thumb, the simpler the transition
the better.  Sometimes that's hard to do, but you need to explain what changes
you're making between the games.}

\cpnote{As usual, start by naming an adversary that you're going to use in the
PRF reduction.}

The main observation is that seeing the representation of a salted, keyed Bloom
filter does nothing to tell the adversary about what the responses to $\QRYO$
will be. Using a conditioning argument, we move from $\game_0$, which is
equivalent to the standard \errep\ game, to the alternate $\game_1$ that uses a
lazily-evaluated random function in place of the PRF $F$ for hashing, with the
random function being different for each representation.
%
Provided that the adversary cannot distinguish the PRF from a random function
and provided that the per-representation salt never repeats (the probability of
which is on the order of $q_R^2/2^\lambda$ by the birthday bound \cpnote{change
``on the order of'' to ``at most''}), the adversary
cannot distinguish this from the original game.
%
\cpnote{Great start. This is a good the ``sketch'', but bear in mind you haven't
proven anything yet. In particular, you need to \emph{exhibit} a PRF adversary
whose advantage upper bounds the probability of~$\advA$ distngusihing between
$\game_0$ and $\game_1$. (This is the usual ``game-playing''
argument~\cite{bellare2006triple}.}

Next,\cpnote{Again, you haven't proven anything yet} since it never benefits the
adversary to re-query an element instead of querying a new one, and because
false negatives do not occur in Bloom filters, we can assume without loss of
generality that the adversary only makes queries to previously-unqueried
elements which are not in the underlying set. But if an element is not in the
underlying set, it must not have been included in the original $\col$ sent to
$\REPO$, and it must never have been inserted with $\UPO$ since Bloom filters do
not support deletion. Furthermore, since it has not been queried before, it has
not been tested with $\QRYO$ either. This means that each element being queried
is a new input to the random function used for hashing, and its output is
therefore indistinguishable from any other input that is provided. We can then
move from $\game_1$ to $\game_2$, which ignores the query given as input and
instead makes a random query to a previously-untested element. Since the outputs
of the $\QRYO$ oracle are indistinguishable from those in $\game_1$ and there
are no other changes, we have
%
%\todo{DC (lead)}{This doesn't make sense.}
%$\Adv{\game_1}_{\struct,r}(\advA) = \Adv{\game_2}_{\struct,r}(\advA)$.
%
But now
that the queries are random, the adversary cannot possibly do better than
producing a representation with maximal (non-adaptive) false-positive
probability and making as many arbitrary queries as possible. Given a threshold
where the proportion of 1 bits is capped at $p$, \cpnote{What's $p$? You mean
$\ell/m$?} the false positive probability for each query is bounded by $p^k$.
%
\cpnote{I'm not clear how you got that.}
%
By the properties of the binomial distribution, the probability of accumulating
at
least $r$ errors given $q_T$ queries is
%$$\Adv{\errep}_{\struct,r,d}(\advA) \le \Adv{\prf}(F) + q_R^2/2^\lambda + I_{p^k}(r, q_T-r+1)$$
\todo{DC}{Complete the bound. Bear in mind that $\Adv{\prf}(F)$ is not a
well-defined quantity! What you mean is $\Adv{\prf}_F(\advB)$, where $\advB$ is
a PRF adversary that you define.}
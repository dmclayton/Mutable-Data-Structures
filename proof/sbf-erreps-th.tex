\begin{figure*}
\twoCols{0.47}
{
  \vspace{-7pt}
  \experimentv{$\game_{0}(\advA)$}\hfill\diffminus{$\game_0(\advA)$}\diffplus{$\game_1(\advB)$}\\[2pt]
    $ct \gets 0$;
    $\setZ \gets \emptyset$;
    $\setP \gets \emptyset$\\
    \diffminus{$i \gets \advA^{\REPO,\QRYO,\UPO,\HASHO_1,\REVO}$}\\
    \diffplus{$i \gets \advB^{\REPO,\QRYO,\UPO,\HASHO_1}$}\\
    return $\big[\sum_x \err_i[x] \geq r\big] \wedge i \not\in \setP$
  \\[6pt]
  \oraclev{$\HASHO_c(\salt \cat x)$}\\[2pt]
    $\vv \getsr [m]^k$\\
    if $\salt \in \setZ$ and $c = 1$ then \com{Caller is adversary}\\
    \tab $\bad_1 \gets 1$\\
    if $T[Z,x] = \bot$ then $\vv \gets T[Z,x]$\\
    $T[Z,x] \gets \vv$; return $\vv$
  \\[6pt]
  \oraclev{$\QRYO(i, \qry_x)$}\\[2pt]
    $X \gets \bmap_m(\HASHO_3(\salt_i \cat x))$;
    $a \gets X = M_i \AND X$\\
    if $\err_i[x] < \delta(a,\qry_x(\col_i))$ then
          $\err_i[x] \gets \delta(a,\qry_x(\col_i))$\\
    return $a$
  \\[6pt]
  \oraclev{$\REPO(\col)$}\hfill \diffplus{$\game_1$}\\[2pt]
    $ct \gets ct+1$;
    $M_{ct} \gets 0^m$;
    $\salt_{ct} \gets \bits^\lambda$\diffplus{$\setminus \setZ$}\\
    \diffplus{$\setZ \gets \setZ \cup \{\salt_{ct}\}$;}
    $\setS_{ct} \gets \col$\\
    for $x \in \col$ do\\
    $\tab\UPO(ct, \up_x)$\\
    return $\top$
  \\[6pt]
  \oraclev{$\UPO(i, \up_x)$}\\[2pt]
    if $w(M) > \ell$ then return $\top$\\
    if $\QRYO(\qry_x) = 1$ then $\err_i[x] \gets 0$\\
    $M_i \gets M_i \vee \bmap_m(\HASHO_2(\salt^* \cat x))$;
    $\setS_i \gets \up_x(\setS_i)$
    return $\top$
  \\[6pt]
  \oraclev{$\REVO(i)$}\\[2pt]
    $\setP \gets \setP \cup \{i\}$\\
    return $\langle M_i, \salt_i\rangle$
}
{
  \vspace{-7pt}
  \oraclev{$\HASHO_c(\salt \cat x)$}\hfill\diffplus{$\game_2$}\\[2pt]
    $\vv \getsr [m]^k$\\
    if $\salt \in \setZ$ and $c = 1$ then \com{Caller is adversary}\\
    \tab $\bad_1 \gets 1$; \diffplus{return $\vv$}\\
    if $T[Z,x] = \bot$ then $\vv \gets T[Z,z]$\\
    $T[Z,x] \gets \vv$; return $\vv$

  \vspace{6pt}\hrule\vspace{3pt}

  \oraclev{$\REPO(\col)$}\hfill\diffplus{$\game_3$}\\[2pt]
    $ct \gets ct+1$;
    $M_{ct} \gets 0^m$;
    $\salt_{ct} \gets \bits^\lambda\setminus \setZ$\\
    $\setZ \gets \setZ \cup \{\salt_{ct}\}$;
    $\setS_{ct} \gets \col$\\
    for $x \in \col$ do\\
    $\tab\UPO(ct, \up_x)$\\
    \diffplusbox{while $w(M_{ct}) < \ell+k$ do\\
    $\tab i \getsr [m]$;
    $M_{ct}[i] \gets 1$}
    return $\top$

  \vspace{6pt}\hrule\vspace{3pt}

  \experimentv{$\game_4(D)$}\\[2pt]
    $M \gets 0^m$\\
    while $w(M_{ct}) < \ell+k$ do\\
    $\tab i \getsr [m]$;
    $M_{ct}[i] \gets 1$\\
    $D^{\QRYO,\HASHO_1}$;
    return $\big[\sum_x \err[x] \geq r\big]$
  \\[6pt]
  \oraclev{$\QRYO(\qry_x)$}\\[2pt]
    $X \gets \bmap_m(\HASHO_3(\salt_i \cat x))$\\
    $a \gets X = M \AND X$\\
    $\err[x] \gets a$\\
    return $a$
}
\caption{Games 0, 1, and 2 for proof of Theorem~\ref{thm:sbf-erreps}.}
\label{fig:sbf-erreps/games}
\end{figure*}

As in previous proofs, we assume without loss of generality that there are no
insertions of or queries for elements of $\col$, and we start with a
game~$\game_0$, defined in Figure~\ref{fig:sbf-erreps/games} that is identical to the \erreps\ game for $\KBF_\mathrm{ft}$.

To avoid the unfortunate $q_R$ factor in the bound, we do not make use of
Lemma~\ref{thm:lemma1} in this proof. Because of that, we must find some other
way to ensure that $\REVO$ is not useful to the adversary. In particular, if
there are unique salts across representations, the $\REPO$, $\QRYO$, and $\UPO$
calls for one representation will be independent of those for other
representations, since the unique salt is passed as part of the input. Therefore
in~$\game_1$ we specify that all salts created will be unique, but deny access
to $\REVO$. By the birthday bound, the probability of salts repeating
in~$\game_0$ is no more than $q_R^2/2^\lambda$. If the representations are
independent, calling $\REVO$ would provide no information about other
representations, and would in fact only weaken the adversary by causing some
possible outputs $i \in [q_R]$ to be automatic losses. So we have
$\Prob{\game_0(\advA) = 1} \le q_R^2/2^\lambda + \Prob{\game_1(\advB) = 1}$,
where $\advB$ is an adversary that performs identically to $\advA$ but is
syntactically distinct because it lacks a $\REVO$ oracle.

Next, we want to ensure that the adversary's $\HASHO_1$ queries are independent
of the $\HASHO_2$ and $\HASHO_3$ queries used for $\REPO$, $\QRYO$, and $\UPO$.
Since the $\HASHO_c$ oracles use random sampling to fill a shared table, this
occurs if and only if the adversary calls
$\HASHO_1(\langle\salt_i \cat x\rangle)$
for some salt $\salt_i$ used by one of the representations created by $\REPO$.
By an argument very similar to that in the previous proofs, the adversary has at
most $q_H/2^\lambda$ probability of calling $\HASHO_1$ with the salt used by
some specific representation. However, since there are now $q_R$
representations, each with a distinct salt, there is at most a
$q_Rq_H/2^\lambda$ probability of the adversary correctly guessing a salt.
In~$\game_1(\advB)$, we set the $\bad_1$ flag if the adversary succeeds in
guessing the salt in this manner, but the flag does not affect the game. The
case of~$\game_2(\advB)$ is identical until the $\bad_1$ flag is set, which
occurs only when a salt is guessed, so we have $\Prob{\game_1(\advB) = 1} \le
q_Rq_H/2^\lambda + \Prob{\game_2(\advB) = 1}$.

Now the adversary derives no advantage from guessing the salts of the
representations, in the sense that the outputs of $\HASHO_1$ are independent of the results
of all calls to $\REPO$, $\QRYO$, and $\UPO$. The next oracle we want to target
is $\UPO$. Now that we have a filter threshold, we want to argue that the
adversary cannot use $\UPO$ to mount an effective attack.
%
In~$\game_3(\advC)$, the $\REPO$ oracle creates the filter as normal and then
randomly sets bits until filter is full (i.e., its Hamming weight is at least
$\ell$), which is $\ell+k$ (since
updates are not allowed when more than $\ell$ bits are set, and a single update
may set at most $k$ bits to 1). For any $\advB$ in~$\game_2$, we can construct
$\advC$ for~$\game_3$ that obtains at least as large an advantage by having $\advC$
simulate $\advB$, forwarding all oracle queries in the natural way except that
$\UPO$ queries are ignored, with $\advC$ simply returning $\top$ to $\advB$ without
performing any additional computations or oracle calls. (Recall that, in the
\erreps setting, the adversry expects $\top$ to be the output of~$\UPO$.) Since $\advB$ does not
query elements which are already in $\col$, the outputs of $\HASHO_3$ calls are
independent of any prior $\HASHO_2$ calls.
%
The probability of such an output producing a false positive is strictly a
function of the number of bits in the filter which have been set to 1.
%
Since at least as many bits have been set to 1 in~$\game_3$ as in~$\game_2$,
every $\QRYO$ call is at least as likely to produce a false positive. Therefore
$\Prob{\game_2(\advB) = 1} \le \Prob{\game_3(\advC) = 1}$.

In~$\game_3$, $\UPO$ calls do not actually change the representation. Since each
representation is created using independent $\HASHO_2$ outputs and then filled
in a uniform random manner until $\ell+k$ bits are set to 1, and is never
modified afterwards, the representations
themselves are random bitmaps which are uniformly distributed over the set of
$m$-length bitmaps with $\ell+k$ bits set to 1. This allows us to move
to~$\game_4(D)$, where the adversary is given a single arbitrary bitmap $\pub$
of length $m$ with $\ell+k$ bits set to 1 and makes $\QRYO$ calls exclusively
for $\pub$, winning if it produces $r$ errors for that `representation'. Given
an adversary $\advC$ for~$\game_3$, we construct a $D$ for~$\game_4$ that simulates
$\advC$. When $\advC$ makes a $\REPO$ call, $D$ immediately returns $\top$ to $\advC$, and
when $\advC$ makes a $\QRYO(i,\qry_x)$ call, $D$ selects the lexicographically first
$y$ that has not yet been queried and calls $\QRYO(y)$, returning the result to
$\advC$. Random oracle calls are forwarded to $D$'s own random oracle.
Since representations are uniformly distributed and $\HASHO_3$ calls are
independent of the $\HASHO_2$ calls used to construct the representation, each
$\QRYO$ call made by $D$ has the same chance of producing a hit as the $\QRYO$
call made by $C$ has of producing a false positive.
%
The win conditions differ
only in that $D$ wins by accumulating $r$ positive results, whereas $C$ must
accumulate $r$ false positives within a single representation, and so $D$ wins
if $\advC$ does. Therefore $\Prob{\game_3(\advC) = 1} \le \Prob{\game_4(D) = 1}$.

However, since $\QRYO$ calls with distinct inputs have independent outputs, each $\QRYO$ call made by $D$ has the same probability of producing a false positive. In particular, the probability of any one of the $k$ outputs of $\HASHO_3$ colliding with a 1 bit is $(\ell+k)/m$, and the probability of all $k$ outputs doing so is then $((\ell+k)/m)^k$. If we let $\setX$ be the set of all inputs made to $\QRYO$, we again have a binomial distribution where $q_T$ queries are made. Letting $p_\ell = ((\ell+k)/m)^k$, we have
\begin{equation}
   \Prob{\game_4(D)=1} \le
     \sum_{i=r}^{q_T} \binom{q_T}{i}p_\ell^i(1-p_\ell)^{q_T-i} \,,
\end{equation}
and we can once more apply a Chernoff bound, as long as $p_\ell q_T < r$, to simplify this to
\begin{equation}
   \Prob{\game_4(D)=1} \le
     e^{r-p_\ell q_T}\left(\frac{p_\ell q_T}{r}\right)^r.
\end{equation}

Substituting this into our earlier inequalities yields the final bound of
\begin{equation}
   \Adv{\erreps}_{\Pi,\delta,r}(\advA) \leq
     \frac{q_R(q_H+q_R)}{2^\lambda} + e^{r-p_\ell q_T}\left(\frac{p_\ell q_T}{r}\right)^r.
\end{equation}
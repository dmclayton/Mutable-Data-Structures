As in previous proofs, we assume without loss of generality that there are no insertions of or queries for elements of $\col$.

To avoid the unfortunate $q_R$ factor in the bound, we do not make use of Lemma~\ref{thm:lemma1} in this proof. Because of that, we must find some other way to ensure that $\REVO$ is not useful to the adversary. In particular, if there are unique salts across representations, the $\REPO$, $\QRYO$, and $\UPO$ calls for one representation will be independent of those for other representations, since the unique salt is passed as part of the input. Therefore in~$\game_1$ we specify that all salts created will be unique. By the birthday bound, we have $\Prob{\game_0(\advA) = 1} \le q_R^2/2^\lambda + \Prob{\game_1(\advA) = 1}$.

In~$\game_1$, all queries within one representation are independent of those in other representations. We can then assume without loss of generality that the adversary never makes use of its $\REVO$ oracle, since doing so provides no information about other representations and can only constrain the adversary in terms of which outputs $i$ can potentially allow it to win. Next, we want to ensure that the adversary's $\HASHO_1$ queries are independent of the $\HASHO_2$ and $\HASHO_3$ queries used for $\REPO$, $\QRYO$, and $\UPO$. Since the $\HASHO_c$ oracles use random sampling to fill a shared table, this occurs if and only if the adversary calls $\HASHO_1(\langle\salt_i,x\rangle)$ for some salt $\salt_i$ used by one of the representations created by $\REPO$. By an argument very similar to that in the previous proofs, the adversary has at most $q_H/2^\lambda$ probability of calling $\HASHO_1$ with the salt used by some specific representation. However, since there are now $q_R$ representations, each with a distinct salt, there is a $q_Rq_H/2^\lambda$ probability of the adversary correctly guessing at least one of these salts. In~$\game_1(\advA)$, we set the $\bad_1$ flag if the adversary succeeds in guessing the salt in this manner, but the flag does not affect the game. When we move to~$\game_2(\advA)$, the %[...]

We may now assume that the adversary never guesses the salt of any of the representations, and so the outputs of $\HASHO_1$ are independent of the results of all calls to $\REPO$, $\QRYO$, and $\UPO$. The next oracle we want to target is $\UPO$. Now that we have a filter threshold, we want to argue that the adversary cannot use $\UPO$ to mount an effective pollution attack.

\ignore{
\cpnote{In general, the same notes as above apply here. I can sort of see how
the argument works, but the lack of sufficient detail makes it difficult for me
to decide whether you're right.}

\cpnote{Tip: When beginning a proof, be very, VERY, \underline{\textbf{VERY!}}
clear about the initial game ($\game_0$). In this case, it should be essentiall
\errep\ with the given experiment paremeters $(\delta, r)$, the given adversary,
and the given scheme. Then you can start simplifyingi $\game_0$ in preparation
for the next step in the proof. As a rule of thumb, the simpler the transition
the better.  Sometimes that's hard to do, but you need to explain what changes
you're making between the games.}

\cpnote{As usual, start by naming an adversary that you're going to use in the
PRF reduction.}

The main observation is that seeing the representation of a salted, keyed Bloom
filter does nothing to tell the adversary about what the responses to $\QRYO$
will be. Using a conditioning argument, we move from $\game_0$, which is
equivalent to the standard \errep\ game, to the alternate $\game_1$ that uses a
lazily-evaluated random function in place of the PRF $F$ for hashing, with the
random function being different for each representation.
%
Provided that the adversary cannot distinguish the PRF from a random function
and provided that the per-representation salt never repeats (the probability of
which is on the order of $q_R^2/2^\lambda$ by the birthday bound \cpnote{change
``on the order of'' to ``at most''}), the adversary
cannot distinguish this from the original game.
%
\cpnote{Great start. This is a good the ``sketch'', but bear in mind you haven't
proven anything yet. In particular, you need to \emph{exhibit} a PRF adversary
whose advantage upper bounds the probability of~$\advA$ distngusihing between
$\game_0$ and $\game_1$. (This is the usual ``game-playing''
argument~\cite{bellare2006triple}.}

Next,\cpnote{Again, you haven't proven anything yet} since it never benefits the
adversary to re-query an element instead of querying a new one, and because
false negatives do not occur in Bloom filters, we can assume without loss of
generality that the adversary only makes queries to previously-unqueried
elements which are not in the underlying set. But if an element is not in the
underlying set, it must not have been included in the original $\col$ sent to
$\REPO$, and it must never have been inserted with $\UPO$ since Bloom filters do
not support deletion. Furthermore, since it has not been queried before, it has
not been tested with $\QRYO$ either. This means that each element being queried
is a new input to the random function used for hashing, and its output is
therefore indistinguishable from any other input that is provided. We can then
move from $\game_1$ to $\game_2$, which ignores the query given as input and
instead makes a random query to a previously-untested element. Since the outputs
of the $\QRYO$ oracle are indistinguishable from those in $\game_1$ and there
are no other changes, we have
%
%\todo{DC (lead)}{This doesn't make sense.}
%$\Adv{\game_1}_{\struct,r}(\advA) = \Adv{\game_2}_{\struct,r}(\advA)$.
%
But now
that the queries are random, the adversary cannot possibly do better than
producing a representation with maximal (non-adaptive) false-positive
probability and making as many arbitrary queries as possible. Given a threshold
where the proportion of 1 bits is capped at $p$, \cpnote{What's $p$? You mean
$\ell/m$?} the false positive probability for each query is bounded by $p^k$.
%
\cpnote{I'm not clear how you got that.}
%
By the properties of the binomial distribution, the probability of accumulating
at
least $r$ errors given $q_T$ queries is
%$$\Adv{\errep}_{\struct,r,d}(\advA) \le \Adv{\prf}(F) + q_R^2/2^\lambda + I_{p^k}(r, q_T-r+1)$$
\todo{DC}{Complete the bound. Bear in mind that $\Adv{\prf}(F)$ is not a
well-defined quantity! What you mean is $\Adv{\prf}_F(\advB)$, where $\advB$ is
a PRF adversary that you define.}}
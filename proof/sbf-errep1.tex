We will use the following lemma for keyless structures.
%
\todo{DC (lead)}{Prove this in the appendix.}

\begin{lemma}\label{thm:lemma1}
  For every $q_R, q_T, q_U, q_H, r, t \geq 0$ and keyless structure~$\Gamma$ it
  holds that
  \begin{equation*}
    \Adv{\errep}_{\Gamma,\delta,r}(t, q_R, q_T, q_U, q_H) \leq
    q_R\Adv{\errep1}_{\Gamma,\delta,r}(O(f(t)), q_T, q_U, q_H) \,,
  \end{equation*}
  where $f(t) = t + (q_R-1)\ticks(\Rep,t) + q_T\ticks(\Qry,t) + q_U\ticks(\Up,t)$.
  \todo{Any}{Define $\Adv{\errep1}_{\cdot,\cdot,\cdot}(\cdots)$.}
\end{lemma}
%
%
\noindent
The proof is by a fairly straight-forward hybrid argument. Because~$\Gamma$ is
keyless, in the reduction we simulate $q_R-1$ $\REPO$-queries experiment and use
our own oracles for the remaining query. The best we can do with this strategy
is to ``guess'' which representation the \errep\ adversary willl use in its
attack, which results in the~$q_R$ factor in the bound.
%
We defer the full details to Appendix~\ref{app:iproof/lemma1}.

Let $\advA$ be an \errep\ adversary making~$1$ query to~$\REPO$, $q_T$ queries
to $\QRYO$, $0$ queries to $\UPO$, and $q_H$ queries to the random
oracle~$\HASHO$.
%
We make the following assumptions, all of which are without loss of generality.
%
First, all of~$\advA$'s $\QRYO$ queries proceed its $\REPO$ query.
%
Second, we assume that $x\not\in\setS$ for all queries $\qry_x$ to $\QRYO$,
where~$\setS$ was the input to~$\advA$'s $\REPO$ query. This is without loss
because Bloom filtersadmit false positives, but not false negatives
%
Third, we we assume that $|\setS| \leq n$; this is without loss because
otherwise~$\REPO$ outputs~$\bot$ and~$\advA$ gets no advantage.

Fourth, we assume that all of~$\advA$'s $\HASHO$ queries are of the form $Z\cat
x$, where $|Z| = \lambda$.
%
finally, we assume that~$\advA$'s $\QRYO$ queries are all distinct.

We begin with a game-playing argument~\cite{bellare2006triple}, then obtain the
final bound via applicaition of Lemma~\ref{thm:lemma1}.
%
\cpnote{Here's the proof sketch.}
%
The high-level goal is to rewrite the game so that the probability that one
of~$\advA$'s queries runs up the score is is precisely the false positive
probability for Bloom filters in the usual analytical
setting~\cite{kirsch2006less}. In other words, our goal is transistion into a
setting in which the Bloom filter output by~$\REPO$ is independent of the
outcome of~$\advA$'s other queries.

We begin with the game~$\game_0(\advB)$ defined in
Figure~\ref{fig:sbf-errep1/games}. It is similar to the \errep\ experiment when
executed with~$\advA$, $\Pi$, $\delta$, and~$r$. Indeed, it is not difficult to
see that for every~$\advA$ there exists an adversary~$\advB$ such that
\begin{equation}
  \Adv{\errep}_{\Pi,\delta,r}(\advA) \leq \Prob{\game_0(\advB) = 1}
\end{equation}
and~$\advB$ has the same query resources as~$\advA$.
%
Adversary~$\advB$ executes~$\advA$, forwarding~$\advA$'s oracle queries
to its own oracles in the natural way.

Observe that in game~$\game_0$ the salt used for the representation of~$\setS^*$
is generated prior to executing~$\advB$. Game~$\game_1$ is identical
to~$\game_0$ until the flag~$\bad_1$ gets set by oracle~$\HASHO$. This occurs
if~$\advB$ asks $\HASHO_1(\salt^* \cat x)$, where~$\salt^*$ is the salt generated
at the beginning of the game.
%
By the Fundamental Lemma of Geame Playing~\cite{bellare2006triple} it follows
that
%
\begin{eqnarray}
  \Prob{\game_0(\advB)=1} &\leq&
    \Prob{\game_1(\advB)=1} + \Prob{\game_1(\advB) \sets \bad_1}\\
  &\leq&
    \Prob{\game_1(\advB)=1} + q_H/2^\lambda \,.
\end{eqnarray}
%
Note that in $\game_1$, the value of~$M^*$ is independent of~$\advB$'s
$\HASHO_1$ queries.


\begin{figure*}
\threeColsUnbalanced{0.40}{0.27}{0.27}
{
  \vspace{-7pt}
  \experimentv{$\game_{0}(\advB)$}
      \hfill \diffplus{$\game_1(\advB)$}\\[2pt]
    $\salt^* \getsr \bits^\lambda$;
    $\advB^{\REPO,\QRYO,\HASHO_1}$;
    return $\big[\sum_x \err[x] \geq r\big]$
  \\[6pt]
  \oraclev{$\REPO(\col)$}\\[2pt]
    $M^* \gets \bigvee_{x \in \col} \bmap_m(\HASHO_2(\salt^* \cat x))$;
    $\setS^* \gets \col$;
    return $\langle M^*, Z^* \rangle$
  \\[6pt]
  \oraclev{$\QRYO(i, \qry_x)$}\\[2pt]
    $X \gets \bmap_m(\HASHO_3(\salt^* \cat x))$;
    $a \gets X = M^* \AND X$\\
    if $\err[x] < \delta(a,\qry_x(\col^*))$ then;
          $\err[x] \gets \delta(a,\qry_x(\col^*))$\\
    return $a$
  \\[6pt]
  \oraclev{$\HASHO_c(\salt \cat x)$}\\[2pt]
    $\vv \getsr [m]^k$\\
    if $\salt=\salt^*$ and $c=1$ then \com{Caller is~$\advB$}\\
    \tab $\bad_1 \gets 1$; \diffplus{return $\vv$}\\
    if $T[Z,x] = \bot$ then $\vv \gets T[Z,z]$\\
    $T[Z,x] \gets \vv$; return $\vv$
}
{
  hella
}
{
  cool
}
\caption{Games for proof of Theorem~\ref{thm:sbf-errep1}.}
\label{fig:sbf-errep1/games}
\end{figure*}

\ignore{
\begin{figure*}
  \boxThmBFSaltCorrect{0.48}
  {
    \underline{$\game_0(\advA)$}\\[2pt]
      $\col \getsr \advA^H$; $\setC \gets \emptyset$; $\err \gets 0$\\
      $\pub \getsr \Rep[\hashbf[H]](\col)$\\
      $\bot \getsr \advA^{H,\QRYO}(\pub)$;
      return $(\err \geq r)$
    \\[6pt]
    \oraclev{$\QRYO(\qry_x)$}\\[2pt]
      if $\qry_x \in \mathcal{C}$ then return $\bot$\\
      $\setC \gets \setC \union \{\qry_x\}$\\
      $a \gets \Qry[\hashbf[H]](\pub, \qry_x)$\\
      if $a \neq \qry_x(\col)$ then $\err \gets \err + 1$\\
      return~$a$
    \\[4pt]
    \hspace*{-4pt}\rule{1.043\textwidth}{.4pt}
    \\[5pt]
    \oraclev{$\HASHO_1(\salt,x)$} \hfill\diffplus{$\game_2$}\;{$\game_1$}\hspace*{3pt}\\
      $\hh \getsr [m]^2$; $\vv \gets \fff(\hh$)\\
      if $\salt = \salt^*$ then\\
      \tab $\bad_1 \gets 1$; \diffplus{return $\vv$}\\
      if $T[\salt,x]$ is defined then $\vv \gets T[\salt,x]$\\
      $T[\salt,x] \gets \vv$;
      return $\vv$
  }
  { % New RO semantics
    % Change of hashing scheme (iid since inputs are distinct)
    \underline{$\game_1(\advB)$}\\[2pt]
      $\salt^* \getsr \bits^\lambda$;
      $\col \getsr \advB^{\HASHO_1}$\\
      $\pub \gets \Repx[\HASHO_2](\col, \salt^*)$\\
      $\setC \gets \emptyset$;
      $\err \gets 0$\\
      $\bot \getsr \advB^{\HASHO_3,\QRYO}(\pub)$;
      return $(\err \geq r)$
    \\[6pt]
    \oraclev{$\QRYO(\qry_x)$}\\[2pt]
      if $\qry_x \in \mathcal{C}$ then return $\bot$\\
      $\setC \gets \setC \cup \{\qry_x\}$\\
      $a \gets \Qry[\HASHO_3](\pub, \qry_x)$\\
      if $a \neq \qry_x(\col)$ then $\err \gets \err + 1$\\
      return~$a$
    \\[6pt]
    \oraclev{$\HASHO_i(\salt,x)$}\\[2pt]
      $\hh \getsr [m]^2$; $\vv \gets \fff(\hh$)\\
      if $\salt = \salt^*$ then $\bad_i \gets 1$\\
      if $T[\salt,x]$ is defined then\\
      \tab $\vv \gets T[\salt,x]$\\
      $T[\salt,x] \gets \vv$;
      return $\vv$
  }
  {
    \underline{$\game_3(\advB)$}\\[2pt]
      $\salt^* \getsr \bits^\lambda$;
      $\col \getsr \advB^{\HASHO_1}$\\
      $\pub \gets \Repx[\HASHO_2](\col, \salt^*)$\\
      $\setC \gets \emptyset$;
      $\err \gets 0$\\
      $\bot \getsr \advB^{\HASHO_3,\QRYO}(\pub)$;
      return $(\err \geq r)$
    \\[9pt]
    \oraclev{$\QRYO(\qry_x)$}\\[2pt]
      if $\qry_x \in \setC$ then return $\bot$\\
      $\setC \gets \setC \union \{\qry_x\}$\\
      if $\Ans[x]$ is undefined then\\
      \tab $\vv \gets \HASHO_3(\salt^*,x)$\\
      return $\Ans[x]$
  }
  {
    \oraclev{$\HASHO_3(\salt,x)$}\\[2pt]
      $\hh \getsr [m]^2$; $\vv \gets \fff(\hh$)\\
      if $T[\salt,x]$ is defined then\\
      \tab $\vv \gets T[\salt,x]$\\
      $T[\salt,x] \gets \vv$\\
      if $\salt \ne \salt^*$ or $\qry_x \in \setC$ then return~$\vv$\\
      $\setC \gets \setC \union \{\qry_x\}$\\
      $\Ans[x] \gets 1$;
      for $j \gets 1$ to $k$ do\\
      \tab\tab if $M[\vv_j] \neq 1$ then $\Ans[x] \gets 0$\\
      if $\Ans[x] \ne \qry_x(\col)$ then\\
      \tab $\err \gets \err + 1$\\
      return $\vv$
  }
  \caption{Games 0--3 for proof of Theorem~\ref{thm:bf-salt-correct}.}
  \label{fig:bf-salt-correct}
  %\vspace{6pt}
  %\hrule
\end{figure*}

The proof is by a game-playing argument; see Figure~\ref{fig:bf-salt-correct}.
%
Game~$\game_0$ is precisely the \errepone game with~$\struct_\saltybloom$,
adversary~$\advA$, and error parameter~$r$, and so
$\Adv{\errepone}_{\struct_\saltybloom,r}(\advA) = \Prob{\game_0(\advA)=1}$.
%
Next, $\game_1$ differs from~$\game_0$ in three important ways.
%
First, the random oracle~$H$ is replaced with three oracles: $\HASHO_1$ is
given to the adversary's first stage, $\HASHO_2$ is used to compute the
representation, and $\HASHO_3$ is given to the adversary's second stage and
is also used by~$\Qry$ to answer the $\QRYO$ queries.  All of them
populate the same table~$T$ mapping inputs to random values, thereby simulating
a single random oracle via lazy evaluation.
%
Second, whereas~$H$ maps strings to $[m]$, oracles $\HASHO_1$,
$\HASHO_2$, and $\HASHO_3$ map elements of $\bits^\lambda \cross
\bits^*$ to $[m]^k$.
%
Third, the salt~$\salt^*$ corresponding to the representation of~$\col$
is generated before~$\advA$ selects~$\col$.

There exists an adversary~$\advB$ such that $\Prob{\game_0(\advA)=1} =
\Prob{\game_1(\advB)=1}$.
%
Adversary~$\advB$ maintains a table~$R$, initially empty.
%
In its first stage, $\advB$ executes~$\advA$ in its first stage, answering each
query~$w$ to~$H$ as follows:
%
Check if $R[w]$ is defined; if so, then return $R[w]$ to~$\advA$.
%
Otherwise, if $w = \str(\salt, j, x)$ for some $\salt \in \bits^\lambda$, $j \in
[k]$, and $x \in \bits^*$, then ask $(\salt, x)$ of $\HASHO_1$, getting~$\vv$ in
return, and let $R[\str(\salt,j,x)] = \vv_j$ for each $j\in[k]$.
%
Otherwise, sample~$r$ uniformly from $[m]$ and let $R[w] = r$.
%
Finally, return~$R[w]$ to~$\advA$.
%
When~$\advA$ outputs~$\col$, output~$\col$.
%
In its second stage, $\advB$ executes $\advA$ in its second stage, answering
its~$H$ queries as above (except using~$\HASHO_3$), and answering each
query~$\QRYO$ query using its own oracle.
%
Adversary~$\advB$'s simulation is perfect because the outputs of
$\Rep[\hashbf[H]](\col)$ and $\Rep[\HASHO_2](\col)$ are
identically distributed.

Next, games~$\game_2$ and~$\game_1$ are identical until the flag~$\bad_1$ gets set.
This occurs if~$\advB$ queries~$\HASHO_1$ on a point $(\salt^*,x)$. If this
occurs, the revised oracle samples~$\hh$ uniformly from $[m]^2$, computes $\vv
\gets \fff(\hh)$, and immediately returns~$\vv$, without updating the table by
setting $T[\salt^*, x] = \vv$.
%
Since $\salt^*$ is sampled uniformly from~$\bits^\lambda$, and~$\advB$ makes at
at most $q_1$ distinct queries to~$\HASHO_1$ in its first stage, we have
that $\Prob{\game_1(\advB) \sets \bad_1} \leq q_1/2^{\lambda}$, and
\begin{equation*}
\Adv{\errepone}_{\struct_\saltybloom,r}(\advA) \leq \Prob{\game_2(\advB)=1} + q_1/2^{\lambda} \,.
\end{equation*}
At this point, the representation $\pub$ is independent of~$\advB$'s stage-1
queries.

Finally, $\game_3$ is semantically the same as~$\game_2$ (that is, the adversary's
view does not change), but the logic for checking if a~$\QRYO$ query
causes a false positive is moved to the~$\HASHO_3$ oracle.
%
On input of $(\salt, x)$, $\HASHO_3$ makes sure that $T[\salt,x]$ is
defined as usual. Note that if $\salt=\salt^*$, where~$\salt^*$ is the salt
corresponding to the challenge representation $\pub = \str(M, \salt^*)$, it now
checks to see if~$x$ is a false positive for~$\pub$; if so it lets~$\Ans[x]=1$;
otherwise it lets $\Ans[x]=0$.
%
Note, too, that on input of $\qry_x$, $\QRYO$ simply looks up $\Ans[x]$; if it is
undefined, then it asks $(\salt^*, x)$ of $\HASHO_3$.

We now consider $\Prob{\game_3(\advB)=1}$.
%
Observe that $\game_3(\advB)$ outputs~$1$ only in the case that there are~$r$ calls to the
$\HASHO_3$ oracle that increment~$\err$, i.e., queries $x \not\in
\col$ such that $\bigwedge_{j=1}^{k} M[\vv_j]=1$.
%
By construction, for each of adversary~$\advB$'s queries $(\salt^*, x)$
to~$\HASHO_3$ for which $x \not\in \col$, the value $T[\salt^*, x]$
is undefined. Similarly, for each of~$\advB$'s queries $\qry_x$ to~$\QRYO$
for which $x \not\in \col$, the value $T[\salt^*, x]$ is undefined.
%
Now, there are at most $q_T + q_2$ queries made to~$\HASHO_3$, either
directly by~$\advB$ or indirectly by~$\advB$ calling~$\QRYO$, that are
potentially false positives for the representation $\str(M, \salt^*)$ and would
result in~$\err$ getting incremented. These are queries of the form~$(\salt^*,
x)$, where $x \not\in \col$.
%
We can assume without loss of generality that~$\advB$ halts as soon as~$\err=r$
(which it can easily track), so there are at most $\binom{q_T+q_2}{r}$ possible
``winning'' query patterns to consider.
%
Moreover, as each error $\err$-incrementing event is independent of all previous
choices made by~$\HASHO_3$, we need only consider the probability of a single
error. As a corollary of Theorem~\ref{thm:mitz2} (Appendix~\ref{sec:mitz}),
along with a union bound and the independence of $\err$-incrementing events, we
can conclude that \[
\Prob{\game_3(\advB)=1} \leq \dbinom{q_T + q_2}{r} \left( (1-e^{-kn/m})^k + O(1/n) \right)^r
\]
%so that the final bound becomes
%\[
%  \Adv{\errepone}_{\struct_\saltybloom,r}(\advB) \leq
%    \frac{q_1}{2^\lambda} +
%    {\dbinom{q_T + q_2}{r}} \left( (1-e^{-kn/m})^k + O(1/n) \right)^r\,.
%\]
%
Letting $q_H = q_1 + q_2$, the claim follows.
}

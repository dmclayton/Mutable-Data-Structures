\heading{Keyless structures}

In a keyless structure, we assume all details of the algorithm used are known beforehand by the adversary, except any which might be generated `on the fly' as representations are created and modified. For example, many of these structures will assume the use of a random salt which is picked at the time the representation is created. A structure with neither randomization nor a private key is usually vulnerable to pre-computation and offline attacks, where the adversary can search for errors to produce without having to interact with the structure itself.

In many cases, it is easier to reason about an adversary which only has the chance to create and manipulate a single representation, rather than being able to call a $\REPO$ oracle as many times as it likes. Fortunately, in the case of keyless structures we can show that the adversary's advantage in a single-representation case is bounded above by a multiple of the advantage in the general case. We use $\errep1$ and $\erreps1$ to denote the public-representation and private-representation experiments where the adversary makes a single $\REPO$ query. In writing the advantage for these games we omit the $q_R$ parameter since it is fixed at 1.

%There are two equivalent ways of defining the $\errep1$ experiment. First, we may define it as a modified version of $\errep$ in which the adversary is constrained to make no more than one call to $\REPO$. We may equivalently define it as a two-stage adversary in the experiment below.

%\begin{figure}[t]
  \twoColsNoDivide{0.48}
  {
     \raisebox{-5pt}{\experimentv{$\Exp{\errep1}_{\struct,r}(\advA)$ {\colorbox{lightgray}{$\Exp{\erreps1}_{\struct,r}(\advA)$}}}}\\[2pt]
      $\setP \gets \emptyset$\\
      $\key \getsr \keys$\\
      $\col \getsr \advA$\\
      $i \getsr \advA^{\UPO,\QRYO}(\REPO(\col))$\\
      return $[\sum_\qry \err[\qry] \geq r]$ 
    \\[6pt]
    \oraclev{$\REPO(\col)$}\\[2pt]
      $\pub \getsr \Rep_\key(\col)$\\
      $\setC \gets \emptyset$\\
      $\mathrm{rv} \gets \pub$ \highlight{;\mathrm{rv} \gets \top} \\
      return $\mathrm{rv}$
  }
  {
    \oraclev{$\UPO(\up)$}\\[2pt]
%      $X \gets \col^{v[i]}_i
%      $v[i] \gets v[i]+1$\\
      $X \gets \col$ \\
      $\col \gets \up(X)$\\
      $\pub \getsr \Up_\key(\pub, \up)$\\
      $\setC \gets \emptyset$\\
      for $\qry$ in $\err$ do\\
      \tab $a \gets \Qry_K(\pub, \qry)$\\
      \tab if $\err[\qry] > d(a,\qry(\col))$ then\\
      \tab\tab$\err[\qry] \gets d(a,\qry(\col))$\\
      $\mathrm{rv} \gets \pub$ \highlight{;\mathrm{rv} \gets \top}\\
      return $\mathrm{rv}$
      \medskip

    \oraclev{$\QRYO(\qry)$}\\[2pt]
      if $\qry \in \setC$ then return $\bot$\\
      $\setC \gets \setC \union \{\qry\}$\\
      $a \gets \Qry_K(\pub, \qry)$\\
      if $\err[\qry] < d(a,\qry(\col))$ then\\
      \tab$\err[\qry] \gets d(a,\qry(\col))$\\
      return $a$
  }
  \caption{The $\errep1$ and $\erreps1$ notions of adversarial correctness. These are equivalent to the standard $\errep$ and $\erreps$ notions where the adversary is only allowed to make a single call to $\REPO$.}
  \vspace{6pt}\hrule
  \label{fig:security}
\end{figure}

Note that in the $\erreps1$ scenario we need not provide the adversary with a $\REVO$ oracle. If the adversary uses only a single representation, may assume without loss of generality that they make no call to $\REVO$, since doing so would prevent the adversary from having any possibility of winning. Since the case of a single $\REPO$ query is considerably simpler to handle, the first step in each of our proofs will be to reduce $\errep$ and $\erreps$ to $\errep1$ and $\erreps1$ respectively.% In addition to this, we wish to move from actual hash functions to true random functions. Using the following lemmas, we may reduce the case of $\errep$ for any structure using a salted hash to the case of $\errep1$ using a true random function, and we may reduce $\erreps$ for a structure using a secret-keyed hash to the case of $\erreps1$ using a true random function.

\begin{lemma}[\errep1 and \erreps1 imply \errep and \erreps for keyless structures]\label{lemma:errep}
  Let $\struct = (\Rep, \Qry, \Up)$ be a data structure with key
  space $\{\emptystr\}$. For every $t, q_R, q_T, q_U, q_H, r\geq 0$, it holds that
  \[
    \Adv{\errep}_{\struct,r}(t, q_R, q_T, q_U, q_H) \leq
    q_R\cdot\Adv{\errep1}_{\struct,r}(O(f(t)), q_T, q_U, q_H) \,,
  \]
  where $f(t) = t + (q_R-1)\ticks(\Rep,t) + q_T\ticks(\Qry,t) + q_U\ticks(\Up,t)$.
\end{lemma}

\begin{proof}For a fixed $r \ge 0$, let $\advA$ be an $\errep$ or $\erreps$ adversary which runs in $t$ time steps and makes $q_R$ $\REPO$ queries, $q_T$ $\QRYO$ queries, $q_U$ $\UPO$ queries, and $q_H$ RO queries. We construct an adversary $B$ for $\errep1$ or $\erreps1$, respectively, as follows.

First, $\advB$ initializes a counter $ct \gets 0$ and a set $\setC \gets \emptyset$, and samples $q \getsr [q_R]$. Next $\advB$ executes $\advA$, simulating the answers to its oracle queries as follows. When $\advA$ asks the query $\REPO(\col)$, $\advB$ sets $ct \gets ct + 1$ and stores $\col_{ct} \gets \col$. Then, if $ct = q$, $B$ forwards $\col$ to its own $\REPO$ oracle, returning the resulting value ($\pub$ in the public-representation case, or $\top$ in the private-representation case) to $\advA$. Otherwise, $\advB$ computes $\pub_{ct} \getsr \Rep(\col)$ and returns either $\pub_{ct}$ or $\top$. When $\advA$ asks for the query $\QRYO(i,\qry)$, $\advB$ first checks if $(i,\qry) \in \setC$ and returns $\bot$ if this condition holds. Otherwise $\advB$ forwards $(i,\qry)$ to its $\QRYO$ oracle and returns $a$ if $i = q$, and reutrns $\Qry(\pub_i,\qry)$ otherwise. Similarly, when $\advA$ makes an $\UPO(i,\up)$ query, $\advB$ forwards $(i,\up)$ to its $\UPO$ oracle if $i = q$ and evaluates $\Up(\pub_i,\up)$ otherwise. Finally, queries from $\advA$ to its RO are simply forwarded to $\advB$'s RO. When $\advA$ halts and outputs $j$, $\advB$ does the same.

If $j = q$, then $B$ wins if $A$ does, since all queries from $A$ to $\pub_j$ were forwarded to $B$'s $\QRYO$ oracle. Given that $q$ is sampled uniformly from the range $[q_R]$, it follows that

$$\Adv{\errep}_{\struct,r}(t, q_R, q_T, q_U, q_H) \leq q_R\cdot\Adv{\errep1}_{\struct,r}(O(f(t)), q_T, q_H).$$

Note that $B$ makes at most $q_T$ queries to $\QRYO$ and $q_H$ queries to its RO. Since $A$ runs in $t$ time steps and writing a bit takes 1 time step, the input length to any $\Rep$, $\Qry$, or $\Up$ evaluated by $B$ is at most $t$ bits. Hence, adversary $B$ runs in time $O(t+(q_R-1)\ticks(\Rep,t)+q_T\ticks(\Qry,t)+q_U\ticks(\Up,t))$.
\end{proof}
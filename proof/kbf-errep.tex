\begin{figure*}
\todo{DC}{Rename $\HASHO$ to something else! It looks like a random oracle, but
it's not!}
\todo{DC}{nit: Here and throughout the rest of paper, change $ct$ to
$\mathit{ct}$. The former looks like $c\cdot t$.}
\twoCols{0.47}
{
  \vspace{-7pt}
  \experimentv{$\game_{0}(\advA)$}\\[2pt]
    $\key \getsr \keys$;
    $ct \gets 0$\\
    $i \getsr \advA^{\REPO,\QRYO,\UPO}$;
    return $\big[\sum_x \err_i[x] \geq r\big]$
  \\[6pt]
  \oraclev{$\HASHO(\salt \cat x)$}\hfill\diffminus{$\game_0$}\diffplus{$\game_1$}\\[2pt]
    \diffminus{$\vv \gets F_K(\salt \cat x)$}\\
    \diffplusbox{$\vv \getsr [m]^k$\\
    if $T[Z,x] = \bot$ then $\vv \gets T[Z,z]$\\
    $T[Z,x] \gets \vv$; return $\vv$}
  \\[6pt]
  \oraclev{$\QRYO(i, \qry_x)$}\\[2pt]
    $X \gets \bmap_m(\HASHO(\salt_i \cat x))$;
    $a \gets X = M_i \AND X$\\
    if $\err_i[x] < \delta(a,\qry_x(\col_i))$ then
          $\err_i[x] \gets \delta(a,\qry_x(\col_i))$\\
    return $a$
  \\[6pt]
  \oraclev{$\REPO(\col)$}\\[2pt]
    $ct \gets ct+1$;
    $\setS_{ct} \gets \col$;
    $\salt_{ct} \getsr \bits^\lambda$;
    $c_{ct} \gets |\col|$\\
    $M_{ct} \gets \bigvee_{x \in \col} \bmap_m(\HASHO(\salt_{ct} \cat x))$;
    return $\langle M_{ct}, \salt_{ct}, c_{ct} \rangle$
  \\[6pt]
  \oraclev{$\UPO(i, \up_x)$}\\[2pt]
    if $w(M) > \ell$ then return $\top$\\
    if $\QRYO(\qry_x) = 1$ then $\err_i[x] \gets 0$\\
    $M_i \gets M_i \vee \bmap_m(\HASHO(\salt_i \cat x))$;
    $\setS_i \gets \up_x(\setS_i)$;
    return $\langle M_i, \salt_i, c_i+1\rangle$
}
{
  \vspace{-7pt}
  \experimentv{$\game_{2}(\advA)$}\hfill\diffplus{$\game_3$}\\[2pt]
    $\key \getsr \keys$;
    $ct \gets 0$;
    $\setZ \gets \emptyset$\\
    $i \getsr \advA^{\REPO,\QRYO,\UPO}$;
    return $\big[\sum_x \err_i[x] \geq r\big]$
  \\[6pt]
  \oraclev{$\REPO(\col)$}\\[2pt]
    $ct \gets ct+1$;
    $\setS_{ct} \gets \col$;
    $\salt_{ct} \getsr \bits^\lambda \setminus \setZ$;
    $c_{ct} \gets |\col|$\\
    $\setZ \gets \setZ \cup \{\salt_{ct}\}$\\
    $M_{ct} \gets \bigvee_{x \in \col} \bmap_m(\HASHO(\salt_{ct} \cat x))$;
    return $\langle M_{ct}, \salt_{ct}, c_{ct} \rangle$
  \\[6pt]
  \oraclev{$\QRYO(i, \qry_x)$}\\[2pt]
    $X \gets \bmap_m(\HASHO(\salt_i \cat x))$;
    $a \gets X = M_i \AND X$\\
    if $\err_i[x] < \delta(a,\qry_x(\col_i))$ then
          $\err_i[x] \gets \delta(a,\qry_x(\col_i))$\\
    \diffplus{$\UPO(i, \up_x)$;}
    return $a$
  \\[6pt]
  \experimentv{$\game_{4}(\advB)$}\\[2pt]
    $\key \getsr \keys$;
    $ct \gets 0$;
    $\setZ \gets \emptyset$\\
    $i \getsr \advB^{\REPO,\QRYO}$;
    return $\big[\sum_x \err_i[x] \geq r\big]$
  \\[6pt]
  \oraclev{$\QRYO(i, \qry_x)$}\\[2pt]
    for $i \in [ct]$ do\\
    $\tab X \gets \bmap_m(\HASHO(\salt_i \cat x))$;
    $a \gets X = M_i \AND X$\\
    $\tab$if $\err_i[x] < \delta(a,\qry_x(\col_i))$ then
          $\err_i[x] \gets \delta(a,\qry_x(\col_i))$\\
    $\tab\UPO(i, \up_x)$;
    return $a$
}
\caption{Games 0--4 for proof of Theorem~\ref{thm:bf-key-bound}.}
\label{fig:kbf-errep/games}
\end{figure*}

We start with a game~$\game_0$ which is essentially the same as the standard
\errep\ experiment on a Bloom filter, given the assumption (without loss of
generality) that the adversary never attempts to construct a representation for
a set with more than $n$ elements. As with the other proofs, it is easy to see
that for any such \errep\ adversary we can make an adversary $\advA$
for~$\game_0$ with the same resources that achieves the same advantage.

Unlike in the previous two proofs, we cannot use Lemma~\ref{thm:lemma1} because
an adversary cannot simulate the oracles without knowing the private key. We use
an alternate approach to gradually reduce to the standard binomial bound
deriving from the non-adaptive false positive probabilities. The first thing we
want to do is to bound the probability that the adversary can break the PRF.

The number of times the PRF is evaluated on distinct inputs is bounded by the
number of queries available to the adversary. In particular, $\QRYO$ and $\UPO$
each call the PRF once, while $\REPO$ may call the PRF up to $n$ times. If the
adversary runs in $t$ time steps, then, the probability it can distinguish the
PRF from a random function is bounded by $\Adv{\prf}_F(t,nq_R+q_T+q_U)$.
%
In~$\game_1$, we have a game which is identical to~$\game_0$ except that it uses
random sampling in place of the PRF. If $\advA$ cannot distinguish the PRF from
a random function then these games are indistinguishable from the adversary's
perspective, so $\Prob{\game_0(\advA) = 1} \le \Adv{\prf}_F(t,nq_R+q_T+q_U) +
\Prob{\game_1(\advA) = 1}$.
%
\cpnote{In fat, this isn't immediate. You show this by a reduction. You want to
show that for every $\advA$ there exists an adversary~$D$ such that
$\Prob{\game_0(\advA)=1} - \Prob{\game_1(\advA)=1} \leq \Adv{\prf}_F(D)$. You
don't need to be super formal about it, but you do need to say how~$D$
executes~$\advA$ and what outputs.}

Our goal is to argue, in a similar manner as to the previous theorems, that all of the oracle calls are independent. In order to guarantee this we must deal with the possibility of a salt collision between different representations. In~$\game_2(\advA)$ we require that all salts be distinct between representations. By the birthday bound, collisions between randomly-generated salts occur with frequency at most $q_R^2/2^\lambda$, so $\Prob{\game_1(\advA) = 1} \le q_R^2/2^\lambda + \Prob{\game_2(\advA) = 1}$.

With guaranteed-unique salts, the result of each $\REPO$, $\UPO$, and $\QRYO$
call for a given representation is independent of the calls for all other
representations. By an almost identical argument to the proof of
Theorem~\ref{thm:sbf-erreps}, we can assume without loss of generality that the
adversary follows any $\QRYO$ call that does not find a false positive with an
$\UPO$ call to insert that element, and therefore move to~$\game_3(\advA)$,
which as in the previous proof automatically performs an update after each query
is made. Since the adversary never inserts the same element multiple times, we
can again conclude that without loss of generality the adversary never directly
invokes the $\UPO$ oracle.
%
\cpnote{As in the previous result, I think it's better to give an explicit
reduction, rather than say ``we can assume without loss of generality ..''}

Finally, we must deal with the possibility that the adversary chooses which
representations to target with $\UPO$ and $\QRYO$ calls based on the result of
$\REPO$, since some representations may be more full than others. In
game~$\game_4$,we deny the adversary direct access to the $\UPO$ oracle
because it is never needed, but we allow the adversary credit if a call to
$\QRYO$ produces an error in any of the representations that have been
constructed. Furthermore, the updates made by $\QRYO$ apply to all
representations that are not already full. Since all $\UPO$ calls are
identically and independently distributed, and having more elements in a filter
cannot decrease the false positive rate, the fact that some representations may
become full more quickly than they otherwise would have can only help the
adversary. Similarly, having $\QRYO$ count errors across all representations
never harms the adversary, and so the adversary's advantage may only increase
when moving to~$\game_4(\advB)$ \todo{DC}{$\advB$ is undefined at this point}. Therefore $\Prob{\game_3(\advA) = 1} \le
\Prob{\game_4(\advB) = 1}$, where $\advB$ is an adversary which behaves
identically to $\advA$ but which is syntactically distinct because it lacks the
unused $\UPO$ oracle.
%
\cpnote{Instead of making assumptions about~$\advA$'s behavior and arguing that
they're not without loss, just give an explicit reduction.}

We are now in a situation where we can apply the standard, non-adaptive error
bound. Let $\setX$ be the set of all queries $\qry_x$ made by the adversary over
the course of the game. As in the previous proof, we have $|\setX| \le q_T$.
However, $\qry_x$ may now cause a false positive in any of the representations.
The probability of causing a false positive in a specific representation is
still given by the non-adaptive false positive probability $p'$ for a Bloom
filter containing $n+r$ elements. Since the representations are independent of
each other, the probability of a false positive occurring in any of up to $q_R$
representations is at most $p'q_R$. We can therefore bound the adversary's
success probability using a binomial distribution, similar to before:
\begin{equation}
   \Prob{\game_4(\advB)=1} \le
     \sum_{i=r}^{q_T} \binom{q_T}{i}(p'q_R)^i(1-p'q_R)^{q_T-i} \,.
\end{equation}

Applying the usual Chernoff bound, we find
\begin{equation}
   \Prob{\game_4(\advB)=1} \le
     e^{r-p'q_Rq_T}\left(\frac{p'q_Rq_T}{r}\right)^r.
\end{equation}

So, substituting this bound back into the earlier advantage inequalities, we find the final bound of
\begin{equation*}
  \begin{aligned}
    \Adv{\errep}_{\Pi,\delta,r}(t, q_R,q_T,q_U,q_H) &\leq \\
      \Adv{\prf}_F(t,nq_R+q_T+q_U) & +
    \frac{q_R^2}{2^\lambda} +
    \left(\frac{p'q_Rq_T}{r}\right)^re^{r-p'q_Rq_T}
  \end{aligned}
\end{equation*}

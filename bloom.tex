\begin{figure}
  \twoColsNoDivide{0.22}
  {
    \underline{$\Rep^R_K(\col)$}\\[2pt]
      if $|\col| > n$ return $\bot$\\
      $\salt \getsr \bits^\lambda$\\
      $M \gets \bigvee_{x \in \col} \bmap_m(R_K(\salt \cat x))$\\
      if $w(M) > \ell$ then return $\bot$\\
      return $\langle M, \salt \rangle$
  }
  {
    \underline{$\Qry^R_K(\langle M, \salt \rangle,\qry_x)$}\\[2pt]
      $X \gets \bmap_m(R_K(\salt \cat x))$\\
      return $M \AND X = X$
    \\[6pt]
    \underline{$\Up^R_K(\langle M, \salt \rangle,\qry_x)$}\\[2pt]
      if $\hw(M) > \ell$ then return $\bot$\\
      return $\langle M \vee \bmap_m(R_K(\salt \cat x)), \salt \rangle$
  }
  \caption{The keyless structure $\bloom[R,\ell,n,\lambda]$ given by
  $(\Rep^R,\Qry^R,\Up^R)$ is used to define Bloom filter variants. The
  parameters are a function $R: \keys\by\bits^* \to [m]^k$ and integers $\ell, n,
  \lambda \geq0$. A concrete scheme is given by a particular choice of
  parameters.  Functions~$\hw$ and~$\bmap_m$ are defined in
  Section~\ref{sec:prelims}.
  }
  \label{fig:bf-def}
\end{figure}


\medskip
We specify three Bloom filter variants using the keyed structure
$\bloom[R,\ell,n,\lambda] = (\Rep^R,\Qry^R,\Up^R)$ specified in
Figure~\ref{fig:bf-def}.
%
The construction has four paraemters: a function~$R:\keys\by\bits^*\to[m]^k$, the
\emph{update threshold} $\ell\geq0$, the \emph{initial threshold} $n\geq0$, and
the \emph{salt length}~$\lambda\geq0$.
%
Let $H:\bits^*\to[m]^k$ be a hash function and let $\ell, n, \lambda\geq0$ be
integers.
%
The standard Bloom filter is the structure $\BF[H,\ell,n] =
\bloom[\id^H,\ell,n,0]$, which we will term the \emph{basic} Bloom filter. It
has no key (the key sapce of $\id^H$ is $\{\emptystr\}$, see
Section~\ref{sec:prelims}) and does not use a salt.
%
The \emph{salted} Bloom filter $\SBF[H,\ell,n,\lambda] =
\bloom[\id^H,\ell,n,\lambda]$ is the same except that it allows a nonempty salt.
%
Finally, we also consider a salted variant that uses a PRF instead of a hash
function. The \emph{keyed} Bloom filter $\KBF[F,\ell,n,\lambda]$ is the
structure $\bloom[F,\ell,n,\lambda]$, where $F:\keys\by\bits^*\to[m]^k$ is a
PRF.
%
Note that the basic and salted BFs have key spaces $\{\emptystr\}$ and the keyed
BF has key space~$\keys$.

In this section, we will show that the basic Bloom filter construction
$\BF[H,\ell,n]$ is flawed, allowing the adversary to make an offline attack that
has a high probability of success while using a minimal number of queries. In
the immutable setting, where the adversary is constrained to never use the
$\UPO$ oracle, i.e. $q_U = 0$, it suffices to use the $\SBF$ construction in
order to provide a good security guarantee in either the public-representation
or private-representation settings. However, in the case where we allow
$q_U > 0$ so that the adversary can make updates, we will find that $\SBF$ is
not secure in the public-representation setting. Instead, $\KBF$ must be used to
guarantee security.

\ignore{%
The standard Bloom filter shows a variety of different behaviors depending on
its exact implementation. If the hash functions used are chosen beforehand and
potentially known to the adversary, this public information allows offline
attacks to be mounted against the data structure which can produce potentially
damaging false positives. In the case of immutable Bloom filters, making use of
a per-representation salt is sufficient to prevent these attacks, though
depending on the use case the use of non-fixed per-representation randomness may
or may not be feasible. Furthermore, in the case of mutable Bloom filters there
are additional difficulties with offline attacks due to adversarially-chosen
updates. To guarantee correctness in this case we must additionally guarantee
that representations can be kept private from the adversary.
}

In the non-adaptive setting, the false positive rate for a Bloom filter of $m$
bits using $k$ hash functions and storing $n$ items can be closely approximated
as $(1-e^{-kn/m})^k + O(1/m)$.~\cite{kirsch2006less} Over the course of $q_T$
random evaluations of $\Qry$, we would then expect about
$q_T((1-e^{-kn/m})^k + O(1/m))$ false positives to occur. Our goal is to
minimize the probability that even an adaptive adversary can perform
significantly better than this lower bound.

The standard definition of a Bloom filter does not make use of a threshold
$\ell$ on the number of 1s in the filter. In practice the close relationshop
between the size of the set and the number of 1s in the representation given by
the Kirsch and Mitzenmacher bound means that that the behavior of the filter in
the absence of an adversary will be almost identical for reasonable choices of
$\ell$. On the other hand, in the presence of an adversary that has access to
$\UPO$ the use of $\ell$ allows us to make stronger guarantees with cleaner
error bounds than would otherwise be possible. If the adversary does not have
access to $\UPO$, i.e. if $q_U = 0$, we will assume an $\ell$ which is large
enough to not affect the behavior of the filter at all, in order to model actual
practice as closely as possible.

\heading{Error function for set-mempership queries}
%
Throughout this section we will use the error function $\delta:\bits^2\to\N$
defined as
\begin{equation*}
  \delta(a, b) =
  \begin{cases}
    0 & \text{if}\ a=b \\
    1 & \text{otherwise.}
  \end{cases}
\end{equation*}

\subsection{Insecurity of unsalted BFs}
Fix a hash function $H:\bits^*\to[m]^k$ and integers $n\geq0$,$\ell = kn$ and
let $\Pi = \BF[H,\ell,n] = (\Rep^H, \Qry^H, \Up^H)$.
%
Suppose the adversary is interacting with a system representing a dataset
with~$\Pi$ and that it is able to choose some fraction of the input data.
For example, some web spiders use Bloom filters to store the set of webpages
which have already been visited during a crawl. Suppose the adversary controls
at least one such webpage and wishes to deny the spider access to a different
webpage, the `target webpage'. The adversary can choose the links present on its
own webpage, which will cause the spider to visit the chosen webpages and set
the corresponding bits of its Bloom filter to 1. If those links are chosen in
such a way that they produce a false positive for the target webpage, the spider
will then erroneously believe it has already visited the target webpage. The
target webpage will therefore never be visited during the spider's crawl.

\heading{Pollution attacks}
%
The goal of the attacks pointed out by Gerbet \etal~~\cite{gerbet2015power} is
to pollute the filter such that the ``average'' set-membership query yields a
false positive with high probability; to do so, the attacker chooses a set of
inputs that maximize the number of 1s in the filter. This strategy is especially
effective when the structure of the hash function is known to the adversary. In
particular, if a secretly-keyed cryptographically-secure function is not used,
the adversary can compute the hash function on its own in order to determine
which choices will set the maximum number of bits to 1, or which choices will
set certain target bits to 1 in order to cause specific false positives. They
show that with some reasonable parameter choices, the adversary can double the
false positive rate if they control a third of the inputs to the filter.
%

Gerbet \etal discuss multiple ways to mitigate pollution attacks, including
using alternate data structures besides Bloom filters or choosing the parameters
of the filter pesimistically, so that even if a pollution attack occurs, the
false positive rate is kept below some threshold of acceptability. Both of these
possibilities are potentially viable, but can significantly increase the amount
of memory required to store the data structure. Since Bloom filters are designed
to store large amounts of data compactly, these solutions may be unnaceptable in
practice. The authors also discuss the possibility of using a secretly-keyed
hash function. Since they assume that the filter itself is kept entirely
private, with no potential for long-term leakage or other phenomena modeled by
our $\REVO$ oracle, this does not fall prey to the attacks against keyed but
unsalted filters that occur in our model. In this paper we consider the
possibility that the structure may not be secret, either in the sense that it
may eventually be recovered (in the private-representation setting) or that it
is publicly available to potential adversaries (in the public-representation
setting). Simply using a MAC with a long-term secret key does not suffice to
secure a Bloom filter in these scenarios.

\heading{Target-set coverage attacks}
%
Of course, exhibiting a high false positive rate is not the only way a Bloom
filter might fail to be correct. In particular, it would be undesirable if the
filter were consistently incorrect on a \emph{particular set of inputs}. Rather
than pollute the filter, the adversary's goal might be to craft a set of
legitimate looking inputs that cover some disjoint target set of inputs.
%
This type of attack is nicely captured by our adversarial model.
%
In a \emph{target-set coverage attack}, the adversary is given a small target set
$\setT\subseteq\bits^*$ and searches for a cover set $\setR\subseteq\bits^*$
such that $\Qry^H(\Rep^H(\setR),x)=1$ for each $x\in\setT$.
%
Once a suitable cover set is found, the adversary queries $\Rep(\setR)$. Then
for each $x\in\setT$, it asks $\Qry(x)$, achieving a score of $r = |\setT|$.

This \erreps1 attack succeeds with probability~$1$ assuming a covering set can
be found.  If $|\setT| \leq |\setR|$, then such a set exists; but finding it may be
computationally infeasible, depending on the size of the cover set, the size of
the target set, and the parameters of the Bloom filter.
%
In Appendix~\ref{app:unsalted-attack} we demonstrate that target-set coverage
attacks are feasible for practical BF parameters. We do so by simulating the
attack when~$H$ is a random function (i.e., for each distinct input we choose
$k$ integers from $[m]$ at random) for typical choices of $k$, $m$, and~$n$.
%
\todo{CP}{Add simulation to appendix.}

%\todo{DS (lead)}{Double check that Gerbet \etal don't suggest anything like this
%attack. If they do it's not a big deal; we just need to ensure, to the best of
%our ability, that we give credit where credit is due.}

The key to pollution attacks and target-set coverage attacks is that the
adversary can compute the representation of the set on its own. In the remainder
of this section, we examine ways of enhancing the basic BF structure so that it
avoids this pitfall.

\subsection{Salted BFs in the (im)mutable setting}
%
Here we consider the correctness of Bloom filters when the hashed input is
prepended with a salt.
%
Fix $H:\bits^*\to[m]^k$, $n,\lambda\geq0$, $\ell = kn$, and let
$\Pi = \SBF[H,\ell,n,\lambda] = (\Rep^H, \Qry^H, \Up^H)$ as defined above.

In the mutable setting, where the adversary is allowed to make $\UPO$ queries,
we can perform an $\errep$ attack against~$\Pi$ as follows. First, the adversary
calls $\REPO(\emptyset)$, receiving an empty filter and the salt. Now that the
adversary has the salt, it performs the exact same pollution attack as in the
unsalted case, with any evaluations of $H(x)$ replaced with evaluations of
$H(\salt \cat x)$. After a pollution set is found that sets many bits to 1, the
adversary calls $\UPO$ repeatedly to insert these elements into the filter. This
attack succeeds whenever the original pollution attack succeeds against an
unsalted filter.

%

The attack succeeds because the adversary can search for false positives on its
own as soon as it learns the salt. An adversary in the real world may not be
able to perform this exact attack, since even in the streaming setting it is
possible that the salt is not immediately revealed to the adversary. However, as
soon as the adversary does learn the salt, it can immediately launch a pollution
attack against the filter, without having to make any queries directly to the
filter. Just as in the immutable setting the adversary can exploit its knowledge
of the hash functions to find false positives without needing to make queries,
in the mutable and public-representation setting the adversary can identically
exploit its knowledge of the hash functions \textit{and salt} to find false
positives without needing to make queries.

\ignore{
The security of this structure depends upon how it is used.
%
\todo{DC (lead)}{As concisely as you can, articulate the \errep\ attack against~$\Pi$.
Remember that the adversarial model (i.e., the security experiment) is supposed
to reflect how the primitive is supposed to be used. So when giving attacks, try
to make it clear what the attack is. You should answer the following questions:
What is the sequence of queries made by the adversary?
How efficient is the attack?
Is the attack devastating? Is it a real attack, or is it more theoretical?
What about the scheme and setting make it possible?
%
Note that I've left your text in a ignore\{\} so that you can lift from it as
needed.
}
The use of a salt without a private key in the \errep\ setting is
insufficient to defeat [the attacka above]. In this setting, the adversary need only
make its $\REPO$ query for $\setT$ in advance, at which point it will receive
both the representation $\pub$ and the salt $\salt$ used to construct it. Using
this known salt, the adversary is still able to simulate $\REPO$ for arbitrary
singleton representations. The previous attack therefore still works with the
same (minimal) number of $\QRYO$ calls at the very end of the experiment, after
it has determined $\setR$ using offline computations.
\cpnote{I don't think this is true, since each $\ROPO$ query returns a filter
with a different, independently generated salt.}

The opposite of this, using a private key without a salt, does weaken the attack
somewhat. Even with public representations, the adversary cannot locally
simulate $\REPO$ without guessing the private key. However, they can still
outperform random $\QRYO$ calls by making $\REPO$ queries for singleton elements
without fixing any $\setT$ in advance. After selecting a random set $\col$ of
size $q_R-1$, the adversary performs offline computations to find $\setT$,
$\setR \subseteq \col$ such that the elements of $\setR$ are false positives for
the representation of $\setT$ (which can be computed from the representations of
the singleton subsets of $\setT$). The adversary wins if there is a partition
where $\setR$ produces at least $r$ errors on $\REPO(\setT)$.
\cpnote{This might be the case, but I don't think it's worth spending too much
time on. We're going to have other reasons for needing to add salt even when
there's a key.}

Using a salted Bloom filter in the private representation setting, however, does
provide some security. At the time a representation is created, the structure
chooses a salt $\salt$ which it will use for all further queries and updates. In
order for maximum security to be guaranteed, we must ensure that the
representation, and in particular the salt, is kept secret from the adversary.
We define this structure $\SBF[H,k,m,n,\lambda]$ as the Bloom filter structure
that uses $H(s) = (h_1(s),\ldots,h_k(s))$ for hashing inputs to $k$ values in
$[m]$. Furthermore, each call of $\Rep$ first involves picking a salt $\salt$
from the salt space $\bits^\lambda$, and all hashes made to insert or query for
an element $x$ are determined using $H(x \Vert \salt)$. Finally, the parameter
$n$ means that any attempts to represent sets with more than $n$ elements fail.
\todo{DC lead}{Specify what exactly is being analyzed.  What are the updates?}
}

Without this exploitation of mutability, and in particular the ability to insert
elements even after the salt has been seen, the above attack fails. Indeed, when
we restrict ourselves to the immutable setting, we can prove the following.
%
\begin{theorem}[Immutable \errep\ security of salted BFs]\label{thm:sbf-errep-immutable}
Let $p$ be the standard (non-adaptive) false positive probability for a Bloom
filter with the chosen parameters. For all integers $q_R, q_T, q_H, r, t \geq 0$
and for $q = q_T + q_H$, if $r > pq$, then it holds that
  \begin{equation*}
    \begin{aligned}
            \Adv{\errep}_{\Pi,\delta,r}(t,\,&q_R,q_T,0,q_H) \leq \\
        & q_R \cdot \left[\frac{q_H}{2^\lambda} +
        \left(\frac{pq}{r}\right)^re^{r-pq}\right] \,,
    \end{aligned}
  \end{equation*}
  where $H$ is modeled as a random oracle.  %
\end{theorem}
We consider only the case of $r > pq$ because $pq$ is the expected number of
false positives obtained by an adversary that simply uses its knowledge of the
salt (after the representation is created) to guess as many random elements as
possible. Because this simple adversary can get $pq$ successes on average, we
can only hope to provide good security bounds against arbitrary adversaries in
the case that $r > pq$.

This bound can be broken down into three main components. The external $q_R$
term means that the security guarantee will be weak in the case that the
adversary is able to view a large number of representations. The $q_H/2^\lambda$
term corresponds to the probability of the adversary guessing the salt before
the representation is constructed, but this will be negligibleas long as
$\lambda$ is chosen to be sufficiently large (say, $\lambda=128$). The final,
messier term comes from applying a Chernoff bound to the non-adaptive
adversary's probability of succeeding in the experiment given $q = q_H+q_T$
guesses.

By way of clarifying the performance of our bound, we have plotted the last
component for various parameters of interest. Let
%
\begin{equation}
  \zeta_{k,m,n}(q,r) = \left(\frac{pq}{r}\right)^re^{r-pq}
\end{equation}
%
and refer to Figure~\ref{fig:bf-bound}. This shows values
of~$\zeta_{k,m,n}(q,r)$ for varying~$m$ (filter length in kilobytes). What this
plot shows that, for a given error capacity~$r$, once a certain filter-length
threshold is reached, the $\zeta$ value decreases quite quickly. Moreover, the
rate at which~$\zeta$ decreases scales quite nicely with the error capacity. For
example, if one is willing to tolerate up to~$r=10$ false positives when for a
single filter (i.e., $q_R=1$) representing $n=100$ elements, then picking a
filter length of $3$ kilobytes is sufficient to ensure that observing~$10$ false
positives occurs with probability less than $2^{-17}$, even when the adversary
can make~$2^{64}$ RO queries.


\begin{figure}
  \hspace*{-10pt}
  \includegraphics{fig/bf-bound}
  \vspace{-24pt}
  \caption{
    The value of $\zeta_{k,m,n}(q,r)$ for $q=2^{64}$, $k=16$, $n=100$, varying
    values of~$r$ (one line per $r$-value) and filter length~$m$ (the x-axis).
    Note the log-2 scale on the y-axis.
  }
  \label{fig:bf-bound}
\end{figure}

\begin{proof}[Proof of Theorem~\ref{thm:sbf-errep-immutable}]
  \begin{figure*}
\threeColsOneDivideUnbalanced{0.40}{0.27}{0.27}
{
  \vspace{-7pt}
  \experimentv{$\game_{0}(\advB)$}
      \hfill \diffplus{$\game_1$}\\[2pt]
    $M^* \gets \bot$;
    $\salt^* \getsr \bits^\lambda$\\
    $\advB^{\REPO,\QRYO,\HASHO_1}$;
    return $\big[\sum_x \err[x] \geq r\big]$
  \\[6pt]
  \oraclev{$\REPO(\col)$}\\[2pt]
    $M^* \gets \bigvee_{x \in \col} \bmap_m(\HASHO_2(\salt^* \cat x))$;
    $\setS^* \gets \col$;
    return $\langle M^*, Z^* \rangle$
  \\[6pt]
  \oraclev{$\QRYO(\qry_x)$}\\[2pt]
    $X \gets \bmap_m(\HASHO_3(\salt^* \cat x))$;
    $a \gets X = M^* \AND X$\\
    if $\err[x] < \delta(a,\qry_x(\col^*))$ then
          $\err[x] \gets \delta(a,\qry_x(\col^*))$\\
    return $a$
  \\[6pt]
  \oraclev{$\HASHO_c(\salt \cat x)$}\\[2pt]
    $\vv \getsr [m]^k$\\
    if $M^*=\bot$ and $\salt=\salt^*$ and $c=1$ then \com{Caller is~$\advB$}\\
    \tab $\bad_1 \gets 1$; \diffplus{return $\vv$}\\
    if $T[Z,x] = \bot$ then $\vv \gets T[Z,x]$\\
    $T[Z,x] \gets \vv$; return $\vv$
}
{
  \vspace{-2pt}
  \oraclev{$\HASHO_c(\salt \cat x)$}\\[2pt]
    $\vv \getsr [m]^k$\\
    if $M^*=\bot$ and $\salt=\salt^*$ and $c=1$ then\\
    \tab $\bad_1 \gets 1$; return $\vv$\\
    if $T[Z,x] = \bot$ then $\vv \gets T[Z,z]$\\
    $T[Z,x] \gets \vv$\\[2pt]
    \diffplusbox{
    \com{Caller is~$\advB$ or $\QRYO$}\\
    if $c=1$ or $c=3$ then\\
    \tab if $\salt \ne \salt^*$  then return $\vv$\\
    \tab $\Ans[x] \gets \bmap_m(\vv) = M^* \AND \bmap_m(\vv)$\\
    \tab if $\err[x] < \delta(\Ans[x],\qry_x(\col^*))$ then
    \tab\tab $\err[x] \gets \delta(\Ans[x],\qry_x(\col^*))$
    }
    return $\vv$
}
{
  \vspace{-7pt}
  \oraclev{$\QRYO(\qry_x)$}\
      \hfill \diffminus{$\game_1$} \diffplus{$\game_2$}\\[2pt]
    \diffminusbox{%
      $X \gets \bmap_m(\HASHO_3(\salt^* \cat x))$\\
      $a \gets X = M^* \AND X$\\
      if $\err[x] < \delta(a,\qry_x(\col^*))$ then\\
      \tab $\err[x] \gets \delta(a,\qry_x(\col^*))$
    }\\[2pt]
    \diffplusbox{
      $\HASHO_3(Z^* \cat x)$\\
      $a \gets \Ans[x]$
    }
    return $a$
}
\caption{Games 0, 1, and 2 for proof of Theorem~\ref{thm:sbf-errep-immutable}.}
\label{fig:sbf-errep-immutable/games}
\end{figure*}

We will use the following lemma for keyless structures, which is proved in
Appendix~\ref{sec:keyless-proof}.

\begin{lemma}\label{thm:lemma1}
  For every $q_R, q_T, q_U, q_H, r, t \geq 0$ and keyless structure~$\Gamma$ it
  holds that
  \begin{eqnarray*}
    \begin{aligned}
      \Adv{\errep}_{\Gamma,\delta,r}(t,\,&q_R, q_T, q_U, q_H) \leq \\
      & q_R \cdot \Adv{\errep1}_{\Gamma,\delta,r}(O(t), q_T, q_U, q_H) \,,
    \end{aligned}
  \end{eqnarray*}
\end{lemma}
%
\noindent
The proof is by a fairly straightforward hybrid argument. Because~$\Gamma$ is
keyless, in the reduction we simulate $q_R-1$ of the calls to $\REPO$ experiment
and use our own oracles for the remaining query. The best we can do with this
strategy is to ``guess'' which representation the \errep\ adversary will use in
its attack, which results in the~$q_R$ factor in the bound.
%
We defer the full details to Appendix~\ref{app:deferred}

Let $\advA$ be an \errep\ adversary making~$1$ query to~$\REPO$, $q_T$ queries
to $\QRYO$, $0$ queries to $\UPO$, and $q_H$ queries to the random
oracle~$\HASHO$.
%
We make the following assumptions, all of which are without loss of generality.
%
First, all of~$\advA$'s $\QRYO$ queries follow its $\REPO$ query.
%
Second, we assume that $x\not\in\setS$ for all queries $\qry_x$ to $\QRYO$,
where~$\setS$ was the input to~$\advA$'s $\REPO$ query. This is without loss
because Bloom filters admit false positives, but not false negatives
%
Third, we we assume that $|\setS| \leq n$; this is without loss because
otherwise~$\REPO$ outputs~$\bot$ and~$\advA$ gets no advantage.
%
Fourth, we assume that all of~$\advA$'s $\HASHO$ queries are of the form $Z\cat
x$, where $|Z| = \lambda$.

We begin with a game-playing argument~\cite{bellare2006triple}, then obtain the
final bound via applicaition of Lemma~\ref{thm:lemma1}.
%
The high-level goal is to rewrite the game so that the probability that one
of~$\advA$'s queries runs up the score is precisely non-adaptive false positive
probability.
%
In other words, our goal is to transistion into a setting in which the Bloom
filter output by~$\REPO$ is independent of the outcome of~$\advA$'s other
queries.

Consider the game~$\game_0(\advB)$ defined in
Figure~\ref{fig:sbf-errep-immutable/games}. It is similar to the \errep\
experiment when executed with~$\advA$, $\Pi$, $\delta$, and~$r$, but the
pseudocode has been simplified to clarify our argumen. Indeed, it is not
difficult to see that for every~$\advA$ there exists an adversary~$\advB$ such
that
\begin{equation}
  \Adv{\errep}_{\Pi,\delta,r}(\advA) \leq \Prob{\game_0(\advB) = 1}
\end{equation}
and~$\advB$ has the same query resources as~$\advA$.
%
Adversary~$\advB$ executes~$\advA$, forwarding~$\advA$'s oracle queries
to its own oracles in the natural way.

Observe that in game~$\game_0$ the salt used for the representation of~$\setS^*$
is generated prior to executing~$\advB$. Game~$\game_1$ is identical
to~$\game_0$ until the flag~$\bad_1$ gets set by oracle~$\HASHO$. This occurs
if~$\advB$ asks $\HASHO_1(\salt^* \cat x)$, where~$\salt^*$ is the salt generated
at the beginning of the game, and it has not yet called $\QRYO$ (i.e.,
$M^*=\bot$).
%
By the Fundamental Lemma of Game Playing~\cite{bellare2006triple} it follows
that
%
\begin{eqnarray}
  \Prob{\game_0(\advB)=1} &\leq&
    \Prob{\game_1(\advB)=1} + \Prob{\game_1(\advB) \sets \bad_1}\\
  &\leq&
    \Prob{\game_1(\advB)=1} + q_H/2^\lambda \,.
\end{eqnarray}
%
Note that in $\game_1$, the value of~$M^*$ is independent of~$\advB$'s
$\HASHO_1$ queries. In particular, the probability that some bit of~$M^*$ is set
is independent of the choices of~$\advB$.

In game $\game_2$ the $\HASHO$ and $\QRYO$ oracles have been rewritten so that
the winning-condition is computed by $\HASHO$ instead of $\QRYO$. The former
oracle maintains a set~$\Ans$ such that $\Ans[x] = \Qry^{\HASHO_3}(M^*, \qry_x)$ for
each query $\salt^* \cat x$; on input of $\qry_x$, oracle~$\QRYO$ simply runs
$\HASHO_3(\salt^* \cat x)$ and returns $\Ans[x]$.
%
We are effectively giving the adverseary credit for RO queries that result in
false positives for the representation of~$\setS^*$, but which it does not
explicitly ask of the~$\QRYO$. Because~$\advB$'s advantage is at least that
of~$\advA$'s, it holds that
%
\begin{equation}
  \Prob{\game_1(\advB)=1} \leq \Prob{\game_2(\advB)=1} \,.
\end{equation}

We now consider $\Prob{\game_2(\advB)=1}$.
%
Let $\setX$ be the set $\{ x \in \bits^* : \Ans[x] \ne \bot \}$ and $\setT = \{x
\in\setX: \Ans[x] = 1\}$, where $\Ans$ is at is defined when~$\advB$ halts. We
will call~$\setX$ the set of attempts and~$\setT$ the set of false positives.
%
Note that $\setX\intersection\setS^*=\emptyset$ and
$|\setX| \leq q_H + q_T$.
%
Hence, the probability that~$\game_2(\advB)=1$ is equal to the probability
that~$|\setT| \geq r$.

For each $x\in\setX$, let $T(x)$ denote the event that $x\in\setT$.
%
In the random oracle model for~$H$, the set of random random variables $T(x)$
for each $x\in X$ are independently and identically distributed.
%
Hence, the probability that~$\advB$ succeeds is binomially distributed:
%
\begin{equation}
   \Prob{\game_2(\advB)=1} = \Prob{ |\setT| \geq r } =
     \sum_{i=r}^{q} \binom{q}{i}p^i(1-p)^{q-i} \,,
\end{equation}
%
where $q \leq q_H + q_T$ and $p = \Pr[T(x)=1]$. Here we can apply a Chernoff
bound which states that, for any $\delta > 0$,
%
\begin{equation}
  \Prob{X \geq (1+\delta)\mu} < \left(\frac{e^\delta}{(1+\delta)^{1+\delta}}\right)^\mu
\end{equation}
%
We set $\delta = r\mu^{-1}-1$ and note that $\mu = pq$.
This yields
%
\begin{equation}
 \Prob{|\setT| \geq r} < \left(\frac{e^{r\mu^{-1}-1}}{(r\mu^{-1})^{r\mu^{-1}}}\right)^\mu = \left(\frac{e^{r-\mu}}{(r\mu^{-1})^r}\right) = e^{r-pq}\left(\frac{pq}{r}\right)^r
\end{equation}
%
So we have
%
\begin{equation}
  \Adv{\errep}_{\Pi,\delta,r}(\advA) < \frac{q_H}{2^\lambda} + \left(\frac{pq}{r}\right)^re^{r-pq}
\end{equation}
%
Applying Lemma~\ref{thm:lemma1} to move from the single-representation case to the
general case, we get our final bound of
\begin{equation}
  \Adv{\errep}_{\Pi,\delta,r}(\advA) \leq
    q_R \cdot \left[
      \frac{q_H}{2^\lambda} +
      \left(\frac{pq}{r}\right)^re^{r-pq}
    \right] \,.
\end{equation}

\end{proof}


Recall that the attack against mutable salted filters exploited the fact that
the adversary learned the salt as soon as the filter was created, and that from
this it could compute the hash function on its own. Even if the filter is
mutable, we can prevent this attack from working as long as we require that the
filter under attack be kept secret from adversaries. In fact, we can attain the
following \erreps\ bound.

\begin{theorem}[\erreps\ security of salted BFs]\label{thm:sbf-erreps}
  For every $q_R, q_T, q_U, q_H, t, r\geq 0$, it holds that
  \begin{eqnarray*}
    \begin{aligned}
      \Adv{\erreps}_{\Pi,\delta,r}(t,q_R, q_T, q_U, q_H) &\leq \\
          \text{some cool bound} \,,
    \end{aligned}
\end{eqnarray*}
where $H$ is modeled as a random oracle
\end{theorem}

\todo{DC (lead)}{Turn this into a short summary of what's different from
Theorem~\ref{thm:sbf-errep-immutable}. In an earlier iteration this was the first result
in the paper, hence the length. But now it's the second result.}
%
The main idea behind the proof is to remove the adversary's
adaptivity a step at a time. We isolate the possibility of the adversary
guessing the salt, which would allow it to mount its own offline attack on the
filter without relying on the $\QRYO$ oracle. If the adversary does not guess
the salt, the outputs of the $\REPO$, $\QRYO$, and $\UPO$ oracles are
unpredictable to the adversary, producing uniformly randomly distributed bits to
set (for $\REPO$ and $\UPO$) or to check (for $\QRYO$). Under the assumption
that the adversary does not predict the salt, queries made to distinct elements
are independent of each other. The only remaining issue is that the adversary
can potentially gain an advantage by testing whether some object $x$ is a false
positive for the filter, and then updating the filter to include $x$ only if the
test query returned `false'.
%
An analysis shows that this is now (once imperfect
pseudorandom functions and salt collisions have been dealt with \cpnote{This
doesn't quite make sense ... there's no PRF in~$\Pi$})
%
the only way for the adversary to gain an advantage over making queries to an
immutable Bloom filter. Because this adaptive strategy introduces tricky
conditional possibilities, we cannot compute an exact value for the adversary's
advantage.  Instead, we move to an alternate scenario where each $\QRYO$ also
produces a free update and every $\UPO$ first performs a free query. This makes
$\QRYO$ and $\UPO$ calls indistinguishable, so that the adversary is effectively
making a series of independent random queries that each have a chance to
increment the error counter. Because the number of 1s in the filter can only
increase, the probability of a false positive from any one of these queries is
bounded above by the probability of a false positive on the final
maximally-sized filter, a probability which is given by the Kirsch and
Mitzenmacher bound.

\cpnote{I muted a ``In the ROM, salted hashing is almost as good as distinct
random functions'' lemma that appeared. As stated, I don't think it cleanly
isolates the part of the proof you're attempting to isolate. This is the
$\game_0$ to $\game_1$ transition in Theorem~\ref{thm:sbf-errep-immutable}. I think it's
better to leave this argument in-line in the proof.}
\ignore{
\begin{lemma}[In the ROM, salted hashing is almost as good as distinct random
  functions in \errep1]\label{lemma:salttorand}
  %
  Let $\struct = (\Rep, \Qry, \Up)$ be a data structure with key space
  $\{\emptystr\}$ and salt space $\bits^\lambda$, and let $\struct'$ be the same
  structure using true random functions in place of salted hash functions. For
  every $t, q_R, q_T, q_U, q_H, r \geq 0$, it holds that
  \[
    \Adv{\errep1}_{\struct,r}(t, q_R, q_T, q_U, q_H) \leq \frac{q_H}{2^\lambda} + \Adv{\errep1}_{\struct',r}(t, q_R, q_T, q_U, q_H)
  \]
\end{lemma}

\begin{proof}
Let $\game_0$ be the standard $\errep1$ game for the structure $\struct$, and
let $\game_1$ be the same game for $\struct'$. There exists an adversary $\advB$
such that $\Prob{\game_0(\advA) = 1} \le \Prob{\game_1(\advB) = 1} + q_H/2^\lambda$.
This adversary initializes an empty table $R$ and simulates $\advA$. When a
query $w$ is sent to $\HASHO$, $\advB$ returns $R[w]$ if this entry in the table
is defined. Otherwise, if $w = \langle\salt, x\rangle$ for some
$\salt \in \bits^\lambda$ and $x \in \bits^*$, forward $(\salt, x)$ to $\HASHO$,
store the result in $R[w]$, and return this value. Finally, if $R[w]$ is not
defined and $w$ is not of this form, sample $r$ uniformly from the range of the
hash function, store the result in $R[w]$ and return that result. Queries to all
other oracles are simply forwarded to $\advB$'s oracle. Assuming the output of
the hash function is uniformly distributed, this simulation is perfect unless
$\advA$ guesses the salt correctly, which happens with probability
$q_H/2^\lambda$.
\end{proof}
}

\begin{proof}[Proof of Theorem~\ref{thm:sbf-erreps}]
  \begin{figure*}
  \cpnote{I suggest re-doing this from scratch. The security experiment
  (\erreps) has changed and so has the construction. $\Repx$ and $\fff$ are not
  defined. Where's the $\REVO$ oracle?}
  \boxThmBFSaltCorrect{0.48}
  {
    \underline{$\game_0(\advA)$}\\[2pt]
      $\col \getsr \advA^H$; $\setC \gets \emptyset$; $\err \gets 0$\\
      $\pub \getsr \Rep[H](\col)$\\
      $\bot \getsr \advA^{H,\QRYO,\UPO}$\\
      return $(\err \geq r)$
    \\[6pt]
    \oraclev{$\QRYO(\qry_x)$}\\[2pt]
      if $\qry_x \in \mathcal{C}$ then return $\bot$\\
      $\setC \gets \setC \union \{\qry_x\}$\\
      $a \gets \Qry[H](\pub, \qry_x)$\\
      if $a \neq \qry_x(\col)$ then $\err \gets \err + 1$\\
      return~$a$
    \\[6pt]
    \oraclev{$\UPO(\up_x)$}\\[2pt]
      $\setC \gets \emptyset$\\
      $a \gets \Qry[H](\pub, \qry_x)$\\
      if $\qry_x \in \setC$ and $a \neq \qry_x(\col)$ then\\
      \tab $\err \gets \err-1$\\
      $\col \gets \col \union \{x\}$\\
      $\pub \gets \Up[H](\pub,\up_x)$\\
      return~$\bot$
    \\[4pt]
    \hspace*{-4pt}\rule{1.043\textwidth}{.4pt}
    \\[5pt]
    \oraclev{$\HASHO_1(\salt,x)$} \hfill\diffplus{$\game_2$}\;{$\game_1$}\hspace*{3pt}\\
      $\hh \getsr [m]^2$; $\vv \gets \fff(\hh)$\\
      if $\salt = \salt^*$ then\\
      \tab $\bad_1 \gets 1$; \diffplus{return $\vv$}\\
      if $T[\salt,x]$ is defined then $\vv \gets T[\salt,x]$\\
      $T[\salt,x] \gets \vv$;
      return $\vv$
  }
  {
    \underline{$\game_1(\advB)$}\\[2pt]
      $\salt^* \getsr \bits^\lambda$;
      $\col \getsr \advB^{\HASHO_1}$\\
      $\pub \gets \Repx[\HASHO_2](\col, \salt^*)$\\
      $\setC \gets \emptyset$;
      $\err \gets 0$\\
      $\bot \getsr \advB^{\HASHO_1,\QRYO,\UPO}$\\
      return $(\err \geq r)$
    \\[6pt]
    \oraclev{$\QRYO(\qry_x)$}\\[2pt]
      if $\qry_x \in \mathcal{C}$ then return $\bot$\\
      $\setC \gets \setC \cup \{\qry_x\}$\\
      $a \gets \Qry[\HASHO_2](\pub, \qry_x)$\\
      if $a \neq \qry_x(\col)$ then $\err \gets \err + 1$\\
      return~$a$
    \\[6pt]
    \oraclev{$\UPO(\up_x)$}\\[2pt]
      $\setC \gets \emptyset$\\
      $a \gets \Qry[H](\pub, \qry_x)$\\
      if $a \neq \qry_x(\col)$ and $\qry_x \in \setC$ then\\
      \tab $\err \gets \err-1$\\
      $\col \gets \col \union \{x\}$\\
      $\pub \gets \Up[\HASHO_2](\pub,\up_x)$\\
      return~$\bot$
    \\[6pt]
    \oraclev{$\HASHO_2(\salt,x)$}\\[2pt]
      $\hh \getsr [m]^2$; $\vv \gets \fff(\hh)$\\
      if $T[\salt,x]$ is defined then\\
      \tab $\vv \gets T[\salt,x]$\\
      $T[\salt,x] \gets \vv$;
      return $\vv$
  }
  {
    \underline{$\game_3(\advB)$}\\[2pt]
    \oraclev{$\QRYO(\qry_x)$}\\[2pt]
      $a \gets \Qry[\HASHO_3](\pub, \qry_x)$\\
      if $a \neq \qry_x(\col)$ then $\err \gets \err + 1$\\
      $\col \gets \col \union \{x\}$
      $\pub \gets \Up[\HASHO_2](\pub,\up_x)$\\
      return~$a$
  }
  {
    \oraclev{$\UPO(\up_x)$}\\[2pt]
      $a \gets \Qry[\HASHO_3](\pub, \qry_x)$\\
      if $a \neq \qry_x(\col)$ then $\err \gets \err + 1$\\
      $\col \gets \col \union \{x\}$
      $\pub \gets \Up[\HASHO_2](\pub,\up_x)$\\
      return~$\bot$
    \\[6pt]
    \oraclev{$\HASHO_i(\salt,x)$}\\[2pt]
      $\hh \getsr [m]^2$; $\vv \gets \fff(\hh)$\\
      return $\vv$
  }
  \caption{Games 0--3 for proof of Theorem~\ref{thm:sbf-erreps}.}
  \label{fig:sbf-erreps/games}
\end{figure*}

\cpnote{I understand the crux of the argument of how you deal with interleaved
updates/queries. It's a clever idea and I think it's believable. That said,
there's a lot of details that are omitted addressed.
%
Right now the biggest problem with this argument is that the games have
virtually nothing to do with the security notions and nothing to do with the
scheme being analyzed. They seem to be a carry-over from the old paper, but
things have change significantly. They will need to be rewritten. Try using the
games in Figure~\ref{fig:sbf-errep1/games} as a reference.}

\cpnote{I'm not clear on how update thresholding is used in the argument. From
my read it seems you're assuming some maximum represented set size, but we're
not maintaining a counter in the construction. Here's a hint: if thresholding is
necessary for security, then I'd expect $\ell$ to come up in the bound; if it
doesn't, then there'd better be a good reason.}

\cpnote{The argument silently assumes that there's no $\REVO$ oracle.}

\cpnote{Try starting this way:}
%
Just as in the proof of Theorem~\ref{thm:sbf-errep1} we will assume the
advwersary just makes a single query to~$\QRYO$ and use Lemma~\ref{thm:lemma1}
to complete the bound.
%
Let $\advA$ be an \erreps adversary making exactly~$1$ query to~$\REPO$, $q_T$
queries to~$\QRYO$, $q_U$ queries to~$\UPO$, and $q_H$ queries to~$\HASHO$.


\cpnote{Tip: If an claim follows easily from an argument made earlier than the
proof, then feel free to move quickly through it and refer the reader to the
argument for detials. The best you can do is say something like ``Equation (X)
follows nearly the same argument as used to deerive Equation (Y) ...''}

\cpnote{It'll be easier to apply Lemma~\ref{thm:lemma1} at the end.}
We first reduce from the \erreps case to the \erreps1 case, which by
lemma~\ref{lemma:errep} may scale the adversary's advantage only by a factor of
$q_R$. The game~$\game_0$ is exactly equivalent to the $\erreps1$ experiment, so
$\Adv{\errep1}_{\struct_s,r}(\advA) = \Prob{\game_0(\advA) = 1}$.%

In~$\game_1$ we split the hash oracle into three, giving the adversary access
$\HASHO_1$ in both stages of the game, while $\HASHO_2$ is reserved for oracular
use by $\Repx$ \cpnote{Undefined!}, $\QRYO$, and $\UPO$. For any $\advA$ for~$\game_0$, there is
$\advB$ for~$\game_1$ which produces the same advantage by simulating $\advA$.
This adversary first creates its own table $R$ with all values initially
undefined.  When $\advA$ makes a query $w$ to $H$, $\advB$ returns $R[w]$ if
that entry in the table is defined. Otherwise, if there are $\salt \in
\bits^\lambda$, $j \in [k]$, and $x \in \bits^*$ such that $w = \langle\salt, j,
x\rangle$, forward $(\salt,x)$ to $\HASHO_1$. For each $j \in [k]$, set
$R[\langle\salt, j, x\rangle] = \vv_j$, where $\vv$ is the output of the
$\HASHO_1$ oracle. If there is no such triple $\langle\salt, j, x\rangle$, just
sample $r$ from $[m]$ uniformly and set $R[w] = r$.
%
\cpnote{All of this is out date. This is about the linear hashing scheme of
Kirsch-Mitzenmacher, which we're no longer analyzing.}
%
In either case, return
$R[w]$ to $\advA$. When $\advA$ outputs its collection $\col$, $\advB$ outputs
$\col$ as well. Any queries by $\advA$ to $\QRYO$ or $\UPO$ are forwarded to
$\advB$'s corresponding oracle. The simulation is perfect because
$\Rep[H](\col)$ and $\Up[H](\col,\up)$ are identically distributed to
$\Rep[\HASHO_2](\col)$ and
$\Up[\HASHO_2](\col,\up)$. Because we have a perfect simulation,
$\Adv{\erreps1}_{\struct_s,r}(\advA) = \Prob{\game_1(\advA) = 1}$.

The game~$\game_2$ is the same as~$\game_1$ until $\bad_1$ is set, which occurs
exactly when $\advB$ sends $(\salt^*,x)$ to $\HASHO_1$ for some $x$. In the
first phase, there is again a $q_1/2^\lambda$ chance of the adversary guessing
the salt. In the second phase, the random sampling used by $\HASHO_i$ ensures
that each call the adversary makes to the $\HASHO_i$ oracle is independent of
all previous calls. We therefore have a $q_2/2^\lambda$ chance of the adversary
guessing the salt during this phase, for a total chance of $q_H/2^\lambda$
chance of the adversary guessing the salt at some point during the experiment.
Then $\Adv{\erreps}_{\struct_s,r}(\advA) \le \Prob{\game_2(\advB) = 1} +
q_H/2^\lambda$. Having taken this into account, we may now assume the adversary
never guesses the salt.
%
\cpnote{I would move quickly through this point. Just refer back to the step in
Theorem~\ref{thm:sbf-errep1}.}


\cpnote{The rest of this seems like the meat of the argument.}

We want to show that alternating between sequences of queries and sequences of
updates is no better than making one long series of updates and then one long
sequence of queries. There are three types of updates the adversary can make:
updates to add elements that have been queried and found to be false positives;
updates to add elements that have been queried and found not to be false
positives; and updates to add elements that have not been queried yet. We may
assume without loss of generality that the adversary never makes the first type
of update, since doing so is never beneficial (it does not change the
representation at all and decreases the number of errors the adversary has
found). \cpnote{Explicitly name these type-1, type-2, and type-3 updates, since
you refer to them below. IN fact, it might be beneficial to put them in a
bultted list.}

Note that the choices of $\vv$ constructed by the $\HASHO_i$ oracles are
independent of all previous queries. Because of this, any update of type-3 is
equivalent to any other update of type-3%
%
\cpnote{Careful with the word ``equivalent'': I think you mean ``have the same
distribution'' or something?}%
%
; the probability of any bit being flipped by one update is the same as the
probability of the bit being flipped by the other update. Similarly, any update
of type-2 is equivalent to any other update of type-2, but is not the same as
type-3 since the probability is conditioned on $\vv$ not being a false positive.
We assume the worst case, namely that all updates are type-2 (i.e. at least one
bit is flipped by each update).

Because the adversary never guesses the salt, $\HASHO_1$ simply functions as a
random oracle.
%
\cpnote{This isn't quite true. Go back to Thereom~\ref{fig:sbf-errep1} and think
about the semantics of~$\HASHO_1$ in games~0 and~1.}
%
Furthermore, we can assume the adversary never adds an element of
$\col$ to $\col$%
%
\cpnote{Word this differently. You mean update the data structure with an
element that is already in it?}
%
and never makes a $\QRYO$ call for an element which is already
in $\col$, since neither of these provides any additional information and
neither affects the rest of the experiment in any way.

Now we move to the game~$\game_3$. Here each $\QRYO$ query also calls $\UPO$ to
add that element to $\col$. \cpnote{Clever.}  Additionally, the penalty for adding known false
positives is removed. To avoid penalizing the adversary by prematurely maxing
out the number of elements in $\col$ because of added false positives, we also
increase the maximum size of $\col$ from $n$ to $n+s$, where $s = \min(r,q_U)$.
Because the adversary (without loss of generality) stops after accumulating $r$
errors, only $\min(r,q_U)$ false positives will be added to $\col$ and so a
maximum size of $n+s$ is sufficient to produce no penalty for the adversary.
Furthermore, each $\UPO$ call is preceded by a $\QRYO$ call. Neither of these
changes can produce a worse result for the adversary, so $\Prob{\game_2(\advB) =
1} \le \Prob{\game_3(\advB) = 1}$. Now, however, there is no longer any
distinction between $\QRYO$ and $\UPO$ calls. All calls to either oracle are
independent of each other and produce the same effect, querying and then
updating $\col$. Each of these queries for false positives is at most as
successful as a query to a Bloom filter with $n+r$ elements, so the adversary's
probability of finding a false positive on any query is bounded above by the
standard success rate for a Bloom filter with those parameters. The adversary is
required to produce $r$ errors over the course of $q_T+q_U$ queries, which by
the binomial theorem gives an advantage bound of $\Prob{\game_3(\advB) = 1} \leq
...$.
%
\cpnote{What about~$q_H$?}

\todo{DC (lead)}{Finish the bound by applying Lemma~\ref{thm:lemma1}.}
\end{proof}

\subsection{Keyed BFs}

Salted BFs are secure in the immutable setting \emph{or} the private
representation setting, but not both. Our argument for the \erreps\ security of
salted Bloom filters is made possible by virtue of the structure under attack
not being revealed to the adversary. While this is realistic in many settings,
it may be desirable for the Bloom filter to be public \emph{and} updatable.
%
Here we show that building a Bloom filter from a PRF suffices for security in
this setting.
%
So far our bounds have lost a factor of~$q_R$ in order to move from the \errep1\
to the \errep\ setting; as a bonus, here we show that it is possible to do
better.
%
Let $F:\keys\by\bits^*\to[m]^k$ be a function and fix
integers~$\ell,n,\lambda\geq0$.
%
Let $\Pi = \KBF[F,\ell,n,\lambda] = (\Rep^F, \Qry^F, \Up^H)$ as defined in
Figure~\ref{fig:bf-def}.

\begin{theorem}[\errep\ security of keyed BFs]\label{thm:bf-key-bound}
  For integers $q_R, q_T, q_H, r, t \geq 0$ it holds that.
  \begin{equation*}
    \begin{aligned}
      \Adv{\errep}_{\Pi,\delta,r}(t, q_R,q_T,q_U,q_H) &\leq \\
        \Adv{\prf}_{F}(\cdots) + \text{some cool bound} \,.
    \end{aligned}
  \end{equation*}
\end{theorem}
\begin{proof}
  \begin{figure*}
\todo{DC}{Rename $\HASHO$ to something else! It looks like a random oracle, but
it's not!}
\todo{DC}{nit: Here and throughout the rest of paper, change $ct$ to
$\mathit{ct}$. The former looks like $c\cdot t$.}
\twoCols{0.47}
{
  \vspace{-7pt}
  \experimentv{$\game_{0}(\advA)$}\\[2pt]
    $\key \getsr \keys$;
    $ct \gets 0$\\
    $i \getsr \advA^{\REPO,\QRYO,\UPO}$;
    return $\big[\sum_x \err_i[x] \geq r\big]$
  \\[6pt]
  \oraclev{$\HASHO(\salt \cat x)$}\hfill\diffminus{$\game_0$}\diffplus{$\game_1$}\\[2pt]
    \diffminus{$\vv \gets F_K(\salt \cat x)$}\\
    \diffplusbox{$\vv \getsr [m]^k$\\
    if $T[Z,x] = \bot$ then $\vv \gets T[Z,z]$\\
    $T[Z,x] \gets \vv$; return $\vv$}
  \\[6pt]
  \oraclev{$\QRYO(i, \qry_x)$}\\[2pt]
    $X \gets \bmap_m(\HASHO(\salt_i \cat x))$;
    $a \gets X = M_i \AND X$\\
    if $\err_i[x] < \delta(a,\qry_x(\col_i))$ then
          $\err_i[x] \gets \delta(a,\qry_x(\col_i))$\\
    return $a$
  \\[6pt]
  \oraclev{$\REPO(\col)$}\\[2pt]
    $ct \gets ct+1$;
    $\setS_{ct} \gets \col$;
    $\salt_{ct} \getsr \bits^\lambda$;
    $c_{ct} \gets |\col|$\\
    $M_{ct} \gets \bigvee_{x \in \col} \bmap_m(\HASHO(\salt_{ct} \cat x))$;
    return $\langle M_{ct}, \salt_{ct}, c_{ct} \rangle$
  \\[6pt]
  \oraclev{$\UPO(i, \up_x)$}\\[2pt]
    if $w(M) > \ell$ then return $\top$\\
    if $\QRYO(\qry_x) = 1$ then $\err_i[x] \gets 0$\\
    $M_i \gets M_i \vee \bmap_m(\HASHO(\salt_i \cat x))$;
    $\setS_i \gets \up_x(\setS_i)$;
    return $\langle M_i, \salt_i, c_i+1\rangle$
}
{
  \vspace{-7pt}
  \experimentv{$\game_{2}(\advA)$}\hfill\diffplus{$\game_3$}\\[2pt]
    $\key \getsr \keys$;
    $ct \gets 0$;
    $\setZ \gets \emptyset$\\
    $i \getsr \advA^{\REPO,\QRYO,\UPO}$;
    return $\big[\sum_x \err_i[x] \geq r\big]$
  \\[6pt]
  \oraclev{$\REPO(\col)$}\\[2pt]
    $ct \gets ct+1$;
    $\setS_{ct} \gets \col$;
    $\salt_{ct} \getsr \bits^\lambda \setminus \setZ$;
    $c_{ct} \gets |\col|$\\
    $\setZ \gets \setZ \cup \{\salt_{ct}\}$\\
    $M_{ct} \gets \bigvee_{x \in \col} \bmap_m(\HASHO(\salt_{ct} \cat x))$;
    return $\langle M_{ct}, \salt_{ct}, c_{ct} \rangle$
  \\[6pt]
  \oraclev{$\QRYO(i, \qry_x)$}\\[2pt]
    $X \gets \bmap_m(\HASHO(\salt_i \cat x))$;
    $a \gets X = M_i \AND X$\\
    if $\err_i[x] < \delta(a,\qry_x(\col_i))$ then
          $\err_i[x] \gets \delta(a,\qry_x(\col_i))$\\
    \diffplus{$\UPO(i, \up_x)$;}
    return $a$
  \\[6pt]
  \experimentv{$\game_{4}(\advB)$}\\[2pt]
    $\key \getsr \keys$;
    $ct \gets 0$;
    $\setZ \gets \emptyset$\\
    $i \getsr \advB^{\REPO,\QRYO}$;
    return $\big[\sum_x \err_i[x] \geq r\big]$
  \\[6pt]
  \oraclev{$\QRYO(i, \qry_x)$}\\[2pt]
    for $i \in [ct]$ do\\
    $\tab X \gets \bmap_m(\HASHO(\salt_i \cat x))$;
    $a \gets X = M_i \AND X$\\
    $\tab$if $\err_i[x] < \delta(a,\qry_x(\col_i))$ then
          $\err_i[x] \gets \delta(a,\qry_x(\col_i))$\\
    $\tab\UPO(i, \up_x)$;
    return $a$
}
\caption{Games 0--4 for proof of Theorem~\ref{thm:bf-key-bound}.}
\label{fig:kbf-errep/games}
\end{figure*}

We start with a game~$\game_0$ which is essentially the same as the standard
\errep\ experiment on a Bloom filter, given the assumption (without loss of
generality) that the adversary never attempts to construct a representation for
a set with more than $n$ elements. As with the other proofs, it is easy to see
that for any such \errep\ adversary we can make an adversary $\advA$
for~$\game_0$ with the same resources that achieves the same advantage.

Unlike in the previous two proofs, we cannot use Lemma~\ref{thm:lemma1} because
an adversary cannot simulate the oracles without knowing the private key. We use
an alternate approach to gradually reduce to the standard binomial bound
deriving from the non-adaptive false positive probabilities. The first thing we
want to do is to bound the probability that the adversary can break the PRF.

The number of times the PRF is evaluated on distinct inputs is bounded by the
number of queries available to the adversary. In particular, $\QRYO$ and $\UPO$
each call the PRF once, while $\REPO$ may call the PRF up to $n$ times. If the
adversary runs in $t$ time steps, then, the probability it can distinguish the
PRF from a random function is bounded by $\Adv{\prf}_F(t,nq_R+q_T+q_U)$.
%
In~$\game_1$, we have a game which is identical to~$\game_0$ except that it uses
random sampling in place of the PRF. If $\advA$ cannot distinguish the PRF from
a random function then these games are indistinguishable from the adversary's
perspective, so $\Prob{\game_0(\advA) = 1} \le \Adv{\prf}_F(t,nq_R+q_T+q_U) +
\Prob{\game_1(\advA) = 1}$.
%
\cpnote{In fat, this isn't immediate. You show this by a reduction. You want to
show that for every $\advA$ there exists an adversary~$D$ such that
$\Prob{\game_0(\advA)=1} - \Prob{\game_1(\advA)=1} \leq \Adv{\prf}_F(D)$. You
don't need to be super formal about it, but you do need to say how~$D$
executes~$\advA$ and what outputs.}

Our goal is to argue, in a similar manner as to the previous theorems, that all of the oracle calls are independent. In order to guarantee this we must deal with the possibility of a salt collision between different representations. In~$\game_2(\advA)$ we require that all salts be distinct between representations. By the birthday bound, collisions between randomly-generated salts occur with frequency at most $q_R^2/2^\lambda$, so $\Prob{\game_1(\advA) = 1} \le q_R^2/2^\lambda + \Prob{\game_2(\advA) = 1}$.

With guaranteed-unique salts, the result of each $\REPO$, $\UPO$, and $\QRYO$
call for a given representation is independent of the calls for all other
representations. By an almost identical argument to the proof of
Theorem~\ref{thm:sbf-erreps}, we can assume without loss of generality that the
adversary follows any $\QRYO$ call that does not find a false positive with an
$\UPO$ call to insert that element, and therefore move to~$\game_3(\advA)$,
which as in the previous proof automatically performs an update after each query
is made. Since the adversary never inserts the same element multiple times, we
can again conclude that without loss of generality the adversary never directly
invokes the $\UPO$ oracle.
%
\cpnote{As in the previous result, I think it's better to give an explicit
reduction, rather than say ``we can assume without loss of generality ..''}

Finally, we must deal with the possibility that the adversary chooses which
representations to target with $\UPO$ and $\QRYO$ calls based on the result of
$\REPO$, since some representations may be more full than others. In
game~$\game_4$,we deny the adversary direct access to the $\UPO$ oracle
because it is never needed, but we allow the adversary credit if a call to
$\QRYO$ produces an error in any of the representations that have been
constructed. Furthermore, the updates made by $\QRYO$ apply to all
representations that are not already full. Since all $\UPO$ calls are
identically and independently distributed, and having more elements in a filter
cannot decrease the false positive rate, the fact that some representations may
become full more quickly than they otherwise would have can only help the
adversary. Similarly, having $\QRYO$ count errors across all representations
never harms the adversary, and so the adversary's advantage may only increase
when moving to~$\game_4(\advB)$ \todo{DC}{$\advB$ is undefined at this point}. Therefore $\Prob{\game_3(\advA) = 1} \le
\Prob{\game_4(\advB) = 1}$, where $\advB$ is an adversary which behaves
identically to $\advA$ but which is syntactically distinct because it lacks the
unused $\UPO$ oracle.
%
\cpnote{Instead of making assumptions about~$\advA$'s behavior and arguing that
they're not without loss, just give an explicit reduction.}

We are now in a situation where we can apply the standard, non-adaptive error
bound. Let $\setX$ be the set of all queries $\qry_x$ made by the adversary over
the course of the game. As in the previous proof, we have $|\setX| \le q_T$.
However, $\qry_x$ may now cause a false positive in any of the representations.
The probability of causing a false positive in a specific representation is
still given by the non-adaptive false positive probability $p'$ for a Bloom
filter containing $n+r$ elements. Since the representations are independent of
each other, the probability of a false positive occurring in any of up to $q_R$
representations is at most $p'q_R$. We can therefore bound the adversary's
success probability using a binomial distribution, similar to before:
\begin{equation}
   \Prob{\game_4(\advB)=1} \le
     \sum_{i=r}^{q_T} \binom{q_T}{i}(p'q_R)^i(1-p'q_R)^{q_T-i} \,.
\end{equation}

Applying the usual Chernoff bound, we find
\begin{equation}
   \Prob{\game_4(\advB)=1} \le
     e^{r-p'q_Rq_T}\left(\frac{p'q_Rq_T}{r}\right)^r.
\end{equation}

So, substituting this bound back into the earlier advantage inequalities, we find the final bound of
\begin{equation*}
  \begin{aligned}
    \Adv{\errep}_{\Pi,\delta,r}(t, q_R,q_T,q_U,q_H) &\leq \\
      \Adv{\prf}_F(t,nq_R+q_T+q_U) & +
    \frac{q_R^2}{2^\lambda} +
    \left(\frac{p'q_Rq_T}{r}\right)^re^{r-p'q_Rq_T}
  \end{aligned}
\end{equation*}

\end{proof}

\todo{DC}{Add discussion of how the $\REVO$ oracle rules out the security of
keyed, unsalted Bloom filters. Be sure to mention that this was something
considered by Gerbet \etal~\cite{gerbet2015power}.}
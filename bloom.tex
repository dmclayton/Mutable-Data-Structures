%\subsection{Bloom filter lower bounds}

%\begin{table}[thp]
\begin{center}
  \begin{tabular}{ | c | c |c | c | c | }
    \hline
    Salt & Key & \parbox{.75in}{Private\\ Representation} & \parbox{.75in}{Irreversible\\ Updates} & Result \\ \hline\hline
    0 & 0 & 0 & 0 & Attack 1 \\ \hline
    0 & 0 & 0 & 1 & Attack 1 \\ \hline
    0 & 0 & 1 & 0 & Attack 1 \\ \hline
    0 & 0 & 1 & 1 & Attack 1 \\ \hline
    0 & 1 & 0 & 0 & Attack 2 \\ \hline
    0 & 1 & 0 & 1 & Attack 2 \\ \hline
    0 & 1 & 1 & 0 & Attack 2 \\ \hline
    0 & 1 & 1 & 1 & Attack 2 \\ \hline
    1 & 0 & 0 & 0 & Attack 3 \\ \hline
    1 & 0 & 0 & 1 & Attack 3 \\ \hline
    1 & 0 & 1 & 0 & ? \\ \hline
    1 & 0 & 1 & 1 & Theorem~\ref{thm:bf-priv-salt-bound} \\ \hline
    1 & 1 & 0 & 0 & Attack 4 \\ \hline
    1 & 1 & 0 & 1 & Theorem~\ref{thm:bf-key-bound} \\ \hline
    1 & 1 & 1 & 0 & ? \\ \hline
    1 & 1 & 1 & 1 & Theorem~\ref{thm:bf-key-bound} \\
    \hline
  \end{tabular}
\end{center}
\tsnote{This table feels unintutive, somehow. ``Salt'', ``Key'', ``Irreversible updates'' are all syntactic properties; ``Private Representation'' is an artifact of the attack model.  We need a better way to organize and present these results.}
\caption{Summary table for Bloom filters. \textbf{Attack 1:}The adversary chooses a maximally large test set $\col$ and simulates $\Rep$ to produce a representation $\pub$. The adversary then simulates $\Rep$ for many arbitrarily chosen singleton sets disjoint from $\col$, checking each one to see if its element is a false positive for $\pub$. Once it has accumulated $r$ false positives, call $\REPO$ on $\col$, call $\QRYO$ for each false positive found, and return 1.  \textbf{Attack 2:} The adversary chooses a maximally large test set $\col$ and calls $\REPO$ to produce a representation $\pub$. The adversary then calls $\REPO$ for many arbitrarily chosen singleton sets disjoint from $\col$, using $\QRYO$ on each to determine if its element is a false positive for the representation constructed for $\col$. Once it has accumualated $r$ false positives, return 1. \textbf{Attack 3:}  The adversary chooses a maximally large test set $\col$ and calls $\REPO$ to receive a representation $\pub$ together with the salt $\salt$. Using the known salt, the adversary simulates $\Rep$ for many arbitrarily chosen singleton sets disjoint from $\col$, checking each one to see if its element is a false positive for $\pub$. Once it has accumulated $r$ false positives, call $\QRYO$ for each such false positive and return 1. \textbf{Attack 4:}The adversary chooses a test set $\col_0$ and a target set $\col_1$, performing a search on representable subsets of $\col_0$ represented as a tree ordered by $\subseteq$. Moving up or down in the tree is accomplished using $\UPO$, and at each node $\QRYO$ is called for each element of $\col_1$ to determine which are false positives. Once $r$ false positives are found, halt and return 1. }
\label{tab:main}
\end{table}
\todo{DC (lead)}{Explain away all cases except the one(s) for which we give
security (upperbound) theorems.} The standard Bloom filter shows a variety of
different behaviors depending on its exact implementation. If the hash functions
used are chosen beforehand and potentially known to the adversary, this public
information allows offline attacks to be mounted against the data structure
which can produce potentially damaging false positives. In the case of immutable
Bloom filters, making use of a per-representation salt is sufficient to prevent
these attacks, though depending on the use case the use of non-fixed
per-representation randomness may or may not be feasible. Furthermore, in the
case of mutable Bloom filters there are additional difficulties with offline
attacks due to adversarially-chosen updates. To guarantee correctness in this
case we must additionally guarantee that representations can be kept private
from the adversary.

A generic attack against unsalted, unkeyed data structures handling set
membership queries is for the adversary to choose a large set $\col$ of
potential filter elements and simulate $\REPO$ to produce representations for
$\{x\}$ for each $x \in \col$. Given these, it can perform offline computations
to determine disjoint $\setT$, $\setR \subseteq \col$ such that the elements of
$\setR$ all produce errors when queried for membership in $\REPO(\setT)$. After
a sufficiently large $\setR$ has been found, the adversary makes a single
$\REPO$ call on $\setT$ and makes one $\QRYO$ call per element of $\setR$,
resulting in guaranteed success for the adversary with only a minimal number of
queries performed.

In general, such an attack may not be computationally feasible in the real
world. However, because the attack is entirely offline, many structures with
non-negligible error probabilities are vulnerable to these attacks. For example,
consider a Bloom filter with false positive probability $10^{-5}$ with an
adversary wishing to construct $r = 10$ false positives. The adversary can fix
$\setT$ in advance and perform somewhere on the order of a million hash queries
to random elements in order to construct a $\setR$ of size 10, and then perform
only a single $\REPO$ calls and ten $\QRYO$ calls to the service hosting the
Bloom filter. This is likely to be much more feasible for the adversary than
performing on the order of a million $\QRYO$ calls in an online attack which
randomly guesses elements until it accumulates 10 false positives.

The use of a salt without a private key in the public representation setting is
insufficient to defeat this attack. In this setting, the adversary need only
make its $\REPO$ query for $\setT$ in advance, at which point it will receive
both the representation $\pub$ and the salt $\salt$ used to construct it. Using
this known salt, the adversary is still able to simulate $\REPO$ for arbitrary
singleton representations. The previous attack therefore still works with the
same (minimal) number of $\QRYO$ calls at the very end of the experiment, after
it has determined $\setR$ using offline computations.

The opposite of this, using a private key without a salt, does weaken the attack
somewhat. Even with public representations, the adversary cannot locally
simulate $\REPO$ without guessing the private key. However, they can still
outperform random $\QRYO$ calls by making $\REPO$ queries for singleton elements
without fixing any $\setT$ in advance. After selecting a random set $\col$ of
size $q_R-1$, the adversary performs offline computations to find $\setT$,
$\setR \subseteq \col$ such that the elements of $\setR$ are false positives for
the representation of $\setT$ (which can be computed from the representations of
the singleton subsets of $\setT$). The adversary wins if there is a partition
where $\setR$ produces at least $r$ errors on $\REPO(\setT)$.

% ...

Using a salted Bloom filter in the private representation setting, however, does
provide some security. At the time a representation is created, the structure
chooses a salt $\salt$ which it will use for all further queries and updates. In
order for maximum security to be guaranteed, we must ensure that the
representation, and in particular the salt, is kept secret from the adversary.
We define this structure $\SBF[H,k,m,n,\lambda]$ as the Bloom filter structure
that uses $H(s) = (h_1(s),\ldots,h_k(s))$ for hashing inputs to $k$ values in
$[m]$. Furthermore, each call of $\Rep$ first involves picking a salt $\salt$
from the salt space $\bits^\lambda$, and all hashes made to insert or query for
an element $x$ are determined using $H(x \Vert \salt)$. Finally, the parameter
$n$ means that any attempts to represent sets with more than $n$ elements fail.
\todo{DC lead}{Specify what exactly is being analyzed.  What are the updates?}

\begin{theorem}[Correctness Bound for Private-Representation Salted Bloom Filters]\label{thm:bf-priv-salt-bound}
Fix integers $k, m, n, \lambda, r\geq 0$, let $H \colon \bits^* \to [m]$ be a function, and let $\struct_s = \SBF[H,k,m,n,\lambda]$.
  For every $t, q_R, q_T, q_U, q_H \geq 0$, it holds that
  \begin{eqnarray*}
    \Adv{\erreps}_{\struct_\saltybloom,r}(t,&q_R,& q_T, q_U, q_H) \leq \\ && q_R \cdot
     \left[
      \frac{q_H}{2^\lambda} +
      {\dbinom{q_T+q_U}{r}} p(k, m, n+s)^r
    \right] \,,
\end{eqnarray*}
where $s$ is defined to be $\min(r,q_U)$, $H$ is modeled as a random oracle, and $p(k, m, n+s)$ is the standard, non-adaptive false-positive probability on a Bloom filter with the given parameters.
\end{theorem}

This proof first reduces to the single-representation case, which as shown in
lemma~\ref{lemma:errep} will reduce the adversary's advantage by at most a
factor of $q_R$. The main idea behind the proof is to remove the adversary's
adaptivity a step at a time. We isolate the possibility of the adversary
guessing the salt, which would allow it to mount its own offline attack on the
filter without relying on the $\QRYO$ oracle. If the adversary does not guess
the salt, the outputs of the $\REPO$, $\QRYO$, and $\UPO$ oracles are
unpredictable to the adversary, producing uniformly randomly distributed bits to
set (for $\REPO$ and $\UPO$) or to check (for $\QRYO$). Under the assumption
that the adversary does not predict the salt, queries made to distinct elements
are independent of each other. The only remaining issue is that the adversary
can potentially gain an advantage by testing whether some object $x$ is a false
positive for the filter, and then updating the filter to include $x$ only if the
test query returned `false'. An analysis shows that this is now (once imperfect
pseudorandom functions and salt collisions have been dealt with) the only way
for the adversary to gain an advantage over making queries to an immutable Bloom
filter. Because this adaptive strategy introduces tricky conditional
possibilities, we cannot compute an exact value for the adversary's advantage.
Instead, we move to an alternate scenario where each $\QRYO$ also produces a
free update and every $\UPO$ first performs a free query. This makes $\QRYO$ and
$\UPO$ calls indistinguishable, so that the adversary is effectively making a
series of independent random queries that each have a chance to increment the
error counter. Because the number of 1s in the filter can only increase, the
probability of a false positive from any one of these queries is bounded above
by the probability of a false positive on the final maximally-sized filter, a
probability which is given by the Kirsch and Mitzenmacher bound.

\begin{figure*}
  \boxThmBFSaltCorrect{0.48}
  {
    \underline{$\game_0(\advA)$}\\[2pt]
      $\col \getsr \advA^H$; $\setC \gets \emptyset$; $\err \gets 0$\\
      $\pub \getsr \Rep[H](\col)$\\
      $\bot \getsr \advA^{H,\QRYO,\UPO}$\\
      return $(\err \geq r)$
    \\[6pt]
    \oraclev{$\QRYO(\qry_x)$}\\[2pt]
      if $\qry_x \in \mathcal{C}$ then return $\bot$\\
      $\setC \gets \setC \union \{\qry_x\}$\\
      $a \gets \Qry[H](\pub, \qry_x)$\\
      if $a \neq \qry_x(\col)$ then $\err \gets \err + 1$\\
      return~$a$
    \\[6pt]
    \oraclev{$\UPO(\up_x)$}\\[2pt]
      $\setC \gets \emptyset$\\
      $a \gets \Qry[H](\pub, \qry_x)$\\
      if $\qry_x \in \setC$ and $a \neq \qry_x(\col)$ then\\
      \tab $\err \gets \err-1$\\
      $\col \gets \col \union \{x\}$\\
      $\pub \gets \Up[H](\pub,\up_x)$\\
      return~$\bot$
    \\[4pt]
    \hspace*{-4pt}\rule{1.043\textwidth}{.4pt}
    \\[5pt]
    \oraclev{$\HASHO_1(\salt,x)$} \hfill\diffplus{$\game_2$}\;{$\game_1$}\hspace*{3pt}\\
      $\hh \getsr [m]^2$; $\vv \gets \fff(\hh$)\\
      if $\salt = \salt^*$ then\\
      \tab $\bad_1 \gets 1$; \diffplus{return $\vv$}\\
      if $T[\salt,x]$ is defined then $\vv \gets T[\salt,x]$\\
      $T[\salt,x] \gets \vv$;
      return $\vv$
  }
  {
    \underline{$\game_1(\advB)$}\\[2pt]
      $\salt^* \getsr \bits^\lambda$;
      $\col \getsr \advB^{\HASHO_1}$\\
      $\pub \gets \Repx[\HASHO_2](\col, \salt^*)$\\
      $\setC \gets \emptyset$;
      $\err \gets 0$\\
      $\bot \getsr \advB^{\HASHO_1,\QRYO,\UPO}$\\
      return $(\err \geq r)$
    \\[6pt]
    \oraclev{$\QRYO(\qry_x)$}\\[2pt]
      if $\qry_x \in \mathcal{C}$ then return $\bot$\\
      $\setC \gets \setC \cup \{\qry_x\}$\\
      $a \gets \Qry[\HASHO_2](\pub, \qry_x)$\\
      if $a \neq \qry_x(\col)$ then $\err \gets \err + 1$\\
      return~$a$
    \\[6pt]
    \oraclev{$\UPO(\up_x)$}\\[2pt]
      $\setC \gets \emptyset$\\
      $a \gets \Qry[H](\pub, \qry_x)$\\
      if $a \neq \qry_x(\col)$ and $\qry_x \in \setC$ then\\
      \tab $\err \gets \err-1$\\
      $\col \gets \col \union \{x\}$\\
      $\pub \gets \Up[\HASHO_2](\pub,\up_x)$\\
      return~$\bot$
    \\[6pt]
    \oraclev{$\HASHO_2(\salt,x)$}\\[2pt]
      $\hh \getsr [m]^2$; $\vv \gets \fff(\hh$)\\
      if $T[\salt,x]$ is defined then\\
      \tab $\vv \gets T[\salt,x]$\\
      $T[\salt,x] \gets \vv$;
      return $\vv$
  }
  {
    \underline{$\game_3(\advB)$}\\[2pt]
    \oraclev{$\QRYO(\qry_x)$}\\[2pt]
      $a \gets \Qry[\HASHO_3](\pub, \qry_x)$\\
      if $a \neq \qry_x(\col)$ then $\err \gets \err + 1$\\
      $\col \gets \col \union \{x\}$
      $\pub \gets \Up[\HASHO_2](\pub,\up_x)$\\
      return~$a$
  }
  {
    \oraclev{$\UPO(\up_x)$}\\[2pt]
      $a \gets \Qry[\HASHO_3](\pub, \qry_x)$\\
      if $a \neq \qry_x(\col)$ then $\err \gets \err + 1$\\
      $\col \gets \col \union \{x\}$
      $\pub \gets \Up[\HASHO_2](\pub,\up_x)$\\
      return~$\bot$
    \\[6pt]
    \oraclev{$\HASHO_i(\salt,x)$}\\[2pt]
      $\hh \getsr [m]^2$; $\vv \gets \fff(\hh$)\\
      return $\vv$
  }
  \caption{Games 0--3 for proof of Theorem~\ref{thm:bf-priv-salt-bound}.}
  \label{fig:bf-priv-salt-bound}
\end{figure*}

\begin{proof}

We first reduce from the $\erreps$ case to the $\erreps1$ case, which by
lemma~\ref{lemma:errep} may scale the adversary's advantage only by a factor of
$q_R$. The game~$\game_0$ is exactly equivalent to the $\erreps1$ experiment, so
$\Adv{\errep1}_{\struct_s,r}(\advA) = \Prob{\game_0(\advA) = 1}$. In~$\game_1$
we split the hash oracle into three, giving the adversary access $\HASHO_1$ in
both stages of the game, while $\HASHO_2$ is reserved for oracular use by
$\Repx$, $\QRYO$, and $\UPO$. For any $\advA$ for~$\game_0$, there is $\advB$
for~$\game_1$ which produces the same advantage by simulating $\advA$. This
adversary first creates its own table $R$ with all values initially undefined.
When $\advA$ makes a query $w$ to $H$, $\advB$ returns $R[w]$ if that entry in
the table is defined. Otherwise, if there are $\salt \in \bits^\lambda$, $j \in
[k]$, and $x \in \bits^*$ such that $w = \langle\salt, j, x\rangle$, forward
$(\salt,x)$ to $\HASHO_1$. For each $j \in [k]$, set $R[\langle\salt, j,
x\rangle] = \vv_j$, where $\vv$ is the output of the $\HASHO_1$ oracle. If there
is no such triple $\langle\salt, j, x\rangle$, just sample $r$ from $[m]$
uniformly and set $R[w] = r$. In either case, return $R[w]$ to $\advA$. When
$\advA$ outputs its collection $\col$, $\advB$ outputs $\col$ as well. Any
queries by $\advA$ to $\QRYO$ or $\UPO$ are forwarded to $\advB$'s corresponding
oracle. The simulation is perfect because $\Rep[H](\col)$ and $\Up[H](\col,\up)$
are identically distributed to $\Rep[\HASHO_2](\col)$ and
$\Up[\HASHO_2](\col,\up)$. Because we have a perfect simulation,
$\Adv{\erreps1}_{\struct_s,r}(\advA) = \Prob{\game_1(\advA) = 1}$.

The game~$\game_2$ is the same as~$\game_1$ until $\bad_1$ is set, which occurs
exactly when $\advB$ sends $(\salt^*,x)$ to $\HASHO_1$ for some $x$. In the
first phase, there is again a $q_1/2^\lambda$ chance of the adversary guessing
the salt. In the second phase, the random sampling used by $\HASHO_i$ ensures
that each call the adversary makes to the $\HASHO_i$ oracle is independent of
all previous calls. We therefore have a $q_2/2^\lambda$ chance of the adversary
guessing the salt during this phase, for a total chance of $q_H/2^\lambda$
chance of the adversary guessing the salt at some point during the experiment.
Then $\Adv{\erreps_1}_{\struct_s,r}(\advA) \le \Prob{\game_2(\advB) = 1} +
q_H/2^\lambda$. Having taken this into account, we may now assume the adversary
never guesses the salt.

We want to show that alternating between sequences of queries and sequences of
updates is no better than making one long series of updates and then one long
sequence of queries. There are three types of updates the adversary can make:
updates to add elements that have been queried and found to be false positives;
updates to add elements that have been queried and found not to be false
positives; and updates to add elements that have not been queried yet. We may
assume without loss of generality that the adversary never makes the first type
of update, since doing so is never beneficial (it does not change the
representation at all and decreases the number of errors the adversary has
found).

Note that the choices of $\vv$ constructed by the $\HASHO_i$ oracles are
independent of all previous queries. Because of this, any update of type 3 is
equivalent to any other update of type 3; the probability of any bit being
flipped by one update is the same as the probability of the bit being flipped by
the other update. Similarly, any update of type 2 is equivalent to any other
update of type 2, but is not the same as type 3 since the probability is
conditioned on $\vv$ not being a false positive. We assume the worst case,
namely that all updates are type 2 (i.e. at least one bit is flipped by each
update).

Because the adversary never guesses the salt, $\HASHO_1$ simply functions as a
random oracle. Furthermore, we can assume the adversary never adds an element of
$\col$ to $\col$ and never makes a $\QRYO$ call for an element which is already
in $\col$, since neither of these provides any additional information and
neither affects the rest of the experiment in any way.

Now we move to the game~$\game_3$. Here each $\QRYO$ query also calls $\UPO$ to
add that element to $\col$. Additionally, the penalty for adding known false
positives is removed. To avoid penalizing the adversary by prematurely maxing
out the number of elements in $\col$ because of added false positives, we also
increase the maximum size of $\col$ from $n$ to $n+s$, where $s = \min(r,q_U)$.
Because the adversary (without loss of generality) stops after accumulating $r$
errors, only $\min(r,q_U)$ false positives will be added to $\col$ and so a
maximum size of $n+s$ is sufficient to produce no penalty for the adversary.
Furthermore, each $\UPO$ call is preceded by a $\QRYO$ call. Neither of these
changes can produce a worse result for the adversary, so $\Prob{\game_2(\advB) =
1} \le \Prob{\game_3(\advB) = 1}$. Now, however, there is no longer any
distinction between $\QRYO$ and $\UPO$ calls. All calls to either oracle are
independent of each other and produce the same effect, querying and then
updating $\col$. Each of these queries for false positives is at most as
successful as a query to a Bloom filter with $n+r$ elements, so the adversary's
probability of finding a false positive on any query is bounded above by the
standard success rate for a Bloom filter with those parameters. The adversary is
required to produce $r$ errors over the course of $q_T+q_U$ queries, which by
the binomial theorem gives an advantage bound of $\Prob{\game_3(\advB) = 1} \le
\binom{q_T+q_U}{r}p(k, m, n+s)^r$.

The full adversarial advantage is then
$$\Adv{\erreps_1}_{\struct_s,r}(\advA) \le q_R \cdot \left(\frac{q_H}{2^\lambda} + \binom{q_T+q_U}{r}p(k, m, n+s)^r\right).$$

\missingqed
\end{proof}

%Without `thresholding', where the addition to a filter fails if a certain proportion of bits are set to 1, the bound improves somewhat.

%Once we have reduced to the case of $\HASHO_1$ functioning as a random oracle and $\QRYO$ calls only being made to distinct elements not in $\col$, the adversary has only three ways of interacting with the filter: querying an element not in $\col$ which has not yet been queried, inserting a new element into $\col$ which has not yet been queried, and inserting a new element into $\col$ which has been queried and was found to not be a false positive. In the non-threshold case, the last of these is exploitable, since an element which is confirmed not to be a false positive will necessarily flip at least one bit when inserted into the representation. However, when a maximum proportion $p$ of the bits are allowed to be set to 1, adding known non-false-positives rather than unqueried elements only causes the threshold to be reached faster, without providing any additional benefit to the adversary. In any case the adversary can do no better than setting exactly $pm$ bits to 1, in which case the false positive rate will be exactly $p^k$ and the adversarial advantage will be

%$$\Adv{\erreps_1}_{\struct_s,r}(\advA) \le q_R \cdot \left(\frac{q_H}{2^\lambda} + \binom{q_T}{r}p^{kr}\right).$$

For the salted and keyed case with a built-in threshold, we use the notion of query independence to demonstrate that, if a thresholding assumption is used to prevent the filter from becoming too full, we can attain a good error bound even in the public-representation case.

%NOTE: n is the THRESHOLD here
\begin{theorem}[Correctness Bound for Salted and Keyed Bloom Filters]\label{thm:bf-key-bound}
Fix integers $k, m, n, \lambda, r\geq 0$, let $\Pi$ be the salted and keyed Bloom filter described above with PRF $F$, and let $d$ be the error function assigning a weight of 1 to a false positive and 0 to a correct response.
  For every $t, q_R, q_T, q_U, q_H \geq 0$ such that $q_T \geq r$, it holds that
  \begin{eqnarray*}
    \Adv{\errep}_{\struct,r,d}(t,&q_R,& q_T, q_U, q_H) \leq \\ && I_{(n/m)^k}(r, q_T-r+1) \,.
  \end{eqnarray*}
where $I$ is the regularized incomplete beta function.
\end{theorem}

\begin{proof}
In order to show that a salted, secretly-keyed Bloom filter is secure, we want
to show that the structure has query independence up to leakage that is defined
to be the number of bits in the filter which are set to 1. If the PRF $F$ is
good, and the key is unknown to the adversary, the adversary cannot use the
random oracle to predict what representations will look like ahead of time. In
particular, by a straightforward conditioning argument,
$\Adv{\game_0}_{\struct,r}(\advA) \le \Adv{\prf}(F) +
\Adv{\game_1}_{\struct,r}(\advA)$, where $\game_1$ is the same game but with the
pseudorandom function replaced by a lazily-evaluated random function. In this
game, the adversary cannot possibly guess queries ahead of time with better than
50/50 odds, since the output of the query depends on the output of a true random
function on an untested input. We can then immediately apply the independence
lemma to see that $\Adv{\game_1}_{\struct,r}(\advA) \le I_p(r, q_T-r+1)$, where
$p$ is the false positive probability. In a filter where $\textsf{lk}(\pub)$ out
of $m$ bits are set to 1, this probability is simply $(\textsf{lk}(\pub)/m)^k$.
The adversary can maximize this value by setting as many bits to 1 as possible.
With a threshold of $n$ bits allowed to be set to 1, the overall advantage is
then
$$\Adv{\errep}_{\struct,r,d}(\advA) \le \Adv{\prf}(F) + I_{(n/m)^k}(r, q_T-r+1)$$
\end{proof}

%The case of a Bloom filter whose representations are always public but which uses a private key in addition to a salt is almost identical. In practice the bound is actually somewhat stronger: the instance of $q_H$ in the bound is replaced with $q_R$, which means that the adversary which attempts to guess the salt must rely entirely on `online' queries as opposed to `offline' queries to $q_H$.

%These two theorems together show that in order to guarantee maximal correctness in the streaming setting, one must ensure that representations are kept private from potential adversaries, and possibly use a secret key in addition to a salt when implementing a Bloom filter. In either case, the hidden information is necessary to protect the filter from an adversary.

We can use this final and best bound to demonstrate some of the possible bounds
with various parameter settings. Figure ? shows the upper bound for a various
combination of parameters, starting from the default parameters of $k = 4$, $m =
1024$, $n = 100$, $r = 10$, $q_R = 1$, $q_T = 100$, and $q_U = 100$. The most
significant factor in the adversarial advantage is $q_R$, with even small
increases producing a drastic increase in the adversarial advantage. Since this
is the only part of the error bound that is not clearly associated with an
attack, it is possible that this term could be reduced or eliminated.

%\includegraphics[scale=0.75]{BF_Fig}

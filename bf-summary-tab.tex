\begin{table}[thp]
\begin{center}
  \begin{tabular}{ | c | c |c | c | c | }
    \hline
    Salt & Key & \parbox{.75in}{Private\\ Representation} & \parbox{.75in}{Irreversible\\ Updates} & Result \\ \hline\hline
    0 & 0 & 0 & 0 & Attack 1 \\ \hline
    0 & 0 & 0 & 1 & Attack 1 \\ \hline
    0 & 0 & 1 & 0 & Attack 1 \\ \hline
    0 & 0 & 1 & 1 & Attack 1 \\ \hline
    0 & 1 & 0 & 0 & Attack 2 \\ \hline
    0 & 1 & 0 & 1 & Attack 2 \\ \hline
    0 & 1 & 1 & 0 & Attack 2 \\ \hline
    0 & 1 & 1 & 1 & Attack 2 \\ \hline
    1 & 0 & 0 & 0 & Attack 3 \\ \hline
    1 & 0 & 0 & 1 & Attack 3 \\ \hline
    1 & 0 & 1 & 0 & ? \\ \hline
    1 & 0 & 1 & 1 & Theorem~\ref{thm:bf-priv-salt-bound} \\ \hline
    1 & 1 & 0 & 0 & Attack 4 \\ \hline
    1 & 1 & 0 & 1 & Theorem~\ref{thm:bf-key-bound} \\ \hline
    1 & 1 & 1 & 0 & ? \\ \hline
    1 & 1 & 1 & 1 & Theorem~\ref{thm:bf-key-bound} \\
    \hline
  \end{tabular}
\end{center}
\tsnote{This table feels unintutive, somehow. ``Salt'', ``Key'', ``Irreversible updates'' are all syntactic properties; ``Private Representation'' is an artifact of the attack model.  We need a better way to organize and present these results.}
\caption{Summary table for Bloom filters. \textbf{Attack 1:}The adversary chooses a maximally large test set $\col$ and simulates $\Rep$ to produce a representation $\pub$. The adversary then simulates $\Rep$ for many arbitrarily chosen singleton sets disjoint from $\col$, checking each one to see if its element is a false positive for $\pub$. Once it has accumulated $r$ false positives, call $\REPO$ on $\col$, call $\QRYO$ for each false positive found, and return 1.  \textbf{Attack 2:} The adversary chooses a maximally large test set $\col$ and calls $\REPO$ to produce a representation $\pub$. The adversary then calls $\REPO$ for many arbitrarily chosen singleton sets disjoint from $\col$, using $\QRYO$ on each to determine if its element is a false positive for the representation constructed for $\col$. Once it has accumualated $r$ false positives, return 1. \textbf{Attack 3:}  The adversary chooses a maximally large test set $\col$ and calls $\REPO$ to receive a representation $\pub$ together with the salt $\salt$. Using the known salt, the adversary simulates $\Rep$ for many arbitrarily chosen singleton sets disjoint from $\col$, checking each one to see if its element is a false positive for $\pub$. Once it has accumulated $r$ false positives, call $\QRYO$ for each such false positive and return 1. \textbf{Attack 4:}The adversary chooses a test set $\col_0$ and a target set $\col_1$, performing a search on representable subsets of $\col_0$ represented as a tree ordered by $\subseteq$. Moving up or down in the tree is accomplished using $\UPO$, and at each node $\QRYO$ is called for each element of $\col_1$ to determine which are false positives. Once $r$ false positives are found, halt and return 1. }
\label{tab:main}
\end{table}
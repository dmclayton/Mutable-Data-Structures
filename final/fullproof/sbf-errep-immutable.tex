\begin{figure*}
\threeColsOneDivideUnbalanced{0.34}{0.33}{0.27}
{
  \vspace{-7pt}
  \experimentv{$\game_{0}(\advB)$}
      \hfill \diffplus{$\game_1$}\\[2pt]
    $M^* \gets \bot$;
    $\salt^* \getsr \bits^\lambda$\\
    $\advB^{\REPO,\QRYO,\HASHO_1}$\\
    return $\big[\sum_x \err[x] \geq r\big]$
  \\[6pt]
  \oraclev{$\REPO(\col)$}\\[2pt]
    $M^* \gets \bigvee_{x \in \col} \bmap_m(\HASHO_2(\salt^* \cat x))$\\
    $\setS^* \gets \col$\\
    return $\langle M^*, Z^* \rangle$
  \\[6pt]
  \oraclev{$\QRYO(\qry_x)$}\\[2pt]
    $X \gets \bmap_m(\HASHO_3(\salt^* \cat x))$\\
    $a \gets X = M^* \AND X$\\
    if $\err[x] < \delta(a,\qry_x(\col^*))$ then\\
          $\err[x] \gets \delta(a,\qry_x(\col^*))$\\
    return $a$
  \\[6pt]
  \oraclev{$\HASHO_c(\salt \cat x)$}\\[2pt]
    $\vv \getsr [m]^k$\\
    if $M^*=\bot$ and $\salt=\salt^*$ and $c=1$ then\\
    \tab $\bad_1 \gets 1$; \diffplus{return $\vv$} \com{Caller is~$\advB$}\\
    if $T[Z,x] = \bot$ then\\
    \tab $\vv \gets T[Z,x]$\\
    $T[Z,x] \gets \vv$; return $\vv$
}
{
  \vspace{-2pt}
  \oraclev{$\HASHO_c(\salt \cat x)$}\\[2pt]
    $\vv \getsr [m]^k$\\
    if $M^*=\bot$ and $\salt=\salt^*$ and $c=1$ then\\
    \tab $\bad_1 \gets 1$; return $\vv$\\
    if $T[Z,x] = \bot$ then\\
    \tab $\vv \gets T[Z,z]$\\
    $T[Z,x] \gets \vv$\\[2pt]
    \diffplusbox{
    \com{Caller is~$\advB$ or $\QRYO$}\\
    if $c=1$ or $c=3$ then\\
    \tab if $\salt \ne \salt^*$  then\\
    \tab return $\vv$\\
    \tab $\Ans[x] \gets \bmap_m(\vv) = M^* \AND \bmap_m(\vv)$\\
    \tab $\epsilon \gets \delta(\Ans[x],\qry_x(\col^*))$\\
    \tab if $\err[x] < \epsilon$ then\\
    \tab\tab $\err[x] \gets \epsilon$
    }
    return $\vv$
}
{
  \vspace{-7pt}
  \oraclev{$\QRYO(\qry_x)$}\
      \hfill \diffminus{$\game_1$} \diffplus{$\game_2$}\\[2pt]
    \diffminusbox{%
      $X \gets \bmap_m(\HASHO_3(\salt^* \cat x))$\\
      $a \gets X = M^* \AND X$\\
      if $\err[x] < \delta(a,\qry_x(\col^*))$ then\\
      \tab $\err[x] \gets \delta(a,\qry_x(\col^*))$
    }\\[2pt]
    \diffplusbox{
      $\HASHO_3(Z^* \cat x)$\\
      $a \gets \Ans[x]$
    }
    return $a$
}
\caption{Games 0, 1, and 2 for proof of Theorem~\ref{thm:sbf-errep-immutable}.}
\label{fig:sbf-errep-immutable/games}
\end{figure*}

We will use the following lemma for keyless structures.

\begin{lemma}\label{thm:lemma1}
  For every $q_R, q_T, q_U, q_H, r, t \geq 0$ and keyless structure~$\Gamma$ it
  holds that
  \begin{eqnarray*}
    \begin{aligned}
      \Adv{\errep}_{\Gamma,\delta,r}(t,\,&q_R, q_T, q_U, q_H) \leq q_R \cdot \Adv{\errep1}_{\Gamma,\delta,r}(O(t), q_T, q_U, q_H) \,,
    \end{aligned}
  \end{eqnarray*}
\end{lemma}
%
\noindent
The proof is by a fairly straightforward hybrid argument. Because~$\Gamma$ is
keyless, in the reduction we simulate $q_R-1$ of the calls to $\REPO$ experiment
and use our own oracles for the remaining query. The best we can do with this
strategy is to guess the representation the \errep\ adversary will use in
its attack, which results in the~$q_R$ factor in the bound.

Let $\advA$ be an \errep\ adversary making~$1$ query to~$\REPO$, $q_T$ queries
to $\QRYO$, $0$ queries to $\UPO$, and $q_H$ queries to the random
oracle~$\HASHO$.
%
We make the following assumptions, all of which are without loss of generality.
%
First, all of~$\advA$'s $\QRYO$ queries follow its $\REPO$ query.
%
Second, we assume that $x\not\in\setS$ for all queries $\qry_x$ to $\QRYO$,
where~$\setS$ was the input to~$\advA$'s $\REPO$ query. This is without loss
because Bloom filters admit false positives, but not false negatives.
%
Third, we we assume that $|\setS| \leq n$; this is without loss because
otherwise~$\REPO$ outputs~$\bot$ and~$\advA$ gets no advantage.
%
Fourth, we assume that all of~$\advA$'s $\HASHO$ queries are of the form $Z\cat
x$, where $|Z| = \lambda$.

We begin with a game-playing argument~\cite{bellare2006triple}, then obtain the
final bound via application of Lemma~\ref{thm:lemma1}.
%
The high-level goal is to rewrite the game so that the probability that one
of~$\advA$'s queries runs up the score is precisely the non-adaptive false
positive probability.
%
In other words, our goal is to transition into a setting in which the Bloom
filter output by~$\REPO$ is independent of the outcome of~$\advA$'s other
queries.

Consider the game~$\game_0(\advB)$ defined in
Figure~\ref{fig:sbf-errep-immutable/games}. It is similar to the \errep\
experiment when executed with~$\advA$, $\Pi$, $\delta$, and~$r$, but the
pseudocode has been simplified to clarify our argument. Indeed, it is not
difficult to see that for every~$\advA$ there exists an adversary~$\advB$ such
that
\begin{equation}
  \Adv{\errep}_{\Pi,\delta,r}(\advA) \leq \Prob{\game_0(\advB) = 1}
\end{equation}
and~$\advB$ has the same resources as~$\advA$.
%
Adversary~$\advB$ executes~$\advA$, forwarding~$\advA$'s oracle queries
to its own oracles in the natural way.

Observe that in game~$\game_0$ the salt used for the representation of~$\setS^*$
is generated prior to executing~$\advB$. Game~$\game_1$ is identical
to~$\game_0$ until the flag~$\bad_1$ gets set by oracle~$\HASHO$. This occurs
if~$\advB$ asks $\HASHO_1(\salt^* \cat x)$, where~$\salt^*$ is the salt generated
at the beginning of the game, and it has not yet called its $\QRYO$ oracle (i.e.,
$M^*=\bot$).
%
By the Fundamental Lemma of Game Playing~\cite{bellare2006triple} it follows
that
%
\begin{eqnarray}
  \Prob{\game_0(\advB)=1} &\leq&
    \Prob{\game_1(\advB)=1} + \Prob{\game_1(\advB) \sets \bad_1}\\
  &\leq&
    \Prob{\game_1(\advB)=1} + q_H/2^\lambda \,.
\end{eqnarray}
%
Note that in $\game_1$, the value of~$M^*$ is independent of~$\advB$'s
$\HASHO_1$ queries. In particular, the probability that some bit of~$M^*$ is set
is independent of random coins of~$\advB$.

In game $\game_2$ the $\HASHO$ and $\QRYO$ oracles have been rewritten so that
the winning condition is computed by $\HASHO$ instead of $\QRYO$. The former
oracle maintains a set~$\Ans$ such that $\Ans[x] = \Qry^{\HASHO_3}(M^*, \qry_x)$ for
each query $\salt^* \cat x$; on input of $\qry_x$, oracle~$\QRYO$ simply runs
$\HASHO_3(\salt^* \cat x)$ and returns $\Ans[x]$.
%
We are effectively giving the adversary credit for RO queries that result in
false positives for the representation of~$\setS^*$, but which it does not
explicitly ask of~$\QRYO$. Because~$\advB$'s advantage in the new game is at
least as much as it gets in the old one, we have that.
%
\begin{equation}
  \Prob{\game_1(\advB)=1} \leq \Prob{\game_2(\advB)=1} \,.
\end{equation}

We now consider $\Prob{\game_2(\advB)=1}$.
%
Let $\setX$ be the set $\{ x \in \bits^* : \Ans[x] \ne \bot \}$ and $\setT = \{x
\in\setX: \Ans[x] = 1\}$, where $\Ans$ is at is defined when~$\advB$ halts. We
will call~$\setX$ the set of attempts and~$\setT$ the set of false positives.
%
Note that $\setX\intersection\setS^*=\emptyset$ by assumption, and
$|\setX| \leq q_H + q_T$ by definition.
%
Hence, the probability that~$\game_2(\advB)=1$ is equal to the probability
that~$|\setT| \geq r$.

For each $x\in\setX$, let $T(x)$ denote the event that $x\in\setT$.
%
In the random oracle model for~$H$, the set of random random variables $T(x)$
for each $x\in X$ are independently and identically distributed.
%
Hence, the probability that~$\advB$ succeeds is binomially distributed:
%
\begin{equation}
   \Prob{\game_2(\advB)=1} = \Prob{ |\setT| \geq r } =
     \sum_{i=r}^{q} \binom{q}{i}p^i(1-p)^{q-i} \,,
\end{equation}
%
where $q \leq q_H + q_T$ and $p = \Pr[T(x)=1]$. Here we can apply a Chernoff
bound which states that, for any $\delta > 0$,
%
\begin{equation}
  \Prob{X \geq (1+\delta)\mu} < \left(\frac{e^\delta}{(1+\delta)^{1+\delta}}\right)^\mu
  \,.
\end{equation}
%
We set $\delta = r\mu^{-1}-1$ and note that $\mu = pq$.
This yields
%
\begin{equation}
 \Prob{|\setT| \geq r} < \left(\frac{e^{r\mu^{-1}-1}}{(r\mu^{-1})^{r\mu^{-1}}}\right)^\mu = \left(\frac{e^{r-\mu}}{(r\mu^{-1})^r}\right) = e^{r-pq}\left(\frac{pq}{r}\right)^r
\end{equation}
%
and so
%
\begin{equation}
  \Adv{\errep}_{\Pi,\delta,r}(\advA) < \frac{q_H}{2^\lambda} + \left(\frac{pq}{r}\right)^re^{r-pq}
  \,.
\end{equation}
%
Applying Lemma~\ref{thm:lemma1} to move from the single-representation case to the
general case, we get our final bound of
\begin{equation}
  \Adv{\errep}_{\Pi,\delta,r}(\advA) \leq
    q_R \cdot \left[
      \frac{q_H}{2^\lambda} +
      \left(\frac{pq}{r}\right)^re^{r-pq}
    \right] \,.
\end{equation}

\begin{figure*}[t]
\twoCols{0.47}
{
  \vspace{-7pt}
  \experimentv{$\game_{0}(\advA)$}\\[2pt]
    $\key \getsr \keys$;
    $\ct \gets 0$\\
    $i \getsr \advA^{\REPO,\QRYO,\UPO}$;
    return $\big[\sum_x \err_i[x] \geq r\big]$
  \\[6pt]
  \oraclev{$\PRFO(\salt \cat x)$}\hfill\diffminus{$\game_0$}\diffplus{$\game_1$}\\[2pt]
    \diffminus{$\vv \gets F_K(\salt \cat x)$}\\
    \diffplusbox{$\vv \getsr [m]^k$\\
    if $T[Z,x] = \bot$ then $\vv \gets T[Z,x]$\\
    $T[Z,x] \gets \vv$; return $\vv$}
  \\[6pt]
  \oraclev{$\QRYO(i, \qry_x)$}\\[2pt]
    $X \gets \bmap_m(\PRFO(\salt_i \cat x))$;
    $a \gets X = M_i \AND X$\\
    if $\err_i[x] < \delta(a,\qry_x(\col_i))$ then
          $\err_i[x] \gets \delta(a,\qry_x(\col_i))$\\
    return $a$
  \\[6pt]
  \oraclev{$\REPO(\col)$}\\[2pt]
    $\ct \gets \ct+1$;
    $\setS_{\ct} \gets \col$;
    $\salt_{\ct} \getsr \bits^\lambda$;
    $c_{\ct} \gets |\col|$\\
    $M_{\ct} \gets \bigvee_{x \in \col} \bmap_m(\PRFO(\salt_{\ct} \cat x))$;
    return $\langle M_{\ct}, \salt_{\ct}, c_{\ct} \rangle$
  \\[6pt]
  \oraclev{$\UPO(i, \up_x)$}\\[2pt]
    if $w(M) > \ell$ then return $\top$\\
    if $\QRYO(\qry_x) = 1$ then $\err_i[x] \gets 0$\\
    $M_i \gets M_i \vee \bmap_m(\PRFO(\salt_i \cat x))$;
    $\setS_i \gets \up_x(\setS_i)$\\
    return $\langle M_i, \salt_i, c_i+1\rangle$
}
{
  \vspace{-7pt}
  \experimentv{$\game_2(\advA)$}\hfill\diffplus{$\game_3$}\\[2pt]
    $\key \getsr \keys$;
    $\ct \gets 0$;
    $\setZ \gets \emptyset$\\
    $i \getsr \advA^{\REPO,\QRYO,\UPO}$;
    return $\big[\sum_x \err_i[x] \geq r\big]$
  \\[6pt]
  \oraclev{$\REPO(\col)$}\\[2pt]
    $\ct \gets \ct+1$;
    $\setS_{\ct} \gets \col$;
    $\salt_{\ct} \getsr \bits^\lambda \setminus \setZ$;
    $c_{\ct} \gets |\col|$\\
    $\setZ \gets \setZ \cup \{\salt_{\ct}\}$\\
    $M_{\ct} \gets \bigvee_{x \in \col} \bmap_m(\PRFO(\salt_{\ct} \cat x))$;
    return $\langle M_{\ct}, \salt_{\ct}, c_{\ct} \rangle$
  \\[6pt]
  \oraclev{$\QRYO(i, \qry_x)$}\\[2pt]
    $X \gets \bmap_m(\PRFO(\salt_i \cat x))$;
    $a \gets X = M_i \AND X$\\
    if $\err_i[x] < \delta(a,\qry_x(\col_i))$ then
          $\err_i[x] \gets \delta(a,\qry_x(\col_i))$\\
    \diffplus{$\UPO(i, \up_x)$;}
    return $a$
  \\[4pt]
  \hrule
  \vspace{2pt}
  \oraclev{$\QRYO(i, \qry_x)$}\hfill\diffminus{$\game_3$}\diffplus{$\game_4$}\\[2pt]
    \diffminusbox{
    $X \gets \bmap_m(\PRFO(\salt_i \cat x))$;
    $a \gets X = M_i \AND X$\\
    if $\err_i[x] < \delta(a,\qry_x(\col_i))$ then\\
    \tab$\err_i[x] \gets \delta(a,\qry_x(\col_i))$\\
    $\UPO(i, \up_x)$;
    return $a$
    }
    \diffplusbox{
    for $j \in [\ct]$ do\\
    $\tab X \gets \bmap_m(\PRFO(\salt_j \cat x))$;
    $a_j \gets X = M_j \AND X$\\
    $\tab$if $\err_j[x] < \delta(a_j,\qry_x(\col_j))$ then\\
    \tab\tab$\err_j[x] \gets \delta(a_j,\qry_x(\col_j))$\\
    $\tab\UPO(j, \up_x)$\\
    return $a_i$}
}
\caption{Games 0--4 for proof of Theorem~\ref{thm:bf-key-bound}.}
\label{fig:kbf-errep/games}
\end{figure*}

We start with a game~$\game_0$, shown in Figure~\ref{fig:kbf-errep/games}, which is essentially the same as the standard
\errep\ experiment on a Bloom filter, given the assumption (without loss of
generality) that the adversary never attempts to construct a representation for
a set with more than~$n$ elements.
%
%As with the other proofs, it is easy to see
%that for any such \errep\ adversary we can make an adversary $\advA$
%for~$\game_0$ with the same resources that achieves the same advantage.
%
Unlike in the previous two proofs, we cannot use Lemma~\ref{thm:lemma1} because
an adversary cannot simulate the oracles without knowing the private key. We use
an alternate approach to gradually reduce to the standard binomial bound
deriving from the non-adaptive false positive probabilities. The first thing we
want to do is to bound the probability that the adversary can break the PRF.

The number of times the PRF is evaluated on distinct inputs is bounded by the
number of queries available to the adversary. In particular, $\QRYO$ and $\UPO$
each call the PRF once, while $\REPO$ may call the PRF up to $n$ times. Thus,
when executed with~$\advA$, game~$\game_0$ makes at most~$Q = q_T + q_U + nq_R$
queries to~$\PRFO$.
%
In~$\game_1$, we have a game which is identical to~$\game_0$ except that it uses
random sampling in place of the PRF. If $\advA$ cannot distinguish the PRF from
a random function then these games are indistinguishable from the adversary's
perspective.
%
We exhibit a $O(t)$-time, \prf-adversary~$D$ making at most~$Q$ queries to its
oracle such that
%
\begin{equation}
  \Adv{\prf}_F(D) = \Prob{\game_0(\advA)=1} - \Prob{\game_1(\advA)=1} \,.
\end{equation}
%
Adversary~$D^{\,\PRFO}$ works by executing~$\advA$ in game~$\game_1$, except
that whenever the game calls \emph{its}~$\PRFO$, adversary~$D$ uses its
own oracle to compute the response.
%
Finally, when~$\advA$ halts, if the winning condition in~$\game_1$ is satisfied,
then~$D$ outputs~$1$ as its guess; otherwise it outputs~$0$.
%
Then conditioning on the outcome of the coin flip~$b$ in~$D$'s game, we have that
%
\begin{eqnarray}
  \Adv{\prf}_F(D) &=&
    2\Prob{\Exp{\prf}_F(D)=1}-1 \\
  &=&
    2\,\Big(
      1/2\Prob{\Exp{\prf}_F(D)=1 \given b=1} + 1/2\Prob{\Exp{\prf}_F(D)=1 \given b=0}
    \Big) - 1 \\
  &=&
    \Prob{\Exp{\prf}_F(D)=1 \given b=1} + \Prob{\Exp{\prf}_F(D)=1 \given b=0}
     - 1 \\
  &=& \Prob{\game_0(\advA)=1} - \Prob{\game_1(\advA)=1} \,.
\end{eqnarray}
%
Next, our goal is to argue, in a similar manner as to the previous theorems, that all
of the oracle calls are independent. In order to guarantee this we must deal
with the possibility of a salt collision between different representations.
In~$\game_2(\advA)$ we require that all salts be distinct between
representations. By the birthday bound, collisions between randomly-generated
salts occur with frequency at most $q_R^2/2^\lambda$, so $\Prob{\game_1(\advA) =
1} \le q_R^2/2^\lambda + \Prob{\game_2(\advA) = 1}$.

With guaranteed-unique salts, the result of each $\REPO$, $\UPO$, and $\QRYO$
call for a given representation is independent of the calls for all other
representations. By an almost identical argument to the one in the proof of
Theorem~\ref{thm:sbf-erreps}, we can reduce from any $\advA$ to an adversary
$\advB$ which follows any $\QRYO$ call that finds a true negative with an
$\UPO$ call to insert that element, and therefore move to~$\game_3(\advB)$,
which as in the Theorem~\ref{thm:sbf-erreps} proof performs an update after each
query is made, with the guarantee that $\Prob{\game_1(\advA) = 1} \le
\Prob{\game_2(\advB) = 1}$.

Finally, we must deal with the possibility that the adversary chooses which
representations to target with $\UPO$ and $\QRYO$ calls based on the result of
$\REPO$, since some representations may be more full than others. In
game~$\game_4$, we allow the adversary credit if a call to
$\QRYO$ produces an error in any of the representations that have been
constructed. Furthermore, the updates made by $\QRYO$ apply to all
representations that are not already full. Since all $\UPO$ calls are
identically and independently distributed, and having more elements in a filter
cannot decrease the false positive rate, the fact that some representations may
become full more quickly than they otherwise would have can only help the
adversary. Similarly, having $\QRYO$ count errors across all representations
never harms the adversary, and so the adversary's advantage may only increase
when moving to~$\game_4$,
i.e. $\Prob{\game_3(\advB) = 1} \le \Prob{\game_4(\advB) = 1}$.

We are now in a situation where we can apply the standard, non-adaptive error
bound. Let $\setX$ be the set of all queries $\qry_x$ made by the adversary over
the course of the game. As in the previous proof, we have $|\setX| \le q_T$.
However, $\qry_x$ may now cause a false positive in any of the representations.
The probability of causing a false positive in a specific representation is
still given by the non-adaptive false positive probability $p'$ for a Bloom
filter containing $n+r$ elements. Since the representations are independent of
each other, the probability of a false positive occurring in any of up to $q_R$
representations is at most $p'q_R$. We can therefore bound the adversary's
as before. In particular,
\begin{equation}
   \Prob{\game_4(\advB)=1} \le
     e^{r-p'q_Rq_T}\left(\frac{p'q_Rq_T}{r}\right)^r.
\end{equation}

So, substituting this bound back into the earlier advantage inequalities, we find the final bound of
\begin{equation*}
  \begin{aligned}
    \Adv{\errep}_{\Pi,\delta,r}(\advA) &\leq \Adv{\prf}_F(D) &  +
    \frac{q_R^2}{2^\lambda} +
    \left(\frac{p'q_Rq_T}{r}\right)^re^{r-p'q_Rq_T} \,.
  \end{aligned}
\end{equation*}

Probabilistic data structures use space-efficient representations of data in
order to (approximately) respond to queries about the data.  Traditionally,
these structures are accompanied by probabilistic bounds on query-response
errors. These bounds implicitly assume benign attack models, in which the data
and the queries are chosen non-adaptively, and independent of the randomness
used to construct the representation. Yet probabilistic data structures are
increasingly used in settings where these assumptions may be violated.

This work provides a provable-security treatment of probabilistic data
structures in adversarial environments. We give a syntax that captures a wide
variety of in-use structures, and our security notions support derivation of
error bounds in the presence of powerful attacks.

We use our formalisms to analyze Bloom filters, counting (Bloom) filters and
count-min sketch data structures.  For the traditional version of these, our
security findings are largely negative; however, we show that simple
embellishments (e.g., using salts or secret keys) yields structures that
provide provable security, and with little overhead.

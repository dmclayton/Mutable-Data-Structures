\label{sec:count}
\begin{figure}
  \twoColsNoDivide{0.33}
  {
    \underline{$\Rep^R_K(\col)$}\\[2pt]
      $\salt \getsr \bits^\lambda$ \com{Choose a salt $\salt$}\\
      $\pub \gets \langle \zeroes(m), \salt\rangle$\\
      for $x \in \col$ do \\
        $\tab \pub \gets \Up^R_K(\pub, \up_{x,1})$\\
        $\tab$if $\pub = \bot$ then return $\bot$\\
      return $\pub$
    \\[6pt]
      \underline{$\Qry^R_K(\langle \v.M, \salt\rangle,x)$}\\[2pt]
      $\v.X \gets R_K(\salt \cat x)$\\
      for $i \in \v.X$ do\\
        $\tab$if $\v.M[i] = 0$ then return 0\\
      return 1
  }
  {
    \underline{$\Up^R_K(\langle \v.M, \salt\rangle, \up_{x,b})$}\\[2pt]
      if $c \geq n$ then return $\bot$\\
      $\v.M' \gets \v.M$;
      $\v.X \gets R_K(\salt \cat x)$\\
      for $i$ in $\v.X$ do\\
      $\tab$ $a \gets \v.M'[i]$\\
      $\tab$ if $a = 0 \wedge b < 0$ then return $\bot$\\
      $\tab \v.M'[i] \gets \v.M'[i] + b$\\
      $\v.M \gets \v.M'$\\
      return $\langle \v.M, \salt \rangle$
  }
  \caption{Keyed structure $\countbloom[R,\ell,\lambda]$ given by
  $(\Rep^R,\Qry^R,\Up^R)$ is used to define the $\ell$-thresholded version of a
  counting filter. The parameters are a function $R:
  \keys\by\bits^* \to [m]^k$ and integers $\ell, \lambda \geq0$. A concrete scheme
  is given by a particular choice of parameters. The function $\hw'$, used to
  determine if the filter is full, is defined in Section~\ref{sec:prelims}.}
  \label{fig:cbf-def}
\end{figure}

Counting filters are a modified version of Bloom filters which are designed to,
like a count min-sketch, allow for deletion as well as insertion~\cite{fan2000summary}.
Unlike CMSes,
however, counting filters are designed to handle set membership queries rather
than frequency queries. Despite this, the two structures are closely related in
terms of security properties. We show that \errep\ security is similarly
impossible, but employing $\ell$-thresholding allows for \erreps\ security with a
bound that is close to count min-sketch. This $\ell$-thresholded filter is
formally defined in Figure~\ref{fig:cbf-def}.

\heading{Error function for frequency queries}
%
Unlike with a Bloom filter or count min-sketch, counting filters must account
for two different types of errors: false positives and false negatives. To be as
general as possible, we define a parametrized error function~$\delta$ for
positive $\delta^+, \delta^- \in \R$ as
\begin{equation}
  \delta(x, y) =
  \begin{cases}
    0 & \text{if}\ x = y \\
    \delta^+ & \text{if}\ x = 1, y = 0 \\
    \delta^- & \text{if}\ x = 0, y = 1
  \end{cases}
\end{equation}
This means that false positives are given a weight of $\delta^+$ while false
negatives are given a weight of $\delta^-$, and correct responses are given a
weight of 0.

\subsection{Insecurity of public counting filters}
Any counting filter construction necessarily fails to satisfy \errep\
correctness for the same reasons as in the case of count min-sketch. In
particular, the adversary can call $\REPO(\emptyset)$ to receive an empty
representation, insert an element $x$, observe which counters are incremented by
this insertion, and then delete $x$. By doing this repeatedly, the adversary can
gain information about which elements overlap with which combinations of other
elements, and can therefore mount the same attack described in
Section~\ref{sec:pub-sketch-bad}.

\subsection{Private Thresholded Counting Filters}

\begin{theorem}\label{thm:counting-erreps}
Fix integers $q_R,q_T,q_U,q_H,q_V, r, t \geq 0$, let $p_\ell = ((\ell+1)/m)^k$,
and let
$r' = \lfloor r/\max(\delta^+,k\delta^-) \rfloor$. For all such
$q_R,q_T,$ $q_U,q_H,q_V,r$, and~$t$, if $r' > p_\ell q_T$ then
  \begin{equation*}
  \begin{aligned}
   \Adv{\erreps}_{\Pi,\delta,r}(O(t),\,&q_R,q_T,q_U,q_H,q_V) \leq q_R \cdot \left[\frac{q_H}{2^\lambda} + e^{r'-p_\ell q_T}\left(\frac{p_\ell q_T}{r'}\right)^{r'}\right],
  \end{aligned}
\end{equation*}
where $H$ is modeled as a random oracle.
\end{theorem}
The proof combines details from the proofs of count min-sketch bounds and the
proofs of Bloom filter bounds. In particular, we begin by following an argument
as in the count min-sketch case to limit the advantage the adversary can get
from deleting elements and from re-inserting elements of the original set. There
is a difference in the counting filter case in that \emph{inserting} a known
false positive cannot benefit the adversary, while \emph{deleting} it can. This
is the opposite of the count min-sketch case, but the effect on the bound is
quite similar. In each case we give the adversary additional credit for finding
these false positives while constraining them to not modify false positives they
find. This causes $r'$ to appear in the proof bound rather than $r$ in each
case, but because of the different error functions the definitions of $r'$ are
slightly different.
%
After this step, we observe that a counting filter without
deletion behaves the same as an ordinary Bloom filter in terms of how it
responds to queries. We can therefore borrow the arguments used to establish the
bounds in Theorem~\ref{thm:sbf-erreps-th} to finish the proof.

\proc{
  \begin{proof}[Proof Sketch of Theorem~\ref{thm:counting-erreps}]
  \begin{figure*}
\twoCols{0.47}
{
  \vspace{-7pt}
  \experimentv{$\game_{0}(\advA)$}\hfill\diffplus{$\game_1$}\\[2pt]
    $\v.M^* \gets \bot$;
    $\setS \gets \emptyset$;
    $\salt^* \getsr \bits^\lambda$\\
    $\advB^{\REPO,\QRYO,\UPO,\HASHO_1}$;
    return $\big[\sum_x \err[x] \geq r\big]$
  \\[6pt]
  \oraclev{$\HASHO_c(\salt \cat x)$}\\[2pt]
    $\vv \getsr [m]^k$\\
    if $\salt=\salt^*$ and $c = 1$ then \com{Caller is~$\advB$}\\
    \tab $\bad_1 \gets 1$; \diffplus{return $\vv$}\\
    if $T[Z,x] = \bot$ then $\vv \gets T[Z,x]$\\
    $T[Z,x] \gets \vv$; return $\vv$
  \\[6pt]
  \oraclev{$\QRYO(\qry_x)$}\\[2pt]
    $\v.X \gets \HASHO_3(\salt^* \cat x)$;
    $\setS \gets \setS \cup \{x\}$;
    $a = 1$\\
    for $i$ in $\v.X$ do\\
      $\tab$if $\v.M[i] = 0$ then $a = 0$\\
    if $\err[x] < \delta(a,\qry_x(\col^*))$ then
          $\err[x] \gets \delta(a,\qry_x(\col^*))$\\
    return $a$
  \\[6pt]
  \oraclev{$\REPO(\col)$}\\[2pt]
    $\v.M^* \gets 0^m$\\
    $\setS^* \gets \col$\\
    for $x \in \col$ do\\
      $\tab\UPO(\up_x)$\\
    return $\top$
  \\[6pt]
  \oraclev{$\UPO(\up_{x,b})$}\\[2pt]
    if $w'(\v.M^*) > \ell$ then return $\top$\\
    $\v.X \gets \HASHO_3(\salt^* \cat x)$;
    $\v.M' \gets \v.M^*$\\
    for $i$ in $\v.X$ do\\
      $\tab$ if $\v.M'[i] = 0$ and $b < 0$ then return $\top$\\
      $\tab \v.M'[i] \gets \v.M'[i] + b$\\
    if $b > 0$ and $\QRYO(\qry_x) = 1$ then $\err_i[x] \gets 0$\\
    if $b < 0$ and $\QRYO(\qry_x) = 0$ then $\err_i[x] \gets 0$\\
    $\v.M^* \gets \v.M'$;
    $\setS^* \gets \up_{x,b}(\setS^*)$;
    return $\top$
}
{
  \vspace{-7pt}
  \experimentv{$\game_2(\advA)$}\hfill\diffplus{$\game_2$}\\[2pt]
    $\v.M^* \gets \bot$;
    $\setS \gets \emptyset$;
    \diffplus{$\setR \gets \emptyset$; $r' \gets \lfloor r/\max(\delta^+,k\delta^-)\rfloor$}\\
    $\salt^* \getsr \bits^\lambda$\\
    $\advB^{\REPO,\QRYO,\UPO,\HASHO_1}$;
    return $\big[\sum_x \err[x] \geq r\big]$
  \\[6pt]
  \oraclev{$\QRYO(\qry_x)$}\\[2pt]
    $\v.X \gets \HASHO_3(\salt^* \cat x)$;
    $\setS \gets \setS \cup \{x\}$;
    $a = 1$\\
    for $i$ in $\v.X$ do\\
      $\tab$if $\v.M[i] = 0$ then $a = 0$\\
    if $\err[x] < \delta(a,\qry_x(\col^*))$ then
          $\err[x] \gets \delta(a,\qry_x(\col^*))$\\
    if $\err[x] > 0$ then $\setR \gets \setR \cup \{x\}$\\
    return $a$
  \\[6pt]
  \oraclev{$\UPO(\up_{x,b})$}\\[2pt]
    if $w'(\v.M^*) > \ell$\diffplus{$+r'$} then return $\top$\\
    \diffplus{if $x \in \setR$ and $b < 0$ then return $\top$}\\
    $\v.X \gets \HASHO_3(\salt^* \cat x)$;
    $\v.M' \gets \v.M^*$\\
    for $i$ in $\v.X$ do\\
      $\tab$ if $\v.M'[i] = 0$ and $b < 0$ then return $\top$\\
      $\tab \v.M'[i] \gets \v.M'[i] + b$\\
    if $b > 0$ and $\QRYO(\qry_x) = 1$ then $\err_i[x] \gets 0$\\
    if $b < 0$ and $\QRYO(\qry_x) = 0$ then $\err_i[x] \gets 0$\\
    $\v.M^* \gets \v.M'$;
    $\setS^* \gets \up_{x,b}(\setS^*)$;
    return $\top$
  \vspace{6pt}\hrule\vspace{3pt}
  \oraclev{$\UPO(\up_{x,b})$}\hfill\diffminus{$\game_2$}\diffplus{$\game_3$}\\[2pt]
    if $w'(\v.M^*) > \ell+r'$ then return $\top$\\
    if $x \in \setR$ and $b < 0$ then return $\top$\\
    $\v.X \gets \HASHO_3(\salt^* \cat x)$;
    $\v.M' \gets \v.M^*$\\
    for $i$ in $\v.X$ do\\
      $\tab$ if $\v.M'[i] = 0$ and $b < 0$ then return $\top$\\
      \diffminus{$\tab \v.M'[i] \gets \v.M'[i] + b$}\\
      \diffplus{$\tab \v.M'[i] \gets \min(\v.M'[i] + b, 1)$}\\
    if $b > 0$ and $\QRYO(\qry_x) = 1$ then $\err_i[x] \gets 0$\\
    if $b < 0$ and $\QRYO(\qry_x) = 0$ then $\err_i[x] \gets 0$\\
    $\v.M^* \gets \v.M'$;
    $\setS^* \gets \up_{x,b}(\setS^*)$;
    return $\top$
}
\caption{Games 0--3 for proof of Theorem~\ref{thm:scbf-erreps-th}.}
\label{fig:sbf-erreps/games}
\end{figure*}

As with the proof of Theorem~\ref{thm:sbf-errep-immutable}, we derive a bound in
the \erreps1 case and then use Lemma~\ref{thm:lemma1} to move from \erreps1 to
the more general \erreps case. Because we are in the \erreps1 case, we may
assume without loss of generality that the adversary does not call $\REVO$,
since revealing the only representation automatically prevents the adversary
from winning.

We begin with a game~$\game_0$ which has identical behavior to the \erreps1
experiment for a counting filter. As in the proof of
Theorem~\ref{thm:sbf-errep-immutable}, we have a
$\bad_1$ flag that gets set if the adversary ever calls $\HASHO_1$ with the
actual salt used by the representation. By a very similar argument, we can
move to~$\game_1$, where the behavior is different only when the $\bad_1$ flag
is set, with a bound of
\begin{equation}
  \Prob{\game_0(\advA)=1} \leq
    q_H/2^\lambda + \Prob{\game_1(\advA)=1} \,.
\end{equation}

Unlike in the case of a count min-sketch, it is entirely possible for deletions
to benefit the adversary in this game. In particular, if $x$ is found to be a
false positive, deleting $x$ may cause up to $k$ elements of $\col$ to become
false negatives. We therefore move to a game~$\game_2$ where the adversary gets
credit for either a single false positive or for $k$ false negatives whenever it
finds a false positive, but where the adversary cannot delete any false
positives that it finds. We let $r' = \lfloor r/\max(\delta^+,k\delta^-)\rfloor$
represent the number of false positives the adversary has to find in~$\game_2$
in order to win. In order to prevent the adversary from being penalized by the
filter becoming full too early, we also raise the thresold from $\ell$ to
$\ell+r'$ in~$\game_2$. Now for any $\advA$ for~$\game_1$, we can construct
$\advB$ for~$\game_2$ that simulates $\advA$, keeping track of all query
responses and forwarding all oracle queries in the natural way, except that
calls to delete false positives are ignored. Since $\UPO$ never fails for
$\advB$ due to the increased threshold, and since $\advB$ gets automatic credit
for any false negatives that might have been caused by deleting false positives,
$\advB$ succeeds whenever $\advA$ does, i.e.
$\Prob{\game_1(\advA)=1} \le \Prob{\game_2(\advB) = 1}$.

Since the remaining deletions do not cause errors, we can use the same argument
as in the proof of Theorem~\ref{thm:scms-erreps-th} to reduce from $\advB$ to an
adversary $\advC$ which does not make deletions at all. In~$\game_3$, we further
reduce from a counting filter to a normal Bloom filter by capping each of the
counters in the filter at 1. Since no deletions are performed, a counter
in~$\game_3(\advC)$ is nonzero if and only if the same counter
in~$\game_2(\advC)$ is nonzero. So $\QRYO$ behaves the same in~$\game_3$ as it
did in~$\game_2$, and $\Prob{\game_2(\advC)=1} \le \Prob{\game_3(\advC) = 1}$.

Note that~$\game_3$ is actually simulating an ordinary Bloom filter, since all
`counters' in the filter are restricted to the range $\bits$, there are no
deletions, and any insertions just set the corresponding bits to 1. In fact,
this game is identical to~$\game_2$ in the proof of
Theorem~\ref{thm:sbf-erreps-th} except that the adversary need only accumulate
$r'$ errors instead of $r$ errors and the threshold is $\ell+r'$ instead of
$\ell$. An identical argument allows us to reach the binomial bound of
\begin{equation}
   \Prob{\game_3(\advC)=1} \le
     \sum_{i=r'}^{q_T} \binom{q_T}{i}p_\ell^i(1-p_\ell)^{q_T-i} \,,
\end{equation}
where $p_\ell$ is now defined to be $((\ell+k+r')/m)^k$. Then the standard
Chernoff bound, along with Lemma~\ref{thm:lemma1}, yields the final bound of
\begin{equation}
   \Adv{\erreps}_{\Pi,\delta,r}(\advA) \leq
     q_R \cdot \left[\frac{q_H}{2^\lambda} + e^{r'-p_\ell q_T}\left(\frac{p_\ell q_T}{r'}\right)^{r'}\right]s.
\end{equation}
\end{proof}
}

\full{
  \begin{proof}[Proof Sketch of Theorem~\ref{thm:counting-erreps}]
  As with the proof of Theorem~\ref{thm:sbf-errep-immutable}, we derive a bound in
the \erreps1 case and then use Lemma~\ref{thm:lemma1} to move from \erreps1 to
the more general case of \erreps. Because we are in the \erreps1 case, we may
assume without loss of generality that the adversary does not call $\REVO$,
since revealing the only representation automatically prevents the adversary
from winning.

As in the proof of
Theorem~\ref{thm:sbf-errep-immutable}, we first add a
$\bad_1$ flag that gets set if the adversary ever hashes with the
actual salt used by the representation. By a very similar argument, we can
move to~$\game_1$, where the behavior is different only when the $\bad_1$ flag
is set, which happens with probability at most $q_H/2^\lambda$.

Unlike in the case of a count min-sketch, it is entirely possible for deletions
to benefit the adversary in this game. In particular, if $x$ is found to be a
false positive, deleting $x$ may cause up to $k$ elements of $\col$ to become
false negatives. We therefore move to a game~$\game_2$ where the adversary gets
credit for either a single false positive or for $k$ false negatives whenever it
finds a false positive, but where the adversary cannot delete any false
positives that it finds. We let $r' = \lfloor r/\max(\delta^+,k\delta^-)\rfloor$
represent the number of false positives the adversary has to find in~$\game_2$
in order to win. In order to prevent the adversary from being penalized by the
filter becoming full too early, we also raise the threshold from $\ell$ to
$\ell+r'$ in~$\game_2$. Now for any $\advA$ for~$\game_1$, we can construct
$\advB$ for~$\game_2$ that simulates $\advA$, keeping track of all query
responses and forwarding all oracle queries in the natural way, except that
calls to delete false positives are ignored. Since $\UPO$ never fails for
$\advB$ due to the increased threshold, and since $\advB$ gets automatic credit
for any false negatives that might have been caused by deleting false positives,
$\advB$ succeeds whenever $\advA$ does, i.e.
$\Prob{\game_1(\advA)=1} \le \Prob{\game_2(\advB) = 1}$.

Since the remaining deletions do not cause errors, we can use the same argument
as in the proof of Theorem~\ref{thm:scms-erreps-th} to reduce from $\advB$ to an
adversary $\advC$ which does not make deletions at all. In~$\game_3$, we further
reduce from a counting filter to a normal Bloom filter by capping each of the
counters in the filter at 1. Since no deletions are performed, a counter
in~$\game_3(\advC)$ is nonzero if and only if the same counter
in~$\game_2(\advC)$ is nonzero. So $\QRYO$ behaves the same in~$\game_3$ as it
did in~$\game_2$, and $\Prob{\game_2(\advC)=1} \le \Prob{\game_3(\advC) = 1}$.

Note that~$\game_3$ is actually simulating an ordinary Bloom filter, since all
`counters' in the filter are restricted to the range $\bits$, there are no
deletions, and any insertions just set the corresponding bits to 1. In fact,
this game is identical to~$\game_2$ in the proof of
Theorem~\ref{thm:sbf-erreps-th} except that the adversary need only accumulate
$r'$ errors instead of $r$ errors and the threshold is $\ell+r'$ instead of
$\ell$. We can therefore use the same argument to produce the final bound of
\begin{equation}
   \Adv{\erreps}_{\Pi,\delta,r}(\advA) \leq
     q_R \cdot \left[\frac{q_H}{2^\lambda} + e^{r'-p_\ell q_T}\left(\frac{p_\ell
     q_T}{r'}\right)^{r'}\right] \,.
\end{equation}

\end{proof}
}

\subsection{Discussion}
The results for counting filters are similar to the results for count-min
sketch, as might be expected given the similarities in terms of both the
supported updates and the structure of the representations themselves (any
count-min sketch can be transformed into a counting filter by adding all the
rows together element-wise.) In particular, we again see that counting filters
which are publicly visible cannot provide good security guarantees. This means
that counting filters intended for a security-sensitive setting should be kept
hidden from potential adversaries. Furthermore, our bound relies on
per-representation random salts and $\ell$-thresholding, so these changes should
also be taken into account when constructing secure counting filters. The size
increase of the filters is comparable to the size increase of count
min-sketches, but is distinct in that it depends on the relative weight of false
positives as opposed to false negatives. False negatives impact the bound more
than false positives due to the scaling factor of $k$ that appears in $r'$,
which indicates that applications seeking to minimize false negatives will
require larger filters than those seeking to minimize false positives. This is
distinctly different than in the non-adaptive setting, where false positives are
much more common in counting filters than false negatives, and therefore much
more relevant in determining the minimum size of the filter.

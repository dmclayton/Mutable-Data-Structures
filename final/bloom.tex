\begin{figure}
  \twoColsNoDivide{0.33}
  {
    \underline{$\Rep^R_K(\col)$}\\[2pt]
      $\salt \getsr \bits^\lambda$ \com{Choose a salt $\salt$}\\
      $\pub \gets \langle 0^m, \salt, 0\rangle$\\
      for $x \in \col$ do \\
        $\tab \pub \gets \Up^R_K(\pub, \up_x)$\\
        $\tab$if $\pub = \bot$ then return $\bot$\\
      return $\pub$
  }
  {
    \underline{$\Qry^R_K(\langle M, \salt, c \rangle,\qry_x)$}\\[2pt]
      $X \gets \bmap_m(R_K(\salt \cat x))$\\
      return $M \AND X = X$
    \\[6pt]
    \underline{$\Up^R_K(\langle M, \salt, c \rangle,\up_x)$}\\[2pt]
      if $c \geq n$ then return $\bot$\\
      $M \gets M \vee \bmap_m(R_K(\salt \cat x))$\\
      return $\langle M, \salt, c+1 \rangle$
  }
  \caption{Keyed structure $\bloom[R,n,\lambda] = (\Rep^R,\Qry^R,\Up^R)$ is used
  to define Bloom filter variants used to represent sets of at most~$n$
  elements. The parameters are a function $R: \keys\by\bits^* \to [m]^k$ and
  integers $n, \lambda \geq0$. A concrete scheme is given by a particular choice
  of parameters. The function~$\bmap_m$ is defined in Section~\ref{sec:prelims}.
  %
  }
  \label{fig:bf-def}
\end{figure}
In this section we consider two classes of Bloom filters, each employing a
different strategy to determine when the filter reaches full capacity. The first
class is specified in Figure~\ref{fig:bf-def}. This class of $n$-capped filters
captures the classical setting in which the filter is used to represent some
fixed number of elements $n\geq0$. Our construction $\bloom[R,n,\lambda] =
(\Rep^R,\Qry^R,\Up^R)$ has two additional parameters besides the cap: a
function~$R:\keys\by\bits^*\to[m]^k$ and the \emph{salt length}~$\lambda\geq0$.
%
Let $H:\bits^*\to[m]^k$ be a hash function and let $\ell, n, \lambda\geq0$ be
integers.
%
The standard Bloom filter is the structure $\BF[H,n] =
\bloom[\id^H,n,0]$, which we will term the \emph{basic} Bloom filter. It
has no key (the key space of $\id^H$ is $\{\emptystr\}$, see
Section~\ref{sec:prelims}) and does not use a salt.
%
The \emph{salted} Bloom filter $\SBF[H,n,\lambda] =
\bloom[\id^H,n,\lambda]$ is the same except that it allows a non-empty salt.
%
Finally, we consider a salted variant that uses a PRF instead of a hash
function. The \emph{keyed} Bloom filter $\KBF[F,n,\lambda]$ is the
structure $\bloom[F,n,\lambda]$, where $F:\keys\by\bits^*\to[m]^k$ is a
PRF.
%
Note that the basic and salted BFs have key spaces $\{\emptystr\}$ and the keyed
BF has key space~$\keys$.

In this section, we will show that the basic Bloom filter construction
$\BF[H,n]$ is insecure in our setting. This is because it allows the adversary to
make an offline attack that has a high probability of success while using a
minimal number of queries. In the immutable setting, where the adversary is
constrained to never use the $\UPO$ oracle, i.e. $q_U = 0$, it suffices to use
the $\SBF$ construction in order to provide a security guarantee in either the
public-representation or private-representation settings. However, in the case
where we allow $q_U > 0$ so that the adversary can make updates, we will find
that $\SBF$ is only secure in the \erreps\ setting. To provide \errep\ security
when updates are needed, $\KBF$ must be used instead.

At the end of the section, we discuss the second class of filters that we call the
\emph{$\ell$-thresholded}. Instead of rejecting updates after a pre-determined
number of elements are added to the set, a thresholded filter is deemed full
once at least $\ell\geq0$ bits of the filter are set.
%
In the usual, non-adaptive setting, this strategy performs
similarly to the standard Bloom filter, but we find that a filter threshold
allows us to obtain better bounds.
%
We will demonstrate this for salted BFs in the \erreps\ setting.
%
\ignore{[...] and this construction has the additional advantage of not
requiring a separate counter to keep track of the number of elements in the
filter.}


\heading{Non-adaptive false-positive probability}
Let~$\rho:\bits^*\to[m]^k$ be a function, $\lambda\geq0$ be an integer, and
define $\bloom[\id^\rho,n,\lambda] = (\Rep^\rho, \Qry^\rho, \Up^\rho)$ as in
Figure~\ref{fig:bf-def}. (Note the mild abuse of notation by which we write
``$\rho$'' instead of ``$\id^\rho$''.)
%
Let $\setS\subseteq\bits^*$ be a set of length~$n$. We define the non-adaptive, false positive
probability for Bloom filters as
\begin{equation}\label{eq:bf-fp-prob}
  \begin{aligned}
    P_{k,m}(n) =
      \Pr\big[&\rho \getsr \Func(\bits^*,[m]^k);
              \pub \getsr \Rep^\rho(\setS); \\
              &x \getsr \bits^*\setminus\setS: \Qry^\rho(\pub, \qry_x) = 1 \given \pub \ne \bot
      \big] \,.
  \end{aligned}
\end{equation}
%
%(Note that, since the probability is conditioned on the event that
%$\pub\ne\bot$, this quantity is the same for both classes of filter.)
%
That is, $P_{k,m}(n)$ is the probability that some~$x$ is a false positive for
the representation of some~$\setS$ for which $|\setX|=n$ and $x\not\in\setS$,
when a random function is used for hashing. Because of the randomization
provided by~$\rho$, this probability is independent of~$\setS$ and~$x$.
%
Finding a tight, concrete upper bound for $P_{k,m}(n)$ has proven challenging,
but we do understand its asymptotic behavior. Kirsch and
Mitzenmacher~\cite{kirsch2006less} prove that, for certain choices of~$k$
and~$m$ as functions of~$n$, it holds that
$
  P_{k,m}(n) = \lim_{n\goesto\infty} (1-e^{-kn/m})^k \,.
$
%
Moreover, they demonstrate via simulation that this is a very good approximation
of the false positive probability.
%
In lieu of a concrete upper bound, we will refer to $P_{k,m}(n)$ as defined in
Equation~(\ref{eq:bf-fp-prob}) in the remainder.

\heading{Error function for set-membership queries}
%
Throughout this section we will use the error function~$\delta$ defined as
\begin{equation}
  \delta(x, y) =
  \begin{cases}
    0 & \text{if}\ x=y \\
    1 & \text{otherwise.}
  \end{cases}
\end{equation}
This simply indicates whether the query result matched the correct response.

\subsection{Insecurity of unsalted BFs}\label{sec:bad-bfs}
The performance of basic Bloom filters is well-understood, assuming the choice of
set~$\setS$ being represented is independent of the choice of hash function. When
this assumption is violated, however, their performance can be substantially
degraded~\cite{gerbet2015power}.
%
Here we show that, even when we (optimistically) model the hash function as a
random oracle, basic BFs cannot achieve security in our setting.
%
The basic Bloom filter has no salt and no secret key.
Let $H:\bits^*\to[m]^k$ be a function, fix $n\geq0$, and let
$\Pi = \BF[H,n]=(\Rep^H,\Qry^H,\Up^H)$ as defined above.
%
With no per-representation randomness and no secret key to be concealed from the
adversary, there is no difference between \errep\ and \erreps\ security, as the
adversary can easily compute the representation of any set for itself. This
ability of the adversary to reconstruct the set without making queries allows
for various attacks that badly harm the accuracy of the filter.

\heading{Pollution attacks}
Gerbet \etal~\cite{gerbet2015power} provide the following example of an attack
setting and a potential attack against Bloom filters.
Suppose the adversary is interacting with a system representing a dataset
with~$\Pi$ and that it is able to choose some fraction of the input data.  For
example, consider a web crawler which performs a ``crawl'' of webpages~\cite{mapreduce}, following
the links on each page it visits in order to index, archive, or otherwise
analyze websites. In order to keep track of the set of webpages which have
already been visited during a crawl, some crawlers use a Bloom filter which is
updated to include each new page the crawler visits.
Suppose the adversary controls at least one such webpage along the crawl's path
and wishes to deny the spider access to a different webpage, the `target
webpage'. The adversary can choose the links present on its own webpage, which
will cause the spider to visit the chosen webpages and set the corresponding
bits of its Bloom filter to~$1$. If those links are chosen in such a way that they
produce a false positive for the target webpage, the spider will then
erroneously believe it has already visited the target webpage. The target
webpage will therefore never be visited during the spider's crawl.

In cases where the adversary is able to control at least some of the filter inputs,
Gerbet \etal describe an attack where the adversary chooses a set of
inputs that maximizes the number of 1s in the filter. This strategy is especially
effective when the structure of the hash function is known to the adversary. In
particular, as long as the choice of hash function and any associated parameters
are public, the adversary can compute the hash function on its own in order to
determine which choices will set the maximum number of bits to 1, or which
choices will set certain target bits to 1 in order to cause specific false
positives. They show that with $m = 3200$ and $k = 4$, the adversary can double
the false positive rate if they control 200 out of a total of $n = 600$
insertions, under the assumption that~$H$ is known to and
computable by the attacker.

Gerbet \etal suggest various ways to mitigate pollution attacks, such as choosing
the parameters~$k, m, n$ so that even if a pollution attack
occurs, the false positive rate is kept below some threshold of acceptability.
This strategy is  potentially viable, but may significantly increase the amount
of memory required to store the data structure.  The bounds we provide show how
the parameters of a filter can be tweaked to keep the error rate low not just in
the presence of this specific type of attack, but in the presence of any
adversary covered by our more general attack model; doing so, however, will
require altering the structure.

Gerbet \etal also discuss the possibility of using a secretly-keyed
hash function. In the attack model they consider, where representations are kept
private indefinitely, this suffices to prevent the pollution attack they
describe. However, under the more general attack models where the representation
may eventually be recovered (in the private-representation setting via~$\REVO$)
or is public, simply using a PRF \emph{without per-representation randomness}
does not suffice for security in our setting.

\heading{Target-set coverage attacks}
%
Of course, exhibiting a high false positive rate is not the only way a Bloom
filter might fail to be correct. In particular, it would be undesirable if the
filter were consistently incorrect on a \emph{particular set of inputs}. Rather
than pollute the filter, the adversary's goal might be to craft a set of
legitimate looking inputs that cover some disjoint target set of inputs.
%
This type of attack is nicely captured by our adversarial model.
%
In a \emph{target-set coverage attack}, the adversary is given a small target set
$\setT\subseteq\bits^*$ and searches for a cover set $\setR\subseteq\bits^*$
such that $\Qry^H(\Rep^H(\setR),\qry_x)=1$ for each $x\in\setT$.
%
Once a suitable cover set is found, the adversary queries $\REPO(\setR)$. Then
for each $x\in\setT$, it asks $\QRYO(\qry_x)$, achieving a score of $r = |\setT|$.

This \erreps1 attack succeeds with probability~$1$ assuming a covering set can
be found.  If $|\setT| \leq |\setR|$, then such a set exists; but finding it may be
computationally infeasible, depending on the size of the cover set, the size of
the target set, and the parameters of the Bloom filter.
%
\ignore{
In Appendix~\ref{app:unsalted-attack} we demonstrate that target-set coverage
attacks are feasible for practical BF parameters. We do so by simulating the
attack when~$H$ is a random function (i.e., for each distinct input we choose
$k$ integers from $[m]$ at random) for typical choices of $k$, $m$, and~$n$.}%
%
\todo{CP}{Decide whether we want to include the simualtion code. Given how
simple the attack is, I think we can get away with not including it.}

Fix inetgers $\ell, n, \lambda \geq0$, let $H:\bits^*\to[m]^k$ be a function,  and let
$\Pi = \BF[H,t,\lambda]$ as specified in Figure~\ref{fig:bf-def}.
%
The possibility of pre-computing the structure in the \errep1 games yields the
following attack. Given a set $\setT\subseteq\bits^*$ of target queries, the
adversary searches for a set $\setS\subseteq\bits^*$ such that
$\Qry^H(\Rep^H(\setS), x) = 1$ for all $x \in \setT$. We call such a set a
\emph{cover} set.
%
Once a suitable~$\setT$ is found, the adversary queries $\REPO(\setS)$ followed
by $\QRYO(x)$ for each $x\in\setT$.
%
Assuming $|\setT| \geq r$, where~$r$ is the error capacity, the attack succeeds
with probability~$1$.
%
In this appendix we describe a strategy for finding a covering set and evalute
its performance in terms of success rate and computational cost. We will
(conservatively) model~$H$ as a random oracle.

Fix a set of~$s$ \emph{test queries}. Our goal is to find a set~$\col$ of at
most~$n \leq s$ test queries that covers~$\setT$.
%
For each test query~$x$, we ompute $X = \bmap_m(\HASHO(x))$. If we have the the
resources to compute $X_1 \OR \cdots \OR X_n$ for each of the ${s}\choose{n}$
size-$n$ subsets of targets, twe will eventually find a suitable covering set if
such a set exists.
%
The query complexity of this appraoch is modest; we need to make $s+r$ queries
to~$\HASHO$ ($s$ for the test set, $r$ for the target set), one query
to~$\REPO$, and~$r$ queries to~$\QRYO$. However, checking $\binom{s}{n}$ sets
would be infeasible, even for modest choices of $k$, $m$, and $n$. Observe,
hwoever, that there is a lot of sub-structure to exploit in the search.
%
In the remainder, we describe a \emph{heuristic} strategy for finding a
satisfying set (if it exists) in which we need only check about $O(kr)$ sets on
average.

Let $\setT = \{x_1, \ldots, x_s\}$ be the set of test queries, and let $\setR =
\{x_{s+1}, \ldots, x_{s+r}\}$ be the set of target queries.
%
We construct a tree whose vertices are labeled with subsets of~$[s]$ as follows.
%
Let~$\emptyset$ be the root.
%
For each node $I$ and each $w \in [s] \setminus I$, if $\setlen{I} < n$, then let $I
\union \{ w \}$ be a child of~$I$.
%
The new attack works as follows: traverse the tree depth-first, beginning
at~$\emptyset$, until a vertex~$I$ is reached such that each element of $\setR$ is
a false positive for the representation of $\setS_I = \{ x_i : i \in I \}
\subseteq \setT$.
%
Then for every child~$J$ of $I$, the elements of~$\setR$ are false positives
for~$\setS_J$. The adversary may choose any one of these as its query
to~$\REPO$.

The tree has $\binom{s}{n}$ leaves, and it will traverse the entire tree if
there is no solution. Hence, the worst-case runtime is the same as before.
However, we can reduce the search space dramatically in two ways.
%
First, if there is no solution, then often this can be determined without
traversing the tree. Let $M_\setT$ and $M_\setR$ denote the filters
corresponding to the set of test and target queries, respectively.  If for some
$i \in [m]$, the $i$-th bit of $M_\setR$ is set, but the $i$-th bit of $M_\setT$
is not set, then it is immediate that there is no covering set.
%
The second is a greedy heuristic for eliminating branches of the search tree.
The idea is that we only take a branch if it results in covering an additional
bit in~$M_\setR$. More precisely, for each child $J$ of~$I$, we do as follows:
if there is a bit~$i$ such that the $i$-th bit of $M_{\col_J}$ and $M_{\setR}$
are set, but the $i$-th bit of~$M_{\col_I}$ is \emph{not set}, then the
branch~$J$ is taken; otherwise it is not.
%
This optimization results in a heuristic attack, since the search may miss the
optimal solution.

We implemented this attack and evaluated its performance.
Figure~\ref{fig:bf-correct-attk-sim} shows the success rate and average number
of sets checked for a number of simulations and various parameters.
%
Unsurprisingly, the success rate decreases as we increase the number of filter
bits (top-left of Figure~\ref{fig:bf-correct-attk-sim}); however, with $k=4$,
$m=1024$, $n=100$, and $r=1$, a test set of just $512$ elements suffices for a
success rate of nearly $60\%$ (top-right). It is also worth noting that
increasing the error parameter only slightly decreases the success rate
(bottom-left). These results show that even for very pessimistic parameter
choices, the basic Bloom filter is not secure in the \errep1\ sense.
%
Finally, we find that the average number of sets that were tested is about
$O(kr)$ within all of the parameter ranges we tested.

%
\newcommand{\mydot}{{\scriptsize\ding{108}}}
\newcommand{\myex}{{\scriptsize\ding{53}}}
\begin{SCfigure*}
\centering
\includegraphics[page=1,scale=0.675]{fig/bf-correct-attk-sim}
\caption{
  %
  Success rate and average-case time complexity for the optimized attack on classic
  Bloom filter.
  %
  Each plot shows the success rate (\mydot), as well as the average number
  of tested sets (\myex), for 1000 executions of
  the attack on simulated inputs.
  %
  Default parameters are $k=4$, $m=2^{10}$, $n=100$, $r=1$, and $s=2^{10}$.
  In each plot, one of these parameters is varied.
}
\label{fig:bf-correct-attk-sim}
\end{SCfigure*}

\ignore{
\heading{Privacy.}
%
\ignore{
  \begin{figure}[t]
    \boxTwoCols{0.48}
    {
      \algorithmv{$\dist[n,\mu]^H$}\\
        $\col \gets \emptyset$;
        for $i\gets 1$ to $n$ do\\
        \tab for $j \gets 1$ to $\mu$ do $\vv_j \gets H(\str(i,j))$\\
        \tab $\col \gets \col \multiunion \{ \str(\vv) \}$\\
        return $\col$
    }
    {
      \adversaryv{$\advA^H$}\\
        for $j \gets 1$ to $\mu$ do $\vv_j \gets H(\str(1,j))$\\
        return $\str(\vv)$\\
    }
    \caption{Distribution~$\dist[n,\mu]$, where $n,\mu \geq0$ are integers, and
    adversary~$\advA$ for attack on ow-rep privacy of~$\BF[k,m,n]$.}
    \label{fig:bf-ow-rep}
  \end{figure}
}
%
The structure~$\BF[k,m,n]$ does not meet our strong notion of \ssrep privacy.
(Indeed, as noted earlier, no keyless data structure can satisfy that
definition.)
%
Even worse, we cannot achieve \owrep privacy for inputs that depend on the
random oracle.
%
Recall that, in the ROM, the distribution~$\dist$ may depend on the random
oracle~$H$.
%
This being the case, we can easily exhibit a distribution with high min-entropy
against which an attacker gets advantage~$1$.
%
Let $\dist$ be a distribution over sets of strings,
parameterized by integers $n,\mu \geq 0$, defined as follows:
%
define a sequence of vectors $\vv_1, \ldots, \vv_n$ so that $\vv_{i}[j] =
H(\str(i,j))$ for each $i\in[n]$ and $j \in [\mu]$.
%
Let $\col = \{\str(\vv_1), \ldots, \str(\vv_n)\}$.
%
Since each query to~$H$ is distinct, the probability that $x \in \col$
for any $x \in \bits^*$ is at most $m^{-\mu}$. Therefore, the point-wise
min-entropy of the distribution is $\mu \cdot \log m$.
%
Now, the attacker can easily compute any $x \in \col$, given access
to~$H$, despite the fact that~$\dist$ has high min-entropy.

Though this attack is theoretical and not particularly interesting in any
practical sense, it is easy to see that \emph{rainbow
tables}~\cite{oechslin2003rainbow}, a practical technique for breaking password
hashes, could be used to deduce the contents.

These negative results for classic BFs in the random oracle model are
reminiscent of the random oracle model \emph{with auxiliary
input}~\cite{unruh2007romaux,dodis2017filling}. In this setting, the adversary
is given a bounded amount of information about the random oracle. This allows us
to model offline computation performed by the adversary before beginning its
attack. Both attacks exploit the fact that the adversary is given the random
oracle \emph{before} choosing its set.
}

\if{0}{
  \heading{Security of RO-independent collections. }
  %
  \cpnote{I vote for cutting this heading entirely. It feels redundant, and the
  restrictions we need to make on the adversary (no stage-1 RO queries) do not
  seem realistic.}
  %
  Both the correctness and privacy attack exploit the fact that the collection may
  depend on the random oracle.
  %
  We remark that, in both settings, security is achievable for collections chosen
  independently of the RO.
  %For correctness, we do not allow the adversary to make
  %queries to~$H$ in its first stage:
  %
  \begin{theorem}[$\BF$ is correct for RO-independent collections]\label{thm:bf-correct*}
    Fix integers $k,m,n,r\ge0$, let $H \colon \bits^* \to [m]$ be a function, modeled as a
    random oracle, and let $\struct_\bloom = \BF[k,m,n]$.
    %
    Let~$\advA$ be an adversary making~$0$ queries to~$H$ in its first stage, and
    $q_{2} \geq 0$ queries to~$H$ and $q_T \geq 0$ queries to~$\QRYO$ in its
    second stage, such that $q_T + q_2 \geq r$.
    %
    Then
    \[
      \Adv{\errep}_{\struct_\bloom,r}(\advA) \leq
      {\dbinom{q_T + q_{2}}{r}} \left( (1-e^{-kn/m})^k + O(1/m) \right)^r\,,
    \]
    \cpnote{We can't use $\Adv{\errep}_{\struct,r}(t,q_T,q_H)$ notation without
    modification.}\tsnote{Added footnote}\cpnote{Muted footnote}
  \end{theorem}
  %
  To be clear, all of the adversary's RO queries are made in its second
  stage.
  %
  The bound reflects the probability that exactly~$r$ of the adversary's stage-2
  queries are a false positive for the set's representation.
  %
  The proof is technically similar to Theorem~\ref{thm:bf-salt-correct}, and so we
  omit the details.
  %

  For privacy, we must restrict the distribution from making any RO queries.  The
  proof is technically similar to Theorem~\ref{thm:bf-salt-ow-rep}, so we again
  omit the details.

  \begin{theorem}[$\BF$ is ow-rep for RO-independent collections]\label{thm:bf-ow-rep*}
    Fix $k,m,n\geq 0$, let $H \colon \bits^* \to [m]$ be a function,
    and let $\struct_\bloom = \BF[k,m,n]$.
    %
    Then for any $t,q_2,\mu \geq 0$, it holds that
    \[
      \Adv{\owrep}_{\struct_\bloom,\mu}(t,0,q_2) \leq
        \frac{(q_2+1)n}{2^\mu} \,,
    \]
    where~$H$ is modeled as a random oracle.
  \end{theorem}
%
}\fi


\ignore{
The key to pollution attacks and target-set coverage attacks is that the
adversary can compute the representation of the set on its own. In the remainder
of this section, we examine ways of enhancing the basic BF structure so that it
avoids this pitfall.
}

\subsection{Salted BFs in the (im)mutable setting}\label{sec:sbf}
%
Here we consider the correctness of Bloom filters when the hashed input is
prepended with a salt.
%
Fix $H:\bits^*\to[m]^k$ and $n,\lambda\geq0$ and let
$\Pi = \SBF[H,n,\lambda]$.

If the adversary can update the representation via~$\UPO$, then it can perform
an \errep1 attack against~$\Pi$ that is closely related to the attacks in the
previous section.  The adversary calls $\REPO(\emptyset)$, getting an empty
filter and the salt in response.  It may then use the salt to construct
representations on its own just as described in the target-set coverage attack.
%
The works because the adversary can test for errors on its own because it knows
the salt.  In practice, an adversary may not be able to perform this exact
attack, since even in the streaming setting it is possible that the salt is not
immediately revealed to the adversary. However, as soon as the adversary does
learn the salt, it can immediately launch a pollution attack against the filter,
without having to make any queries directly to the filter.
%
\ignore{Just as in the immutable setting the adversary can exploit its knowledge
of the hash functions to find false positives without needing to make queries,
in the mutable and public-representation setting the adversary can identically
exploit its knowledge of the hash functions \textit{and salt} to find false
positives without needing to make queries.
}

Without the ability to insert elements even after the salt has been seen, the
above attack fails. Indeed, when we restrict ourselves to the immutable setting,
we can prove the following.
%
\begin{theorem}[Immutable \errep\ security of salted BFs]\label{thm:sbf-errep-immutable}
  Let $p=P_{k,m}(n)$.  For all integers $q_R, q_T, q_H, r, t \geq 0$ it holds
  that
  \begin{equation*}
    \begin{aligned}
            \Adv{\errep}_{\Pi,\delta,r}(t,\,&q_R,q_T,0,q_H) \leq q_R \cdot \left[\frac{q_H}{2^\lambda} +
        \left(\frac{pq}{r}\right)^re^{r-pq}\right] \,,
    \end{aligned}
  \end{equation*}
  where $H$ is modeled as a random oracle, $q = q_T + q_H$, and
  $r > pq$.
\end{theorem}
We consider only the case of $r > pq$ because $pq$ is the expected number of
false positives obtained by an adversary that simply uses its knowledge of the
salt (after the representation is created) to guess as many random elements as
possible. Because this simple adversary can get $pq$ successes on average, we
can only hope to provide good security bounds against arbitrary adversaries in
the case that $r > pq$.

Before giving the proof, let us take a moment to unpack the result a bit.  The
bound can be broken down into three main components. The factor
of~$q_R$ means that the bound we can prove is weakened somewhat when a number of
representations are observed by the adversary. (In Section~\ref{sec:bf-thresh},
we will show that we can do better by thresholding rather than capping.) The $q_H/2^\lambda$ term
corresponds to the probability of the adversary guessing the salt before the
representation is constructed, but this will be negligible as long as $\lambda$
is chosen to be sufficiently large (say, $\lambda=128$). The final, messier term
comes from applying a Chernoff bound to the non-adaptive adversary's probability
of succeeding in the experiment given $q = q_H+q_T$ guesses.
%
By way of clarifying the performance of our bound, we have plotted the last
component for various parameters of interest. Let
%
\begin{equation}\label{eq:zeta}
  \zeta_{k,m,n}(q,r) = \left(\frac{p^*q}{r}\right)^re^{r-p^*q}
\end{equation}
%
where
$
  p^* = (1-e^{-kn/m})^k \,,
$
the approximation of the non-adaptive false positive probability given by Kirsch
and Mitzenmacher~\cite{kirsch2006less}.
%
Figure~\ref{fig:bf-bound} shows values
of~$\zeta_{k,m,n}(q,r)$ for varying~$m$.
%
What this plot shows is that, for a given error capacity~$r$, once a certain lower
bound on the filter size is reached, the $\zeta$ term decreases quite quickly.
Moreover, the rate at which~$\zeta$ decreases scales nicely with the error
capacity.  For example, if one is willing to tolerate up to~$r=10$ false
positives for a filter representing $n=100$ elements, then picking a filter
length of $3$ kilobytes is sufficient to ensure that observing~$10$ false
positives occurs with probability less than $2^{-17}$, even when the adversary
can make~$q =2^{64}$ $\HASHO$ or $\QRYO$ queries.

We concede that requiring a~$3$KB for a filter a set of $100$ elements may be
prohibitive in some applications. We would require a substantially smaller
filter for smaller~$q$, but unfortunately, a query complexity of $q=2^{64}$ in
the \errep\ setting is quite realistic, since the attack can be carried out
offline.
%
In the \erreps\ setting, or in the \errep\ setting when we use a PRF instead of a
hash function, the adversary's attack is largely \emph{online}, rendering
the~$q$ term quite conservative. In these settings, a significantly smaller
filter will do. For example, if we assume that an adversary making an online
attack will make no more than $q=2^{32}$ online queries, we get the results seen
in Figure~\ref{fig:bf-bound-online}. Beyond this, if a larger error rate is acceptable, the filters can
again be made substantially smaller.

\begin{figure}
  \hspace*{-10pt}
  \includegraphics[scale=0.9]{fig/bf-bound}
  \includegraphics[scale=0.9]{fig/bf-bound-big}
  \includegraphics[scale=0.9]{fig/bf-bound-online}
  \includegraphics[scale=0.9]{fig/bf-bound-big-online}
  \caption{
    The value of $\zeta_{k,m,n}(q,r)$ (Equation~(\ref{eq:zeta})) for $q=2^{64}$ (top) or $q = 2^{32}$,
    $k=16$, $n=100$ (left) or $n=10^9$ (right), varying values of~$r$ (one line per $r$-value) and filter
    length~$m$ (the x-axis).  Note the log-2 scale on the y-axis.
  }
  \label{fig:bf-bound}
\end{figure}

\proc{
\begin{proof}[Proof Sketch of Theorem~\ref{thm:sbf-errep-immutable}]
  \begin{figure*}
\threeColsOneDivideUnbalanced{0.40}{0.27}{0.27}
{
  \vspace{-7pt}
  \experimentv{$\game_{0}(\advB)$}
      \hfill \diffplus{$\game_1$}\\[2pt]
    $M^* \gets \bot$;
    $\salt^* \getsr \bits^\lambda$\\
    $\advB^{\REPO,\QRYO,\HASHO_1}$;
    return $\big[\sum_x \err[x] \geq r\big]$
  \\[6pt]
  \oraclev{$\REPO(\col)$}\\[2pt]
    $M^* \gets \bigvee_{x \in \col} \bmap_m(\HASHO_2(\salt^* \cat x))$;
    $\setS^* \gets \col$;
    return $\langle M^*, Z^* \rangle$
  \\[6pt]
  \oraclev{$\QRYO(\qry_x)$}\\[2pt]
    $X \gets \bmap_m(\HASHO_3(\salt^* \cat x))$;
    $a \gets X = M^* \AND X$\\
    if $\err[x] < \delta(a,\qry_x(\col^*))$ then
          $\err[x] \gets \delta(a,\qry_x(\col^*))$\\
    return $a$
  \\[6pt]
  \oraclev{$\HASHO_c(\salt \cat x)$}\\[2pt]
    $\vv \getsr [m]^k$\\
    if $M^*=\bot$ and $\salt=\salt^*$ and $c=1$ then \com{Caller is~$\advB$}\\
    \tab $\bad_1 \gets 1$; \diffplus{return $\vv$}\\
    if $T[Z,x] = \bot$ then $\vv \gets T[Z,x]$\\
    $T[Z,x] \gets \vv$; return $\vv$
}
{
  \vspace{-2pt}
  \oraclev{$\HASHO_c(\salt \cat x)$}\\[2pt]
    $\vv \getsr [m]^k$\\
    if $M^*=\bot$ and $\salt=\salt^*$ and $c=1$ then\\
    \tab $\bad_1 \gets 1$; return $\vv$\\
    if $T[Z,x] = \bot$ then $\vv \gets T[Z,z]$\\
    $T[Z,x] \gets \vv$\\[2pt]
    \diffplusbox{
    \com{Caller is~$\advB$ or $\QRYO$}\\
    if $c=1$ or $c=3$ then\\
    \tab if $\salt \ne \salt^*$  then return $\vv$\\
    \tab $\Ans[x] \gets \bmap_m(\vv) = M^* \AND \bmap_m(\vv)$\\
    \tab if $\err[x] < \delta(\Ans[x],\qry_x(\col^*))$ then
    \tab\tab $\err[x] \gets \delta(\Ans[x],\qry_x(\col^*))$
    }
    return $\vv$
}
{
  \vspace{-7pt}
  \oraclev{$\QRYO(\qry_x)$}\
      \hfill \diffminus{$\game_1$} \diffplus{$\game_2$}\\[2pt]
    \diffminusbox{%
      $X \gets \bmap_m(\HASHO_3(\salt^* \cat x))$\\
      $a \gets X = M^* \AND X$\\
      if $\err[x] < \delta(a,\qry_x(\col^*))$ then\\
      \tab $\err[x] \gets \delta(a,\qry_x(\col^*))$
    }\\[2pt]
    \diffplusbox{
      $\HASHO_3(Z^* \cat x)$\\
      $a \gets \Ans[x]$
    }
    return $a$
}
\caption{Games 0, 1, and 2 for proof of Theorem~\ref{thm:sbf-errep-immutable}.}
\label{fig:sbf-errep-immutable/games}
\end{figure*}

We will use the following lemma for keyless structures, which is proved in
Appendix~\ref{sec:keyless-proof}.

\begin{lemma}\label{thm:lemma1}
  For every $q_R, q_T, q_U, q_H, r, t \geq 0$ and keyless structure~$\Gamma$ it
  holds that
  \begin{eqnarray*}
    \begin{aligned}
      \Adv{\errep}_{\Gamma,\delta,r}(t,\,&q_R, q_T, q_U, q_H) \leq \\
      & q_R \cdot \Adv{\errep1}_{\Gamma,\delta,r}(O(t), q_T, q_U, q_H) \,,
    \end{aligned}
  \end{eqnarray*}
\end{lemma}
%
\noindent
The proof is by a fairly straightforward hybrid argument. Because~$\Gamma$ is
keyless, in the reduction we simulate $q_R-1$ of the calls to $\REPO$ experiment
and use our own oracles for the remaining query. The best we can do with this
strategy is to ``guess'' which representation the \errep\ adversary will use in
its attack, which results in the~$q_R$ factor in the bound.
%
We defer the full details to Appendix~\ref{app:deferred}

Let $\advA$ be an \errep\ adversary making~$1$ query to~$\REPO$, $q_T$ queries
to $\QRYO$, $0$ queries to $\UPO$, and $q_H$ queries to the random
oracle~$\HASHO$.
%
We make the following assumptions, all of which are without loss of generality.
%
First, all of~$\advA$'s $\QRYO$ queries follow its $\REPO$ query.
%
Second, we assume that $x\not\in\setS$ for all queries $\qry_x$ to $\QRYO$,
where~$\setS$ was the input to~$\advA$'s $\REPO$ query. This is without loss
because Bloom filters admit false positives, but not false negatives
%
Third, we we assume that $|\setS| \leq n$; this is without loss because
otherwise~$\REPO$ outputs~$\bot$ and~$\advA$ gets no advantage.
%
Fourth, we assume that all of~$\advA$'s $\HASHO$ queries are of the form $Z\cat
x$, where $|Z| = \lambda$.

We begin with a game-playing argument~\cite{bellare2006triple}, then obtain the
final bound via applicaition of Lemma~\ref{thm:lemma1}.
%
The high-level goal is to rewrite the game so that the probability that one
of~$\advA$'s queries runs up the score is precisely non-adaptive false positive
probability.
%
In other words, our goal is to transistion into a setting in which the Bloom
filter output by~$\REPO$ is independent of the outcome of~$\advA$'s other
queries.

Consider the game~$\game_0(\advB)$ defined in
Figure~\ref{fig:sbf-errep-immutable/games}. It is similar to the \errep\
experiment when executed with~$\advA$, $\Pi$, $\delta$, and~$r$, but the
pseudocode has been simplified to clarify our argumen. Indeed, it is not
difficult to see that for every~$\advA$ there exists an adversary~$\advB$ such
that
\begin{equation}
  \Adv{\errep}_{\Pi,\delta,r}(\advA) \leq \Prob{\game_0(\advB) = 1}
\end{equation}
and~$\advB$ has the same query resources as~$\advA$.
%
Adversary~$\advB$ executes~$\advA$, forwarding~$\advA$'s oracle queries
to its own oracles in the natural way.

Observe that in game~$\game_0$ the salt used for the representation of~$\setS^*$
is generated prior to executing~$\advB$. Game~$\game_1$ is identical
to~$\game_0$ until the flag~$\bad_1$ gets set by oracle~$\HASHO$. This occurs
if~$\advB$ asks $\HASHO_1(\salt^* \cat x)$, where~$\salt^*$ is the salt generated
at the beginning of the game, and it has not yet called $\QRYO$ (i.e.,
$M^*=\bot$).
%
By the Fundamental Lemma of Game Playing~\cite{bellare2006triple} it follows
that
%
\begin{eqnarray}
  \Prob{\game_0(\advB)=1} &\leq&
    \Prob{\game_1(\advB)=1} + \Prob{\game_1(\advB) \sets \bad_1}\\
  &\leq&
    \Prob{\game_1(\advB)=1} + q_H/2^\lambda \,.
\end{eqnarray}
%
Note that in $\game_1$, the value of~$M^*$ is independent of~$\advB$'s
$\HASHO_1$ queries. In particular, the probability that some bit of~$M^*$ is set
is independent of the choices of~$\advB$.

In game $\game_2$ the $\HASHO$ and $\QRYO$ oracles have been rewritten so that
the winning-condition is computed by $\HASHO$ instead of $\QRYO$. The former
oracle maintains a set~$\Ans$ such that $\Ans[x] = \Qry^{\HASHO_3}(M^*, \qry_x)$ for
each query $\salt^* \cat x$; on input of $\qry_x$, oracle~$\QRYO$ simply runs
$\HASHO_3(\salt^* \cat x)$ and returns $\Ans[x]$.
%
We are effectively giving the adverseary credit for RO queries that result in
false positives for the representation of~$\setS^*$, but which it does not
explicitly ask of the~$\QRYO$. Because~$\advB$'s advantage is at least that
of~$\advA$'s, it holds that
%
\begin{equation}
  \Prob{\game_1(\advB)=1} \leq \Prob{\game_2(\advB)=1} \,.
\end{equation}

We now consider $\Prob{\game_2(\advB)=1}$.
%
Let $\setX$ be the set $\{ x \in \bits^* : \Ans[x] \ne \bot \}$ and $\setT = \{x
\in\setX: \Ans[x] = 1\}$, where $\Ans$ is at is defined when~$\advB$ halts. We
will call~$\setX$ the set of attempts and~$\setT$ the set of false positives.
%
Note that $\setX\intersection\setS^*=\emptyset$ and
$|\setX| \leq q_H + q_T$.
%
Hence, the probability that~$\game_2(\advB)=1$ is equal to the probability
that~$|\setT| \geq r$.

For each $x\in\setX$, let $T(x)$ denote the event that $x\in\setT$.
%
In the random oracle model for~$H$, the set of random random variables $T(x)$
for each $x\in X$ are independently and identically distributed.
%
Hence, the probability that~$\advB$ succeeds is binomially distributed:
%
\begin{equation}
   \Prob{\game_2(\advB)=1} = \Prob{ |\setT| \geq r } =
     \sum_{i=r}^{q} \binom{q}{i}p^i(1-p)^{q-i} \,,
\end{equation}
%
where $q \leq q_H + q_T$ and $p = \Pr[T(x)=1]$. Here we can apply a Chernoff
bound which states that, for any $\delta > 0$,
%
\begin{equation}
  \Prob{X \geq (1+\delta)\mu} < \left(\frac{e^\delta}{(1+\delta)^{1+\delta}}\right)^\mu
\end{equation}
%
We set $\delta = r\mu^{-1}-1$ and note that $\mu = pq$.
This yields
%
\begin{equation}
 \Prob{|\setT| \geq r} < \left(\frac{e^{r\mu^{-1}-1}}{(r\mu^{-1})^{r\mu^{-1}}}\right)^\mu = \left(\frac{e^{r-\mu}}{(r\mu^{-1})^r}\right) = e^{r-pq}\left(\frac{pq}{r}\right)^r
\end{equation}
%
So we have
%
\begin{equation}
  \Adv{\errep}_{\Pi,\delta,r}(\advA) < \frac{q_H}{2^\lambda} + \left(\frac{pq}{r}\right)^re^{r-pq}
\end{equation}
%
Applying Lemma~\ref{thm:lemma1} to move from the single-representation case to the
general case, we get our final bound of
\begin{equation}
  \Adv{\errep}_{\Pi,\delta,r}(\advA) \leq
    q_R \cdot \left[
      \frac{q_H}{2^\lambda} +
      \left(\frac{pq}{r}\right)^re^{r-pq}
    \right] \,.
\end{equation}

\end{proof}
}
\full{
\begin{proof}[Proof of Theorem~\ref{thm:sbf-errep-immutable}]
  \begin{figure*}
\threeColsOneDivideUnbalanced{0.40}{0.27}{0.27}
{
  \vspace{-7pt}
  \experimentv{$\game_{0}(\advB)$}
      \hfill \diffplus{$\game_1$}\\[2pt]
    $M^* \gets \bot$;
    $\salt^* \getsr \bits^\lambda$\\
    $\advB^{\REPO,\QRYO,\HASHO_1}$;
    return $\big[\sum_x \err[x] \geq r\big]$
  \\[6pt]
  \oraclev{$\REPO(\col)$}\\[2pt]
    $M^* \gets \bigvee_{x \in \col} \bmap_m(\HASHO_2(\salt^* \cat x))$;
    $\setS^* \gets \col$;
    return $\langle M^*, Z^* \rangle$
  \\[6pt]
  \oraclev{$\QRYO(\qry_x)$}\\[2pt]
    $X \gets \bmap_m(\HASHO_3(\salt^* \cat x))$;
    $a \gets X = M^* \AND X$\\
    if $\err[x] < \delta(a,\qry_x(\col^*))$ then
          $\err[x] \gets \delta(a,\qry_x(\col^*))$\\
    return $a$
  \\[6pt]
  \oraclev{$\HASHO_c(\salt \cat x)$}\\[2pt]
    $\vv \getsr [m]^k$\\
    if $M^*=\bot$ and $\salt=\salt^*$ and $c=1$ then \com{Caller is~$\advB$}\\
    \tab $\bad_1 \gets 1$; \diffplus{return $\vv$}\\
    if $T[Z,x] = \bot$ then $\vv \gets T[Z,x]$\\
    $T[Z,x] \gets \vv$; return $\vv$
}
{
  \vspace{-2pt}
  \oraclev{$\HASHO_c(\salt \cat x)$}\\[2pt]
    $\vv \getsr [m]^k$\\
    if $M^*=\bot$ and $\salt=\salt^*$ and $c=1$ then\\
    \tab $\bad_1 \gets 1$; return $\vv$\\
    if $T[Z,x] = \bot$ then $\vv \gets T[Z,z]$\\
    $T[Z,x] \gets \vv$\\[2pt]
    \diffplusbox{
    \com{Caller is~$\advB$ or $\QRYO$}\\
    if $c=1$ or $c=3$ then\\
    \tab if $\salt \ne \salt^*$  then return $\vv$\\
    \tab $\Ans[x] \gets \bmap_m(\vv) = M^* \AND \bmap_m(\vv)$\\
    \tab if $\err[x] < \delta(\Ans[x],\qry_x(\col^*))$ then
    \tab\tab $\err[x] \gets \delta(\Ans[x],\qry_x(\col^*))$
    }
    return $\vv$
}
{
  \vspace{-7pt}
  \oraclev{$\QRYO(\qry_x)$}\
      \hfill \diffminus{$\game_1$} \diffplus{$\game_2$}\\[2pt]
    \diffminusbox{%
      $X \gets \bmap_m(\HASHO_3(\salt^* \cat x))$\\
      $a \gets X = M^* \AND X$\\
      if $\err[x] < \delta(a,\qry_x(\col^*))$ then\\
      \tab $\err[x] \gets \delta(a,\qry_x(\col^*))$
    }\\[2pt]
    \diffplusbox{
      $\HASHO_3(Z^* \cat x)$\\
      $a \gets \Ans[x]$
    }
    return $a$
}
\caption{Games 0, 1, and 2 for proof of Theorem~\ref{thm:sbf-errep-immutable}.}
\label{fig:sbf-errep-immutable/games}
\end{figure*}

We will use the following lemma for keyless structures, which is proved in
Appendix~\ref{sec:keyless-proof}.

\begin{lemma}\label{thm:lemma1}
  For every $q_R, q_T, q_U, q_H, r, t \geq 0$ and keyless structure~$\Gamma$ it
  holds that
  \begin{eqnarray*}
    \begin{aligned}
      \Adv{\errep}_{\Gamma,\delta,r}(t,\,&q_R, q_T, q_U, q_H) \leq \\
      & q_R \cdot \Adv{\errep1}_{\Gamma,\delta,r}(O(t), q_T, q_U, q_H) \,,
    \end{aligned}
  \end{eqnarray*}
\end{lemma}
%
\noindent
The proof is by a fairly straightforward hybrid argument. Because~$\Gamma$ is
keyless, in the reduction we simulate $q_R-1$ of the calls to $\REPO$ experiment
and use our own oracles for the remaining query. The best we can do with this
strategy is to ``guess'' which representation the \errep\ adversary will use in
its attack, which results in the~$q_R$ factor in the bound.
%
We defer the full details to Appendix~\ref{app:deferred}

Let $\advA$ be an \errep\ adversary making~$1$ query to~$\REPO$, $q_T$ queries
to $\QRYO$, $0$ queries to $\UPO$, and $q_H$ queries to the random
oracle~$\HASHO$.
%
We make the following assumptions, all of which are without loss of generality.
%
First, all of~$\advA$'s $\QRYO$ queries follow its $\REPO$ query.
%
Second, we assume that $x\not\in\setS$ for all queries $\qry_x$ to $\QRYO$,
where~$\setS$ was the input to~$\advA$'s $\REPO$ query. This is without loss
because Bloom filters admit false positives, but not false negatives
%
Third, we we assume that $|\setS| \leq n$; this is without loss because
otherwise~$\REPO$ outputs~$\bot$ and~$\advA$ gets no advantage.
%
Fourth, we assume that all of~$\advA$'s $\HASHO$ queries are of the form $Z\cat
x$, where $|Z| = \lambda$.

We begin with a game-playing argument~\cite{bellare2006triple}, then obtain the
final bound via applicaition of Lemma~\ref{thm:lemma1}.
%
The high-level goal is to rewrite the game so that the probability that one
of~$\advA$'s queries runs up the score is precisely non-adaptive false positive
probability.
%
In other words, our goal is to transistion into a setting in which the Bloom
filter output by~$\REPO$ is independent of the outcome of~$\advA$'s other
queries.

Consider the game~$\game_0(\advB)$ defined in
Figure~\ref{fig:sbf-errep-immutable/games}. It is similar to the \errep\
experiment when executed with~$\advA$, $\Pi$, $\delta$, and~$r$, but the
pseudocode has been simplified to clarify our argumen. Indeed, it is not
difficult to see that for every~$\advA$ there exists an adversary~$\advB$ such
that
\begin{equation}
  \Adv{\errep}_{\Pi,\delta,r}(\advA) \leq \Prob{\game_0(\advB) = 1}
\end{equation}
and~$\advB$ has the same query resources as~$\advA$.
%
Adversary~$\advB$ executes~$\advA$, forwarding~$\advA$'s oracle queries
to its own oracles in the natural way.

Observe that in game~$\game_0$ the salt used for the representation of~$\setS^*$
is generated prior to executing~$\advB$. Game~$\game_1$ is identical
to~$\game_0$ until the flag~$\bad_1$ gets set by oracle~$\HASHO$. This occurs
if~$\advB$ asks $\HASHO_1(\salt^* \cat x)$, where~$\salt^*$ is the salt generated
at the beginning of the game, and it has not yet called $\QRYO$ (i.e.,
$M^*=\bot$).
%
By the Fundamental Lemma of Game Playing~\cite{bellare2006triple} it follows
that
%
\begin{eqnarray}
  \Prob{\game_0(\advB)=1} &\leq&
    \Prob{\game_1(\advB)=1} + \Prob{\game_1(\advB) \sets \bad_1}\\
  &\leq&
    \Prob{\game_1(\advB)=1} + q_H/2^\lambda \,.
\end{eqnarray}
%
Note that in $\game_1$, the value of~$M^*$ is independent of~$\advB$'s
$\HASHO_1$ queries. In particular, the probability that some bit of~$M^*$ is set
is independent of the choices of~$\advB$.

In game $\game_2$ the $\HASHO$ and $\QRYO$ oracles have been rewritten so that
the winning-condition is computed by $\HASHO$ instead of $\QRYO$. The former
oracle maintains a set~$\Ans$ such that $\Ans[x] = \Qry^{\HASHO_3}(M^*, \qry_x)$ for
each query $\salt^* \cat x$; on input of $\qry_x$, oracle~$\QRYO$ simply runs
$\HASHO_3(\salt^* \cat x)$ and returns $\Ans[x]$.
%
We are effectively giving the adverseary credit for RO queries that result in
false positives for the representation of~$\setS^*$, but which it does not
explicitly ask of the~$\QRYO$. Because~$\advB$'s advantage is at least that
of~$\advA$'s, it holds that
%
\begin{equation}
  \Prob{\game_1(\advB)=1} \leq \Prob{\game_2(\advB)=1} \,.
\end{equation}

We now consider $\Prob{\game_2(\advB)=1}$.
%
Let $\setX$ be the set $\{ x \in \bits^* : \Ans[x] \ne \bot \}$ and $\setT = \{x
\in\setX: \Ans[x] = 1\}$, where $\Ans$ is at is defined when~$\advB$ halts. We
will call~$\setX$ the set of attempts and~$\setT$ the set of false positives.
%
Note that $\setX\intersection\setS^*=\emptyset$ and
$|\setX| \leq q_H + q_T$.
%
Hence, the probability that~$\game_2(\advB)=1$ is equal to the probability
that~$|\setT| \geq r$.

For each $x\in\setX$, let $T(x)$ denote the event that $x\in\setT$.
%
In the random oracle model for~$H$, the set of random random variables $T(x)$
for each $x\in X$ are independently and identically distributed.
%
Hence, the probability that~$\advB$ succeeds is binomially distributed:
%
\begin{equation}
   \Prob{\game_2(\advB)=1} = \Prob{ |\setT| \geq r } =
     \sum_{i=r}^{q} \binom{q}{i}p^i(1-p)^{q-i} \,,
\end{equation}
%
where $q \leq q_H + q_T$ and $p = \Pr[T(x)=1]$. Here we can apply a Chernoff
bound which states that, for any $\delta > 0$,
%
\begin{equation}
  \Prob{X \geq (1+\delta)\mu} < \left(\frac{e^\delta}{(1+\delta)^{1+\delta}}\right)^\mu
\end{equation}
%
We set $\delta = r\mu^{-1}-1$ and note that $\mu = pq$.
This yields
%
\begin{equation}
 \Prob{|\setT| \geq r} < \left(\frac{e^{r\mu^{-1}-1}}{(r\mu^{-1})^{r\mu^{-1}}}\right)^\mu = \left(\frac{e^{r-\mu}}{(r\mu^{-1})^r}\right) = e^{r-pq}\left(\frac{pq}{r}\right)^r
\end{equation}
%
So we have
%
\begin{equation}
  \Adv{\errep}_{\Pi,\delta,r}(\advA) < \frac{q_H}{2^\lambda} + \left(\frac{pq}{r}\right)^re^{r-pq}
\end{equation}
%
Applying Lemma~\ref{thm:lemma1} to move from the single-representation case to the
general case, we get our final bound of
\begin{equation}
  \Adv{\errep}_{\Pi,\delta,r}(\advA) \leq
    q_R \cdot \left[
      \frac{q_H}{2^\lambda} +
      \left(\frac{pq}{r}\right)^re^{r-pq}
    \right] \,.
\end{equation}

\end{proof}
}

Recall that the \errep1 attack against mutable salted filters exploited the fact that
the adversary learned the salt as soon as the filter was created, and that from
this it could compute the hash function on its own. Even if the filter is
mutable, we can prevent this attack from working as long as we require that the
filter under attack be kept secret from adversaries. In fact, we can attain the
following \erreps\ bound for~$\Pi$.

\begin{theorem}[\erreps\ security of salted BFs]\label{thm:sbf-erreps}
  Let $p' = P_{k,m}(n+r)$.
  For all integers $q_R, q_T, q_U q_H, q_V, r, t \geq 0$, if
  $r > p'q_T$, then it holds that
  \begin{eqnarray*}
    \begin{aligned}
      \Adv{\erreps}_{\Pi,\delta,r}(t,\,&q_R, q_T, q_U, q_H, q_V) \leq q_R \cdot \left[
      \frac{q_H}{2^\lambda} +
      \left(\frac{p'q_T}{r}\right)^re^{r-p'q_T}\right]\,,
    \end{aligned}
\end{eqnarray*}
  where $H$ is modeled as a random oracle.
\end{theorem}

The proof follows a similar structure to that of
Theorem~\ref{thm:sbf-errep-immutable}. The main differences come from arguing
that without a ``lucky'' guess of the salt, the adversary cannot use $\HASHO$ to
find false positives, and from having to show that the adversary's access to
$\UPO$ does not substantially change the security bound that can be derived. The
first of these is straightforward given the private-representation setting, but
the second requires investigating how much of an advantage the $\UPO$ oracle can
give, then moving to games where this advantage is taken into account.

\proc{
  \begin{proof}[Proof Sketch of Theorem~\ref{thm:sbf-erreps}]
  \begin{figure*}
  \cpnote{I suggest re-doing this from scratch. The security experiment
  (\erreps) has changed and so has the construction. $\Repx$ and $\fff$ are not
  defined. Where's the $\REVO$ oracle?}
  \boxThmBFSaltCorrect{0.48}
  {
    \underline{$\game_0(\advA)$}\\[2pt]
      $\col \getsr \advA^H$; $\setC \gets \emptyset$; $\err \gets 0$\\
      $\pub \getsr \Rep[H](\col)$\\
      $\bot \getsr \advA^{H,\QRYO,\UPO}$\\
      return $(\err \geq r)$
    \\[6pt]
    \oraclev{$\QRYO(\qry_x)$}\\[2pt]
      if $\qry_x \in \mathcal{C}$ then return $\bot$\\
      $\setC \gets \setC \union \{\qry_x\}$\\
      $a \gets \Qry[H](\pub, \qry_x)$\\
      if $a \neq \qry_x(\col)$ then $\err \gets \err + 1$\\
      return~$a$
    \\[6pt]
    \oraclev{$\UPO(\up_x)$}\\[2pt]
      $\setC \gets \emptyset$\\
      $a \gets \Qry[H](\pub, \qry_x)$\\
      if $\qry_x \in \setC$ and $a \neq \qry_x(\col)$ then\\
      \tab $\err \gets \err-1$\\
      $\col \gets \col \union \{x\}$\\
      $\pub \gets \Up[H](\pub,\up_x)$\\
      return~$\bot$
    \\[4pt]
    \hspace*{-4pt}\rule{1.043\textwidth}{.4pt}
    \\[5pt]
    \oraclev{$\HASHO_1(\salt,x)$} \hfill\diffplus{$\game_2$}\;{$\game_1$}\hspace*{3pt}\\
      $\hh \getsr [m]^2$; $\vv \gets \fff(\hh)$\\
      if $\salt = \salt^*$ then\\
      \tab $\bad_1 \gets 1$; \diffplus{return $\vv$}\\
      if $T[\salt,x]$ is defined then $\vv \gets T[\salt,x]$\\
      $T[\salt,x] \gets \vv$;
      return $\vv$
  }
  {
    \underline{$\game_1(\advB)$}\\[2pt]
      $\salt^* \getsr \bits^\lambda$;
      $\col \getsr \advB^{\HASHO_1}$\\
      $\pub \gets \Repx[\HASHO_2](\col, \salt^*)$\\
      $\setC \gets \emptyset$;
      $\err \gets 0$\\
      $\bot \getsr \advB^{\HASHO_1,\QRYO,\UPO}$\\
      return $(\err \geq r)$
    \\[6pt]
    \oraclev{$\QRYO(\qry_x)$}\\[2pt]
      if $\qry_x \in \mathcal{C}$ then return $\bot$\\
      $\setC \gets \setC \cup \{\qry_x\}$\\
      $a \gets \Qry[\HASHO_2](\pub, \qry_x)$\\
      if $a \neq \qry_x(\col)$ then $\err \gets \err + 1$\\
      return~$a$
    \\[6pt]
    \oraclev{$\UPO(\up_x)$}\\[2pt]
      $\setC \gets \emptyset$\\
      $a \gets \Qry[H](\pub, \qry_x)$\\
      if $a \neq \qry_x(\col)$ and $\qry_x \in \setC$ then\\
      \tab $\err \gets \err-1$\\
      $\col \gets \col \union \{x\}$\\
      $\pub \gets \Up[\HASHO_2](\pub,\up_x)$\\
      return~$\bot$
    \\[6pt]
    \oraclev{$\HASHO_2(\salt,x)$}\\[2pt]
      $\hh \getsr [m]^2$; $\vv \gets \fff(\hh)$\\
      if $T[\salt,x]$ is defined then\\
      \tab $\vv \gets T[\salt,x]$\\
      $T[\salt,x] \gets \vv$;
      return $\vv$
  }
  {
    \underline{$\game_3(\advB)$}\\[2pt]
    \oraclev{$\QRYO(\qry_x)$}\\[2pt]
      $a \gets \Qry[\HASHO_3](\pub, \qry_x)$\\
      if $a \neq \qry_x(\col)$ then $\err \gets \err + 1$\\
      $\col \gets \col \union \{x\}$
      $\pub \gets \Up[\HASHO_2](\pub,\up_x)$\\
      return~$a$
  }
  {
    \oraclev{$\UPO(\up_x)$}\\[2pt]
      $a \gets \Qry[\HASHO_3](\pub, \qry_x)$\\
      if $a \neq \qry_x(\col)$ then $\err \gets \err + 1$\\
      $\col \gets \col \union \{x\}$
      $\pub \gets \Up[\HASHO_2](\pub,\up_x)$\\
      return~$\bot$
    \\[6pt]
    \oraclev{$\HASHO_i(\salt,x)$}\\[2pt]
      $\hh \getsr [m]^2$; $\vv \gets \fff(\hh)$\\
      return $\vv$
  }
  \caption{Games 0--3 for proof of Theorem~\ref{thm:sbf-erreps}.}
  \label{fig:sbf-erreps/games}
\end{figure*}

\cpnote{I understand the crux of the argument of how you deal with interleaved
updates/queries. It's a clever idea and I think it's believable. That said,
there's a lot of details that are omitted addressed.
%
Right now the biggest problem with this argument is that the games have
virtually nothing to do with the security notions and nothing to do with the
scheme being analyzed. They seem to be a carry-over from the old paper, but
things have change significantly. They will need to be rewritten. Try using the
games in Figure~\ref{fig:sbf-errep1/games} as a reference.}

\cpnote{I'm not clear on how update thresholding is used in the argument. From
my read it seems you're assuming some maximum represented set size, but we're
not maintaining a counter in the construction. Here's a hint: if thresholding is
necessary for security, then I'd expect $\ell$ to come up in the bound; if it
doesn't, then there'd better be a good reason.}

\cpnote{The argument silently assumes that there's no $\REVO$ oracle.}

\cpnote{Try starting this way:}
%
Just as in the proof of Theorem~\ref{thm:sbf-errep1} we will assume the
advwersary just makes a single query to~$\QRYO$ and use Lemma~\ref{thm:lemma1}
to complete the bound.
%
Let $\advA$ be an \erreps adversary making exactly~$1$ query to~$\REPO$, $q_T$
queries to~$\QRYO$, $q_U$ queries to~$\UPO$, and $q_H$ queries to~$\HASHO$.


\cpnote{Tip: If an claim follows easily from an argument made earlier than the
proof, then feel free to move quickly through it and refer the reader to the
argument for detials. The best you can do is say something like ``Equation (X)
follows nearly the same argument as used to deerive Equation (Y) ...''}

\cpnote{It'll be easier to apply Lemma~\ref{thm:lemma1} at the end.}
We first reduce from the \erreps case to the \erreps1 case, which by
lemma~\ref{lemma:errep} may scale the adversary's advantage only by a factor of
$q_R$. The game~$\game_0$ is exactly equivalent to the $\erreps1$ experiment, so
$\Adv{\errep1}_{\struct_s,r}(\advA) = \Prob{\game_0(\advA) = 1}$.%

In~$\game_1$ we split the hash oracle into three, giving the adversary access
$\HASHO_1$ in both stages of the game, while $\HASHO_2$ is reserved for oracular
use by $\Repx$ \cpnote{Undefined!}, $\QRYO$, and $\UPO$. For any $\advA$ for~$\game_0$, there is
$\advB$ for~$\game_1$ which produces the same advantage by simulating $\advA$.
This adversary first creates its own table $R$ with all values initially
undefined.  When $\advA$ makes a query $w$ to $H$, $\advB$ returns $R[w]$ if
that entry in the table is defined. Otherwise, if there are $\salt \in
\bits^\lambda$, $j \in [k]$, and $x \in \bits^*$ such that $w = \langle\salt, j,
x\rangle$, forward $(\salt,x)$ to $\HASHO_1$. For each $j \in [k]$, set
$R[\langle\salt, j, x\rangle] = \vv_j$, where $\vv$ is the output of the
$\HASHO_1$ oracle. If there is no such triple $\langle\salt, j, x\rangle$, just
sample $r$ from $[m]$ uniformly and set $R[w] = r$.
%
\cpnote{All of this is out date. This is about the linear hashing scheme of
Kirsch-Mitzenmacher, which we're no longer analyzing.}
%
In either case, return
$R[w]$ to $\advA$. When $\advA$ outputs its collection $\col$, $\advB$ outputs
$\col$ as well. Any queries by $\advA$ to $\QRYO$ or $\UPO$ are forwarded to
$\advB$'s corresponding oracle. The simulation is perfect because
$\Rep[H](\col)$ and $\Up[H](\col,\up)$ are identically distributed to
$\Rep[\HASHO_2](\col)$ and
$\Up[\HASHO_2](\col,\up)$. Because we have a perfect simulation,
$\Adv{\erreps1}_{\struct_s,r}(\advA) = \Prob{\game_1(\advA) = 1}$.

The game~$\game_2$ is the same as~$\game_1$ until $\bad_1$ is set, which occurs
exactly when $\advB$ sends $(\salt^*,x)$ to $\HASHO_1$ for some $x$. In the
first phase, there is again a $q_1/2^\lambda$ chance of the adversary guessing
the salt. In the second phase, the random sampling used by $\HASHO_i$ ensures
that each call the adversary makes to the $\HASHO_i$ oracle is independent of
all previous calls. We therefore have a $q_2/2^\lambda$ chance of the adversary
guessing the salt during this phase, for a total chance of $q_H/2^\lambda$
chance of the adversary guessing the salt at some point during the experiment.
Then $\Adv{\erreps}_{\struct_s,r}(\advA) \le \Prob{\game_2(\advB) = 1} +
q_H/2^\lambda$. Having taken this into account, we may now assume the adversary
never guesses the salt.
%
\cpnote{I would move quickly through this point. Just refer back to the step in
Theorem~\ref{thm:sbf-errep1}.}


\cpnote{The rest of this seems like the meat of the argument.}

We want to show that alternating between sequences of queries and sequences of
updates is no better than making one long series of updates and then one long
sequence of queries. There are three types of updates the adversary can make:
updates to add elements that have been queried and found to be false positives;
updates to add elements that have been queried and found not to be false
positives; and updates to add elements that have not been queried yet. We may
assume without loss of generality that the adversary never makes the first type
of update, since doing so is never beneficial (it does not change the
representation at all and decreases the number of errors the adversary has
found). \cpnote{Explicitly name these type-1, type-2, and type-3 updates, since
you refer to them below. IN fact, it might be beneficial to put them in a
bultted list.}

Note that the choices of $\vv$ constructed by the $\HASHO_i$ oracles are
independent of all previous queries. Because of this, any update of type-3 is
equivalent to any other update of type-3%
%
\cpnote{Careful with the word ``equivalent'': I think you mean ``have the same
distribution'' or something?}%
%
; the probability of any bit being flipped by one update is the same as the
probability of the bit being flipped by the other update. Similarly, any update
of type-2 is equivalent to any other update of type-2, but is not the same as
type-3 since the probability is conditioned on $\vv$ not being a false positive.
We assume the worst case, namely that all updates are type-2 (i.e. at least one
bit is flipped by each update).

Because the adversary never guesses the salt, $\HASHO_1$ simply functions as a
random oracle.
%
\cpnote{This isn't quite true. Go back to Thereom~\ref{fig:sbf-errep1} and think
about the semantics of~$\HASHO_1$ in games~0 and~1.}
%
Furthermore, we can assume the adversary never adds an element of
$\col$ to $\col$%
%
\cpnote{Word this differently. You mean update the data structure with an
element that is already in it?}
%
and never makes a $\QRYO$ call for an element which is already
in $\col$, since neither of these provides any additional information and
neither affects the rest of the experiment in any way.

Now we move to the game~$\game_3$. Here each $\QRYO$ query also calls $\UPO$ to
add that element to $\col$. \cpnote{Clever.}  Additionally, the penalty for adding known false
positives is removed. To avoid penalizing the adversary by prematurely maxing
out the number of elements in $\col$ because of added false positives, we also
increase the maximum size of $\col$ from $n$ to $n+s$, where $s = \min(r,q_U)$.
Because the adversary (without loss of generality) stops after accumulating $r$
errors, only $\min(r,q_U)$ false positives will be added to $\col$ and so a
maximum size of $n+s$ is sufficient to produce no penalty for the adversary.
Furthermore, each $\UPO$ call is preceded by a $\QRYO$ call. Neither of these
changes can produce a worse result for the adversary, so $\Prob{\game_2(\advB) =
1} \le \Prob{\game_3(\advB) = 1}$. Now, however, there is no longer any
distinction between $\QRYO$ and $\UPO$ calls. All calls to either oracle are
independent of each other and produce the same effect, querying and then
updating $\col$. Each of these queries for false positives is at most as
successful as a query to a Bloom filter with $n+r$ elements, so the adversary's
probability of finding a false positive on any query is bounded above by the
standard success rate for a Bloom filter with those parameters. The adversary is
required to produce $r$ errors over the course of $q_T+q_U$ queries, which by
the binomial theorem gives an advantage bound of $\Prob{\game_3(\advB) = 1} \leq
...$.
%
\cpnote{What about~$q_H$?}

\todo{DC (lead)}{Finish the bound by applying Lemma~\ref{thm:lemma1}.}
\end{proof}
}

\full{
  \begin{proof}[Proof of Theorem~\ref{thm:sbf-erreps}]
  \begin{figure*}
  \cpnote{I suggest re-doing this from scratch. The security experiment
  (\erreps) has changed and so has the construction. $\Repx$ and $\fff$ are not
  defined. Where's the $\REVO$ oracle?}
  \boxThmBFSaltCorrect{0.48}
  {
    \underline{$\game_0(\advA)$}\\[2pt]
      $\col \getsr \advA^H$; $\setC \gets \emptyset$; $\err \gets 0$\\
      $\pub \getsr \Rep[H](\col)$\\
      $\bot \getsr \advA^{H,\QRYO,\UPO}$\\
      return $(\err \geq r)$
    \\[6pt]
    \oraclev{$\QRYO(\qry_x)$}\\[2pt]
      if $\qry_x \in \mathcal{C}$ then return $\bot$\\
      $\setC \gets \setC \union \{\qry_x\}$\\
      $a \gets \Qry[H](\pub, \qry_x)$\\
      if $a \neq \qry_x(\col)$ then $\err \gets \err + 1$\\
      return~$a$
    \\[6pt]
    \oraclev{$\UPO(\up_x)$}\\[2pt]
      $\setC \gets \emptyset$\\
      $a \gets \Qry[H](\pub, \qry_x)$\\
      if $\qry_x \in \setC$ and $a \neq \qry_x(\col)$ then\\
      \tab $\err \gets \err-1$\\
      $\col \gets \col \union \{x\}$\\
      $\pub \gets \Up[H](\pub,\up_x)$\\
      return~$\bot$
    \\[4pt]
    \hspace*{-4pt}\rule{1.043\textwidth}{.4pt}
    \\[5pt]
    \oraclev{$\HASHO_1(\salt,x)$} \hfill\diffplus{$\game_2$}\;{$\game_1$}\hspace*{3pt}\\
      $\hh \getsr [m]^2$; $\vv \gets \fff(\hh)$\\
      if $\salt = \salt^*$ then\\
      \tab $\bad_1 \gets 1$; \diffplus{return $\vv$}\\
      if $T[\salt,x]$ is defined then $\vv \gets T[\salt,x]$\\
      $T[\salt,x] \gets \vv$;
      return $\vv$
  }
  {
    \underline{$\game_1(\advB)$}\\[2pt]
      $\salt^* \getsr \bits^\lambda$;
      $\col \getsr \advB^{\HASHO_1}$\\
      $\pub \gets \Repx[\HASHO_2](\col, \salt^*)$\\
      $\setC \gets \emptyset$;
      $\err \gets 0$\\
      $\bot \getsr \advB^{\HASHO_1,\QRYO,\UPO}$\\
      return $(\err \geq r)$
    \\[6pt]
    \oraclev{$\QRYO(\qry_x)$}\\[2pt]
      if $\qry_x \in \mathcal{C}$ then return $\bot$\\
      $\setC \gets \setC \cup \{\qry_x\}$\\
      $a \gets \Qry[\HASHO_2](\pub, \qry_x)$\\
      if $a \neq \qry_x(\col)$ then $\err \gets \err + 1$\\
      return~$a$
    \\[6pt]
    \oraclev{$\UPO(\up_x)$}\\[2pt]
      $\setC \gets \emptyset$\\
      $a \gets \Qry[H](\pub, \qry_x)$\\
      if $a \neq \qry_x(\col)$ and $\qry_x \in \setC$ then\\
      \tab $\err \gets \err-1$\\
      $\col \gets \col \union \{x\}$\\
      $\pub \gets \Up[\HASHO_2](\pub,\up_x)$\\
      return~$\bot$
    \\[6pt]
    \oraclev{$\HASHO_2(\salt,x)$}\\[2pt]
      $\hh \getsr [m]^2$; $\vv \gets \fff(\hh)$\\
      if $T[\salt,x]$ is defined then\\
      \tab $\vv \gets T[\salt,x]$\\
      $T[\salt,x] \gets \vv$;
      return $\vv$
  }
  {
    \underline{$\game_3(\advB)$}\\[2pt]
    \oraclev{$\QRYO(\qry_x)$}\\[2pt]
      $a \gets \Qry[\HASHO_3](\pub, \qry_x)$\\
      if $a \neq \qry_x(\col)$ then $\err \gets \err + 1$\\
      $\col \gets \col \union \{x\}$
      $\pub \gets \Up[\HASHO_2](\pub,\up_x)$\\
      return~$a$
  }
  {
    \oraclev{$\UPO(\up_x)$}\\[2pt]
      $a \gets \Qry[\HASHO_3](\pub, \qry_x)$\\
      if $a \neq \qry_x(\col)$ then $\err \gets \err + 1$\\
      $\col \gets \col \union \{x\}$
      $\pub \gets \Up[\HASHO_2](\pub,\up_x)$\\
      return~$\bot$
    \\[6pt]
    \oraclev{$\HASHO_i(\salt,x)$}\\[2pt]
      $\hh \getsr [m]^2$; $\vv \gets \fff(\hh)$\\
      return $\vv$
  }
  \caption{Games 0--3 for proof of Theorem~\ref{thm:sbf-erreps}.}
  \label{fig:sbf-erreps/games}
\end{figure*}

\cpnote{I understand the crux of the argument of how you deal with interleaved
updates/queries. It's a clever idea and I think it's believable. That said,
there's a lot of details that are omitted addressed.
%
Right now the biggest problem with this argument is that the games have
virtually nothing to do with the security notions and nothing to do with the
scheme being analyzed. They seem to be a carry-over from the old paper, but
things have change significantly. They will need to be rewritten. Try using the
games in Figure~\ref{fig:sbf-errep1/games} as a reference.}

\cpnote{I'm not clear on how update thresholding is used in the argument. From
my read it seems you're assuming some maximum represented set size, but we're
not maintaining a counter in the construction. Here's a hint: if thresholding is
necessary for security, then I'd expect $\ell$ to come up in the bound; if it
doesn't, then there'd better be a good reason.}

\cpnote{The argument silently assumes that there's no $\REVO$ oracle.}

\cpnote{Try starting this way:}
%
Just as in the proof of Theorem~\ref{thm:sbf-errep1} we will assume the
advwersary just makes a single query to~$\QRYO$ and use Lemma~\ref{thm:lemma1}
to complete the bound.
%
Let $\advA$ be an \erreps adversary making exactly~$1$ query to~$\REPO$, $q_T$
queries to~$\QRYO$, $q_U$ queries to~$\UPO$, and $q_H$ queries to~$\HASHO$.


\cpnote{Tip: If an claim follows easily from an argument made earlier than the
proof, then feel free to move quickly through it and refer the reader to the
argument for detials. The best you can do is say something like ``Equation (X)
follows nearly the same argument as used to deerive Equation (Y) ...''}

\cpnote{It'll be easier to apply Lemma~\ref{thm:lemma1} at the end.}
We first reduce from the \erreps case to the \erreps1 case, which by
lemma~\ref{lemma:errep} may scale the adversary's advantage only by a factor of
$q_R$. The game~$\game_0$ is exactly equivalent to the $\erreps1$ experiment, so
$\Adv{\errep1}_{\struct_s,r}(\advA) = \Prob{\game_0(\advA) = 1}$.%

In~$\game_1$ we split the hash oracle into three, giving the adversary access
$\HASHO_1$ in both stages of the game, while $\HASHO_2$ is reserved for oracular
use by $\Repx$ \cpnote{Undefined!}, $\QRYO$, and $\UPO$. For any $\advA$ for~$\game_0$, there is
$\advB$ for~$\game_1$ which produces the same advantage by simulating $\advA$.
This adversary first creates its own table $R$ with all values initially
undefined.  When $\advA$ makes a query $w$ to $H$, $\advB$ returns $R[w]$ if
that entry in the table is defined. Otherwise, if there are $\salt \in
\bits^\lambda$, $j \in [k]$, and $x \in \bits^*$ such that $w = \langle\salt, j,
x\rangle$, forward $(\salt,x)$ to $\HASHO_1$. For each $j \in [k]$, set
$R[\langle\salt, j, x\rangle] = \vv_j$, where $\vv$ is the output of the
$\HASHO_1$ oracle. If there is no such triple $\langle\salt, j, x\rangle$, just
sample $r$ from $[m]$ uniformly and set $R[w] = r$.
%
\cpnote{All of this is out date. This is about the linear hashing scheme of
Kirsch-Mitzenmacher, which we're no longer analyzing.}
%
In either case, return
$R[w]$ to $\advA$. When $\advA$ outputs its collection $\col$, $\advB$ outputs
$\col$ as well. Any queries by $\advA$ to $\QRYO$ or $\UPO$ are forwarded to
$\advB$'s corresponding oracle. The simulation is perfect because
$\Rep[H](\col)$ and $\Up[H](\col,\up)$ are identically distributed to
$\Rep[\HASHO_2](\col)$ and
$\Up[\HASHO_2](\col,\up)$. Because we have a perfect simulation,
$\Adv{\erreps1}_{\struct_s,r}(\advA) = \Prob{\game_1(\advA) = 1}$.

The game~$\game_2$ is the same as~$\game_1$ until $\bad_1$ is set, which occurs
exactly when $\advB$ sends $(\salt^*,x)$ to $\HASHO_1$ for some $x$. In the
first phase, there is again a $q_1/2^\lambda$ chance of the adversary guessing
the salt. In the second phase, the random sampling used by $\HASHO_i$ ensures
that each call the adversary makes to the $\HASHO_i$ oracle is independent of
all previous calls. We therefore have a $q_2/2^\lambda$ chance of the adversary
guessing the salt during this phase, for a total chance of $q_H/2^\lambda$
chance of the adversary guessing the salt at some point during the experiment.
Then $\Adv{\erreps}_{\struct_s,r}(\advA) \le \Prob{\game_2(\advB) = 1} +
q_H/2^\lambda$. Having taken this into account, we may now assume the adversary
never guesses the salt.
%
\cpnote{I would move quickly through this point. Just refer back to the step in
Theorem~\ref{thm:sbf-errep1}.}


\cpnote{The rest of this seems like the meat of the argument.}

We want to show that alternating between sequences of queries and sequences of
updates is no better than making one long series of updates and then one long
sequence of queries. There are three types of updates the adversary can make:
updates to add elements that have been queried and found to be false positives;
updates to add elements that have been queried and found not to be false
positives; and updates to add elements that have not been queried yet. We may
assume without loss of generality that the adversary never makes the first type
of update, since doing so is never beneficial (it does not change the
representation at all and decreases the number of errors the adversary has
found). \cpnote{Explicitly name these type-1, type-2, and type-3 updates, since
you refer to them below. IN fact, it might be beneficial to put them in a
bultted list.}

Note that the choices of $\vv$ constructed by the $\HASHO_i$ oracles are
independent of all previous queries. Because of this, any update of type-3 is
equivalent to any other update of type-3%
%
\cpnote{Careful with the word ``equivalent'': I think you mean ``have the same
distribution'' or something?}%
%
; the probability of any bit being flipped by one update is the same as the
probability of the bit being flipped by the other update. Similarly, any update
of type-2 is equivalent to any other update of type-2, but is not the same as
type-3 since the probability is conditioned on $\vv$ not being a false positive.
We assume the worst case, namely that all updates are type-2 (i.e. at least one
bit is flipped by each update).

Because the adversary never guesses the salt, $\HASHO_1$ simply functions as a
random oracle.
%
\cpnote{This isn't quite true. Go back to Thereom~\ref{fig:sbf-errep1} and think
about the semantics of~$\HASHO_1$ in games~0 and~1.}
%
Furthermore, we can assume the adversary never adds an element of
$\col$ to $\col$%
%
\cpnote{Word this differently. You mean update the data structure with an
element that is already in it?}
%
and never makes a $\QRYO$ call for an element which is already
in $\col$, since neither of these provides any additional information and
neither affects the rest of the experiment in any way.

Now we move to the game~$\game_3$. Here each $\QRYO$ query also calls $\UPO$ to
add that element to $\col$. \cpnote{Clever.}  Additionally, the penalty for adding known false
positives is removed. To avoid penalizing the adversary by prematurely maxing
out the number of elements in $\col$ because of added false positives, we also
increase the maximum size of $\col$ from $n$ to $n+s$, where $s = \min(r,q_U)$.
Because the adversary (without loss of generality) stops after accumulating $r$
errors, only $\min(r,q_U)$ false positives will be added to $\col$ and so a
maximum size of $n+s$ is sufficient to produce no penalty for the adversary.
Furthermore, each $\UPO$ call is preceded by a $\QRYO$ call. Neither of these
changes can produce a worse result for the adversary, so $\Prob{\game_2(\advB) =
1} \le \Prob{\game_3(\advB) = 1}$. Now, however, there is no longer any
distinction between $\QRYO$ and $\UPO$ calls. All calls to either oracle are
independent of each other and produce the same effect, querying and then
updating $\col$. Each of these queries for false positives is at most as
successful as a query to a Bloom filter with $n+r$ elements, so the adversary's
probability of finding a false positive on any query is bounded above by the
standard success rate for a Bloom filter with those parameters. The adversary is
required to produce $r$ errors over the course of $q_T+q_U$ queries, which by
the binomial theorem gives an advantage bound of $\Prob{\game_3(\advB) = 1} \leq
...$.
%
\cpnote{What about~$q_H$?}

\todo{DC (lead)}{Finish the bound by applying Lemma~\ref{thm:lemma1}.}
\end{proof}
}

%\begin{proof}[Proof of Theorem~\ref{thm:sbf-erreps}]
%  \begin{figure*}
  \cpnote{I suggest re-doing this from scratch. The security experiment
  (\erreps) has changed and so has the construction. $\Repx$ and $\fff$ are not
  defined. Where's the $\REVO$ oracle?}
  \boxThmBFSaltCorrect{0.48}
  {
    \underline{$\game_0(\advA)$}\\[2pt]
      $\col \getsr \advA^H$; $\setC \gets \emptyset$; $\err \gets 0$\\
      $\pub \getsr \Rep[H](\col)$\\
      $\bot \getsr \advA^{H,\QRYO,\UPO}$\\
      return $(\err \geq r)$
    \\[6pt]
    \oraclev{$\QRYO(\qry_x)$}\\[2pt]
      if $\qry_x \in \mathcal{C}$ then return $\bot$\\
      $\setC \gets \setC \union \{\qry_x\}$\\
      $a \gets \Qry[H](\pub, \qry_x)$\\
      if $a \neq \qry_x(\col)$ then $\err \gets \err + 1$\\
      return~$a$
    \\[6pt]
    \oraclev{$\UPO(\up_x)$}\\[2pt]
      $\setC \gets \emptyset$\\
      $a \gets \Qry[H](\pub, \qry_x)$\\
      if $\qry_x \in \setC$ and $a \neq \qry_x(\col)$ then\\
      \tab $\err \gets \err-1$\\
      $\col \gets \col \union \{x\}$\\
      $\pub \gets \Up[H](\pub,\up_x)$\\
      return~$\bot$
    \\[4pt]
    \hspace*{-4pt}\rule{1.043\textwidth}{.4pt}
    \\[5pt]
    \oraclev{$\HASHO_1(\salt,x)$} \hfill\diffplus{$\game_2$}\;{$\game_1$}\hspace*{3pt}\\
      $\hh \getsr [m]^2$; $\vv \gets \fff(\hh)$\\
      if $\salt = \salt^*$ then\\
      \tab $\bad_1 \gets 1$; \diffplus{return $\vv$}\\
      if $T[\salt,x]$ is defined then $\vv \gets T[\salt,x]$\\
      $T[\salt,x] \gets \vv$;
      return $\vv$
  }
  {
    \underline{$\game_1(\advB)$}\\[2pt]
      $\salt^* \getsr \bits^\lambda$;
      $\col \getsr \advB^{\HASHO_1}$\\
      $\pub \gets \Repx[\HASHO_2](\col, \salt^*)$\\
      $\setC \gets \emptyset$;
      $\err \gets 0$\\
      $\bot \getsr \advB^{\HASHO_1,\QRYO,\UPO}$\\
      return $(\err \geq r)$
    \\[6pt]
    \oraclev{$\QRYO(\qry_x)$}\\[2pt]
      if $\qry_x \in \mathcal{C}$ then return $\bot$\\
      $\setC \gets \setC \cup \{\qry_x\}$\\
      $a \gets \Qry[\HASHO_2](\pub, \qry_x)$\\
      if $a \neq \qry_x(\col)$ then $\err \gets \err + 1$\\
      return~$a$
    \\[6pt]
    \oraclev{$\UPO(\up_x)$}\\[2pt]
      $\setC \gets \emptyset$\\
      $a \gets \Qry[H](\pub, \qry_x)$\\
      if $a \neq \qry_x(\col)$ and $\qry_x \in \setC$ then\\
      \tab $\err \gets \err-1$\\
      $\col \gets \col \union \{x\}$\\
      $\pub \gets \Up[\HASHO_2](\pub,\up_x)$\\
      return~$\bot$
    \\[6pt]
    \oraclev{$\HASHO_2(\salt,x)$}\\[2pt]
      $\hh \getsr [m]^2$; $\vv \gets \fff(\hh)$\\
      if $T[\salt,x]$ is defined then\\
      \tab $\vv \gets T[\salt,x]$\\
      $T[\salt,x] \gets \vv$;
      return $\vv$
  }
  {
    \underline{$\game_3(\advB)$}\\[2pt]
    \oraclev{$\QRYO(\qry_x)$}\\[2pt]
      $a \gets \Qry[\HASHO_3](\pub, \qry_x)$\\
      if $a \neq \qry_x(\col)$ then $\err \gets \err + 1$\\
      $\col \gets \col \union \{x\}$
      $\pub \gets \Up[\HASHO_2](\pub,\up_x)$\\
      return~$a$
  }
  {
    \oraclev{$\UPO(\up_x)$}\\[2pt]
      $a \gets \Qry[\HASHO_3](\pub, \qry_x)$\\
      if $a \neq \qry_x(\col)$ then $\err \gets \err + 1$\\
      $\col \gets \col \union \{x\}$
      $\pub \gets \Up[\HASHO_2](\pub,\up_x)$\\
      return~$\bot$
    \\[6pt]
    \oraclev{$\HASHO_i(\salt,x)$}\\[2pt]
      $\hh \getsr [m]^2$; $\vv \gets \fff(\hh)$\\
      return $\vv$
  }
  \caption{Games 0--3 for proof of Theorem~\ref{thm:sbf-erreps}.}
  \label{fig:sbf-erreps/games}
\end{figure*}

\cpnote{I understand the crux of the argument of how you deal with interleaved
updates/queries. It's a clever idea and I think it's believable. That said,
there's a lot of details that are omitted addressed.
%
Right now the biggest problem with this argument is that the games have
virtually nothing to do with the security notions and nothing to do with the
scheme being analyzed. They seem to be a carry-over from the old paper, but
things have change significantly. They will need to be rewritten. Try using the
games in Figure~\ref{fig:sbf-errep1/games} as a reference.}

\cpnote{I'm not clear on how update thresholding is used in the argument. From
my read it seems you're assuming some maximum represented set size, but we're
not maintaining a counter in the construction. Here's a hint: if thresholding is
necessary for security, then I'd expect $\ell$ to come up in the bound; if it
doesn't, then there'd better be a good reason.}

\cpnote{The argument silently assumes that there's no $\REVO$ oracle.}

\cpnote{Try starting this way:}
%
Just as in the proof of Theorem~\ref{thm:sbf-errep1} we will assume the
advwersary just makes a single query to~$\QRYO$ and use Lemma~\ref{thm:lemma1}
to complete the bound.
%
Let $\advA$ be an \erreps adversary making exactly~$1$ query to~$\REPO$, $q_T$
queries to~$\QRYO$, $q_U$ queries to~$\UPO$, and $q_H$ queries to~$\HASHO$.


\cpnote{Tip: If an claim follows easily from an argument made earlier than the
proof, then feel free to move quickly through it and refer the reader to the
argument for detials. The best you can do is say something like ``Equation (X)
follows nearly the same argument as used to deerive Equation (Y) ...''}

\cpnote{It'll be easier to apply Lemma~\ref{thm:lemma1} at the end.}
We first reduce from the \erreps case to the \erreps1 case, which by
lemma~\ref{lemma:errep} may scale the adversary's advantage only by a factor of
$q_R$. The game~$\game_0$ is exactly equivalent to the $\erreps1$ experiment, so
$\Adv{\errep1}_{\struct_s,r}(\advA) = \Prob{\game_0(\advA) = 1}$.%

In~$\game_1$ we split the hash oracle into three, giving the adversary access
$\HASHO_1$ in both stages of the game, while $\HASHO_2$ is reserved for oracular
use by $\Repx$ \cpnote{Undefined!}, $\QRYO$, and $\UPO$. For any $\advA$ for~$\game_0$, there is
$\advB$ for~$\game_1$ which produces the same advantage by simulating $\advA$.
This adversary first creates its own table $R$ with all values initially
undefined.  When $\advA$ makes a query $w$ to $H$, $\advB$ returns $R[w]$ if
that entry in the table is defined. Otherwise, if there are $\salt \in
\bits^\lambda$, $j \in [k]$, and $x \in \bits^*$ such that $w = \langle\salt, j,
x\rangle$, forward $(\salt,x)$ to $\HASHO_1$. For each $j \in [k]$, set
$R[\langle\salt, j, x\rangle] = \vv_j$, where $\vv$ is the output of the
$\HASHO_1$ oracle. If there is no such triple $\langle\salt, j, x\rangle$, just
sample $r$ from $[m]$ uniformly and set $R[w] = r$.
%
\cpnote{All of this is out date. This is about the linear hashing scheme of
Kirsch-Mitzenmacher, which we're no longer analyzing.}
%
In either case, return
$R[w]$ to $\advA$. When $\advA$ outputs its collection $\col$, $\advB$ outputs
$\col$ as well. Any queries by $\advA$ to $\QRYO$ or $\UPO$ are forwarded to
$\advB$'s corresponding oracle. The simulation is perfect because
$\Rep[H](\col)$ and $\Up[H](\col,\up)$ are identically distributed to
$\Rep[\HASHO_2](\col)$ and
$\Up[\HASHO_2](\col,\up)$. Because we have a perfect simulation,
$\Adv{\erreps1}_{\struct_s,r}(\advA) = \Prob{\game_1(\advA) = 1}$.

The game~$\game_2$ is the same as~$\game_1$ until $\bad_1$ is set, which occurs
exactly when $\advB$ sends $(\salt^*,x)$ to $\HASHO_1$ for some $x$. In the
first phase, there is again a $q_1/2^\lambda$ chance of the adversary guessing
the salt. In the second phase, the random sampling used by $\HASHO_i$ ensures
that each call the adversary makes to the $\HASHO_i$ oracle is independent of
all previous calls. We therefore have a $q_2/2^\lambda$ chance of the adversary
guessing the salt during this phase, for a total chance of $q_H/2^\lambda$
chance of the adversary guessing the salt at some point during the experiment.
Then $\Adv{\erreps}_{\struct_s,r}(\advA) \le \Prob{\game_2(\advB) = 1} +
q_H/2^\lambda$. Having taken this into account, we may now assume the adversary
never guesses the salt.
%
\cpnote{I would move quickly through this point. Just refer back to the step in
Theorem~\ref{thm:sbf-errep1}.}


\cpnote{The rest of this seems like the meat of the argument.}

We want to show that alternating between sequences of queries and sequences of
updates is no better than making one long series of updates and then one long
sequence of queries. There are three types of updates the adversary can make:
updates to add elements that have been queried and found to be false positives;
updates to add elements that have been queried and found not to be false
positives; and updates to add elements that have not been queried yet. We may
assume without loss of generality that the adversary never makes the first type
of update, since doing so is never beneficial (it does not change the
representation at all and decreases the number of errors the adversary has
found). \cpnote{Explicitly name these type-1, type-2, and type-3 updates, since
you refer to them below. IN fact, it might be beneficial to put them in a
bultted list.}

Note that the choices of $\vv$ constructed by the $\HASHO_i$ oracles are
independent of all previous queries. Because of this, any update of type-3 is
equivalent to any other update of type-3%
%
\cpnote{Careful with the word ``equivalent'': I think you mean ``have the same
distribution'' or something?}%
%
; the probability of any bit being flipped by one update is the same as the
probability of the bit being flipped by the other update. Similarly, any update
of type-2 is equivalent to any other update of type-2, but is not the same as
type-3 since the probability is conditioned on $\vv$ not being a false positive.
We assume the worst case, namely that all updates are type-2 (i.e. at least one
bit is flipped by each update).

Because the adversary never guesses the salt, $\HASHO_1$ simply functions as a
random oracle.
%
\cpnote{This isn't quite true. Go back to Thereom~\ref{fig:sbf-errep1} and think
about the semantics of~$\HASHO_1$ in games~0 and~1.}
%
Furthermore, we can assume the adversary never adds an element of
$\col$ to $\col$%
%
\cpnote{Word this differently. You mean update the data structure with an
element that is already in it?}
%
and never makes a $\QRYO$ call for an element which is already
in $\col$, since neither of these provides any additional information and
neither affects the rest of the experiment in any way.

Now we move to the game~$\game_3$. Here each $\QRYO$ query also calls $\UPO$ to
add that element to $\col$. \cpnote{Clever.}  Additionally, the penalty for adding known false
positives is removed. To avoid penalizing the adversary by prematurely maxing
out the number of elements in $\col$ because of added false positives, we also
increase the maximum size of $\col$ from $n$ to $n+s$, where $s = \min(r,q_U)$.
Because the adversary (without loss of generality) stops after accumulating $r$
errors, only $\min(r,q_U)$ false positives will be added to $\col$ and so a
maximum size of $n+s$ is sufficient to produce no penalty for the adversary.
Furthermore, each $\UPO$ call is preceded by a $\QRYO$ call. Neither of these
changes can produce a worse result for the adversary, so $\Prob{\game_2(\advB) =
1} \le \Prob{\game_3(\advB) = 1}$. Now, however, there is no longer any
distinction between $\QRYO$ and $\UPO$ calls. All calls to either oracle are
independent of each other and produce the same effect, querying and then
updating $\col$. Each of these queries for false positives is at most as
successful as a query to a Bloom filter with $n+r$ elements, so the adversary's
probability of finding a false positive on any query is bounded above by the
standard success rate for a Bloom filter with those parameters. The adversary is
required to produce $r$ errors over the course of $q_T+q_U$ queries, which by
the binomial theorem gives an advantage bound of $\Prob{\game_3(\advB) = 1} \leq
...$.
%
\cpnote{What about~$q_H$?}

\todo{DC (lead)}{Finish the bound by applying Lemma~\ref{thm:lemma1}.}
%\end{proof}

\subsection{Keyed BFs}

Salted BFs are \erreps\ secure in general, and are \errep\ secure in the
immutable setting, but are not \errep\ secure when the adversary has access to
an $\UPO$ oracle. Our argument for the \erreps\ security of
salted Bloom filters is made possible by virtue of the structure under attack
not being revealed to the adversary. While this is realistic in many
applications, it may be desirable for the Bloom filter to be public \emph{and}
updatable.
%
Here we show that building a Bloom filter from a PRF suffices for security in
this setting.
%
Let $F:\keys\by\bits^*\to[m]^k$ be a function, fix
integers~$n,\lambda\geq0$, and let $\Pi = \KBF[F,n,\lambda]$.

\begin{theorem}[\errep\ security of keyed BFs]
\label{thm:bf-key-bound}
\label{thm:kbf-errep}
  Let $p' = P_{k,m}(n+r)$.  For integers $q_R, q_T, q_U, q_H, r, t \geq 0$ such that
  $r > p'q_T$, it holds that
  \begin{equation*}
    \begin{aligned}
      \Adv{\errep}_{\Pi,\delta,r}(t,\,&q_R,q_T,q_U,q_H) \leq \\
        \Adv{\prf}_F(O(t),nq_R+q_T+q_U) & +
      \frac{q_R^2}{2^\lambda} +
      \left(\frac{p'q_Rq_T}{r}\right)^re^{r-p'q_Rq_T} \,.
    \end{aligned}
  \end{equation*}
\end{theorem}

As usual, our strategy will be to move to the non-adaptive setting via a
sequence of game transitions, but the details of how we get there differ from
the case of keyless Bloom filters.  In particular, since we are using a PRF, the
initial parts of the proof deal with the adversary potentially being able to
break the PRF and with the possibility of the salts repeating rather than with
the adversary being able to guess the salt.

\proc{
  \begin{proof}[Proof Sketch of Theorem~\ref{thm:kbf-errep}]
  \begin{figure*}
\todo{DC}{Rename $\HASHO$ to something else! It looks like a random oracle, but
it's not!}
\todo{DC}{nit: Here and throughout the rest of paper, change $ct$ to
$\mathit{ct}$. The former looks like $c\cdot t$.}
\twoCols{0.47}
{
  \vspace{-7pt}
  \experimentv{$\game_{0}(\advA)$}\\[2pt]
    $\key \getsr \keys$;
    $ct \gets 0$\\
    $i \getsr \advA^{\REPO,\QRYO,\UPO}$;
    return $\big[\sum_x \err_i[x] \geq r\big]$
  \\[6pt]
  \oraclev{$\HASHO(\salt \cat x)$}\hfill\diffminus{$\game_0$}\diffplus{$\game_1$}\\[2pt]
    \diffminus{$\vv \gets F_K(\salt \cat x)$}\\
    \diffplusbox{$\vv \getsr [m]^k$\\
    if $T[Z,x] = \bot$ then $\vv \gets T[Z,z]$\\
    $T[Z,x] \gets \vv$; return $\vv$}
  \\[6pt]
  \oraclev{$\QRYO(i, \qry_x)$}\\[2pt]
    $X \gets \bmap_m(\HASHO(\salt_i \cat x))$;
    $a \gets X = M_i \AND X$\\
    if $\err_i[x] < \delta(a,\qry_x(\col_i))$ then
          $\err_i[x] \gets \delta(a,\qry_x(\col_i))$\\
    return $a$
  \\[6pt]
  \oraclev{$\REPO(\col)$}\\[2pt]
    $ct \gets ct+1$;
    $\setS_{ct} \gets \col$;
    $\salt_{ct} \getsr \bits^\lambda$;
    $c_{ct} \gets |\col|$\\
    $M_{ct} \gets \bigvee_{x \in \col} \bmap_m(\HASHO(\salt_{ct} \cat x))$;
    return $\langle M_{ct}, \salt_{ct}, c_{ct} \rangle$
  \\[6pt]
  \oraclev{$\UPO(i, \up_x)$}\\[2pt]
    if $w(M) > \ell$ then return $\top$\\
    if $\QRYO(\qry_x) = 1$ then $\err_i[x] \gets 0$\\
    $M_i \gets M_i \vee \bmap_m(\HASHO(\salt_i \cat x))$;
    $\setS_i \gets \up_x(\setS_i)$;
    return $\langle M_i, \salt_i, c_i+1\rangle$
}
{
  \vspace{-7pt}
  \experimentv{$\game_{2}(\advA)$}\hfill\diffplus{$\game_3$}\\[2pt]
    $\key \getsr \keys$;
    $ct \gets 0$;
    $\setZ \gets \emptyset$\\
    $i \getsr \advA^{\REPO,\QRYO,\UPO}$;
    return $\big[\sum_x \err_i[x] \geq r\big]$
  \\[6pt]
  \oraclev{$\REPO(\col)$}\\[2pt]
    $ct \gets ct+1$;
    $\setS_{ct} \gets \col$;
    $\salt_{ct} \getsr \bits^\lambda \setminus \setZ$;
    $c_{ct} \gets |\col|$\\
    $\setZ \gets \setZ \cup \{\salt_{ct}\}$\\
    $M_{ct} \gets \bigvee_{x \in \col} \bmap_m(\HASHO(\salt_{ct} \cat x))$;
    return $\langle M_{ct}, \salt_{ct}, c_{ct} \rangle$
  \\[6pt]
  \oraclev{$\QRYO(i, \qry_x)$}\\[2pt]
    $X \gets \bmap_m(\HASHO(\salt_i \cat x))$;
    $a \gets X = M_i \AND X$\\
    if $\err_i[x] < \delta(a,\qry_x(\col_i))$ then
          $\err_i[x] \gets \delta(a,\qry_x(\col_i))$\\
    \diffplus{$\UPO(i, \up_x)$;}
    return $a$
  \\[6pt]
  \experimentv{$\game_{4}(\advB)$}\\[2pt]
    $\key \getsr \keys$;
    $ct \gets 0$;
    $\setZ \gets \emptyset$\\
    $i \getsr \advB^{\REPO,\QRYO}$;
    return $\big[\sum_x \err_i[x] \geq r\big]$
  \\[6pt]
  \oraclev{$\QRYO(i, \qry_x)$}\\[2pt]
    for $i \in [ct]$ do\\
    $\tab X \gets \bmap_m(\HASHO(\salt_i \cat x))$;
    $a \gets X = M_i \AND X$\\
    $\tab$if $\err_i[x] < \delta(a,\qry_x(\col_i))$ then
          $\err_i[x] \gets \delta(a,\qry_x(\col_i))$\\
    $\tab\UPO(i, \up_x)$;
    return $a$
}
\caption{Games 0--4 for proof of Theorem~\ref{thm:bf-key-bound}.}
\label{fig:kbf-errep/games}
\end{figure*}

We start with a game~$\game_0$ which is essentially the same as the standard
\errep\ experiment on a Bloom filter, given the assumption (without loss of
generality) that the adversary never attempts to construct a representation for
a set with more than $n$ elements. As with the other proofs, it is easy to see
that for any such \errep\ adversary we can make an adversary $\advA$
for~$\game_0$ with the same resources that achieves the same advantage.

Unlike in the previous two proofs, we cannot use Lemma~\ref{thm:lemma1} because
an adversary cannot simulate the oracles without knowing the private key. We use
an alternate approach to gradually reduce to the standard binomial bound
deriving from the non-adaptive false positive probabilities. The first thing we
want to do is to bound the probability that the adversary can break the PRF.

The number of times the PRF is evaluated on distinct inputs is bounded by the
number of queries available to the adversary. In particular, $\QRYO$ and $\UPO$
each call the PRF once, while $\REPO$ may call the PRF up to $n$ times. If the
adversary runs in $t$ time steps, then, the probability it can distinguish the
PRF from a random function is bounded by $\Adv{\prf}_F(t,nq_R+q_T+q_U)$.
%
In~$\game_1$, we have a game which is identical to~$\game_0$ except that it uses
random sampling in place of the PRF. If $\advA$ cannot distinguish the PRF from
a random function then these games are indistinguishable from the adversary's
perspective, so $\Prob{\game_0(\advA) = 1} \le \Adv{\prf}_F(t,nq_R+q_T+q_U) +
\Prob{\game_1(\advA) = 1}$.
%
\cpnote{In fat, this isn't immediate. You show this by a reduction. You want to
show that for every $\advA$ there exists an adversary~$D$ such that
$\Prob{\game_0(\advA)=1} - \Prob{\game_1(\advA)=1} \leq \Adv{\prf}_F(D)$. You
don't need to be super formal about it, but you do need to say how~$D$
executes~$\advA$ and what outputs.}

Our goal is to argue, in a similar manner as to the previous theorems, that all of the oracle calls are independent. In order to guarantee this we must deal with the possibility of a salt collision between different representations. In~$\game_2(\advA)$ we require that all salts be distinct between representations. By the birthday bound, collisions between randomly-generated salts occur with frequency at most $q_R^2/2^\lambda$, so $\Prob{\game_1(\advA) = 1} \le q_R^2/2^\lambda + \Prob{\game_2(\advA) = 1}$.

With guaranteed-unique salts, the result of each $\REPO$, $\UPO$, and $\QRYO$
call for a given representation is independent of the calls for all other
representations. By an almost identical argument to the proof of
Theorem~\ref{thm:sbf-erreps}, we can assume without loss of generality that the
adversary follows any $\QRYO$ call that does not find a false positive with an
$\UPO$ call to insert that element, and therefore move to~$\game_3(\advA)$,
which as in the previous proof automatically performs an update after each query
is made. Since the adversary never inserts the same element multiple times, we
can again conclude that without loss of generality the adversary never directly
invokes the $\UPO$ oracle.
%
\cpnote{As in the previous result, I think it's better to give an explicit
reduction, rather than say ``we can assume without loss of generality ..''}

Finally, we must deal with the possibility that the adversary chooses which
representations to target with $\UPO$ and $\QRYO$ calls based on the result of
$\REPO$, since some representations may be more full than others. In
game~$\game_4$,we deny the adversary direct access to the $\UPO$ oracle
because it is never needed, but we allow the adversary credit if a call to
$\QRYO$ produces an error in any of the representations that have been
constructed. Furthermore, the updates made by $\QRYO$ apply to all
representations that are not already full. Since all $\UPO$ calls are
identically and independently distributed, and having more elements in a filter
cannot decrease the false positive rate, the fact that some representations may
become full more quickly than they otherwise would have can only help the
adversary. Similarly, having $\QRYO$ count errors across all representations
never harms the adversary, and so the adversary's advantage may only increase
when moving to~$\game_4(\advB)$ \todo{DC}{$\advB$ is undefined at this point}. Therefore $\Prob{\game_3(\advA) = 1} \le
\Prob{\game_4(\advB) = 1}$, where $\advB$ is an adversary which behaves
identically to $\advA$ but which is syntactically distinct because it lacks the
unused $\UPO$ oracle.
%
\cpnote{Instead of making assumptions about~$\advA$'s behavior and arguing that
they're not without loss, just give an explicit reduction.}

We are now in a situation where we can apply the standard, non-adaptive error
bound. Let $\setX$ be the set of all queries $\qry_x$ made by the adversary over
the course of the game. As in the previous proof, we have $|\setX| \le q_T$.
However, $\qry_x$ may now cause a false positive in any of the representations.
The probability of causing a false positive in a specific representation is
still given by the non-adaptive false positive probability $p'$ for a Bloom
filter containing $n+r$ elements. Since the representations are independent of
each other, the probability of a false positive occurring in any of up to $q_R$
representations is at most $p'q_R$. We can therefore bound the adversary's
success probability using a binomial distribution, similar to before:
\begin{equation}
   \Prob{\game_4(\advB)=1} \le
     \sum_{i=r}^{q_T} \binom{q_T}{i}(p'q_R)^i(1-p'q_R)^{q_T-i} \,.
\end{equation}

Applying the usual Chernoff bound, we find
\begin{equation}
   \Prob{\game_4(\advB)=1} \le
     e^{r-p'q_Rq_T}\left(\frac{p'q_Rq_T}{r}\right)^r.
\end{equation}

So, substituting this bound back into the earlier advantage inequalities, we find the final bound of
\begin{equation*}
  \begin{aligned}
    \Adv{\errep}_{\Pi,\delta,r}(t, q_R,q_T,q_U,q_H) &\leq \\
      \Adv{\prf}_F(t,nq_R+q_T+q_U) & +
    \frac{q_R^2}{2^\lambda} +
    \left(\frac{p'q_Rq_T}{r}\right)^re^{r-p'q_Rq_T}
  \end{aligned}
\end{equation*}

\end{proof}
}

\full{
  \begin{proof}[Proof of Theorem~\ref{thm:kbf-errep}]
  Unlike in the previous two proofs, we cannot use Lemma~\ref{thm:lemma1} because
an adversary cannot simulate the oracles without knowing the private key. Instead,
the first thing we
want to do is to bound the probability that the adversary can break the PRF.

The number of times the PRF is evaluated on distinct inputs is bounded by the
number of queries available to the adversary. In particular, $\QRYO$ and $\UPO$
each call the PRF once, while $\REPO$ may call the PRF up to $n$ times. Thus
there are at most~$Q = q_T + q_U + nq_R$
queries to~$\PRFO$.
%
In~$\game_1$, we have a game which is identical to the original except that it
uses
random sampling in place of the PRF. If $\advA$ cannot distinguish the PRF from
a random function then these games are indistinguishable from the adversary's
perspective.
%
We exhibit a $O(t)$-time, \prf-adversary~$D$ making at most~$Q$ queries to its
oracle. Adversary~$D^{\PRFO}$ works by executing~$\advA$ in game~$\game_1$, except
that whenever the game calls \emph{its}~$\PRFO$, adversary~$D$ uses its
own oracle to compute the response.
%
Finally, when~$\advA$ halts, if the winning condition in~$\game_1$ is satisfied,
then~$D$ outputs~$1$ as its guess; otherwise it outputs~$0$.
%
Then conditioning on the outcome of the coin flip~$b$ in~$D$'s game, we have that
%
\begin{eqnarray}
  \Adv{\prf}_F(D) &=& \Prob{\game_0(\advA)=1} - \Prob{\game_1(\advA)=1} \,.
\end{eqnarray}
%
Next, our goal is to argue, in a similar manner as to the previous theorems, that all
of the oracle calls are independent. In order to guarantee this we must deal
with the possibility of a salt collision between different representations.
In~$\game_2(\advA)$ we require that all salts be distinct between
representations. By the birthday bound, collisions between randomly-generated
salts occur with frequency at most $q_R^2/2^\lambda$, so $\Prob{\game_1(\advA) =
1} \le q_R^2/2^\lambda + \Prob{\game_2(\advA) = 1}$.

With guaranteed-unique salts, the result of each $\REPO$, $\UPO$, and $\QRYO$
call for a given representation is independent of the calls for all other
representations. By an almost identical argument to the one in the proof of
Theorem~\ref{thm:sbf-erreps}, we can reduce from any $\advA$ to an adversary
$\advB$ which follows any $\QRYO$ call that finds a true negative with an
$\UPO$ call to insert that element, and therefore move to~$\game_3(\advB)$,
which as in the Theorem~\ref{thm:sbf-erreps} proof performs an update after each
query is made, with the guarantee that $\Prob{\game_1(\advA) = 1} \le
\Prob{\game_2(\advB) = 1}$.

Finally, we must deal with the possibility that the adversary chooses which
representations to target with $\UPO$ and $\QRYO$ calls based on the result of
$\REPO$, since some representations may be more full than others. In
game~$\game_4$, we allow the adversary credit if a call to
$\QRYO$ produces an error in any of the representations that have been
constructed. Furthermore, the updates made by $\QRYO$ apply to all
representations that are not already full. Since all $\UPO$ calls are
identically and independently distributed, and having more elements in a filter
cannot decrease the false positive rate, the fact that some representations may
become full more quickly than they otherwise would have can only help the
adversary.

We come again in a situation where we can apply the standard, non-adaptive error
bound, using the same sort of bound for the binomail distribution:
\begin{equation}
   \Prob{\game_4(\advB)=1} \le
     e^{r-p'q_Rq_T}\left(\frac{p'q_Rq_T}{r}\right)^r.
\end{equation}

So, substituting this bound back into the earlier advantage inequalities, we find the final bound of
\begin{equation*}
  \begin{aligned}
    \Adv{\errep}_{\Pi,\delta,r}(\advA) &\leq \\
      \Adv{\prf}_F(D) &  +
    \frac{q_R^2}{2^\lambda} +
    \left(\frac{p'q_Rq_T}{r}\right)^re^{r-p'q_Rq_T} \,.
  \end{aligned}
\end{equation*}

\end{proof}
}

The fact that both a key and a salt are used in the $\KBF$ construction is
critical. In particular, without the per-representation randomness given by the
salt, we would not be able to argue that $\UPO$ and $\QRYO$ calls are
independent across representations. On the contrary, seeing the representation
of a singleton set $\{x\}$ would immediately allow the adversary to test whether
$x$ was a member in every other representation that had been constructed, simply
by testing whether every bit set to $1$ in the representation of $\{x\}$ was also
set to 1 in other representations. Even in the \erreps\ game, using the $\REVO$
oracle on some representations leaks information about other representations,
and again we cannot use the argument that provides the above bound.

We note that Gerbet \etal~\cite{gerbet2015power} suggest using keyed
hash functions as one possibility for constructing secure filters, which is
equivalent in our terminology to using a keyed but unsalted filter.
%
The distinction is that Gerbet \etal assume that representations are kept
private indefinitely, an assumption similar to that underlying our \erreps\
game, but with the stronger restriction that the adversary has no equivalent of
a $\REVO$ oracle. This makes their notion of security much weaker than ours with
respect to keyed structures.

\subsection{$\ell$-thresholded BFs}\label{sec:bf-thresh}

\begin{figure}
  \twoColsNoDivide{0.33}
  {
    \underline{$\Rep^R_K(\col)$}\\[2pt]
      $\salt \getsr \bits^\lambda$ \com{Choose a salt $\salt$}\\
      $\pub \gets \langle 0^m, \salt\rangle$\\
      for $x \in \col$ do\\
        $\tab \pub \gets \Up^R_K(\pub,\qry_x)$\\
        $\tab$if $\pub = \bot$ then return $\bot$\\
      return $\pub$
  }
  {
    \underline{$\Qry^R_K(\langle M, \salt \rangle,\qry_x)$}\\[2pt]
      $X \gets \bmap_m(R_K(\salt \cat x))$\\
      return $M \AND X = X$
    \\[6pt]
    \underline{$\Up^R_K(\langle M, \salt \rangle,\qry_x)$}\\[2pt]
      if $\hw(M) > \ell$ then return $\bot$\\
      return $\langle M \vee \bmap_m(R_K(\salt \cat x)), \salt \rangle$
  }
  \caption{The class of $\ell$-thresholded Bloom filters is given by
  $\bloom_\mathrm{ft}[R,\ell,\lambda] = (\Rep^R,\Qry^R,\Up^R)$. This is a slight
  variant of $n$-capping wherein we use the Hamming weight of the filter ($\hw$,
  as defined in Section~\ref{sec:prelims}) to decide if the filter is full.}
  \label{fig:bft-def}
  \vspace{-4pt}
\end{figure}

So far we have proven bounds for only $n$-capped BFs. It is important to understand
the security of this class of structures because it is representative of
how BFs are used in practice.
%
In this section we demonstrate that we can improve security bounds by defining
``fullness'' in terms of the Hamming weight of the filter, rather than the number of
elements it represents.
%
The general form of this alternate construction is formally defined in
Figure~\ref{fig:bft-def}. We can define the more specific constructions
$\BF_\mathrm{ft}[H,\ell]$, $\SBF_\mathrm{ft}[H,\ell,\lambda]$, and
$\KBF_\mathrm{ft}[H,\ell,\lambda]$ in an exactly the same way as the $n$-capped
variants. Here we only consider case of
$\Pi = \SBF_\mathrm{ft}[H,\ell,\lambda]$ and compare it to the $\SBF$
construction in Section~\ref{sec:sbf}.
%
The non-adaptive false positive probability is similar is similar to capped
filters, since the number of~$1$ bits in the filter can be closely predicted
from the number of elements in a randomly-selected underlying set. Because of
this, and because we are able to demonstrate better security bounds for an
$\ell$-thresholded filter than for a capped filter, we suggest this as a way
of providing strong security guarantees for even smaller filter sizes.

\begin{theorem}[\erreps\ security of thresholded BFs]
\label{thm:bf-thr-bound}
\label{thm:sbf-erreps-th}
Let $p_\ell = ((\ell+k)/m)^k$. For integers $q_R, q_T, q_U, q_H, q_V, r$, $t \geq 0$ such
that $r > p_\ell q_T$, it holds that
  \begin{equation*}
    \begin{aligned}
      \Adv{\erreps}_{\Pi,\delta,r}(t,\,&,q_R,q_T,q_U,q_H,q_V) \leq \frac{q_R(q_H+q_R)}{2^\lambda} + e^{r-p_\ell q_T}\left(\frac{p_\ell q_T}{r}\right)^r
        \,,
    \end{aligned}
  \end{equation*}
  where~$H$ is modeled as a random oracle.
\end{theorem}

From a technical point of view, the main difference between thresholded and
capped filters is that attacks cannot set more than $\ell+k$ bits of the filter
to 1, regardless of how the attack is conducted. The thrust of the proof is to
conservatively assume the adversary will always be able to produce such a maximally full
filter, and then use a standard binomial-distribution-based bound to place a
limit on the adversarial advantage even in this worst-case scenario.

\proc{
  \begin{proof}[Proof Sketch of Theorem~\ref{thm:sbf-erreps-th}]
  \begin{figure*}
  \cpnote{As above, these games need to be revised. There's no $\HASHO$ oracle
  here, since we're in the standard model!}
  \boxThmBFSaltCorrect{0.48}
  {
    \underline{$\game_0(\advA)$}\\[2pt]
      $\key \getsr \keys$; $i \gets 0$\\
      $\bot \getsr \advA^{\HASHO,\REPO,\QRYO,\UPO}$\\
      return $[\sum \err \geq r]$
    \\[6pt]
    \oraclev{$\REPO(\col)$}\\[2pt]
      $i \gets i + 1$\\
      $\col_i \gets \col$\\
      $\pub_i \gets \Rep_K[\HASHO](\col)$\\
      return $\pub_i$
    \\[6pt]
    \oraclev{$\QRYO(i,\qry_x)$}\\[2pt]
      $a \gets \Qry[\HASHO](\pub_i, \qry_x)$\\
      if $a \neq \qry_x(\col)$ then $\err_i[x] \gets 1$\\
      return~$a$
    \\[6pt]
    \oraclev{$\UPO(i,\up_x)$}\\[2pt]
      $a \gets \Qry[\HASHO](\pub_i, \qry_x)$\\
      if $\err_i[x] = 1$ then $\err_i \gets 0$\\
      $\col_i \gets \col_i \union \{x\}$\\
      $\pub_i \gets \Up[\HASHO](\pub_i,\up_x)$\\
      return~$\pub_i$
    \\[4pt]
    \oraclev{$\HASHO(x)$}\\[2pt]
      return $H(x)$
    %\hspace*{-4pt}\rule{1.043\textwidth}{.4pt}
  }
  {
  \underline{$\game_1(\advB)$}\\[2pt]\\
    \oraclev{$\HASHO(i,x)$}\\
      if $T[i, x]$ is $\undefn$ then $T[i, x] \getsr [m]^k$\\
      return $T[i, x]$
    \\[6pt]
    \underline{$\game_2(\advB)$}\\[2pt]\\
      $\key \getsr \keys$; $i \gets 0$; $\setC \gets \emptyset$\\
      $\bot \getsr \advA^{\HASHO,\REPO,\QRYO,\UPO}$\\
      return $[\sum \err \geq r]$
    \\[6pt]
    \oraclev{$\QRYO(\qry_x)$}\\[2pt]
      do\\
      $\tab y \getsr \mathcal{X}$\\
      while $y \in \col \cup \setC$\\
      $a \gets \Qry[\HASHO](\pub_i, \qry_y)$\\
      if $a \neq \qry_y(\col)$ then $\err_i[y] \gets 1$\\
      return~$a$
  }
  {
  }
  {
  }
  \caption{Games 0--2 for proof of Theorem~\ref{thm:bf-priv-key-bound}.}
  \label{fig:bf-priv-salt-bound}
\end{figure*}

\cpnote{In general, the same notes as above apply here. I can sort of see how
the argument works, but the lack of sufficient detail makes it difficult for me
to decide whether you're right.}

\cpnote{Tip: When beginning a proof, be very, VERY, \underline{\textbf{VERY!}}
clear about the initial game ($\game_0$). In this case, it should be essentiall
\errep\ with the given experiment paremeters $(\delta, r)$, the given adversary,
and the given scheme. Then you can start simplifyingi $\game_0$ in preparation
for the next step in the proof. As a rule of thumb, the simpler the transition
the better.  Sometimes that's hard to do, but you need to explain what changes
you're making between the games.}

\cpnote{As usual, start by naming an adversary that you're going to use in the
PRF reduction.}

The main observation is that seeing the representation of a salted, keyed Bloom
filter does nothing to tell the adversary about what the responses to $\QRYO$
will be. Using a conditioning argument, we move from $\game_0$, which is
equivalent to the standard \errep\ game, to the alternate $\game_1$ that uses a
lazily-evaluated random function in place of the PRF $F$ for hashing, with the
random function being different for each representation.
%
Provided that the adversary cannot distinguish the PRF from a random function
and provided that the per-representation salt never repeats (the probability of
which is on the order of $q_R^2/2^\lambda$ by the birthday bound \cpnote{change
``on the order of'' to ``at most''}), the adversary
cannot distinguish this from the original game.
%
\cpnote{Great start. This is a good the ``sketch'', but bear in mind you haven't
proven anything yet. In particular, you need to \emph{exhibit} a PRF adversary
whose advantage upper bounds the probability of~$\advA$ distngusihing between
$\game_0$ and $\game_1$. (This is the usual ``game-playing''
argument~\cite{bellare2006triple}.}

Next,\cpnote{Again, you haven't proven anything yet} since it never benefits the
adversary to re-query an element instead of querying a new one, and because
false negatives do not occur in Bloom filters, we can assume without loss of
generality that the adversary only makes queries to previously-unqueried
elements which are not in the underlying set. But if an element is not in the
underlying set, it must not have been included in the original $\col$ sent to
$\REPO$, and it must never have been inserted with $\UPO$ since Bloom filters do
not support deletion. Furthermore, since it has not been queried before, it has
not been tested with $\QRYO$ either. This means that each element being queried
is a new input to the random function used for hashing, and its output is
therefore indistinguishable from any other input that is provided. We can then
move from $\game_1$ to $\game_2$, which ignores the query given as input and
instead makes a random query to a previously-untested element. Since the outputs
of the $\QRYO$ oracle are indistinguishable from those in $\game_1$ and there
are no other changes, we have
%
%\todo{DC (lead)}{This doesn't make sense.}
%$\Adv{\game_1}_{\struct,r}(\advA) = \Adv{\game_2}_{\struct,r}(\advA)$.
%
But now
that the queries are random, the adversary cannot possibly do better than
producing a representation with maximal (non-adaptive) false-positive
probability and making as many arbitrary queries as possible. Given a threshold
where the proportion of 1 bits is capped at $p$, \cpnote{What's $p$? You mean
$\ell/m$?} the false positive probability for each query is bounded by $p^k$.
%
\cpnote{I'm not clear how you got that.}
%
By the properties of the binomial distribution, the probability of accumulating
at
least $r$ errors given $q_T$ queries is
%$$\Adv{\errep}_{\struct,r,d}(\advA) \le \Adv{\prf}(F) + q_R^2/2^\lambda + I_{p^k}(r, q_T-r+1)$$
\todo{DC}{Complete the bound. Bear in mind that $\Adv{\prf}(F)$ is not a
well-defined quantity! What you mean is $\Adv{\prf}_F(\advB)$, where $\advB$ is
a PRF adversary that you define.}
\end{proof}
}

\full{
  \begin{proof}[Proof of Theorem~\ref{thm:sbf-erreps-th}]
  As before, we assume without loss of generality that there are no insertions of
or queries for elements of $\col$.
%
To avoid the unfortunate $q_R$ factor in the bound, we do not make use of
Lemma~\ref{thm:lemma1} in this proof. Because of that, we must find some other
way to ensure that $\REVO$ is not useful to the adversary. In particular, if
there are unique salts across representations, the $\REPO$, $\QRYO$, and $\UPO$
calls for one representation will be independent of those for other
representations, since the unique salt is passed as part of the input. Therefore
in~$\game_1$ we specify that all salts created will be unique, but deny access
to $\REVO$. By the birthday bound, the probability of salts repeating
in the original game is no more than $q_R^2/2^\lambda$. Since $\REVO$ still does
not help the adversary, we can also remove the adversary's access to the $\REVO$
oracle in~$\game_1$.

Next, we want to ensure that the adversary's direct $\HASHO$ queries are
independent of those made indirectly through $\REPO$, $\QRYO$, and $\UPO$.
Since $\HASHO$ uses random sampling to fill a shared table, this
occurs if and only if the adversary hashes $\salt_i \cat x$
for some salt $\salt_i$ used by one of the representations created by $\REPO$.
By an argument very similar to that in the previous proofs, the adversary has at
most $q_H/2^\lambda$ probability of doing this for
each representation. However, since there are now $q_R$
representations, each with a distinct salt, there is at most a
$q_Rq_H/2^\lambda$ probability of the adversary correctly guessing a salt.
We therefore add a bad flag which is set if the adversary guesses a salt in this
manner, and bound away the probability as $q_Rq_H/2^\lambda$.

Now the adversary derives no advantage from guessing the salts of the
representations, in the sense that the outputs of $\HASHO$ are independent of the results
of the outputs of $\REPO$, $\QRYO$, and $\UPO$. The next oracle we want to target
is $\UPO$. Now that we have a filter threshold, we want to argue that the
adversary cannot use $\UPO$ to mount an effective attack.
%
To do this, we move to a game where the $\REPO$ oracle creates the filter as normal and then
randomly sets bits until filter is full (i.e., its Hamming weight is at least
$\ell$), which is $\ell+k$ (since
updates are not allowed when more than $\ell$ bits are set, and a single update
may set at most $k$ bits to 1). Due to the independence results for the different oracles,
we can argue that this does not decrease the adversary's probability of winning. % Trim from here:

We next change the game so that calls do not actually change the representation.
Since each
representation is created using independent $\HASHO$ outputs and then filled
in a uniform random manner until $\ell+k$ bits are set to 1, and is never
modified afterwards, the representations
themselves are random bitmaps which are uniformly distributed over the set of
$m$-length bitmaps with $\ell+k$ bits set to 1. This allows us to move
a game where the adversary is given a single arbitrary bitmap $\pub$
of length $m$ with $\ell+k$ bits set to 1 and makes $\QRYO$ calls exclusively
for $\pub$, winning if it produces $r$ errors for that `representation'. Since
the hashes used to construct the representation are independent of the hashes
used to query the representation, this does not decrease the adversary's
probability of winning.

Since $\QRYO$ calls with distinct inputs have independent outputs, each $\QRYO$ call made by $D$ has the same probability of producing a false positive. In particular, the probability of any one of the $k$ hashes colliding with a 1 bit is $(\ell+k)/m$, and the probability of all $k$ outputs doing so is then $((\ell+k)/m)^k$. If we let $\setX$ be the set of all inputs made to $\QRYO$, we again have a binomial distribution where $q_T$ queries are made. We can once more apply a Chernoff bound, as long as $p_\ell q_T < r$, to get the final bound of:
\begin{equation}
   \Adv{\erreps}_{\Pi,\delta,r}(\advA) \leq
     \frac{q_R(q_H+q_R)}{2^\lambda} + e^{r-p_\ell q_T}\left(\frac{p_\ell q_T}{r}\right)^r.
\end{equation}

\end{proof}
}


\subsection{Discussion}

These results show that the standard Bloom filter construction is weak to
adaptive adversaries. Moving to the salted $\SBF$ construction mitigates this,
but if filters are public they must be both large and immutable. In the \erreps\
setting updates do not break security and the minimum size of the filter to
guarantee a fixed error rate is considerably reduced. The guarantee (or,
equivalently, filter size) can be further improved, especially if the number of
representations constructed is large, by using $\ell$-thresholding.
Additionally, if the filters themselves cannot be kept private but a secret key
for the hash functions \emph{can} be concealed from adversaries, the $\KBF$
construction shows how to provide security in the \errep\ setting.

These requirements are more stringent than the mitigations suggested by Gerbet
\etal~\cite{gerbet2015power} due to our stronger attack model (where multiple
filters can be constructed, and sometimes revealed, to the adversary) and our
goal of establishing a general security bound for any adversary rather than
mitigating specific attacks. If~$q_R$ is small, our \erreps\ guarantee for
$\SBF$ and \errep\ guarantee for $\KBF$ show that filters need not be made much
larger than Gerbet \etal's in order to provide comparable security against more
general adversaries. If $q_R$ is large, however, the $q_R$ term in the error
bounds means that the filters must be made large to provide good error
guarantees. In this scenario, however, the $\ell$-thresholding class of filter
provides a way to get strong error guarantees without significantly increasing
the filter size.

\heading{Capping versus thresholding}
%
Figure~\ref{fig:bf-th} shows the dominant terms in the \erreps\ bounds for
$n$-capped and $\ell$-threholded salted BFs (Theorem~\ref{thm:sbf-erreps}
and~\ref{thm:sbf-erreps-th} respectively). This shows us that the bounds are
comparable for $\ell=nk$, which is a reasonable choice of $\ell$ given that a
set of size $n$ can set at most $nk$ bits to 1, and this upper bound only occurs
in the unlikely circumstance that there are no hash collisions during insertion.
When we take into account the factor of~$q_R$ present in the $n$-capped security
bound, we conclude that thresholding provides significantly more security if the
adversary is allowed a non-negligible number of calls to $q_R$.

\begin{figure}
  \begin{center}
  \hspace*{-10pt}
  \includegraphics[scale=0.8]{fig/bf-th-small}
  \includegraphics[scale=0.8]{fig/bf-th}
  \includegraphics[scale=0.8]{fig/bf-th-online}
  \includegraphics[scale=0.8]{fig/bf-th-big-online}
  \end{center}
  \caption{
    Performance of $n$-capped versus $\ell$-thresholded Bloom
    filters. The solid orange line shows the value of $\zeta_{k,m,n}(q,r)$ for
    $k=16$, $n=100$ (left) or $n=10^9$ (right), $q=2^{32}$, $r=1$ (top) or $r = 5$ (bottom), and varying $m$ (on the x-axis).
    %
    The dotted blue line shows the dominant term in the bound of
    Theorem~\ref{thm:sbf-erreps-th} for $\ell=nk$. The bounds are comparable,
    but thresholding would perform much better than capping for even
    modest~$q_R$.
  }
  \label{fig:bf-th}
\end{figure}

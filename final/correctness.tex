\begin{figure*}[t]
\fourColsNoDivide{0.22}{.18}{.275}{.25}
{
  \raisebox{-5pt}{\experimentv{$\Exp{\errep}_{\struct,\delta,r}(\advA)$
          \diffminus{$\Exp{\erreps}_{\struct,\delta,r}(\advA)$}}}\\[2pt]
    $\setP \gets \emptyset$; $\ct \gets 0$; $\key \getsr \keys$\\
    $i \getsr \advA^{\REPO,\UPO,\QRYO,\highlighto{\REVO}}$\\
    \diffminus{if $i \in \setP$ then return 0} \\[2.0pt]
    return $[\sum_\qry \err_i[\qry] \geq r]$
  \\[8pt]
  \diffplusbox{
  \oraclev{$\HASHO(X)$}\\[2pt]
    if $X \not\in \setX$ then return $\bot$\\
    if $T[X]=\bot$ then $T[X] \getsr \setY$\\
    return $T[X]$
  }\vspace{-4pt}
}
{
  \oraclev{$\REPO(\col)$}\\[2pt]
    $\pub \getsr \Rep_\key(\col)$\\
    if $\pub = \bot$ return $\bot$\\
    $\ct\gets\ct+1$ \\
    $\pub_\ct \gets \pub$\\
    %$\setC_\ct \gets \emptyset$\\
    $\col_\ct \gets \col$\\
    $\mathsf{rv} \gets \pub_\ct$; \diffminus{$\mathsf{rv} \gets \top$}\\
    return $\mathsf{rv}$
}
{
  \oraclev{$\UPO(i, \up)$}\\[2pt]
    %$X \gets \col^{v[i]}_i
    %$v[i] \gets v[i]+1$\\
    $\pub \getsr \Up_\key(\pub_i, \up)$\\
    if $\pub = \bot$ return $\bot$\\
    $\col_i \gets \up(\col_i)$\\
    $\pub_i \gets \pub$\\
    for $\qry$ in $\err_i$ do\\
    \tab $a \gets \Qry_K(\pub_i, \qry)$\\
    \tab if $\err_i[\qry] > \delta(a,\qry(\col_i))$ then\\
    \tab\tab$\err_i[\qry] \gets \delta(a,\qry(\col_i))$\\
    $\mathsf{rv} \gets \pub_i$; \diffminus{$\mathsf{rv} \gets \top$}\\
    return $\mathsf{rv}$
}
{
  \oraclev{$\QRYO(i, \qry)$}\\[2pt]
    $a \gets \Qry_K(\pub_i, \qry)$\\
    if $\err_i[\qry] < \delta(a,\qry(\col_i))$ then\\
    \tab$\err_i[\qry] \gets \delta(a,\qry(\col_i))$\\
    return $a$
  \\[6pt]
  \oraclev{$\REVO(i)$}\\[2pt]
    $\setP \gets \setP \cup \{i\}$ \\
    return $\pub_i$
}
  \caption{Two notions of adversarial correctness. The \errep\ notion captures
  correctness when the representation is always known to the adversary, while
  the \erreps\ notion captures correctness when the representation is secret.
  When modeling a function $H:\setX\to\setY$ as a random oracle, the $\HASHO$
  oracle is given to $\advA$, $\Rep$, $\Up$ and $\Qry$.}
  \vspace{6pt}\hrule
  \label{fig:security}
\end{figure*}


\newcommand{\oneColCCS}[2]{
  \makebox[.3\textwidth]{
    \begin{tabular}{|@{\gamespadleft}l@{\gamespad}|}
    \hline
    \rule{0pt}{1\normalbaselineskip}
    \begin{minipage}[t]{#1\textwidth}\gamesfontsize
      #2 \vspace{6pt}
    \end{minipage} \\
    \hline
  \end{tabular}
  }
}

\ignore{
\begin{figure}[t]
\oneColCCS{0.25}
{
\oraclev{$H(X)$}\\[2pt]
$V \getsr \mathcal{V}$\\
if $\mathrm{T}[X] \neq \undefn$ then $V \gets \mathrm{T}[X]$\\
$\mathrm{T}[X]\gets V$\\
return~$V$
}
\caption{Pseudocode for a Random Oracle~$H$ with outputs in set $\mathcal{V}$}
\label{fig:ROM}
\end{figure}
}

%We will measure error with the help of an error function $d: \mathcal{R}^2 \to
%[0,\infty)$, which may depend on the application rather than being given by the
%specification of a data structure alone. The value $\delta(x,y)$ represents the
%`badness' of getting an erroneous result of $x$ from $\Qry$ when $y$ should
%actually have been returned. In general we require $\delta(x,x) = 0$, but otherwise
%place no restrictions on what the error function might look like.

Let $\struct=(\Rep,\Up,\Qry)$ be a data structure with response space~$\results$.
%
We define two adversarial notions of correctness given by a pair of related
experiments for~$\struct$, \emph{error function} $\delta:\results^2 \to \R$,
and \emph{error capacity}~$r$. The values $\delta(x,y)$ of the error function
represent the ``badness'' of getting an erroneous result of $x$ from $\Qry$ when
$y$ should actually have been returned. In general we require
$\delta(x,y) \geq 0$ and $\delta(x,x) = 0$ for all $x$ and $y$, but otherwise
place no restrictions on what the error function might look like. For example,
in the case of Bloom filters we use a very simply error function:
$\delta(x,y) = 1$ for any $x \ne y$.

The two correctness notions are given by the experiments in
Figure~\ref{fig:security}. One corresponds to cases where the representations of
the true data are public (\errep) and the other to where they are private
(\erreps). We will describe the former and then give a brief explanation of how
the latter differs, as the two are closely related to each other.

Both experiments aim to capture the total \emph{weight} of the errors caused by
the adversary's queries. However, because we consider mutable data objects and
representations, we only give the adversary credit for $\QRYO$ calls that
produce errors in the ``current'' data objects $\col_i$ and their
representations $\pub_i$. Because we consider mutable data objects and
representations, the notion of ``current'' is defined by calls to the $\REPO$
and $\UPO$ oracles. In the case of Bloom filters, for example, we want to
keep track of all the false positives which have been found so far, except for
those false positives which have since been turned into true positives.

To track errors, both experiments maintain an array $\err_i[]$ for every data
object~$\col_i$ that has been defined.  Initially, $\err_i[]$ is implicitly
assigned the value of~$\undefn$ at every index. (We will silently adopt the
same convention for all uninitialized arrays.) For purposes of value
comparison, we adopt the convention that $\undefn < n$ for all $n \in \R$.
%
Now, the array~$\err_i$ is indexed by query functions~$\qry$, and the value of
$\err_i[\qry]$ is the weight of the error caused by~$\qry$, with respect to
the \emph{current} data object~$\col_i$ and \emph{current}
representation~$\pub_i$ (of~$\col_i$).
%
The value of~$\err_i[\qry]$ is updated within the $\QRYO$- and $\UPO$-oracles,
but observe that $\err_i[\qry] = \undefn$ until $(i,\qry)$ is queried to the
$\QRYO$-oracle.  Intuitively, a representation~$\pub_i$ of data object~$\col_i$
cannot surface errors until it is queried.

When~$\QRYO(i,\qry)$ executes, the value in $\err_i[\qry]$ is overwritten iff
the error caused by~$\qry$ is larger than the existing value of $\err_i[\qry]$.
The first time $(i,\qry)$ is queried to~$\QRYO$ this is guaranteed, since the
minimum possible value output by~$\delta$ is~$0$. After this, the adversary gets
credit only for making a worse error than the one already found. This prevents
the adversary from trivially winning by repeatedly sending the same
error-producing query to $\QRYO$. In our Bloom filter example, if $\QRYO$ finds
that $\qry$ is a new false positive, it will set $\err_i[\qry]$ to 1, showing
that an additional error has been produced.

When a query $\UPO(i,\up)$ is made, the oracle first updates the data
object~$\col_i$ and its corresponding representation.
%
Now, for each defined value~$\err_i[\qry]$, we re-evaluate the error that
\emph{would} be caused by the previously asked~$\qry$, with respect to the newly updated
$\col_i$ and $\pub_i$. If the existing value of $\err_i[\qry]$ is larger than
the error that~$\qry$ would cause (again, w.r.t.\ the newly updated~$\col_i$ and
$\pub_i$), then we overwrite $\err_i[\qry]$ with the smaller value.  Doing so
ensures that the array~$\err_i$ does not overcredit the attacker for errors
against the current data object and representation. For example, if $x$ was
previously found to be a false positive for a Bloom filter, and the adversary
then inserts $x$ into the data structure, we set $\err_i[\qry_x]$ to 0. Since
$x$ is now a true positive rather than a false positive, it should no longer
be counted as an error.

\erreps\ differs from \errep\ only in that the $\REPO$ and $\UPO$ oracles do not
reveal the representation to the adversary. This models the case where the data
structure is stored privately, where the adversary can ask queries but not see
the full representation. To model the possibility that information about a
representation is eventually leaked, we also give the adversary a $\REVO$ oracle
that reveals a given representation. However, to prevent this from being
trivially equivalent to the public-representation case we do not allow the
adversary to win by finding errors in a representation which has been revealed
using $\REVO$. Since the \errep\ adversary gets access to the same information
that \erreps\ does without the need for $\REVO$ calls, \errep\ security is a
stronger notion than \erreps\ security.

We define the advantage of an \errep-adversary~$\advA$ as
\[\Adv{\errep}_{\struct,\delta,r}(\advA) =
\Prob{\Exp{\errep}_{\struct,\delta,r}(\advA) = 1} \,\]
and write
$\Adv{\errep}_{\struct,\delta,r}(t,q_R,q_T,q_U,q_H)$ as the maximum advantage of
any \errep-adversary running in~$t$ time steps and making~$q_R$ calls to
$\REPO$, $q_T$ calls to $\QRYO$, $q_U$ calls to $\UPO$, and~$q_H$ calls
to~$\HASHO$ in the ROM.
%
We define \erreps\ advantage in kind, except that
we add an extra parameter~$q_V$ representing the number of calls to $\REVO$. We
sometimes use \errep1 or \erreps1 to refer to the restriction of the \errep\ or
\erreps\ games to the case of $q_R = 1$; for these we remove the~$q_R$ parameter
from the advantage function.

\ignore{
The first observation we make is that security in the public setting is a
stronger notion than security in the private setting. The following proof
follows from a simple simulation argument.

\begin{lemma}[\errep\ implies \erreps]\label{thm:errep-to-erreps}
  For every structure~$\Gamma$ and error function $\delta$ and any parameters
  $q_R, q_T, q_U, q_H, r, t \geq 0$, it holds that
  \begin{eqnarray*}
    \begin{aligned}
      \Adv{\erreps}_{\Gamma,\delta,r}(t, q_R, q_T, q_U, q_H, q_V) &\leq \\
      & \Adv{\errep}_{\Gamma,\delta,r}(t, q_R, q_T, q_U, q_H) \,,
    \end{aligned}
  \end{eqnarray*}
  where $f(t) = t + (q_R-1)\ticks(\Rep,t) + q_T\ticks(\Qry,t) + q_U\ticks(\Up,t)$.
\end{lemma}

\begin{proof}
Given an adversary $\advA$ for the \erreps\ game, we construct an adversary
$\advB$ for \errep\ that obtains at least the same advantage. At the beginning
of the experiment, $\advB$ initializes a list $L$ and integer $ct = 0$. It then
begins simulating $\advA$, answering its oracle queries as follows. Whenever
$\advA$ makes a $\REPO(\col)$ call, $\advB$ increments $ct$ and forwards the
request to its own oracle, setting $L[ct] \getsr \REPO(\col)$ and returning
$\top$ to $\advA$. Similarly, when $\advA$ makes an $\UPO(i,\up)$ call, $\advB$
sets $L[i] \getsr \UPO(i,\up)$ and returns $\top$. Any $\QRYO(i,\qry)$ calls
made by $\advA$ are directly forwarded to $\advB$'s own $\QRYO$ oracle, with the
response returned to $\advA$. Finally, when $\advA$ makes a $\REVO(i)$ call,
$\advB$ returns $L[i]$. When $\advA$ halts and returns $i$, $\advB$ does the
same. Since $\advB$ knows the actual representations at all times, it is able
to perfectly simulate the $\REVO$ oracle, and since all other oracle queries are
handled by forwarding to $\advB$'s oracle, the simulation of $\advA$ is perfect.
Then, because $\advB$ produces the exact same representations and makes the
exact same queries that $\advA$ does and produces the same final output, $\advB$
wins whenever $\advA$ wins.
\end{proof}

This lemma shows that if we can demonstrate an attack in the
private-representation setting, we know that the attack will also work in the
public-representation setting, and conversely that if we can establish a security
bound in the public-representation setting then the same bound holds in the
private-representation setting.
}

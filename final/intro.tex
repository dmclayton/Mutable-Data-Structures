Probabilistic data structures, which use space-efficient
representations of data to provide (approximately correct) answers to
queries about the data, find myriad uses in modern communication,
storage, and computational systems.  The Bloom
filter~\cite{bloom1970space}, for example, is
ubiquitous in distributed computing, including web caches (e.g., Squid) and hash
tables (e.g., BigTable and Hadoop), resource and packet routing, and network
measurement. (We refer the reader to the
surveys~\cite{broder2004network,tarkoma2012theory} for a comprehensive list of
applications.) 

The traditional approach to analyzing the correctness of a data structure is to
assume that all inputs, and all queries, are independent of any internal
randomness used to construct it.  But as highlighted by Naor and
Yogev (CRYPTO '15~\cite{naor2015bloom}), there are important use-cases in which the inputs
and queries may be chosen \emph{adversarially} and \emph{adaptively}, based on
partial information and prior observations about the data structure. Attacks of
this sort can be used to disrupt or reduce the availability of real systems
\cite{crosby2003denial,gerbet2015power,lipton1993clocked}.

Naor and Yogev (hereafter NY) formalized a notion of adversarial correctness for 
Bloom-filter-like structures.  Recall that a Bloom filter encodes a
set~$\col$ into a length-$m$ array of bits (initially all zeros), where~$m$ is much less than the
number of bits needed to store~$\col$ in full.  Elements $x \in \col$
are encoded by computing multiple hash values
$h_1(x),h_2(x),\ldots,h_k(x)\in [m]$, then setting the indicated array positions
to~$1$.  This bit-array representation of~$\col$ allows for
set membership queries, i.e., ``is $x\in\col$?'', 
by hashing~$x$ and responding positively iff all of the indicated
positions hold a 1-bit. 
%We will often refer to Bloom filters and similar
%structures as `representations' because they represent the data from a set or
%multiset without actually storing the entire (multi)set in full. 
False-negative respones are not possible, but false-positive responses
are.  Classical results relate $|\col|,m,k$ to the probability of false-positive query
responses~\cite{broder2004network,kirsch2006less}, where the
probability is over the sampling of the hash functions.  (These are
usually modeled as independent random functions.)  
Crucially, these results assume that $\col$ and the $h_1,\ldots,h_k$ are independent of
each other.  Said another way, even if~$\col$ is adversarially chosen,
this choice cannot depend on particular hash functions that are used
to produce the Bloom filter and compute the query responses.
%
The conceptual innovation of NY was remove this assumption an explore
the consequences upon the probability of Bloom filter query-response errors.  In
particular, NY allowed the adversary to specify a (fixed)
set~$\col$ that may depend on the hash functions, and then attempt to
induce errors via set-membership queries.

We expand upon NY in several ways, providing syntax and security
notions that allow analysis of a large class of data structures (not
only Bloom filters), in settings where the data may not be a set and
may change over time, and where the structure's representation of the data may (or may not) be publicly visible.
 
\paragraph{Beyond sets and Bloom filters}
We expand upon NY in several ways.  First, our
attack model allows the adversary to adaptively \emph{update} the
collection~$\col$ during its attack.  This captures settings in which
the data to be represented may change over time, e.g., streaming data applications.
Many data structures are designed for such settings ---~the counting filter~\cite{fan2000summary}, count-min
sketch~\cite{cormode2005improved}, cuckoo filter~\cite{fan2014cuckoo}, and
stable Bloom filter~\cite{deng2006approximately}, to name a few~--- by providing
updatable, or\emph{mutable}, representations.  Our syntactic formalization of
data structures captures this reality. 
%Attacks that treat representations as immutable (after creation) are captured as a special case.

%What all of these have in common is that they are designed to \emph{compactly}
%represent the data so that certain types of mutations and queries are supported,
%but a small amount of error is permitted.
%
Next, while the Bloom filter was designed to represent data collections~$\col$
that are sets, streaming data (for example) is more accurately 
modeled as a multiset.  Here one is often interested in information
about frequency, e.g., ``how many times does~$x$ appear
in~$\col$?''
%As with Bloom filters, the challenge is to answer this question with
%as little space consumption as possible, at the cost of admitting a reasonable
%amount of error.
Thus, in addition to admiting mutable respresentation, our
formalization of data structures allows for rich
query spaces.  Specifically, we define a data structure to be a triple of algorithms $(\Rep,
\Qry, \Up)$ denoting the \emph{representation}, \emph{query-evaluation}, and
\emph{update} algorithms, respectively. Associated to the data structure is a
set of supported query \emph{functions}~$\mathcal{Q}$, and a set~$\mathcal{U}$
of allowed update functions.  For reasons we will elucidate in a moment, all
three algorithms take a key~$\ky$ as input, and both~$\Rep$ and~$\Up$ may be
randomized.

The combination of mutability and rich query spaces has significant implications
for security. Consider the counting filter
structure~\cite{fan2000summary}.  It is similar to a Bloom filter, but 
instead of a bit array, a counting filter represents an updatable
multiset~$\col$ as an array of~$m$ integers; these serve as counters.
To add~$x$ to~$\col$, hash values $h_1(x), \ldots,
h_k(x)\in[m]$ are computed, and the indicated counters are
incremented.  Decrementing the counters implements \emph{deletion} of
an occurrence of~$x$ from~$\col$. %(Counters are typically floored at 0.)
%
Like a Bloom filter, a counting filter provide approximately correct
answers to set-membership queries\footnote{Indeed, they were initially
introduced to support deletions from a \emph{set}, without having to
rebuild the representation, as one would for a Bloom filter.}, where a
query about~$x$ results in a positive response iff all of the
hash-indicated counters are at least one.%
%
%
%
Unlike a Bloom filter, this structure admits both false-positive \emph{and} false-negative responses.
In particular, if the representation is updated by ``removing'' an element~$y$
that does not appear in the underlying~$\col$, one or more of the counters
associated to~$x$ may be decremented, potentially causing~$x$ to become a false
negative.

%In this paper, we consider the behavior of these structures in adversarial
%environments: under what circumstances can an adaptive adversary produce a large
%number of errors? We show that the standard implementations of these structures
%are not secure, but that with a series of simple and efficient embellishments we
%can establish reasonable provable security bounds. While we focus primarily on
%the familiar case of Bloom filters, we also show that our syntax and security
%notions can be used to capture other probabilistic structures by looking at the
%case of the count min-sketch.

Both the Bloom and counting filters have binary query responses,
making the notion of response error easy to define: the response is
correct, or it isn't.  But practically important structures, like the count-min sketch, admit
frequency-of-element queries, which have integer responses.  What
constitutes an error is less clear, when this is the case.
Even in the traditional analyses (i.e., non-adaptive attacks) one is
guaranteed only that responses will be ``close'' to correct,
with probability ``close'' to one. 
%In general, more structures with more complex data objects and queries may require a more sophisticated classification of errors than a simple binary indicator.
We therefore parameterize our security experiments with a specifiable
\emph{error function}~$\delta$.  If the correct response to an adversarial
query is~$a$ and the data structure responds with~$a'$, the
experiments award the adversary with an error weight $\delta(a,a') \geq 0$.
%
Our experiments are additionally parameterized by an \emph{error capacity}
$r\geq0$, and the adversary is considered to ``win'' if the total cost of the
errors it induces is greater than this value.  As it turns out, even calculating
this total cost is not straightforward in our setting: one must determine whether
or not the cost of a given error should be carried across (adaptive,
adversarial) updates to~$\col$ and its representation.

\paragraph{Public vs.\ Private Representations}
We define two experiments, one in which representations are shown to
the adversary, and one in which they are not.
%
In the \errep\ game, the adversary is given a
representation-oracle~$\REPO$ that, on input a collection~$\col$,
returns the resulting representation $\Rep_K(\col)$. Note that the
key~$K$ (which may be the empty string, to capture unkeyed structures)
is fixed across all calls; however, per-representation randomness
(e.g. salts) may be present.  The adversary is permitted to
(adaptively) update any established representation via an
update-oracle~$\UPO$ and, at any time, it may query a representation
via a query-oracle~$\QRYO$. The adversary is given credit (determined
by~$\delta$) for each $\QRYO$-query that results in an error.

The \erreps\ game is defined in much the same way, except that representations
are not shown to the adversary unless it explicitly asks for them to be
revealed.

Intuitively, \errep\
security is stronger than \erreps, so would seem the more desirable
target.  But for many of the structures we analyze here, security is only achievable in
the \erreps\ setting.
%
We note that the \erreps\ setting is of significant practical interest, as there are many
applications in which the adversary should not have unfettered access to the
structure~\cite{gerbet2015power}.

%To summarize, our high-level contributions are: formal syntax for
%mutable data structures, and two notions of adversarial
%correctness for these.  Our notions capture settings in which representations
%are made public, or kept private, respectively.

\paragraph{Case studies and our findings}
We exercise our syntax and notions by analyzing three important, real-world data
structures: Bloom filters~\cite{bloom1970space} (Section~\ref{sec:bloom}), count
min-sketches~\cite{cormode2005improved} (Section~\ref{sec:sketch}), and counting
filters~\cite{fan2000summary} (Appendix~\ref{sec:count}), summarized in
Figure~\ref{fig:tab-structures}. Each of these supports different queries and
update operations; taken together, they provide interesting coverage
of the structure/attack-model landscape.

\begin{figure}[tp]
\begin{center}
\small
  \begin{tabular}{ |p{2.5cm} | p{5cm}|}
    \hline
    {\bf Structure} & {\bf Results}\\ \hline
    \parbox[c][2.4cm]{2.5cm}{Bloom\;filter\\(Fig.~\ref{fig:bf-def}, Fig.~\ref{fig:bft-def})}
          & \parbox[c][2cm]{5cm}{Traditional implementation insecure.\\\emph{Immutable} structures can be secured with per-representation salt. \emph{Mutable} structures additionally require a secret key or\\keeping representations private, and benefit from thresholding.}
          \\ \hline
    \parbox[c]{2.5cm}{Counting filter (Fig.~\ref{fig:cbf-def})}
          & \parbox[c][1.6cm]{5cm}{Traditional implementation insecure.\\Security can be achieved by combining a per-representation salt, thresholding, and private representations.}
         \\ \hline
     \parbox[c]{2.5cm}{Count-min\;sketch\\(Fig.~\ref{fig:cms-def})}
          & \parbox[c][1.6cm]{5cm}{Traditional implementation insecure.\\Security can be achieved by combining a per-representation salt, thresholding, and private representations.}
          \\ \hline
  \end{tabular}
\caption{A high-level, informal summary of our results.}
  \label{fig:results-summary}
\end{center}
\end{figure}

We find
that \emph{none of these structures meets either of our security notions}, at least as
they are usually deployed. In particular, if the data being represented,
the updates, and the queries all may depend on the choice of hash function, then each of
these structures is susceptible to a class of attacks we call \emph{target-set
coverage attacks} (described in Section~\ref{sec:bad-bfs}). (These are closely
related to \emph{pollution attacks} against standard Bloom
filters~\cite{gerbet2015power}, which we will discuss in some detail.)
%
However, depending on the security setting (public vs.\ private
representation, updates allowed vs. not), these structures can be refined in
simple, intuitive ways that allow us to prove security.

\paragraph{Bloom filters, our in-depth study}
Due to their wide-spread and varied use (and following NY), we begin
with a deep look at Bloom filters.
%
It is well-known that standard Bloom filters do not perform well in adversarial
settings~\cite{naor2015bloom,gerbet2015power}; we first corroborate these
findings via an explicit \erreps\ attack (Section~\ref{sec:bad-bfs}).
%
We then consider the security of several variants of the basic Bloom
filter for which we can derive correctness (i.e., security) bounds.
%
The first idea is to generate a short, random \emph{salt}, which we prepend to
the input of the hash. Thus, instead of computing $h_i(x)$ for each $1\leq i
\leq k$ we compute $h_i(Z \cat x)$, where~$Z$ is a short (say, 128-bit) string
chosen by the representation algorithm.
%
This leads to our first positive result, for this \emph{salted} Bloom filter, in the
public-representation setting when attacks treat representations as
immutable (i.e., updates are forbidden); this is Theorem~\ref{thm:sbf-errep-immutable}.
%
Following the traditional approach~\cite{broder2004network}, we model
the hash functions as random oracles 
(ROM)~\cite{BR93}.  Our security argument must account for any hash-exploiting
precomputation performed by the adversary via the random oracle. This leads to
fairly weak bounds, which means that larger filters must be used to achieve a
reasonable correctness upper bound (Figure~\ref{fig:bf-bound}). 
%
On the other
hand, we find far better bounds, even in the mutable setting, if the
representation is kept private (Theorem~\ref{thm:sbf-erreps}). 

We derive a similarly good bound for \emph{keyed} Bloom filters, which use a
secretly-keyed pseudorandom function (PRF) instead of a hash function
(in addition to salts). This result is in the mutable \emph{and}
public-representation setting (Theorem~\ref{thm:kbf-errep}), the
strongest attack model we formalize.

Normally, Bloom filters are considered to be ``full'' when some
pre-determined set size, or
\emph{capacity}, is reached.  Indeed, Bloom filter parameters are generally chosen
as a function of this maximum capacity~\cite{kirsch2006less}.
%
We explore an alternative definition of fullness, whereby the filter is deemed full
once the Hamming weight of the filter (i.e., the number of 1s) crosses a
pre-determined \emph{threshold}.  While the two definitions are more or
less interchangeable in the non-adaptive, traditional setting, we show
that this alternative definition has substantial
analytical value in adversarial environments.  In
Theorem~\ref{thm:sbf-erreps-th}, we reconsider the security of salted
BFs in the mutable, private setting, and exhibit substantially tighter bounds. In
particular, we find that as long as salts are reasonably large, we can use
a 1 kilobyte filter to store 100 objects, while incurring less than one in a million
chance of a \emph{single} false positive.  This holds even if the adversary is allowed
to completely control the filter's construction and make up to $2^{16}$ queries
(see Figure~\ref{fig:bf-th}).


\paragraph{Count min-sketches}
Following the deep dive into Bloom filters in Section~\ref{sec:bloom}, we then
consider count min-sketches (CMS), which provide a compact representation of a
multiset, allowing additions and deletions, and yielding approximate queries for
approximate frequency of an element in the multiset. While a count-min sketch
hashes in much the same way as a Bloom filter, it uses a 2D array of non-negative
integer counters rather than a linear array of bits, allowing the structure to
keep track of how many times each counter is incremented.
%
We find that, CMSes are not secure in the
public-representation setting, even if we use a salt or use a PRF in place of
the hash function. The fact that the adversary can see exactly which filters are
incremented or decremented with each update, along with the fact that updates can be
trivially reversed (deletion undoes insertion and vice versa) allows the
adversary to mount attacks by trial and error even if it lacks the ability to
predict in advance where an element will be sent by the hash functions.
%
However, we are able to derive a good correctness in the mutable/private setting
(Theorem~\ref{thm:scms-erreps-th}), using a per-representation salt and a
notion of ``fullness'' similar to threshold Bloom filters.

\heading{Counting filters}
In Appendix~\ref{sec:count}, we include a discussion of the security
of counting filters.
%
In a loose sense, these fall somewhere between the CMS and Bloom
filters: they admit both addition and deletion
updates to the representation (like CMS), but only support set-membership queries.
%
Due to their structural similarities, CMS and counting filters
exhibit similar security properties (Theorem~\ref{thm:counting-erreps}).

\ignore{
%
The central aim of this work is to provide a formal framework for
analyzing probabilistic data structures in adversarial environments,
and to establish the first provable security results for real-world
structures. Our treatment of Bloom filters explores, somewhat deeply,
a neighborhood of designs around the classic structure.  Our
analysis of count min-sketches and counting filters exhibit the broader
applicability of our framework.

The security story is nuanced, and multi-dimensional.  Whether or not
a given data structure is secure depends not only on
the cryptographic primitives it employs (i.e., hash functions or PRF), but also on what
sort of queries the data structure supports, and what sorts of updates are
allowed (and how these interact).  Moreover, highly similar structures
can exhibit structural similarities among different schemes does not always
correspond to similar security properties.
%
It is our hope that our work will catalyze further exploration of these kinds of
structures.
}

\begin{figure*}[tp]
\begin{center}
\small
  \begin{tabular}{ |p{1.75cm} | p{2.5cm} | p{2.95cm} | p{4cm} | p{3.7cm}|}
    \hline
    {\bf Structure} & {\bf Data Objects} & {\bf Supported Queries} & {\bf Supported Updates} & {\bf Parameters} \\ \hline
    \parbox[c]{1.5cm}{Bloom\\ filter (Fig.~\ref{fig:bf-def})}
          & \parbox[c][6ex]{2cm}{Sets\\$\col\subseteq \bits^*$} %, or\\ $\col \in \Func(\bits^*,\{0,1\})$}
          & $\qry_x(\col) = [x \in \col]$
          &  $\up_x(\col) = \col \cup \{x\}$
          & \parbox[c]{4cm}{$n$, max $|\col|$\\$k$, \# hash functions\\$m$, array size (bits)}
          \\\hline
     \parbox[c]{2cm}{$\ell$-thresholded\\ Bloom filter\\ (Fig.~\ref{fig:bft-def})}
          & \parbox[c]{2.5cm}{Sets\\ $\col \subseteq \bits^*$}
          & $\qry_x(\col) = [x \in \col]$
          & \parbox[c][10ex]{4cm}{$\up_x(\col) = \col \cup \{x\}$}
          & \parbox[c]{3.75cm}{$\ell$, max no. 1s in array\\$k$, \# hash functions\\$m$, array size (bits)}
          \\ \hline
     \parbox[c]{2cm}{Count-min\\ sketch (Fig.~\ref{fig:cms-def})}
          & \parbox[c]{2.5cm}{Multisets\\ $\col \in \Func(\bits^*,\N)$}
          & $\qry_x(\col) = \col(x)$
          & \parbox[c][10ex]{4cm}{$\up_{x,0}(\col)(x) = \col(x)+1$ \\ $\up_{x,1}(\col)(x) = \col(x)-1$ \\ $\up_{x,b}(\col)(y) = \col(y)$ for $x \neq y$}
          & \parbox[c]{3.75cm}{$\ell$, max no. nonzero counters\\$k$, \# hash functions and arrays\\$m$, array size (counters)}
          \\ \hline
    \parbox[c]{1.5cm}{Counting\\ filter (Fig.~\ref{fig:cbf-def})}
          & \parbox[c]{2.5cm}{Multisets\\ $\col \in \Func(\bits^*,\N)$}
          & $\qry_x(\col) = [\col(x) > 0]$
          & \parbox[c][10ex]{4cm}{$\up_{x,1}(\col)(x) = \col(x)+1$ \\ $\up_{x,-1}(\col)(x) = \col(x)-1$ \\ $\up_{x,b}(\col)(y) = \col(y)$ for $x \neq y$}
          & \parbox[c]{3.5cm}{$\ell$, max no. non-zero counters\\$k$, \# hash functions\\$m$, array size (counters)}
         \\ \hline
    \ignore{\parbox[c]{1.5cm}{Cuckoo\\ filter}
          & \parbox[c]{2.5cm}{Multisets\\ $\col \in \Func(\bits^*,\N)$}
          & $\qry_x(\col) = [\col(x) > 0]$
          & \parbox[c][10ex]{4cm}{$\up_{x,1}(\col)(x) = \col(x)+1$ \\ $\up_{x,-1}(\col)(x) = \col(x)-1$ \\ $\up_{x,b}(\col)(y) = \col(y)$ for $x \neq y$}
          & \parbox[c]{3.5cm}{$n$, max $|\col|$\\$m$, \# buckets\\$b$, bucket size (entries)\\$f$, fingerprint size (bits)}
          \\ \hline}
  \end{tabular}
\caption{The data structures that we consider. Each data structure yields a
space-efficient representation of its input data object and, in the presence of
non-adaptive attacks, provides approximately correct responses to the supported
queries.  For counting filters and count-min sketches, typical
implementations prevent updates that would cause $\col(x)-1 < 0$.}
  \label{fig:structures-summary}
  \label{fig:tab-structures}
\end{center}
\end{figure*}


\heading{Future work}
%
The focus of this work is the data structures themselves.  Even so, we
were only able to consider a handful of (important, real-world)
examples.  We hope that future work will apply our formalisms to the
many probabilistic data structures that exist.

Going into a different direction, future work should
also address how adversarial correctness impacts high-level protocols that use
these probabilistic data structures. A good example is content-distribution
networks~\cite{byers2002informed}, where many servers propagate representations
of their local cache to their neighbors. (In Section~\ref{sec:bloom} we will
touch briefly on the real-world attacks that are possible in this setting.) The
Bloom filter family alone has a wide range of practical applications, for
example in large database query processing~\cite{broder2004network}, routing
algorithms for peer-to-peer networks~\cite{reynolds2003efficient}, protocols for
establishing linkages between medical-record databases~\cite{schnell2011novel},
fair routing of TCP packets~\cite{feng2001stochastic}, and Bitcoin wallet
synchronization~\cite{gervais2014privacy}.
%
Analyzing higher-level primitives or protocols will require establishing
appropriate syntax and security notions for those, too; hence we leave this for
future work.
%
Another interesting direction is to consider what information data structures
leak via their public representations. A large variety of data structures with
interesting privacy properties have been proposed. For example, variants of
Bloom filters that ensure privacy of the \emph{query} have been
studied~\cite{bellovin2004privacy,nojima2009cryptographically}. These prior work
leave open the security of more conventional data structures, like those studied
in this paper.


\subsection{Related work}
\paragraph{Comparison with Naor-Yogev}
As previously noted, Naor and Yogev~\cite{naor2015bloom} were the first to
formalize adversarial correctness of Bloom filters.  Our work extends theirs
significantly in several directions. First, we consider abstract data
structures, rather than only set-membership structures.  Even with respect to
the specific case of correctness for set-membership structures, our work offers
several advantages as compared to the Naor-Yogev treatment.
%
One, our syntax distinguishes between the (secret) key and the public portion of
a data structure, an important distinction that is missing in their work.
%
Two, the Naor-Yogev definition of correctness allows the adversary to make
several queries, some of which may produce incorrect results; the attacker then
succeeds if it outputs a \emph{fresh} query that causes an error. This
separation seems arbitrary, and we propose instead a parameterized definition in
which the attacker succeeds if it can cause a certain number of (distinct)
errors during its entire execution.
%
Three, Naor and Yogev analyze the correctness of a new Bloom filter variant of
their own design. In contrast, we are mainly interested in analyzing existing,
real-world constructions to understand their security.

\paragraph{Other related works}
There is a long tradition in computer science of designing structures that
concisely (but probabilistically) represent data so as to support some set of
queries, and each of these structures has its own interesting security
characterisitcs~\cite{chazelle2004bloomier,cormode2005improved,DP08a,DF03,fredman1984storing,mironov2011sketching}.

We have already mentioned the ubiquity of Bloom filters in support of efficient
network communication and computing protocols.  They also find use in
security-critical environments, including spam filters, (distributed)
denial-of-service attack detection, and deep packet
inspection~\cite{tarkoma2012theory}.  Recently, Bloom filters were proposed as a
means of efficient certificate-revocation list (CRL)
distribution~\cite{larisch2017crlite}, a crucial component of public-key
infrastructures.

% Correctness attacks
Correctness of data structures in adversarial settings is well-motivated in the
security literature and in practice.
%
Perhaps the earliest published attack on the correctness of a data structure was due to
Lipton and Naughton~\cite{lipton1993clocked} who showed that timing analysis of
record insertion in a hash table allows an adversary to adaptively choose
elements so as to increase look-up time, effectively degrading a service's
performance.
%
Crosby and Wallach~\cite{crosby2003denial} exploited hash collisions to increase
the average URL load time in Squid, a web proxy used for caching content in
order to reduce network bandwidth.
%
More recently, Gerbet \etal~\cite{gerbet2015power} described \emph{pollution
attacks} on Bloom filters, whereby an adversary inserts a number of
adaptively-chosen elements with the goal of forcing a high false-positive rate.
Although some of their attacks exploit weak (i.e., non-cryptographic) hash
functions (as do~\cite{crosby2003denial}), their methodology is effective even
for good choices of hash functions.
%
They suggest revised parameter choices for Bloom filters (i.e., filter length and
number of hashes) in order to cope with their attacks.


\ignore{
Finally, we note that the dictionary construction considered in
Section~\ref{sec:dict} bares resemblance (at least structurally) to
\emph{garbled Bloom filters}, a tool used recently for efficient private-set
intersection~\cite{dong2013when,rindal2017improved}.
}

% NOTE(all) Below are notes and references we considered adding to related work.
\if{0}{
  Correctness in adversarial settings has been considered for broader ranges of
  data structures.  Mironov, Naor, and Segev~\cite{mironov2011sketching} studied
  a setting in which non-colluding parties interact with a third-party
  \emph{referee} in order to compute a function of their data: For example,
  whether their sets are equal, or the approximate size of their intersection.
  The parties, which share a common reference string, but otherwise do not
  communicate, send a concise \emph{sketch} of their data to the referee, who
  performs the computation and publishes the result  The adversary is modeled as
  a malicious party attempting to skew the result.
  %
  \cpnote{It would be interesting to see if there's a connection between our
  notion of correctness and their setting.}
}\fi

\if{0}{
  \emph{Secure indexes}, proposed by Eu-Jin Goh~\cite{goh2003secure}, structure
  a document so that it can be searched by keyword if the querying party has a
  special \emph{trapdoor} for the keyword. The party issuing trapdoors has a
  secret key.  \jnote{I'm not sure the work of Goh is super relevant. Or, if it
  is, then so is any searchable encryption scheme.}
  %
  \cpnote{I agree ... I included it since it was cited in the
  survey~\cite{tarkoma2012theory} as an example of a ``secure'' Bloom filter
variant.}
}\fi

\if{0}{
  Other security notions for data structures, beyond correctness and privacy,
  have been considered.  For example, \emph{authenticated data
  structures}~\cite{tamassia2003authenticated} allow a trusted third party to
  certify the validity of a query on a data set maintained by an untrusted
  server.
}\fi

\if{0}{
  We recommend reading the Naor-Yogev paper for a survey of related work and a
  discussion of related papers. Here we mention a few additional practical
  works, but stress that this only scratches the surface.
  %
  \jnote{Rather random collection of papers using Bloom filters and variants. I
  removed it for now, since it's not clear that they have any particular
relevance to us. I kept only the refs that seemed directly relevant.}
  %
  As previously mentioned, Bloom filters and their relatives are some of the most
  widely used data structures supporting set-membership queries. As examples,
  Hbase, the open-source implementation of Google's BigTable storage
  system~\cite{chang2008bigtable}, a Hadoop-based NoSql database designed to
  handle large datasets, includes an implementation of Bloom filters and
  counting Bloom filters, and he Squid proxy~\cite{fan2000summary} uses a Bloom
  filter as a ``summary'' of the set of URLs in its cache in order to improve
  latency for web-object retrieval. Reynolds and
  Vahdat~\cite{reynolds2003efficient} proposed an efficient distributed search
  engine that can be used to search for files containing a particular keyword.
  Their search engine maps the keywords of each file into a Bloom filter; a
  look-up of the keyword in the Bloom filter tells whether the node has files
  containing that keyword or not. Stochastic Fair Blue~\cite{feng2001stochastic}
  uses counting Bloom filter to manage non-responsive TCP traffic.
}\fi

\if{0}{
  \cite{gao2006internet} is an application of BFs for detecting pollution
  attacks on web caches.
  %
  \heading{Related work: attacks}
  \tsnote{Brought these back into the text just to help Chris get up to speed.}
  \begin{itemize}
    \item Niedermeyer et al., ``Cryptanalysis of Basic Bloom Filters Used for
      Privacy-Preserving Record Linkage'', breaking privacy of
      secret-hash-function Bloom filters. \tsnote{Journal of Privacy and
      Confidentiality, 2014}

    \item Gerbet, Kumar and Lauradoux, ``The power of evil choices in bloom
      filters''. \tsnote{DSN'15: Looks like a real goldmine of related work!}

    \item Crosby and Wallach, ``Denial of Service via Algorithmic Complexity
      Attacks'' \tsnote{Gives attacks on Squid}

    \item Gao et al., ``Internet Cache Pollution Attacks and Countermeasures''
  \end{itemize}

  %\ignore{
  \heading{Related work: definitions(?)}
  \begin{itemize}
    \item Nojima and Kadobayashi, ``Cryptographically Secure Bloom Filters''.
      \tsnote{Gives some security definitions for privacy. Quick scan, not super
      clear what they achieve. The definition of client-privacy (Definition 1) for
      example, makes no sense to me.  Actually, likewise for server-privacy
      (Definition 2).  Both seem vague and thoroughly underspecified.}

    \item Naor and Yogev

    \item Eujin Goh, ``Secure Indexes'' \tsnote{A secure index can
      be used for set membership.  Builds a secret-key data
      structure (an Index) that allows searching for keyword~$w$
      if one holds the trapdoor $T_w$ for~$w$, where the trapdoor
      depends on the secret key.  Main construction uses
      traditional Bloom filters and a PRF.  Construction appears
      quite inefficient, needing a very long secret key, turning a
      keyword~$w$ into a bunch of PRF outputs, and then storing
      each of these PRF outputs in the BF.  Haven't read the full
      analysis; don't know if this was ever published. }
      \jnote{Never published. I think this work uses Bloom filters
      for encrypted search; I don't remember the paper having much
      to say about Bloom filters themselves.}
  \end{itemize}

  \heading{Related work: constructions}
  \begin{itemize}
    \item Bellovin and Cheswick, ``Privacy-Enhanced Searches Using Encrypted Bloom
    Filters''.

  \item Kerschbaum , ``Public-Key Encrypted Bloom Filters with
    Applications to Supply Chain Integrity''.

  \item S\"{a}rell\"{a} et al., ``BloomCasting: Security in Bloom Filter Based Multicast''.

  \item Dong, Chen,
      Wen, ``When Private Set Intersection Meets Big Data: An Efficient and
      Scaleable Protocol'' \tsnote{``garbled bloom filters'', which actually store
      the set element by storing~$k$ xor-shares, one at each of the~$k$ hash
      indices (with care for reusing shares if hash collisions occur); also
      and``oblivious bloom intersection''}\tsnote{If the filter and the hash
      functions are public, there is a naive attack that works for some
      interesting parameters.}

  \item Tarkoma, Rothenberg, Lagerspetz ``Theory and Practice of Bloom Filters in Distributed
      Systems''
      %
      \tsnote{Great high-level coverage.  Only found preprint version though.}
      \cpnote{{ieeexplore.ieee.org/iel5/9739/6151681/05751342.pdf}}

    \item Durham, Kantarcioglu, Xue, Kuzu, Malin ``Composite Bloom Filters for
      Secure Record Linkage'' \tsnote{Per-field BFs, sampled and composed into
      single BF that is then permuted by a secret random permutation.  No clear
      statement of the problem that is being solved.  Should pull full version and
      get details.}
  \end{itemize}

  \heading{Related work: tangential}
  \begin{itemize}
    \item Chang and Mitzenmacher ``Privacy Preserving Keyword Searches on Remote Encrypted Data''.

    \item Mitzenmacher and Vadhan. ``Why Simple Hash Functions Work: Exploiting
      the Entropy in a Data Stream''.

    \item Dodis et al. ``Fuzzy Extractors: How to Generate Strong Keys from
      Biometrics and Other Noisy Data'' \tsnote{Introduces ``secure sketches'',
      which is a representation of a single-element set that is information
      theoretically private (up to some function of the min-entropy of the
      element); only tangentially related to ``sketches'' as defined in the Bloom
      filter literature.}
  \end{itemize}
}\fi



\ignore{
\tsnote{old stuff below here}

\ignore{ %possibly move elsewhere in the intro, or the opening to the
         %bloom filter section
Bloom filters are
ubiquitous in distributed computing, including web caches (e.g., Squid) and hash
tables (e.g., BigTable and Hadoop), resource and packet routing, and network
measurement. (We refer the reader to the
surveys~\cite{broder2004network,tarkoma2012theory} for a comprehensive list of
applications.) 
Bloom filters have also been modified and co-opted for security-critical
applications; perhaps unsurprisingly, things go wrong. Schnell
\etal~\cite{schnell2011novel} proposed using secretly-keyed Bloom filters in
order to enable privacy-preserving record linkage (PPRL) across data sets.  This
was deployed in medical-data applications in Australia, Brazil, Germany, and
Switzerland~\cite{niedermeyer2014cryptanalysis}. 
%As one exercise of our
%notions, we study their proposal in detail. % in Section~\ref{sec:bf-bigram}.
%
}


\heading{Data structures and their correctness.}
%
We formalize a data structure as a triple of algorithms $(\Rep, \Qry, \Up)$ denoting
the \emph{representation}, \emph{query-evaluation}, and \emph{Update} algorithms, respectively.
Associated to the data structure is a set of supported queries~$\mathcal{Q}$.
The representation algorithm is randomized, taking as input a
key~$\ky$ and a collection of data~$\col$, and returning a
representation~$\pub$ of~$\col$.  (To capture unkeyed data structures,
one sets $\ky=\varepsilon$.)
%
The deterministic query-evaluation algorithm~$\Qry$ uses~$\ky$ and $\pub$ in
order to respond to a requested query~$\qry \in \queries$ on~$\col$.
\textcolor{blue}{[[...]]}

For better efficiency, many data structures only approximately
represent the collection~$\col$. In this case, the query-evaluation
algorithm~$\Qry$ may err in its response to queries.  \oldstuff{Roughly
speaking,  our notion of adversarial correctness (\errep) captures how
difficult it is for an attacker (given $\pub$) to find~$r>0$ distinct queries on
which $\Qry$ returns an incorrect answer.}

For Bloom filters, the representation~$\pub$ includes a bit array~$M$ that
represents a set~$\col \subseteq \elts$ using hash functions
$h_1,\ldots,h_k$. The supported queries are the predicates
$\{\qry_x\}_{x\in\elts}$, where $\qry_x(\col)=1$ iff $x \in \col$. It is well
known that Bloom filters may have false positives, and their false-positive rate
for \emph{independently chosen} inputs and queries is well understood. (See
Appendix~\ref{sec:mitz}.) Our correctness notion quantitatively captures the
error rate even in the presence of an attacker that adaptively attempts to
induce errors. \textcolor{blue}{[[...]]}

We note that Naor and Yogev~\cite{naor2015bloom} were the first to formalize
adversarial correctness of Bloom filters and, indeed, their work
provided inspiration for this paper.  Our work significantly extends
theirs in several ways, as we will detail, shortly.  \textcolor{blue}{[[...]]}
% ss-rep
\if{0}{
  \anytodo{Several reviewers have made the same complaint : why these notions?
  In particular, are they interesting beyond an academic exercise?  We need to
  address this head-on.  One idea is to try to build something on top of these
  notions, but I really see that as a separate paper.  Unless we can build some
  \emph{well known} primitive... but I'm not sure what it would be, or how
  interesting.}
  %
  \cpnote{Alex Davidson's paper (ia.cr/2017/448) suggests that garbled Bloom
  filters (or some variation of them) can be used for private-set intersection. We
  could ask if privacy in our sense suffices for this application.
  But \ssrep is not the right notion since it requires a key, and \owrep is
  probably too weak. Davidson views GBFs as distributional virtual black-box
  obfuscators, which are stronger than \owrep-secure structures.}
  %
  \cpnote{To my thinking, these notions were originally devised from the
  perspective of what security properties do existing data structures admit. If
  our intention is to use these properties in order to achieve some higher-level
  goal, I don't think we have the right ones. Short of strengthening them, I think
  our best bet  is to \emph{own} our original perspective. To that end, the place
  we need the most motivation is \ssrep privacy of $\SKBF$, the PRF-based BF. See
  my comments in Section~\ref{sec:bf-prf} for two ways we've already thought of.}
}\fi

\heading{Constructions we analyze.}
%
We put our syntax and security notions to work in several case studies.
%
The brief description of Bloom filters given above was silent as to how the hash
functions $h_1, \ldots, h_k$ are chosen, and whether or not they are
public. In fact, these details have a significant effect on what notions of
security the resulting structure satisfies:
\begin{itemize}
  \item
    (Section~\ref{sec:bf}) If the hash functions are fixed and known to the
    attacker prior to the filter being constructed, the data structure offers
    neither correctness nor privacy for any practically interesting parameters.
    We show this by exhibiting explicit attacks and analyzing their performance.

  \item (Section~\ref{sec:bf-salt}) If \emph{salted} hash functions are used,
    and the adversary is given the salt only after the collection $\col$ is
    chosen, then %with modest changes to the parameters (i.e., the filter length and number of hashes), 
    the structure can achieve the same correctness guarantees in the adversarial setting as do Bloom filters in the traditional
    non-adversarial setting. 
    %(Our analysis here treats the hash functions as random oracles; the usual analysis treats them as ideal random functions.)
    We also show that this structure achieves our privacy notion of one-wayness.

  \item (Section~\ref{sec:bf-prf}) We explore a natural, keyed variant of a
    Bloom filter in which the hash functions are derived from a secretly keyed
    pseudorandom function. (This is similar to a construction proposed by Naor
    and Yogev~\cite{naor2015bloom}.) We show that this variant enjoys
    simulation-based privacy, as well as a tighter security bound for
    correctness than the salted Bloom filter.
\end{itemize}
%
\noindent
Our particular realization of the salted and secretly keyed Bloom filters
leverages results from Kirsch and Mitzenmacher~\cite{kirsch2006less} that allow
one to effectively implement $h_1,\ldots, h_k$ by making only two \emph{actual}
evaluations of an underlying hash function or PRF, respectively.
%
In addition to the comprehensive analysis of Bloom filters described above, we
also apply our definitions to:
\begin{itemize}
  \item (Section~\ref{sec:bf-bigram}) A keyed structure for privacy-preserving
    record linkage introduced by Schnell \etal~\cite{schnell2011novel}, and
    subsequently attacked by Niedermeyer
    \etal~\cite{niedermeyer2014cryptanalysis}. In our framework we are able to
    show precisely how their scheme breaks down.

  \item (Section~\ref{sec:dict}) A dictionary proposed by Charles
    and Chellapilla~\cite{charles2008bloomier2} that stores a set of~$n$
    key/value pairs, where the keys are arbitrary bitstrings and the values are
    of length at most~$m$, using just $O(mn)$ bits.
\end{itemize}
}

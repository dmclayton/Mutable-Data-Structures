\ignore{\begin{figure*}
\threeColsOneDivideUnbalanced{0.40}{0.27}{0.27}
{
  \vspace{-7pt}
  \experimentv{$\game_{0}(\advB)$}
      \hfill \diffplus{$\game_1$}\\[2pt]
    $M^* \gets \bot$;
    $\salt^* \getsr \bits^\lambda$\\
    $\advB^{\REPO,\QRYO,\HASHO_1}$;
    return $\big[\sum_x \err[x] \geq r\big]$
  \\[6pt]
  \oraclev{$\REPO(\col)$}\\[2pt]
    $M^* \gets \bigvee_{x \in \col} \bmap_m(\HASHO_2(\salt^* \cat x))$;
    $\setS^* \gets \col$;
    return $\langle M^*, Z^* \rangle$
  \\[6pt]
  \oraclev{$\QRYO(\qry_x)$}\\[2pt]
    $X \gets \bmap_m(\HASHO_3(\salt^* \cat x))$;
    $a \gets X = M^* \AND X$\\
    if $\err[x] < \delta(a,\qry_x(\col^*))$ then
          $\err[x] \gets \delta(a,\qry_x(\col^*))$\\
    return $a$
  \\[6pt]
  \oraclev{$\HASHO_c(\salt \cat x)$}\\[2pt]
    $\vv \getsr [m]^k$\\
    if $M^*=\bot$ and $\salt=\salt^*$ and $c=1$ then \com{Caller is~$\advB$}\\
    \tab $\bad_1 \gets 1$; \diffplus{return $\vv$}\\
    if $T[Z,x] = \bot$ then $\vv \gets T[Z,x]$\\
    $T[Z,x] \gets \vv$; return $\vv$
}
{
  \vspace{-2pt}
  \oraclev{$\HASHO_c(\salt \cat x)$}\\[2pt]
    $\vv \getsr [m]^k$\\
    if $M^*=\bot$ and $\salt=\salt^*$ and $c=1$ then\\
    \tab $\bad_1 \gets 1$; return $\vv$\\
    if $T[Z,x] = \bot$ then $\vv \gets T[Z,z]$\\
    $T[Z,x] \gets \vv$\\[2pt]
    \diffplusbox{
    \com{Caller is~$\advB$ or $\QRYO$}\\
    if $c=1$ or $c=3$ then\\
    \tab if $\salt \ne \salt^*$  then return $\vv$\\
    \tab $\Ans[x] \gets \bmap_m(\vv) = M^* \AND \bmap_m(\vv)$\\
    \tab if $\err[x] < \delta(\Ans[x],\qry_x(\col^*))$ then
    \tab\tab $\err[x] \gets \delta(\Ans[x],\qry_x(\col^*))$
    }
    return $\vv$
}
{
  \vspace{-7pt}
  \oraclev{$\QRYO(\qry_x)$}\
      \hfill \diffminus{$\game_1$} \diffplus{$\game_2$}\\[2pt]
    \diffminusbox{%
      $X \gets \bmap_m(\HASHO_3(\salt^* \cat x))$\\
      $a \gets X = M^* \AND X$\\
      if $\err[x] < \delta(a,\qry_x(\col^*))$ then\\
      \tab $\err[x] \gets \delta(a,\qry_x(\col^*))$
    }\\[2pt]
    \diffplusbox{
      $\HASHO_3(Z^* \cat x)$\\
      $a \gets \Ans[x]$
    }
    return $a$
}
\caption{Games 0, 1, and 2 for proof of Theorem~\ref{thm:sbf-errep-immutable}.}
\label{fig:sbf-errep-immutable/games}
\end{figure*}}

The proof makes use of the following lemma for keyless structures, which is
proved formally in the full version of the paper.

\begin{lemma}\label{thm:lemma1}
  For every $q_R, q_T, q_U, q_H, r, t \geq 0$ and keyless structure~$\Gamma$ it
  holds that
  \begin{eqnarray*}
    \begin{aligned}
      \Adv{\errep}_{\Gamma,\delta,r}(t,\,&q_R, q_T, q_U, q_H) \leq \\
      & q_R \cdot \Adv{\errep1}_{\Gamma,\delta,r}(O(t), q_T, q_U, q_H) \,,
    \end{aligned}
  \end{eqnarray*}
\end{lemma}
%
\noindent
The proof is by a fairly straightforward hybrid argument.
Because~$\Gamma$ is keyless, an \errep1 adversary can simulate $\REPO$, $\UPO$,
and $\QRYO$ calls itself. In the reduction, we choose a random $i \in [q_R]$ and
forward only the $i$th $\REPO$ call to our $\REPO$ oracle, simulating all other
$\REPO$ queries. Similarly, $\UPO$ and $\QRYO$ calls are forwarded only if they
relate to the $i$th representation, and otherwise are simulated. The \errep1
adversary wins in at least those cases where the \errep\ adversary outputs $i$
and then wins. Since there is a $1/q_R$ chance of the randomly-chosen $i$
matching the \errep\ adversary's output we obtain the factor of~$q_R$ in the
bound.

We begin the argument for the full theorem with $\advA$, an \errep\ adversary
making~$1$ query to~$\REPO$, $q_T$ queries to $\QRYO$, $0$ queries to $\UPO$,
and $q_H$ queries to the random oracle~$\HASHO$.
%
We make the following assumptions, all of which are without loss of generality.
%
First, all of~$\advA$'s $\QRYO$ queries follow its $\REPO$ query.
%
Second, we assume that $x\not\in\setS$ for all queries $\qry_x$ to $\QRYO$,
where~$\setS$ was the input to~$\advA$'s $\REPO$ query. This is without loss
because Bloom filters admit false positives, but not false negatives.
%
Third, we we assume that $|\setS| \leq n$; this is without loss because
otherwise~$\REPO$ outputs~$\bot$ and~$\advA$ gets no advantage.
%
Fourth, we assume that all of~$\advA$'s $\HASHO$ queries are of the form $Z\cat
x$, where $|Z| = \lambda$.

We next make a game-playing argument~\cite{bellare2006triple}, and then obtain the
final bound via application of Lemma~\ref{thm:lemma1}.
%
The high-level goal is to rewrite the game so that the probability that one
of~$\advA$'s queries runs up the score is precisely the non-adaptive false
positive probability.
%
In other words, our goal is to transition into a setting in which the Bloom
filter output by~$\REPO$ is independent of the outcome of~$\advA$'s other
queries.

The first new game we move to is $\game_0$, where the salt $\salt$ for the
representation of~$\setS$ is generated prior to executing~$\advB$, and
the~$\HASHO$ oracle sets a bad flag if~$\advB$ calls $\HASHO(\salt \cat x)$
before calling $\REPO$. Since this occurs with probability $q_H/2^\lambda$, we
can bound away the probability of the adversary prematurely guessing the salt
with a term of $q_H/2^\lambda$, as seen in the final bound.

With this probability established, we can move to a game $\game_1$ where the
adversary's random oracle is independent of the hash functions used to respond
to oracle calls. In this game the adversary's random oracle is incapable of
providing useful information.

In game $\game_2$, we alter the $\HASHO$ and $\QRYO$ oracles so that
the winning condition is computed by $\HASHO$ instead of $\QRYO$. The former
oracle maintains a table~$\Ans$, setting $\Ans[x] = \Qry^\HASHO(M, \qry_x)$
whenever $\salt \cat x$ is hashed and giving the adversary credit for an error
if this $\Qry$ does not match with the correct output. The $\QRYO$ oracle in
$\game_2$ simply runs $\HASHO$ and returns $\Ans[x]$ when $x$ is queried.
%
We are effectively giving the adversary credit for RO queries that result in
false positives for the representation of~$\setS$, but which it does not
explicitly ask of~$\QRYO$. Note that this change can only increase the
adversary's chance of success.

In $\game_2$, the adversary may produce an error through either $\HASHO$ or
$\QRYO$, so the number of chances the adversary has to produce an error is no
greater than $q = q_H + q_T$. Since the outputs of $\HASHO$ are uniformly
randomly distributed and independent of the adversary's random oracle, the
probability of any of these calls producing an error is equal to the
non-adaptive error probability for a Bloom filter, $P_{k,m}(n)$. The adversary's
chances of success are given by a binomial distribution, to which we can apply a
Chernoff bound:
%
\begin{equation}
  \Adv{\errep1}_{\Pi,\delta,r}(\advA) < \frac{q_H}{2^\lambda} + \left(\frac{pq}{r}\right)^re^{r-pq}
  \,.
\end{equation}
%
Applying Lemma~\ref{thm:lemma1} to move from the single-representation case to the
general case, we get our final bound of
\begin{equation}
  \Adv{\errep}_{\Pi,\delta,r}(\advA) \leq
    q_R \cdot \left[
      \frac{q_H}{2^\lambda} +
      \left(\frac{pq}{r}\right)^re^{r-pq}
    \right] \,.
\end{equation}

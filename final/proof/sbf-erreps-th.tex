As before, we assume without loss of generality that there are no insertions of
or queries for elements of $\col$.
%
To avoid the unfortunate $q_R$ factor in the bound, we do not make use of
Lemma~\ref{thm:lemma1} in this proof. Because of that, we must find some other
way to ensure that $\REVO$ is not useful to the adversary. In particular, if
there are unique salts across representations, the $\REPO$, $\QRYO$, and $\UPO$
calls for one representation will be independent of those for other
representations, since the unique salt is passed as part of the input. Therefore
in~$\game_1$ we specify that all salts created will be unique, but deny access
to $\REVO$. By the birthday bound, the probability of salts repeating
in the original game is no more than $q_R^2/2^\lambda$. Since $\REVO$ still does
not help the adversary, we can also remove the adversary's access to the $\REVO$
oracle in~$\game_1$.

Next, we want to ensure that the adversary's direct $\HASHO$ queries are
independent of those made indirectly through $\REPO$, $\QRYO$, and $\UPO$.
Since $\HASHO$ uses random sampling to fill a shared table, this
occurs if and only if the adversary hashes $\salt_i \cat x$
for some salt $\salt_i$ used by one of the representations created by $\REPO$.
By an argument very similar to that in the previous proofs, the adversary has at
most $q_H/2^\lambda$ probability of doing this for
each representation. However, since there are now $q_R$
representations, each with a distinct salt, there is at most a
$q_Rq_H/2^\lambda$ probability of the adversary correctly guessing a salt.
We therefore add a bad flag which is set if the adversary guesses a salt in this
manner, and bound away the probability as $q_Rq_H/2^\lambda$.

Now the adversary derives no advantage from guessing the salts of the
representations, in the sense that the outputs of $\HASHO$ are independent of the results
of the outputs of $\REPO$, $\QRYO$, and $\UPO$. The next oracle we want to target
is $\UPO$. Now that we have a filter threshold, we want to argue that the
adversary cannot use $\UPO$ to mount an effective attack.
%
To do this, we move to a game where the $\REPO$ oracle creates the filter as normal and then
randomly sets bits until filter is full (i.e., its Hamming weight is at least
$\ell$), which is $\ell+k$ (since
updates are not allowed when more than $\ell$ bits are set, and a single update
may set at most $k$ bits to 1). Due to the independence results for the different oracles,
we can argue that this does not decrease the adversary's probability of winning. % Trim from here:

We next change the game so that calls do not actually change the representation.
Since each
representation is created using independent $\HASHO$ outputs and then filled
in a uniform random manner until $\ell+k$ bits are set to 1, and is never
modified afterwards, the representations
themselves are random bitmaps which are uniformly distributed over the set of
$m$-length bitmaps with $\ell+k$ bits set to 1. This allows us to move
a game where the adversary is given a single arbitrary bitmap $\pub$
of length $m$ with $\ell+k$ bits set to 1 and makes $\QRYO$ calls exclusively
for $\pub$, winning if it produces $r$ errors for that `representation'. Since
the hashes used to construct the representation are independent of the hashes
used to query the representation, this does not decrease the adversary's
probability of winning.

Since $\QRYO$ calls with distinct inputs have independent outputs, each $\QRYO$ call made by $D$ has the same probability of producing a false positive. In particular, the probability of any one of the $k$ hashes colliding with a 1 bit is $(\ell+k)/m$, and the probability of all $k$ outputs doing so is then $((\ell+k)/m)^k$. If we let $\setX$ be the set of all inputs made to $\QRYO$, we again have a binomial distribution where $q_T$ queries are made. We can once more apply a Chernoff bound, as long as $p_\ell q_T < r$, to get the final bound of:
\begin{equation}
   \Adv{\erreps}_{\Pi,\delta,r}(\advA) \leq
     \frac{q_R(q_H+q_R)}{2^\lambda} + e^{r-p_\ell q_T}\left(\frac{p_\ell q_T}{r}\right)^r.
\end{equation}

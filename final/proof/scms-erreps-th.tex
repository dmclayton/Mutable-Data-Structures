Again we derive a bound in the \erreps1 case and then use Lemma~\ref{thm:lemma1} to
move from \erreps1 to the more general \erreps\ case. Because we are in the
\erreps1 case, we may assume without loss of generality that the adversary does
not call $\REVO$, since revealing the only representation automatically prevents
the adversary from winning.

As usual, we add a
bad flag that gets set if the adversary ever guesses the salt when making
a $\HASHO$ query. By an almost identical argument, we can
move to~$\game_1$, where the behavior is different only when the bad flag
is set, which occurs with probability no more than $q_H/2^\lambda$.

The key differences between this proof and the Bloom filter proof are the more
complex response space of $\QRYO$ ($\N$ rather than $\bits$) and the fact that
both elements of $\col$ and non-elements of $\col$ may produce errors.

In fact, deletion is never helpful to the adversary because deleting an element
does not affect whether that element is overestimated and can only decrease the
chance that other elements are overestimated. So we can assume that the
adversary does not delete elements.

Using a similar argument, we can show that without loss of generality we may
assume the adversary does not insert an element more than once. Since updates are
deterministic, re-inserting $x$ can only increment the same counters that were
incremented by the original insertion of $x$, and so this re-insertion cannot
cause~$y$ to become overestimated if it was not already.

As a third step, we move from~$\game_1$ to a~$\game_2$ where the adversary gains
$k+1$ `points' for finding a query which produces an
overestimate, but which prevents the adversary from querying elements of $\col$.
These extra points are necessary because, unlike in the case of a Bloom
filter, inserting an overestimated element $x$ can cause other elements of
$\col$ to become overestimated. In particular, if one of the counters
incremented by the insertion of $x$ is shared with an element of $\col$ that is
not overestimated, that element may become overestimated. However, if that
counter is shared with multiple elements of $\col$, that counter is already an
overestimate for all of the elements associated with it, and so no more than one
overestimate can be caused per counter incremented by the insertion of $x$.
Since inserting $x$ increments $k$ counters, at most $k$ errors can be caused in
this way. Moving to this game allows us to assume that an adversary does not
insert an element which is known to be overestimated, but does not decrease an
adversary's chances of success.

We are now dealing with an adversary that gains points when it finds any
overestimate, but which only makes queries to $x\not\in\col$. This means that an
error is simply a query that returns a value greater than 1.
Analogously to the proof of Theorem~\ref{thm:sbf-erreps-th}, we now move to a
game~$\game_3$ where $\REPO$ randomly fills the sketch to capacity after
inserting the elements of $\col$, so that each row has $\ell+1$ non-zero
counters. Again this does not reduce the adversary's chances of success, but
allows us to assume that the adversary never calls $\UPO$.

The probability of the adversary winning can now be given by another binomial bound. The
set of nonzero counters in each row is a uniformly random subset of $[m]$ of
size $\ell+1$. Since any query returning a nonzero value is a success for the
adversary, the probability of any particular $\QRYO$ call causing a
collision within a single row $i$ is $(\ell+1)/m$, and the probability of a
collision in every row (i.e. an error) is $((\ell+1)/m)^k$. The adversary has a
total of $q_T$ attempts, and wins if it accumulates $\lfloor r/(k+1) \rfloor$
successes, since each error gives it $k+1$ points. So, letting
$p_\ell = ((\ell+1)/m)^k$ and $r' = \lfloor r/(k+1) \rfloor$, we apply the usual
Chernoff bound and Lemma~\ref{thm:lemma1} to derive the final bound of
\begin{equation}
   \Adv{\erreps}_{\Pi,\delta,r}(\advA) \leq
     q_R \cdot \left[\frac{q_H}{2^\lambda} + e^{r'-p_\ell q_T}\left(\frac{p_\ell q_T}{r'}\right)^r\right].
\end{equation}

As with the proof of Theorem~\ref{thm:sbf-errep-immutable}, we derive a bound in
the \erreps1 case and then use Lemma~\ref{thm:lemma1} to move from \erreps1 to
the more general case of \erreps. Because we are in the \erreps1 case, we may
assume without loss of generality that the adversary does not call $\REVO$,
since revealing the only representation automatically prevents the adversary
from winning.

As in the proof of
Theorem~\ref{thm:sbf-errep-immutable}, we first add a
$\bad_1$ flag that gets set if the adversary ever hashes with the
actual salt used by the representation. By a very similar argument, we can
move to~$\game_1$, where the behavior is different only when the $\bad_1$ flag
is set, which happens with probability at most $q_H/2^\lambda$.

Unlike in the case of a count min-sketch, it is entirely possible for deletions
to benefit the adversary in this game. In particular, if $x$ is found to be a
false positive, deleting $x$ may cause up to $k$ elements of $\col$ to become
false negatives. We therefore move to a game~$\game_2$ where the adversary gets
credit for either a single false positive or for $k$ false negatives whenever it
finds a false positive, but where the adversary cannot delete any false
positives that it finds. We let $r' = \lfloor r/\max(\delta^+,k\delta^-)\rfloor$
represent the number of false positives the adversary has to find in~$\game_2$
in order to win. In order to prevent the adversary from being penalized by the
filter becoming full too early, we also raise the threshold from $\ell$ to
$\ell+r'$ in~$\game_2$. Now for any $\advA$ for~$\game_1$, we can construct
$\advB$ for~$\game_2$ that simulates $\advA$, keeping track of all query
responses and forwarding all oracle queries in the natural way, except that
calls to delete false positives are ignored. Since $\UPO$ never fails for
$\advB$ due to the increased threshold, and since $\advB$ gets automatic credit
for any false negatives that might have been caused by deleting false positives,
$\advB$ succeeds whenever $\advA$ does, i.e.
$\Prob{\game_1(\advA)=1} \le \Prob{\game_2(\advB) = 1}$.

Since the remaining deletions do not cause errors, we can use the same argument
as in the proof of Theorem~\ref{thm:scms-erreps-th} to reduce from $\advB$ to an
adversary $\advC$ which does not make deletions at all. In~$\game_3$, we further
reduce from a counting filter to a normal Bloom filter by capping each of the
counters in the filter at 1. Since no deletions are performed, a counter
in~$\game_3(\advC)$ is nonzero if and only if the same counter
in~$\game_2(\advC)$ is nonzero. So $\QRYO$ behaves the same in~$\game_3$ as it
did in~$\game_2$, and $\Prob{\game_2(\advC)=1} \le \Prob{\game_3(\advC) = 1}$.

Note that~$\game_3$ is actually simulating an ordinary Bloom filter, since all
`counters' in the filter are restricted to the range $\bits$, there are no
deletions, and any insertions just set the corresponding bits to 1. In fact,
this game is identical to~$\game_2$ in the proof of
Theorem~\ref{thm:sbf-erreps-th} except that the adversary need only accumulate
$r'$ errors instead of $r$ errors and the threshold is $\ell+r'$ instead of
$\ell$. We can therefore use the same argument to produce the final bound of
\begin{equation}
   \Adv{\erreps}_{\Pi,\delta,r}(\advA) \leq
     q_R \cdot \left[\frac{q_H}{2^\lambda} + e^{r'-p_\ell q_T}\left(\frac{p_\ell
     q_T}{r'}\right)^{r'}\right] \,.
\end{equation}

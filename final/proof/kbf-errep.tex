Unlike in the previous two proofs, we cannot use Lemma~\ref{thm:lemma1} because
an adversary cannot simulate the oracles without knowing the private key. Instead,
the first thing we
want to do is to bound the probability that the adversary can break the PRF.

The number of times the PRF is evaluated on distinct inputs is bounded by the
number of queries available to the adversary. In particular, $\QRYO$ and $\UPO$
each call the PRF once, while $\REPO$ may call the PRF up to $n$ times. Thus
there are at most~$Q = q_T + q_U + nq_R$
queries to~$\PRFO$.
%
In~$\game_1$, we have a game which is identical to the original except that it
uses
random sampling in place of the PRF. If $\advA$ cannot distinguish the PRF from
a random function then these games are indistinguishable from the adversary's
perspective.
%
We exhibit a $O(t)$-time, \prf-adversary~$D$ making at most~$Q$ queries to its
oracle. Adversary~$D^{\PRFO}$ works by executing~$\advA$ in game~$\game_1$, except
that whenever the game calls \emph{its}~$\PRFO$, adversary~$D$ uses its
own oracle to compute the response.
%
Finally, when~$\advA$ halts, if the winning condition in~$\game_1$ is satisfied,
then~$D$ outputs~$1$ as its guess; otherwise it outputs~$0$.
%
Then conditioning on the outcome of the coin flip~$b$ in~$D$'s game, we have that
%
\begin{eqnarray}
  \Adv{\prf}_F(D) &=& \Prob{\game_0(\advA)=1} - \Prob{\game_1(\advA)=1} \,.
\end{eqnarray}
%
Next, our goal is to argue, in a similar manner as to the previous theorems, that all
of the oracle calls are independent. In order to guarantee this we must deal
with the possibility of a salt collision between different representations.
In~$\game_2(\advA)$ we require that all salts be distinct between
representations. By the birthday bound, collisions between randomly-generated
salts occur with frequency at most $q_R^2/2^\lambda$, so $\Prob{\game_1(\advA) =
1} \le q_R^2/2^\lambda + \Prob{\game_2(\advA) = 1}$.

With guaranteed-unique salts, the result of each $\REPO$, $\UPO$, and $\QRYO$
call for a given representation is independent of the calls for all other
representations. By an almost identical argument to the one in the proof of
Theorem~\ref{thm:sbf-erreps}, we can reduce from any $\advA$ to an adversary
$\advB$ which follows any $\QRYO$ call that finds a true negative with an
$\UPO$ call to insert that element, and therefore move to~$\game_3(\advB)$,
which as in the Theorem~\ref{thm:sbf-erreps} proof performs an update after each
query is made, with the guarantee that $\Prob{\game_1(\advA) = 1} \le
\Prob{\game_2(\advB) = 1}$.

Finally, we must deal with the possibility that the adversary chooses which
representations to target with $\UPO$ and $\QRYO$ calls based on the result of
$\REPO$, since some representations may be more full than others. In
game~$\game_4$, we allow the adversary credit if a call to
$\QRYO$ produces an error in any of the representations that have been
constructed. Furthermore, the updates made by $\QRYO$ apply to all
representations that are not already full. Since all $\UPO$ calls are
identically and independently distributed, and having more elements in a filter
cannot decrease the false positive rate, the fact that some representations may
become full more quickly than they otherwise would have can only help the
adversary.

We come again in a situation where we can apply the standard, non-adaptive error
bound, using the same sort of bound for the binomail distribution:
\begin{equation}
   \Prob{\game_4(\advB)=1} \le
     e^{r-p'q_Rq_T}\left(\frac{p'q_Rq_T}{r}\right)^r.
\end{equation}

So, substituting this bound back into the earlier advantage inequalities, we find the final bound of
\begin{equation*}
  \begin{aligned}
    \Adv{\errep}_{\Pi,\delta,r}(\advA) &\leq \\
      \Adv{\prf}_F(D) &  +
    \frac{q_R^2}{2^\lambda} +
    \left(\frac{p'q_Rq_T}{r}\right)^re^{r-p'q_Rq_T} \,.
  \end{aligned}
\end{equation*}

Just as in the proof of Theorem~\ref{thm:sbf-errep-immutable}, we will assume
the adversary just makes a single call to $\REPO$ and use Lemma~\ref{thm:lemma1}
to complete the bound. Let $\advA$ be an \erreps\ adversary making exactly 1
call to $\REPO$, $q_T$ calls to $\QRYO$, $q_U$ calls to $\UPO$, and $q_H$ calls
to $\HASHO$. Because $\advA$ creates only a single representation, it will
necessarily lose if it calls $\REVO$ on that representation. We may therefore
assume without loss of generality that $\advA$ makes no calls to $\REVO$.

In addition to the assumptions of Theorem~\ref{thm:sbf-errep-immutable}, we
assume without loss of generality that the adversary never uses $\UPO$ to insert
an element into $\col$ which is already present in the set, and never uses
$\UPO$ to insert an element $x$ where $\QRYO(\qry_x)$ has already been called
and has returned a positive result. Since these insertions do not change the
filter, the adversary would gain no advantage from performing these updates.
Furthermore, we assume without loss of generality that an adversary halts as
soon as it determines it has accumulated enough errors to win the experiment.

Since we are seeking a stronger bound, we now wish to isolate the possibility
that the adversary \emph{ever} guesses the salt, as opposed to just guessing the
salt before calling $\REPO$. This is no longer a trivial task for the adversary
because the representations are private, and so $\REPO$ does not directly reveal
the salt. We therefore set a bad flag whenever the adversary manages to guess
the salt. However, since the adversary is still limited to a total of $q_H$
$\HASHO$ queries, regardless of when the queries are made, we can follow nearly
the same argument as in the previous proof to bound away the probability of the
bad flag being set at $q_H/2^\lambda$. This is the same probability because the
probability of the adversary guessing the salt at any time in this experiment is
equal to the probability of the adversary guessing the salt in advance (before
$\REPO$ is called) in the previous game, and the previous game's adversary could
choose to make all of its random oracle queries before calling $\REPO$.

Once we have done this, we have a game $\game_1$ where the adversary's $\HASHO$
queries are independent of the $\HASHO$ queries made by the oracle: it is
irrelevant whether the adversary guesses the salt.
%
By our starting assumptions, the same element is
never inserted twice. We then take a general adversary $\advA$ for $\game_1$ and
construct $\advB$ that mimics $\advA$ but making new queries in place of repeat
queries. Since such an adversary is at least as likely to win, we can assume
without loss of generality that an adversary never queries the same element
twice, either. We then move from~$\advB$ to~$\advC$, which behaves the same
except that upon finding a true negative using $\QRYO$, it immediately calls
$\UPO$ to add the queried element to the data structure. Once~$\advC$ runs out
of $\UPO$ queries, it simply ignores further $\UPO$ calls from the
simulated~$\advB$.

By our earlier assumptions, there are only two types of update $\advB$ may make:
%
\begin{enumerate}
  \item Inserting an element which is not already in $\col$ and has previously been tested with $\QRYO$, returning a negative result.
  \item Inserting an element which is not already in $\col$ and has not previously been tested with $\QRYO$.
\end{enumerate}
The effects of type 1 updates
on the representation are identically distributed, as are the effects of type 2 updates on
the representation. However, the effects of the
two types of update are \emph{not} identically distributed compared to each
other. In particular, making a type 1 update ensures that at least one new bit
in the filter will be set to 1, since the distribution of
$\bmap_m(\HASHO(\salt \cat x))$ is conditioned on not producing a false
positive. On the other hand, making a type 2 update provides no guarantee about
how many bits in the filter might be set to 1. Type 1 updates are therefore
always preferable for an adversary attempting to produce false positives. % Trim from here:

All of the updates made by $\advC$ but not
by $\advB$ are type 1 updates, which are maximally effective at increasing the
error rate. Therefore any $\QRYO$ calls made by $\advB$ have at least the same
probability of causing an error when made by $\advC$, and so
$\advC$ has at least the same success rate with the same resources.

We next change to a game which
automatically inserts $x$ into $\col$ after computing the correct response to the
query, but which does not remove credit for inserting a false positive. Since false
positives being inserted does increment the total number of elements being
represented, to ensure that the adversary is not penalized in this game we
increase the maximum number of elements that can be inserted from $n$ to $n+r$.

We can then bound the adversary's chances of success with another binomial
distribution, where the adversary has $q_T$ chances and the probability of a
success is identical to that with a Bloom filter containing $n+r$ elements.
Applying a Chernoff bound, we get:
\begin{equation}
   \Prob{\game_2(E)=1} \le
     e^{r-p'q_T}\left(\frac{p'q_T}{r}\right)^r.
\end{equation}
%
Again we apply Lemma~\ref{thm:lemma1} to get a final bound of
\begin{equation}
  \Adv{\erreps}_{\Pi,\delta,r}(\advA) \leq
    q_R \cdot \left[
      \frac{q_H}{2^\lambda} +
      \left(\frac{p'q_T}{r}\right)^re^{r-p'q_T}
    \right] \,.
\end{equation}

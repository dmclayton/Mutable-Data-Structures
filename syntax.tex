\subsection{Preliminaries}
\label{sec:prelims}

Let $x \getsr \setX$ denote sampling~$x$ from a set~$\setX$ according to the
distribution associated with~$\setX$; if~$\setX$ is finite and the distribution
is unspecified, then it is uniform.
%
Let $[i..j]$ denote the set of integers $\{i, \ldots, j\}$; if $i > j$, then
define $[i..j] = \emptyset$. For all $m$ let $[m] = [1..m]$.

\heading{Bitstring operations}
Let $\bits^*$ denote the set of bitstrings and let~$\emptystr$ denote the empty
string.
%
Let $X \cat Y$ denote the concatenation of bitstrings~$X$ and~$Y$.
%
For all $m\geq0$ define~$\bmap_m$ as the following function.  For all
$\v.x\in[m]^*$ let $\bmap_m(\v.x) = X_1X_2\cdots X_m\in\bits^m$, where
$X_v=1$ if and only if $\v.x_i = v$ for some $i\in[|\v.x|]$.
%
We call $\bmap_m(\v.x)$ the \emph{bitmap} of~$\v.x$.
%
Let~$X$ and~$Y$ be equal-length bitstrings. We write $X \OR Y$ for their
bitwise-OR, $X \AND Y$ for their bitwise-AND, and $X \XOR Y$ for their
bitwise-XOR. Let $\NOT X = 1^{|X|} \xor X$, and let $\hw(X)$ denote the Hamming
wieght of (i.e., the number of 1s in)~$X$.
%
For an array~$\v.M$ of integers, we analogously define $\hw'(\v.m)$ to be the number
of \emph{nonzero} integers in the array. We also define $\zeroes(m)$ to be a
function taking a non-negative integer and returning a length-$m$ integer vector
with all entries set to~$0$.

Let $\Func(\setX,\setY)$ denote the set of functions $f:\setX\to\setY$. \tsnote{I have no idea what this
next sentence is meant to say.  ``If $\setY$, then...'' } If
$\setY$, then we associate to this set the following distribution. For all
$x\in\setX$ and $y\in\setY$ it holds that
$\Prob{f\getsr\Func(\setX,\setY):f(x)=y} = |\setY|^{-1}$. \tsnote{This
  uniform probability holds for any non-empty~$\setX,\setY$ by virtue of
  sampling $f \getsr\Func(\setX,\setY)$.  After all, $\Func(\setX,\setY)$ is a
  finite set, so if you leave the distribution unspecified, it is
  uniform.  (So you said, just above.)  Ich bin confused.}
%
\todo{TS}{Decide if you're happy with this. Note that it's used to define the
non-adaptive FP prob of stuff.} 
%
For every function~$f$, define $\id^f$ so that
$\id^f(\emptystr, x) = f(x)$ for all $x$ in the domain
of~$f$. \tsnote{This is underdefined.  Why does it take the first
  argument, and how is this defined when the first argument is not~$\emptystring$?}

\heading{Adversaries}
Adversaries are randomized algorithms that expect access to one or more oracles
defined by the experiment in which it is executed. We say that an adversary is
$t$-time if it halts in $t$ time steps (with respect to some model of
computation, which we leave implicit) regardless of its random coins or the
responses to its oracle queries. By convention, the adversary's runtime includes
the time required to evaluate its oracle queries.

\heading{Pseudorandom functions}
%
For sets $\setX$ and $\setY$ and a keyspace $\setK$, we define a pseudorandom
function to be a function $F: \setK \times \setX \to \setY$. The intent is for
the outputs of the function to appear random for a fixed choice of key, which is
formally captured by the game described in Figure~\ref{fig:prf-def}. We define
the advantage of an adversary $\advA$ to be
$\Adv{\prf}_F(\advA) = \Prob{\Exp{\prf}_F(\advA) = 1}$, and the advantage
$\Adv{\prf}_F(t,q)$ to be the maximum advantage of any adversary running for $t$
steps and making $q$ queries to the PRF oracle.
5
\cpnote{We could save a few inches here by changing the definition. I personally
like a nice, concrete PRF notion like this one, however.}

\begin{figure}
  \twoColsNoDivide{0.22}
  {
    \experimentv{$\Exp{\prf}_F(\advA)$}\\[2pt]
      $b \getsr \bits$; $\key \getsr \keys$\\
      $b' \getsr \advA^{\PRFO}$\\
      return $[b = b']$
  }
  {
    \oraclev{$\PRFO(x)$}\\[2pt]
      if $b = 1$ then return $F_\key(x)$\\
      if $T[x] \neq \bot$ then return $T[x]$\\
      $T[x] \getsr \setY$; return $T[x]$
  }
  \caption{The PRF experiment used to define the pseudorandomness of function
  $F$ with key space $\setK$.
  }
  \label{fig:prf-def}
\end{figure}

\subsection{Data Structures}
Fix non-empty sets $\mathcal{D}, \mathcal{R}, \keys$ of \emph{data objects},
\emph{responses} and \emph{keys}, respectively.  Let $\mathcal{Q}\subseteq
\Func(\mathcal{D},\mathcal{R})$ be a set of allowed \emph{queries}, and let
$\mathcal{U} \subseteq \Func(\mathcal{D},\mathcal{D})$ be a set of allowed
data-object \emph{updates}.  A {\em data structure} is a tuple $\Pi =
(\Rep,\Qry,\Up)$, where:

\begin{itemize}[leftmargin=*]
  \item $\Rep\colon \keys \times \mathcal{D} \to \{0,1\}^* \cup \{\bot\}$ is a
  randomized {\em representation algorithm}, taking as input a key $\key \in
  \keys$ and data object $\col \in \mathcal{D}$, and outputting the
  representation $\pub \in \{0,1\}^*$ of $D$, or $\bot$ in the case of a
  failure. We write this as $\pub \getsr \Rep_\key(\col)$.
%
  \item $\Qry\colon \keys \times \{0,1\}^* \times \mathcal{Q} \to \mathcal{R}$
  is a deterministic {\em query-evaluation algorithm}, taking as input $\key \in
  \keys$, $\pub \in \{0,1\}^*$, and $\qry \in \mathcal{Q}$, and outputting an
  answer $a \in \mathcal{R}$. We write this as $a \gets \Qry_\key(\pub,\qry)$.
%
  \item $\Up\colon \keys \times \{0,1\}^* \times \mathcal{U} \to \{0,1\}^* \cup
  \{\bot\}$ is a randomized {\em update algorithm}, taking as input $\key \in
  \keys$, $\pub \in \{0,1\}^*$, and $\up \in \mathcal{U}$, and outputting an
  updated representation $\pub'$, or $\bot$ in the case of a failure. We write
  this as $\pub' \getsr \Up_\key(\pub,\up)$.
\end{itemize}

Allowing each of the algorithms to take a key~$K$ lets us separate (in our
security notions) any secret randomness used across data structure operations,
from per-operation randomness (e.g., salts).  Note that our syntax admits the
common case of \emph{unkeyed} data structures, by setting
$\keys=\{\emptystring\}$.

We formalize $\Rep$ as randomized to admit defenses against offline attacks and,
as we will see, per-representation randomness will play an important role in
achieving our notion of correctness in the presence of adaptive adversaries.
Both~$\Rep$ and the~$\Up$ algorithm can be viewed (informally) as mapping data
objects to representations ---~explicitly so in the case of~$\Rep$, and
implicitly in the case of~$\Up$~--- so we allow~$\Up$ to make per-call random
choices, too.  Many common data structures do not have randomized representation
updates, but some do, e.g. the Cuckoo filter~\cite{fan2014cuckoo} and the stable
Bloom filter~\cite{deng2006approximately}.

The query algorithm $\Qry$ is deterministic.  This reflects the overwhelming
behavior of data structures in practice, in particular those with
space-efficient representations. It also allows us to focus on correctness
errors caused by the actions of an adaptive adversary, without attending to
those caused by randomized query responses.  Randomized query responses may be
of interest from a data privacy perspective, but our focus is on correctness.

\begin{figure*}[tp]
\begin{center}
\small
  \begin{tabular}{ |p{1.75cm} | p{2.5cm} | p{2.95cm} | p{4cm} | p{3.7cm}|}
    \hline
    {\bf Structure} & {\bf Data Objects} & {\bf Supported Queries} & {\bf Supported Updates} & {\bf Parameters} \\ \hline
    \parbox[c]{1.5cm}{Bloom\\ filter}
          & \parbox[c][6ex]{2cm}{Sets\\$\col\subseteq \bits^*$} %, or\\ $\col \in \Func(\bits^*,\{0,1\})$}
          & $\qry_x(\col) = [x \in \col]$
          &  $\up_x(\col) = \col \cup \{x\}$
          & \parbox[c]{4cm}{$n$, max $|\col|$\\$k$, \# hash functions\\$m$, array size (bits)}
          \\\hline
     \parbox[c]{2cm}{Count-min\\ sketch}
          & \parbox[c]{2.5cm}{Multisets\\ $\col \in \Func(\bits^*,\N)$}
          & $\qry_x(\col) = \col(x)$
          & \parbox[c][10ex]{4cm}{$\up_{x,0}(\col)(x) = \col(x)+1$ \\ $\up_{x,1}(\col)(x) = \col(x)-1$ \\ $\up_{x,b}(\col)(y) = \col(y)$ for $x \neq y$}
          & \parbox[c]{3.75cm}{$n$, max $|\col|$\\$k$, \# hash functions and arrays\\$m$, array size (counters)\\$d$, bits per counter}
          \\ \hline
    \parbox[c]{1.5cm}{Counting\\ filter}
          & \parbox[c]{2.5cm}{Multisets\\ $\col \in \Func(\bits^*,\N)$}
          & $\qry_x(\col) = [\col(x) > 0]$
          & \parbox[c][10ex]{4cm}{$\up_{x,0}(\col)(x) = \col(x)+1$ \\ $\up_{x,1}(\col)(x) = \col(x)-1$ \\ $\up_{x,b}(\col)(y) = \col(y)$ for $x \neq y$}
          & \parbox[c]{3.5cm}{$n$, max $|\col|$\\$k$, \# hash functions\\$m$, array size (counters)\\ $d$, bits per counter}
         \\ \hline
    \ignore{\parbox[c]{1.5cm}{Cuckoo\\ filter}
          & \parbox[c]{2.5cm}{Multisets\\ $\col \in \Func(\bits^*,\N)$}
          & $\qry_x(\col) = [\col(x) > 0]$
          & \parbox[c][10ex]{4cm}{$\up_{x,0}(\col)(x) = \col(x)+1$ \\ $\up_{x,1}(\col)(x) = \col(x)-1$ \\ $\up_{x,b}(\col)(y) = \col(y)$ for $x \neq y$}
          & \parbox[c]{3.5cm}{$n$, max $|\col|$\\$m$, \# buckets\\$b$, bucket size (entries)\\$f$, fingerprint size (bits)}
          \\ \hline}
  \end{tabular}
\caption{The data structrues that we consider. Each data structure yields a
space-efficient representation of its input data object and, in the presense of
non-adaptive attacks, provides approximately correct responses to the supported
queries.  For counting filters and count-min sketches, typical
implementations prevent updates that would cause $\col(x)-1 < 0$.}
  \label{fig:structures-summary}
  \label{fig:tab-structures}
\end{center}
\end{figure*}
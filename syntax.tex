\subsection{Preliminaries}
\label{sec:prelims}

Let $x \getsr \setX$ denote sampling~$x$ from a set~$\setX$ according to the
distribution associated with~$\setX$; if~$\setX$ is finite and the distribution
is unspecified, then it is uniform.
%
Let $[i..j]$ denote the set of integers $\{i, \ldots, j\}$; if $i > j$, then
define $[i..j] = \emptyset$. For all $m$ let $[m] = [1..m]$.

\heading{Bitstring operations}
Let $\bits^*$ denote the set of bitstrings and let~$\emptystr$ denote the empty
string.
%
Let $X \cat Y$ denote the concatenation of bitstrings~$X$ and~$Y$.
%
For all $m\geq0$ define~$\bmap_m$ as the following function.  For all
$\v.x\in[m]^*$ let $\bmap_m(\v.x) = X$, where $|X|=m$ and $X_v=1$ if and only if
$\v.x_i = v$ for some $i\in[|\v.x|]$
%
We call $\bmap_m(\v.x)$ the \emph{bitmap} of~$\v.x$.
%
Let~$X$ and~$Y$ be equal-length bitstrings. We write $X \OR Y$ for their
bitwise-OR, $X \AND Y$ for their bitwise-AND, and $X \XOR Y$ for their
bitwise-XOR. Let $\NOT X = 1^{|X|} \xor X$.
%
Finally, let $\hw(X)$ denote the Hamming wieght of (i.e., the number of 1s
in)~$X$.

Let $\Func(\setX,\setY)$ denote the set of functions $f:\setX\to\setY$.

For every function~$f$, define $\id^f$ so that
$\id^f(\emptystr, x) = f(x)$ for all $x$ in the domain of~$f$.

\todo{Any}{Define function $\ticks(A,\cdot)$ for algorithm~$A$. Do we need big-O
notation in bounds?}

\heading{Pseudorandom functions}
%
For sets $\setX$ and $\setY$ and a keyspace $\setK$, we define a pseudorandom
function to be a function $F: \setK \times \setX \to \setY$. The intent is for
the outputs of the function to appear random for a fixed choice of key, which is
formally captured by the game described in Figure~\ref{fig:prf-def}. We define
the advantage of an adversary $\advA$ to be
$\Adv{\prf}_F(\advA) = \Prob{\Exp{\prf}_F(\advA) = 1}$, and the advantage
$\Adv{\prf}_F(t,q)$ to be the maximum advantage of any adversary running for $t$
steps and making $q$ queries to the PRF oracle.

\begin{figure}
  \twoColsNoDivide{0.22}
  {
    \experimentv{$\Exp{\prf}_F(\advA)$}\\[2pt]
      $b \getsr \bits$; $\key \getsr \keys$\\
      $b' \getsr \advA^{\PRFO}$\\
      return $[b = b']$
  }
  {
    \oraclev{$\PRFO(x)$}\\[2pt]
      if $b = 1$ then return $F_\key(x)$\\
      if $T[x] \neq \bot$ then return $T[x]$\\
      $T[x] \getsr \setY$; return $T[x]$
  }
  \caption{The PRF experiment used to define the pseudorandomness of function
  $F$ with key space $\setK$.
  }
  \label{fig:prf-def}
\end{figure}

\subsection{Data Structures}
Fix non-empty sets $\mathcal{D}, \mathcal{R}, \keys$ of \emph{data objects},
\emph{responses} and \emph{keys}, respectively.  Let $\mathcal{Q}\subseteq
\Func(\mathcal{D},\mathcal{R})$ be a set of allowed \emph{queries}, and let
$\mathcal{U} \subseteq \Func(\mathcal{D},\mathcal{D})$ be a set of allowed
data-object \emph{updates}.  A {\em data structure} is a tuple $\Pi =
(\Rep,\Qry,\Up)$, where:

\begin{itemize}
  \item $\Rep\colon \keys \times \mathcal{D} \to \{0,1\}^* \cup \{\bot\}$ is a
  randomized {\em representation algorithm}, taking as input a key $\key \in
  \keys$ and data object $\col \in \mathcal{D}$, and outputting the
  representation $\pub \in \{0,1\}^*$ of $D$, or $\bot$ in the case of a
  failure. We write this as $\pub \getsr \Rep_\key(\col)$.
%
  \item $\Qry\colon \keys \times \{0,1\}^* \times \mathcal{Q} \to \mathcal{R}$
  is a deterministic {\em query-evaluation algorithm}, taking as input $\key \in
  \keys$, $\pub \in \{0,1\}^*$, and $\qry \in \mathcal{Q}$, and outputting an
  answer $a \in \mathcal{R}$. We write this as $a \gets \Qry_\key(\pub,\qry)$.
%
  \item $\Up\colon \keys \times \{0,1\}^* \times \mathcal{U} \to \{0,1\}^* \cup
  \{\bot\}$ is a randomized {\em update algorithm}, taking as input $\key \in
  \keys$, $\pub \in \{0,1\}^*$, and $\up \in \mathcal{U}$, and outputting an
  updated representation $\pub'$, or $\bot$ in the case of a failure. We write
  this as $\pub' \getsr \Up_\key(\pub,\up)$.
\end{itemize}

Allowing each of the algorithm to take a key~$K$ lets us separate (in our
security notions) any secret randomness used across data structure operations,
from per-operation randomness (e.g., salts).  Note that our syntax admits the
common case of \emph{unkeyed} data structures, by setting
$\keys=\{\emptystring\}$.

We formalize $\Rep$ as randomized to admit defenses against offline attacks and,
as we will see, per-representation randomness will play an important role in
achieving our notion of correctness in the presence of adaptive adversaries.
Both~$\Rep$ and the~$\Up$ algorithm can be viewed (informally) as mapping data
objects to representations ---~explicitly so in the case of~$\Rep$, and
implicitly in the case of~$\Up$~--- so we allow~$\Up$ to make per-call random
choices, too.  Many common data structures do not have randomized representation
updates, but some do, e.g. the Cuckoo filter~\cite{fan2014cuckoo} and the stable
Bloom filter~\cite{deng2006approximately}.

The query algorithm $\Qry$ is formalized as deterministic.  This reflects the
overwhelming behavior of data structures in practice, in particular those with
space-efficient representations.  It also allows us to focus on correctness
errors caused by the actions of an adaptive adversary, without attending to
those caused by randomized query responses.  Randomized query responses may be
of interest from a data privacy perspective, but our focus is on correctness.

\heading{Invertibility}
%
Some important data structures admit update operations that are ``inverses'' of
each other.  For example, the count-min sketch structure supports the addition
and deletion of elements from the underlying (multi)set that it represents.
Thus, we say that $\Pi$ is {\em invertible} if, for every representation $\pub$
, update $\up \in \mathcal{U}$, and key~$K \in \keys$, there is $\up' \in
\mathcal{U}$ such that \[\Prob{\Up(K,\Up(K,\pub,\up),\up') = \pub} = 1\,.\]

\ignore{
For example, count-min sketches are invertible, by consequence of being
deterministic and having both insertion and deletion operations. Counting Bloom
filters are not quite invertible because the counters have a minimum of 0.
Consider an element $a$ such that some but not all of the counters associated
with $a$ are 0. A deletion operation for an element $a$ will decrement only the
nonzero counters, after which an insertion operation for $a$ would increment all
of the counters. This shows that insertion and deletion are not perfect
inverses.
}

\todo{Any}{Fix the weird spacing in the last column of
Figure~\ref{fig:tab-structures}. Also, we need to refer to this \emph{somewhere}
in the body.}

\begin{figure*}[tp]
\begin{center}
\small
  \begin{tabular}{ |p{1.75cm} | p{2.5cm} | p{2.95cm} | p{4cm} | p{3.7cm}|}
    \hline
    {\bf Structure} & {\bf Data Objects} & {\bf Supported Queries} & {\bf Supported Updates} & {\bf Parameters} \\ \hline
    \parbox[c]{1.5cm}{Bloom\\ filter}
          & \parbox[c][6ex]{2cm}{Sets\\$\col\subseteq \bits^*$} %, or\\ $\col \in \Func(\bits^*,\{0,1\})$}
          & $\qry_x(\col) = [x \in \col]$
          &  $\up_x(\col) = \col \cup \{x\}$
          & \parbox[c]{4cm}{$n$, max $|\col|$\\$k$, \# hash functions\\$m$, array size (bits)}
          \\\hline
    \parbox[c]{1.5cm}{Counting\\ filter}
          & \parbox[c]{2.5cm}{Multisets\\ $\col \in \Func(\bits^*,\N)$}
          & $\qry_x(\col) = [\col(x) > 0]$
          & \parbox[c][10ex]{4cm}{$\up_{x,0}(\col)(x) = \col(x)+1$ \\ $\up_{x,1}(\col)(x) = \col(x)-1$ \\ $\up_{x,b}(\col)(y) = \col(y)$ for $x \neq y$}
          & \parbox[c]{3.5cm}{$n$, max $|\col|$\\$k$, \# hash functions\\$m$, array size (counters)\\ $d$, bits per counter}
         \\ \hline
    \parbox[c]{1.5cm}{Cuckoo\\ filter}
          & \parbox[c]{2.5cm}{Multisets\\ $\col \in \Func(\bits^*,\N)$}
          & $\qry_x(\col) = [\col(x) > 0]$
          & \parbox[c][10ex]{4cm}{$\up_{x,0}(\col)(x) = \col(x)+1$ \\ $\up_{x,1}(\col)(x) = \col(x)-1$ \\ $\up_{x,b}(\col)(y) = \col(y)$ for $x \neq y$}
          & \parbox[c]{3.5cm}{$n$, max $|\col|$\\$m$, \# buckets\\$b$, bucket size (entries)\\$f$, fingerprint size (bits)}
          \\ \hline
     \parbox[c]{2cm}{Count-min\\ sketch}
          & \parbox[c]{2.5cm}{Multisets\\ $\col \in \Func(\bits^*,\N)$}
          & $\qry_x(\col) = \col(x)$
          & \parbox[c][10ex]{4cm}{$\up_{x,0}(\col)(x) = \col(x)+1$ \\ $\up_{x,1}(\col)(x) = \col(x)-1$ \\ $\up_{x,b}(\col)(y) = \col(y)$ for $x \neq y$}
          & \parbox[c]{3.75cm}{$n$, max $|\col|$\\$k$, \# hash functions and arrays\\$m$, array size (counters)\\$d$, bits per counter}
          \\ \hline
  \end{tabular}
\caption{The data structrues that we consider. Each data structure yields a
space-efficient representation of its input data object and, in the presense of
non-adaptive attacks, provides approximately correct responses to the supported
queries.  For counting filters, cuckoo filters, and count-min sketch, typical
implementations prevent updates that would cause $\col(x)-1 < 0$.}
  \label{fig:structures-summary}
  \label{fig:tab-structures}
\end{center}
\end{figure*}
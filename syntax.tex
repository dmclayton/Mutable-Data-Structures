We start with a universe $\mathcal{D}$ of data objects, a key space $\keys$, a set $\mathcal{R}$ of responses, a set $\mathcal{Q} = \{\qry: \mathcal{D} \to \mathcal{R}\}$ of queries, and a set $\mathcal{U} = \{\up: \mathcal{D} \to \mathcal{D}\}$ of possible updates. A {\em mutable data structure} is a tuple $\Pi = (\Rep,\Qry,\Up)$, where:

\begin{itemize}
  \item $\Rep: \keys \times \mathcal{D} \to \{0,1\}^* \cup \{\bot\}$ is a randomized {\em representation algorithm}, taking as input a key $\key \in \keys$ and data object $\col \in \mathcal{D}$, and outputting the public representation $\pub \in \{0,1\}^*$ of $D$ or $\bot$ in the case of a failure. We write this as $\pub \getsr \Rep_\key(\col)$.
  \item $\Qry: \keys \times \{0,1\}^* \times \mathcal{Q} \to \mathcal{R}$ is a deterministic {\em query-evaluation algorithm}, taking as input $\key \in \keys$, $\pub \in \{0,1\}^*$, and $\qry \in \mathcal{Q}$, and outputting an answer $a \in \mathcal{R}$. We write this as $a \gets \Qry_\key(\pub,\qry)$.
  \item $\Up: \keys \times \{0,1\}^* \times \mathcal{U} \to \{0,1\}^* \cup \{\bot\}$ is a randomized {\em update algorithm}, taking as input $\key \in \keys$, $\pub \in \{0,1\}^*$, and $\up \in \mathcal{U}$, and outputting an updated representation $\pub'$ or $\bot$ in the case of a failure. We write this as $\pub' \getsr \Up_\key(\pub,\up)$.
\end{itemize}

Unkeyed data structures are a special case where $\keys = \{\epsilon\}$, and immutable data structures are a special case where the update algorithm returns $\textsc{Up}(\key,\pub,\up)=\pub$ for all inputs.  In the latter case, we will often drop mention of the update algorithm.


Say that a structure is {\em invertible} if for every representation $\pub$ and update $\up \in \mathcal{U}$ there is $\up' \in \mathcal{U}$ such that $\Prob{\Up(\Up(\pub,\up),\up' = \pub)} = 1$.  For example, count-min sketches are invertible, by consequence of being deterministic and having both insertion and deletion operations. Counting Bloom filters are not quite invertible because the counters have a minimum of 0. Consider an element $a$ such that some but not all of the counters associated with $a$ are 0. A deletion operation for an element $a$ will decrement only the nonzero counters, after which an insertion operation for $a$ would increment all of the counters. This shows that insertion and deletion are not perfect inverses.

%Each also has the additional feature of having a natural choice of {\em empty} structure such that every other structure can be constructed by starting with the empty structure and performing a corresponding sequence of update operations.

\heading{Discussion.} \tsnote{Here you should unpack what has just been formalized. What has been captured? Are any of the choices subtle/restrictive/overly permissive/etc.? If so, explain them.}

The $\Rep$ algorithm being randomized allows for the choosing of random factors such as a salt at the time the algorithm is run, which is important in preventing offline attacks against a structure from an adversary who can simulate the algorithm's performance in advance for any input. The use of a randomized $\Up$ algorithm is not necessary in most mutable data structures, since parameters are usually chosen upon creation of the set, but allowing randomized updates does not prevent a structure from working with the security notions we develop.

The query algorithm $\Qry$, on the other hand, must be deterministic in order for the definition of correctness to be valid. Since the use of randomized updates with deterministic queries is the predominant choice for probabilistic data structures, so that this syntactic choice does not exclude any of the most commonly-used data structures we would like to consider.

We will measure error with the help of an error function $d: \mathcal{R}^2 \to [0,\infty)$, which may depend on the application rather than being given by the specification of a data structure alone. The value $d(x,y)$ represents the `badness' of getting an erroneous result of $x$ from $\Qry$ when $y$ should actually have been returned. In general we require $d(x,x) = 0$, but otherwise place no restrictions on what the error function might look like.
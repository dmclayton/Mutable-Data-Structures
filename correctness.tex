\begin{figure*}[t]
\fourColsNoDivide{0.22}{.18}{.275}{.25}
{
  \raisebox{-5pt}{\experimentv{$\Exp{\errep}_{\struct,\delta,r}(\advA)$
          \diffminus{$\Exp{\erreps}_{\struct,\delta,r}(\advA)$}}}\\[2pt]
    $\setP \gets \emptyset$; $\ct \gets 0$; $\key \getsr \keys$\\
    $i \getsr \advA^{\REPO,\UPO,\QRYO,\highlighto{\REVO}}$\\
    \diffminus{if $i \in \setP$ then return 0} \\[2.0pt]
    return $[\sum_\qry \err_i[\qry] \geq r]$
  \\[8pt]
  \diffplusbox{
  \oraclev{$\HASHO(X)$}\\[2pt]
    if $X \not\in \setX$ then return $\bot$\\
    if $T[X]=\bot$ then $T[X] \getsr \setY$\\
    return $T[X]$
  }\vspace{-4pt}
}
{
  \oraclev{$\REPO(\col)$}\\[2pt]
    $\pub \getsr \Rep_\key(\col)$\\
    if $\pub = \bot$ return $\bot$\\
    $\ct\gets\ct+1$ \\
    $\pub_\ct \gets \pub$\\
    %$\setC_\ct \gets \emptyset$\\
    $\col_\ct \gets \col$\\
    $\mathsf{rv} \gets \pub_\ct$; \diffminus{$\mathsf{rv} \gets \top$}\\
    return $\mathsf{rv}$
}
{
  \oraclev{$\UPO(i, \up)$}\\[2pt]
    %$X \gets \col^{v[i]}_i
    %$v[i] \gets v[i]+1$\\
    $\pub \getsr \Up_\key(\pub_i, \up)$\\
    if $\pub = \bot$ return $\bot$\\
    $\col_i \gets \up(\col_i)$\\
    $\pub_i \gets \pub$\\
    for $\qry$ in $\err_i$ do\\
    \tab $a \gets \Qry_K(\pub_i, \qry)$\\
    \tab if $\err_i[\qry] > \delta(a,\qry(\col_i))$ then\\
    \tab\tab$\err_i[\qry] \gets \delta(a,\qry(\col_i))$\\
    $\mathsf{rv} \gets \pub_i$; \diffminus{$\mathsf{rv} \gets \top$}\\
    return $\mathsf{rv}$
}
{
  \oraclev{$\QRYO(i, \qry)$}\\[2pt]
    $a \gets \Qry_K(\pub_i, \qry)$\\
    if $\err_i[\qry] < \delta(a,\qry(\col_i))$ then\\
    \tab$\err_i[\qry] \gets \delta(a,\qry(\col_i))$\\
    return $a$
  \\[6pt]
  \oraclev{$\REVO(i)$}\\[2pt]
    $\setP \gets \setP \cup \{i\}$ \\
    return $\pub_i$
}
  \caption{Two notions of adversarial correctness. The \errep\ notion captures
  correctness when the representation is always known to the adversary, while
  the \erreps\ notion captures correctness when the representation is secret.
  When modeling a function $H:\setX\to\setY$ as a random oracle, the $\HASHO$
  oracle is given to $\advA$, $\Rep$, $\Up$ and $\Qry$.}
  \vspace{6pt}\hrule
  \label{fig:security}
\end{figure*}


\newcommand{\oneColCCS}[2]{
  \makebox[.3\textwidth]{
    \begin{tabular}{|@{\gamespadleft}l@{\gamespad}|}
    \hline
    \rule{0pt}{1\normalbaselineskip}
    \begin{minipage}[t]{#1\textwidth}\gamesfontsize
      #2 \vspace{6pt}
    \end{minipage} \\
    \hline
  \end{tabular}
  }
}

\ignore{
\begin{figure}[t]
\oneColCCS{0.25}
{
\oraclev{$H(X)$}\\[2pt]
$V \getsr \mathcal{V}$\\
if $\mathrm{T}[X] \neq \undefn$ then $V \gets \mathrm{T}[X]$\\
$\mathrm{T}[X]\gets V$\\
return~$V$
}
\caption{Pseudocode for a Random Oracle~$H$ with outputs in set $\mathcal{V}$}
\label{fig:ROM}
\end{figure}
}
%
We define two adversarial notions of correctness given by a pair of related experiments for a mutable data structure $\struct$ and error capacity $r$, one in which the representations of the true data are public ($\errep$), and one in which they are private ($\erreps$).  We will describe the former, as the latter is a closely related derivative.  Throughout, we assume that the adversary~$\advA$ does not make any pointless queries. 

Both experiments aim to capture the total weight of the errors caused by the adversary, at any point in time, with respect to the current data objects $\col_i$ and their representations $\pub_i$.  Because we consider mutable data objects and representations, the notion of ``current'' is defined by calls to the $\REPO$ and $\UPO$ oracles.  Specifically, for each~$\col_i$, both experiments maintain a set~$\setC_i$ that is initially set to empty (when~$\col_i$ is first assigned via a~$\REPO$-query), and it is reset to empty whenever~$\col_i$ is updated via an $\UPO$-query.  This is capturing the fact that applying a non-empty update function~$\up$ to~$\col_i$ produces a new data object.

To track errors, both experiments maintain an array $\err_i[]$ for every data object~$\col_i$ that has been defined.   Initially, $\err_i[]$ is implicitly assigned the value of~$\undef$ at every index.  For purposes of value comparison, we adopt the convention that $\undef < n$ for all $n \in \mathbb{R}$.
%
Now, the array~$\err_i$ is indexed by query functions~$\qry$, and the value of $\err_i[\qry]$ is the ``weight'' of the error caused by~$\qry$, with respect to the \emph{current} data object~$\col_i$ and \emph{current} representation~$\pub_i$ (of~$\col_i$).  
%
The value of~$\err_i[\qry]$ is updated within the $\QRYO$- and $\UPO$-oracles, but observe that $\err_i[\qry] = \undef$ until $(i,\qry)$ is queried to the $\QRYO$-oracle.  Intuitively, a representation~$\pub_i$ of data object~$\col_i$ cannot surface errors until it is queried.

When~$\QRYO(i,\qry)$ executes, the value in $\err_i[\qry]$ is overwritten iff the error caused by~$\qry$ is larger than the existing value of $\err_i[\qry]$.  The first time $(i,\qry)$ is queried to~$\QRYO$ this is guaranteed, since the minimum possible value of $d$ is 0.  Since the set $\setC_i$ is used to prevent (WLOG) the adversary from repeating a query~$\QRYO(i,\qry)$ for a given~$\col_i$, increases to the value of $\err_i[\qry]$ may only be made ``across'' updates to~$\col_i$.  This may seem to be overly conservative, as an error-heavy~$\col_i$ may become less so after an update.  But we account for this within the $\UPO$-oracle.
In particular, calls to the $\UPO$-oracle may only \emph{decrease} the value of~$\err_i[\qry]$.  

When a query $\UPO(i,\up)$ is made, the oracle first updates the data object~$\col_i$ and its corresponding representation.  The set~$\setC_i$ is reset to empty, because the data object~$\col_i$ is ``new'' again.
%
Now, for each defined value~$\err_i[\qry]$, we reevaluate the error that \emph{would} be caused by the previously asked~$\qry$ w.r.t. the newly updated $\col_i$ and $\pub_i$.  If the existing value of $\err_i[\qry]$ is larger than the error that~$\qry$ would cause to w.r.t. the newly updated~$\col_i$ and $\pub_i$, then we overwrite $\err_i[\qry]$ with the smaller value.  Doing so insures that the array~$\err_i$ does not overcredit the attacker for errors against the current data object and representation.

\tsnote{The next paragraph is a placeholder; it's too informal as written, but gets the point across for future edits.}
Let us give an example to help make concrete these definitional choices.  Say that the adversary queries a set~$\col = \{1,2,3\}$ to the $\REPO$-oracle, and learns that~$4$ is a false-positive value for the representation~$\pub$ of~$\col$.  If it adds~$4$ to~$\col$ via an update query, then it is no longer a false-positive.  Our definition insures that this is properly accounted.

%%%%%%%%%%%%%%%%%%%%%%%%%%%%%%%%%%%%%%%%%%%%%%%%
\subsection{Generic results}
First, we note that any structure which is insecure in the immutable case is also insecure in the mutable case. Any adversary in the immutable case is identical to a corresponding adversary in the mutable case which simply never makes use of the $\UPO$ oracle. From this we know that standard (unsalted, unkeyed) Bloom filters cannot ensure correctness in the mutable case. 


\heading{Keyless structures.}\tsnote{I've tried to organize these results into little sectionettes.  I don't know, yet, what is the right way to organize these generic results, but this is a start.  As you develop results, please try to keep them organized, whatever the organizational schema.}

In a keyless structure, we assume all details of the algorithm used are known beforehand by the adversary, except any which might be generated `on the fly' as representations are created and modified. For example, many of these structures will assume the use of a random salt which is picked at the time the representation is created. A structure with neither randomization nor a private key is usually vulnerable to pre-computation and offline attacks, where the adversary can search for errors to produce without having to interact with the structure itself.

In many cases, it is easier to reason about an adversary which only has the chance to create and manipulate a single representation, rather than being able to call a $\REPO$ oracle as many times as it likes. Fortunately, in the case of keyless structures we can show that the adversary's advantage in a single-representation case is bounded above by a multiple of the advantage in the general case. There are two equivalent ways of defining the $\errep1$ experiment. First, we may define it as a modified version of $\errep$ in which the adversary is constrained to make no more than one call to $\REPO$. We may equivalently define it as a two-stage adversary in the experiment below.

\begin{figure}[t]
  \twoColsNoDivide{0.48}
  {
     \raisebox{-5pt}{\experimentv{$\Exp{\errep1}_{\struct,r}(\advA)$ {\colorbox{lightgray}{$\Exp{\erreps1}_{\struct,r}(\advA)$}}}}\\[2pt]
      $\setP \gets \emptyset$\\
      $\key \getsr \keys$\\
      $\col \getsr \advA$\\
      $i \getsr \advA^{\UPO,\QRYO}(\REPO(\col))$\\
      return $[\sum_\qry \err[\qry] \geq r]$ 
    \\[6pt]
    \oraclev{$\REPO(\col)$}\\[2pt]
      $\pub \getsr \Rep_\key(\col)$\\
      $\setC \gets \emptyset$\\
      $\mathrm{rv} \gets \pub$ \highlight{;\mathrm{rv} \gets \top} \\
      return $\mathrm{rv}$
  }
  {
    \oraclev{$\UPO(\up)$}\\[2pt]
%      $X \gets \col^{v[i]}_i
%      $v[i] \gets v[i]+1$\\
      $X \gets \col$ \\
      $\col \gets \up(X)$\\
      $\pub \getsr \Up_\key(\pub, \up)$\\
      $\setC \gets \emptyset$\\
      for $\qry$ in $\err$ do\\
      \tab $a \gets \Qry_K(\pub, \qry)$\\
      \tab if $\err[\qry] > d(a,\qry(\col))$ then\\
      \tab\tab$\err[\qry] \gets d(a,\qry(\col))$\\
      $\mathrm{rv} \gets \pub$ \highlight{;\mathrm{rv} \gets \top}\\
      return $\mathrm{rv}$
      \medskip

    \oraclev{$\QRYO(\qry)$}\\[2pt]
      if $\qry \in \setC$ then return $\bot$\\
      $\setC \gets \setC \union \{\qry\}$\\
      $a \gets \Qry_K(\pub, \qry)$\\
      if $\err[\qry] < d(a,\qry(\col))$ then\\
      \tab$\err[\qry] \gets d(a,\qry(\col))$\\
      return $a$
  }
  \caption{The $\errep1$ adn $\erreps1$ notions of adversarial correctness. These are equivalent to the standard $\errep$ and $\erreps$ notions where the adversary is only allowed to make a single call to $\REPO$.}
  \vspace{6pt}\hrule
  \label{fig:security}
\end{figure}

Note that in the $\erreps1$ scenario we need not provide the adversary with a $\REVO$ oracle. If the adversary uses only a single representation, may assume without loss of generality that they make no call to $\REVO$, since doing so would prevent the adversary from having any possibility of winning. Using this definition of $\errep1$, we can show the equivalence between $\errep1$ and $\errep$ using a straightforward reduction argument.

\begin{lemma}[$\errep1$ and $\erreps1$ imply \errep and \erreps for keyless structures]\label{lemma:errep}
  Let $\struct = (\Rep, \Qry, \Up)$ be a data structure with key
  space $\{\emptystr\}$. For every $t, q_R, q_T, q_U, q_H, r\geq 0$, it holds that
  \[
    \Adv{\errep}_{\struct,r}(t, q_R, q_T, q_U, q_H) \leq
    q_R\cdot\Adv{\errep1}_{\struct,r}(O(f(t)), q_T, q_U, q_H) \,,
  \]
  where $f(t) = t + (q_R-1)\ticks(\Rep,t) + q_T\ticks(\Qry,t) + q_U\ticks(\Up,t)$.
\end{lemma}

\begin{proof}For a fixed $r \ge 0$, let $A$ be an $\errep$ adversary which runs in $t$ time steps and makes $q_R$ $\REPO$ queries, $q_T$ $\QRYO$ queries, $q_U$ $\UPO$ queries, and $q_H$ RO queries. We construct an adversary $B$ for $\errep1$ as follows.

First, $\advB$ initializes a counter $ct \gets 0$ and a set $\setC \gets \emptyset$, and samples $q \getsr [q_R]$. Next $\advB$ executes $\advA$, simulating the answers to its oracle queries as follows. When $\advA$ asks the query $\REPO(\col)$, $\advB$ sets $ct \gets ct + 1$ and stores $\col_{ct} \gets \col$. Then, if $ct = q$, $B$ forwards $\col$ to its own $\REPO$ oracle, returning the resulting value ($\pub$ in the public-representation case, or $\top$ in the private-representation case) to $\advA$. Otherwise, $\advB$ computes $\pub_{ct} \getsr \Rep(\col)$ and returns either $\pub_{ct}$ or $\top$. When $\advA$ asks for the query $\QRYO(i,\qry)$, $\advB$ first checks if $(i,\qry) \in \setC$ and returns $\bot$ if this condition holds. Otherwise $\advB$ forwards $(i,\qry)$ to its $\QRYO$ oracle and returns $a$ if $i = q$, and reutrns $\Qry(\pub_i,\qry)$ otherwise. Similarly, when $\advA$ makes an $\UPO(i,\up)$ query, $\advB$ forwards $(i,\up)$ to its $\UPO$ oracle if $i = q$ and evaluates $\Up(\pub_i,\up)$ otherwise. Finally, queries from $\advA$ to its RO are simply forwarded to $\advB$'s RO. When $\advA$ halts and outputs $j$, $\advB$ does the same.

If $j = q$, then $B$ wins if $A$ does, since all queries from $A$ to $\pub_j$ were forwarded to $B$'s $\QRYO$ oracle. Given that $q$ is sampled uniformly from the range $[q_R]$, it follows that

$$\Adv{\errep}_{\struct,r}(t, q_R, q_T, q_U, q_H) \leq q_R\cdot\Adv{\errep1}_{\struct,r}(O(f(t)), q_T, q_H).$$

Note that $B$ makes at most $q_T$ queries to $\QRYO$ and $q_H$ queries to its RO. Since $A$ runs in $t$ time steps and writing a bit takes 1 time step, the input length to any $\Rep$, $\Qry$, or $\Up$ evaluated by $B$ is at most $t$ bits. Hence, adversary $B$ runs in time $O(t+(q_R-1)\ticks(\Rep,t)+q_T\ticks(\Qry,t)+q_U\ticks(\Up,t))$.\missingqed
\end{proof}

\begin{lemma}[Salts do not affect \errep for keyless structures]\label{lemma:nosalt}
  Let $\struct = (\Rep, \Qry, \Up)$ be a data structure with a salt randomly initialized at runtime and $\struct'$ be the same structure using a fixed value from the salt space in place of a randomized salt. For every $t, q_R, q_T, q_U, q_H, r\geq 0$, it holds that
  \[
    \Adv{\errep}_{\struct,r}(t, q_R, q_T, q_U, q_H) \leq
    \Adv{\errep}_{\struct',r}(O(t), q_R, r, n, q_H) \,,
  \]
  where $n$ is the longest minimal sequence of update operations needed to generate a data structure in the space.
\end{lemma}
\begin{proof}
Consider an adversary $A$ for a non-salted data structure which makes $q_R$ queries to $\REPO$ and $q_T$ queries to $\QRYO$. We construct an adversary $B$ for the salted case which produces the same errors by simulating $A$. The adversary $B$ maintains a counter $\ct$ initialized to 0. Whenever $A$ makes a $\REPO(\col)$ query, $B$ first forwards a $\REPO(\emptyset)$ query to its own $\REPO$ oracle, stores the salt $\salt_\ct$ and set $\col_\ct$, initializes an error counter $\err_\ct$ to 0 and a set $\setE_\ct$ to $\emptyset$, and increments $\ct$. Then $B$ simply computes the actual representation of $\col$ using the salt $\salt_\ct$ and returns that to $A$. Whenever $A$ makes a $\QRYO(i,\qry)$ query, $B$ computes the result itself. If it would cause an error, $\err_i$ is incremented appropriately and $\qry$ is added to $\setE_i$, and then $B$ returns the result of the query to $A$. Whenever $A$ makes a $\UPO(i, \up)$ query, $B$ resets $\err_i$ to 0 and again computes the result itself before returning it to $A$. Finally, $B$ simply forwards any $\HASHO$ queries to its own oracle. When $A$ halts and outputs $i$, $B$ performs a series of $\UPO(i, \up)$ queries to transform $\pub_i$ into a representation of its stored $\col_i$ and then calls the queries in $\setE_\ct$. Because the salt remains unchanged when the set is updated, these will induce errors for $B$ just as they did for $A$, and $B$ will win if and only if $A$ does. \missingqed
\end{proof}

\heading{Invertible structures.} \dctodo{needs intro text}
\begin{lemma}[Salts do not affect \errep for invertible structures]\label{lemma:noinvsalt}
  Let $\struct = (\Rep, \Qry, \Up)$ be a data structure with a salt randomly initialized at runtime and $\struct'$ be the same structure using a fixed value from the salt space in place of a randomized salt. For every $t, q_R, q_T, q_U, q_H, r\geq 0$, it holds that
  \[
    \Adv{\errep}_{\struct,r}(t, q_R, q_T, q_U, q_H) \leq
    \Adv{\errep}_{\struct',r}(O(t), 1, q_T, q_U+2n(q_R+q_T+q_U), q_H) \,,
  \]
  where $n$ is the longest minimal sequence of update operations needed to generate a data structure in the space.
\end{lemma}
\begin{proof}
Consider an adversary $A$ in the case of a non-salted data structure which makes $q_R$ queries to $\REPO$ and $q_T$ queries to $\QRYO$. We construct an adversary $B$ for the salted case which produces the same errors as follows. First, $B$ initializes a counter $ct$ to 0 and calls the $\REPO$ oracle on the empty set, receiving an empty representation $\pub$ together with the salt used to create the representation. Then $B$ runs $A$, answering its oracle queries as follows. Whenever $A$ makes a query of the form $\REPO(\col)$, $B$ sets $ct \gets ct + 1$ and calls $\UPO$ repeatedly on $\pub$ to transform it into a representation of $\col$. Then $B$ returns the modified $\pub$ to $A$, stores $\col_{ct} \gets \col$, and performs the opposite updates in reverse order to return to the original empty representation $\pub$. If $A$ makes a query of the form $\QRYO(i,\qry)$, $B$ calls $\UPO$ repeatedly to transform $\pub$ into a representation of $\col_i$ and then returns the result of querying its own oracle with $\QRYO(1, \qry)$. Then once again $B$ performs the inverse updates to transform $\pub$ back into the original empty representation. If $A$ queries for $\UPO(i, \up)$, $B$ sets $\col_i \gets \up(\col_i)$ and again uses $\UPO$ queries to transform $\pub$ into a representation of $\col_i$, returns the value of $\pub$, and then performs opposite $\UPO$ queries to return $\pub$ to the empty representation. Finally, $B$ forwards any of $A$'s RO queries to its own RO.

In general this may use as many as $2n(q_R+q_T+q_U)$ update queries, where $n$ is the longest minimal sequence of update operations needed to generate a data structure in the space. Furthermore $B$ succeeds if $A$ does, so the adversaries' advantages are equal.\missingqed
\end{proof}

\begin{lemma}[Salts do not affect \errep for unkeyed invertible structures]\label{lemma:noinvsaltunkeyed}
  Let $\struct = (\Rep, \Qry, \Up)$ be an unkeyed data structure with a salt randomly initialized at runtime and $\struct'$ be the same structure using a fixed value from the salt space in place of a randomized salt. For every $t, q_R, q_T, q_U, q_H, r\geq 0$, it holds that
  \[
    \Adv{\errep}_{\struct,r}(t, q_R, q_T, q_U, q_H) \leq
    q_R \cdot \Adv{\errep1}_{\struct',r}(O(f(t)), q_T, q_U, q_H) \,,
  \]
  where $n$ is the longest minimal sequence of update operations needed to generate a data structure in the space.
\end{lemma}
\begin{proof}
Consider an adversary $A$ in the case of a non-salted data structure which makes $q_R$ queries to $\REPO$ and $q_T$ queries to $\QRYO$. We construct an adversary $B$ for the salted case which produces the same errors as follows. First, $B$ initializes a counter $ct$ to 0 and calls the $\REPO$ oracle on the empty set, receiving an empty representation $\pub$ together with the salt used to create the representation. Then $B$ runs $A$, answering its oracle queries as follows. Whenever $A$ makes a query of the form $\REPO(\col)$, $B$ sets $ct \gets ct + 1$ and calls $\UPO$ repeatedly on $\pub$ to transform it into a representation of $\col$. Then $B$ returns the modified $\pub$ to $A$, stores $\col_{ct} \gets \col$, and performs the opposite updates in reverse order to return to the original empty representation $\pub$. If $A$ makes a query of the form $\QRYO(i,\qry)$, $B$ calls $\UPO$ repeatedly to transform $\pub$ into a representation of $\col_i$ and then returns the result of querying its own oracle with $\QRYO(1, \qry)$. Then once again $B$ performs the inverse updates to transform $\pub$ back into the original empty representation. If $A$ queries for $\UPO(i, \up)$, $B$ sets $\col_i \gets \up(\col_i)$ and again uses $\UPO$ queries to transform $\pub$ into a representation of $\col_i$, returns the value of $\pub$, and then performs opposite $\UPO$ queries to return $\pub$ to the empty representation. Finally, $B$ forwards any of $A$'s RO queries to its own RO.

In general this may use as many as $2n(q_R+q_T+q_U)$ update queries, where $n$ is the longest minimal sequence of update operations needed to generate a data structure in the space. Furthermore $B$ succeeds if $A$ does, so the adversaries' advantages are equal.\missingqed
\end{proof}

\begin{lemma}[]\label{lemma:immutabletomutable}
  Let $\struct = (\Rep, \Qry, \Up)$ be a data structure, and $\struct'$ be the same structure except that any per-representation random parameters are chosen once and fixed across representations. For every $t, q_r, q_T, q_U, q_H, q_V, r \geq 0$, it holds that
  \[
    \Adv{\errep}_{\struct,r}(t, q_R, q_T, q_U, q_H, q_V) \leq
    \Adv{\prf}_F(\advB) + \Adv{\errep}_{\struct',r}(O(t), q_R + q_U, q_T, 0, q_H, 0) \,,
  \]
\end{lemma}
\begin{proof}
Given an adversary $\advA$ making queries $(q_R, q_U, q_T, q_H)$, we can construct $\advB$ making queries $(q_R+q_U, 0, q_T, q_H)$ which achieves the same advantage.

For a structure $\Pi$, we denote by $\Pi'$ the same structure with pseudo-random functions replaced by true-random functions. Using conditional probabilities, we have $\Adv{\errep}_{\struct,r}(\advA) \le \Adv{\prf}_F(\advB) + \Adv{\errep}_{\struct_1,r}(\advA)$.

In the truly random structure $\struct_1$, the representation of a set is simply a randomly-chosen valid representation of that set. Revealing this therefore reveals no new information about any other representations which have been constructed, and so calling $\REVO$ can only decrease an adversary's chance of success. We may now assume $\REVO$ is never called.

We next modify the structure to $\struct_2$, where all randomized parameters are fixed across representations at the beginning of the experiment, but not revealed to the adversary. We create an adversary $C$ for the correctness experiment on $\struct_2$ which simulates $\advA$ as follows. The adversary $C$ keeps track of a counter $ct$ and list $L$ of indices and simulates $\advA$. When $\advA$ calls $\REPO$, $C$ appends $ct$ to the end of $L$ and then increments $ct$. When $\advA$ calls $\UPO(i,\up)$, $C$ calls $\REPO(\up(\col_{I(i)}))$, sets $L(i) \gets ct$, and increments $ct$. Finally, when $\advA$ calls $\QRYO(i,\qry)$, $C$ calls $\QRYO(L(i),\qry)$. Since $\REVO$ is never called, $C$ need not implement this oracle. Since all queries are called on representations of the same sets, $C$ has the same probability of success as $A$.
\end{proof}

\subsection{Immutabilization}

We say that a structure has `independent representations' if knowing the representation of one set does not provide any additional information about the representation of any other set. For example, the standard Bloom filter vacuously has independent representations, since the adversary can perfectly predict the representation of any set without needing to see the representation of other sets. A filter which uses true random sampling in place of hash functions also has independent representations, but a filter which uses a secret key need not have independent representations.

In each of our example structures involving the use of a secret key, we formalize the system using a pseudorandom function. In each correctness proof, the first thing we do is always moving to a game where the pseudorandom functions are replaced by true random functions, since this gives very similar adversarial advantage (assuming the use of a good prf) while making the behavior of the game much easier to reason about.

In the case of a structure with independent representations, we may assume without loss of generality that the adversary never calls the $\REVO$ oracle, since doing so can only decrease their probability of success.

Note that most of the data structures we consider have some degree of per-representation randomness, such as a salt chosen for hash functions. We consider an alternate correctness game that forces this randomness to behave the same across all representations constructed. In particular, any randomly-chosen parameters which would be chosen during the computation of $\REPO$ are selected at the beginning of the game, using the same random distribution, before the adversary makes any oracle queries. The $\REPO$ oracle then uses this fixed value for any random parameters in place of sampling a new random value.

\begin{lemma}[]\label{lemma:immutabletomutablenew}
  Let $\struct = (\Rep, \Qry, \Up)$ be a data structure, and $\struct'$ be the same structure except that any per-representation random parameters are chosen once and fixed across representations. For every $t, q_r, q_T, q_U, q_H, q_V, r \geq 0$, it holds that
  \[
    \Adv{\errep(s)}_{\struct,r}(t, q_R, q_T, q_U, q_H, q_V) \leq
    \Adv{\errep(s)'}_{\struct,r}(O(t), q_R + q_U, q_T, 0, q_H, 0) \,,
  \]
\end{lemma}
\begin{proof}

We create an adversary $\advB$ for the modified correctness experiment which simulates $\advA$ as follows. The adversary $\advB$ keeps track of a counter $ct$ and list $L$ of indices and simulates $\advA$. When $\advA$ calls $\REPO$, $\advB$ appends $ct$ to the end of $L$ and then increments $ct$. When $\advA$ calls $\UPO(i,\up)$, $\advB$ calls $\REPO(\up(\col_{I(i)}))$, sets $L(i) \gets ct$, and increments $ct$. Finally, when $\advA$ calls $\QRYO(i,\qry)$, $\advB$ calls $\QRYO(L(i),\qry)$. Since $\REVO$ is never called, $\advB$ need not implement this oracle. Since all queries are called on representations of the same sets, $C$ has the same probability of success as $A$.

\end{proof}

%Note that in a counting filter, $n$ is equal to the maximum supported multiset size, and in count-min sketch $n$ is equal to the sum of the absolute values of the elements' frequences. While this means a potentially large number of update queries are needed in the general case, for an unkeyed structure it turns out that far fewer updates are required. In the above proof, the only reason we needed to undo each sequence of $\UPO$ queries was because the following query might involve the representation of a different structure. But in the case of a salted but unkeyed structure, Lemma 1 shows that for any adversary there is an adversary making only a single $\REPO$ query with an advantage differing by a factor of at most $q_R$. For an adversary which only ever makes one $\REPO$ query, we never need to reset the representation to the empty representation, and in the above simulation we can omit all $\UPO$ queries except those needed for $A$'s single $\REPO$ query and any $\UPO$ queries made by $A$ itself.


%%%%%%%%%%%%%%%%%%%%%%%%%%%%%%%%%%%%%%%%%%%%%%%%

\subsection{Bloom filter results}
\label{subsec:bloom-filters}

\begin{table}[thp]
\begin{center}
  \begin{tabular}{ | c | c |c | c | c | }
    \hline
    Salt & Key & \parbox{.75in}{Private\\ Representation} & \parbox{.75in}{Irreversible\\ Updates} & Result \\ \hline\hline
    0 & 0 & 0 & 0 & Attack 1 \\ \hline
    0 & 0 & 0 & 1 & Attack 1 \\ \hline
    0 & 0 & 1 & 0 & Attack 1 \\ \hline
    0 & 0 & 1 & 1 & Attack 1 \\ \hline
    0 & 1 & 0 & 0 & Attack 2 \\ \hline
    0 & 1 & 0 & 1 & Attack 2 \\ \hline
    0 & 1 & 1 & 0 & Attack 2 \\ \hline
    0 & 1 & 1 & 1 & Attack 2 \\ \hline
    1 & 0 & 0 & 0 & Attack 3 \\ \hline
    1 & 0 & 0 & 1 & Attack 3 \\ \hline
    1 & 0 & 1 & 0 & ? \\ \hline
    1 & 0 & 1 & 1 & Theorem~\ref{thm:bf-priv-salt-bound} \\ \hline
    1 & 1 & 0 & 0 & Attack 4 \\ \hline
    1 & 1 & 0 & 1 & Theorem~\ref{thm:bf-key-bound} \\ \hline
    1 & 1 & 1 & 0 & ? \\ \hline
    1 & 1 & 1 & 1 & Theorem~\ref{thm:bf-key-bound} \\
    \hline
  \end{tabular}
\end{center}
\tsnote{This table feels unintutive, somehow. ``Salt'', ``Key'', ``Irreversible updates'' are all syntactic properties; ``Private Representation'' is an artifact of the attack model.  We need a better way to organize and present these results.}
\caption{Summary table for Bloom filters. \textbf{Attack 1:}The adversary chooses a maximally large test set $\col$ and simulates $\Rep$ to produce a representation $\pub$. The adversary then simulates $\Rep$ for many arbitrarily chosen singleton sets disjoint from $\col$, checking each one to see if its element is a false positive for $\pub$. Once it has accumulated $r$ false positives, call $\REPO$ on $\col$, call $\QRYO$ for each false positive found, and return 1.  \textbf{Attack 2:} The adversary chooses a maximally large test set $\col$ and calls $\REPO$ to produce a representation $\pub$. The adversary then calls $\REPO$ for many arbitrarily chosen singleton sets disjoint from $\col$, using $\QRYO$ on each to determine if its element is a false positive for the representation constructed for $\col$. Once it has accumualated $r$ false positives, return 1. \textbf{Attack 3:}  The adversary chooses a maximally large test set $\col$ and calls $\REPO$ to receive a representation $\pub$ together with the salt $\salt$. Using the known salt, the adversary simulates $\Rep$ for many arbitrarily chosen singleton sets disjoint from $\col$, checking each one to see if its element is a false positive for $\pub$. Once it has accumulated $r$ false positives, call $\QRYO$ for each such false positive and return 1. \textbf{Attack 4:}The adversary chooses a test set $\col_0$ and a target set $\col_1$, performing a search on representable subsets of $\col_0$ represented as a tree ordered by $\subseteq$. Moving up or down in the tree is accomplished using $\UPO$, and at each node $\QRYO$ is called for each element of $\col_1$ to determine which are false positives. Once $r$ false positives are found, halt and return 1. }
\label{tab:main}
\end{table}

\dctodo{needs intro text}
Our first secure data structure is the salted Bloom filter, which at the time a representation is created chooses a salt $\salt$ which it will use for all further queries and updates. In order for security to be guaranteed, we must ensure that the representation, and in particular the salt, is kept secret from the adversary.

\begin{theorem}[Correctness Bound for Private-Representation Salted Bloom Filters]\label{thm:bf-priv-salt-bound}
Fix integers $k, m, n, \lambda, r\geq 0$, let $H \colon \bits^* \to [m]$ be a function, and let $\struct_s = \SBF[\hashbf[H],k,m,n,\lambda]$ in the private-representation setting.
  For every $t, q_R, q_T, q_H \geq 0$ such that $q_T \geq r$, it holds that
  \begin{eqnarray*}
    \Adv{\erreps}_{\struct_\saltybloom,r}(t,&q_R,& q_T, q_U, q_H) \leq \\ && q_R \cdot
     \left[
      \frac{q_H}{2^\lambda} +
      {\dbinom{q_T+q_U}{r}} p(k, m, n+r)^r
    \right] \,,
\end{eqnarray*}
where $H$ is modeled as a random oracle and $p(k, m, n+r)$ is the false-positive probability on a Bloom filter with the given parameters.
\end{theorem}

This proof first reduces to the single-representation case, which as shown in Lemma 1 will reduce the adversary's advantage by at most a factor of $q_R$. The main idea behind the proof is to remove the adversary's adaptivity a step at a time. We isolate the possibility of the adversary guessing the salt, which would allow it to mount its own offline attack on the filter without relying on the $\QRYO$ oracle. If the adversary does not guess the salt, the outputs of the $\REPO$, $\QRYO$, and $\UPO$ oracles are unpredictable to the adversary, producing uniformly randomly distributed bits to set (for $\REPO$ and $\UPO$) or to check (for $\QRYO$). Under the assumption that the adversary does not predict the salt, queries made to distinct elements are independent of each other. The only remaining issue is that the adversary can potentially gain an advantage by testing whether some object $x$ is a false positive for the filter, and then updating the filter to include $x$ only if the test query returned `false'. An analysis shows that this is now (once imperfect pseudorandom functions and salt collisions have been dealt with) the only way for the adversary to gain an advantage over making queries to an immutable Bloom filter. Because this adaptive strategy introduces tricky conditional possibilities, we cannot compute an exact value for the adversary's advantage. Instead, we move to an alternate scenario where each $\QRYO$ also produces a free update and every $\UPO$ first performs a free query. This makes $\QRYO$ and $\UPO$ calls indistinguishable, so that the adversary is effectively making a series of independent random queries that each have a chance to increment the error counter. Because the size of the filter can only increase, the probability of a false positive from any one of these queries is bounded above by the probability of a false positive on the final maximally-sized filter, a probability which is given by the Kirsch and Mitzenmacher bound.

\begin{figure}
  \boxThmBFSaltCorrect{0.48}
  {
    \underline{$\G_0(\advA)$}\\[2pt]
      $\col \getsr \advA^H$; $\setC \gets \emptyset$; $\err \gets 0$\\
      $\pub \getsr \Rep[\hashbf[H]](\col)$\\
      $\bot \getsr \advA^{H,\QRYO,\UPO}$\\
      return $(\err \geq r)$
    \\[6pt]
    \oraclev{$\QRYO(\qry_x)$}\\[2pt]
      if $\qry_x \in \mathcal{C}$ then return $\bot$\\
      $\setC \gets \setC \union \{\qry_x\}$\\
      $a \gets \Qry[\hashbf[H]](\pub, \qry_x)$\\
      if $a \neq \qry_x(\col)$ then $\err \gets \err + 1$\\
      return~$a$
    \\[6pt]
    \oraclev{$\UPO(\up_x)$}\\[2pt]
      $\setC \gets \emptyset$\\
      $a \gets \Qry[\hashbf[H]](\pub, \qry_x)$\\
      if $\qry_x \in \setC$ and $a \neq \qry_x(\col)$ then\\
      \tab $\err \gets \err-1$\\
      $\col \gets \col \union \{x\}$\\
      $\pub \gets \Up[\hashbf[H]](\pub,\up_x)$\\
      return~$\bot$
    \\[4pt]
    \hspace*{-4pt}\rule{1.043\textwidth}{.4pt}
    \\[5pt]
    \oraclev{$\HASHO_1(\salt,x)$} \hfill\diffplus{$\G_2$}\;{$\G_1$}\hspace*{3pt}\\
      $\hh \getsr [m]^2$; $\vv \gets \fff(\hh$)\\
      if $\salt = \salt^*$ then\\
      \tab $\bad_1 \gets 1$; \diffplus{return $\vv$}\\
      if $T[\salt,x]$ is defined then $\vv \gets T[\salt,x]$\\
      $T[\salt,x] \gets \vv$;
      return $\vv$
  }
  {
    \underline{$\G_1(\advB)$}\\[2pt]
      $\salt^* \getsr \bits^\lambda$;
      $\col \getsr \advB^{\HASHO_1}$\\
      $\pub \gets \Repx[\HASHO_2](\col, \salt^*)$\\
      $\setC \gets \emptyset$;
      $\err \gets 0$\\
      $\bot \getsr \advB^{\HASHO_1,\QRYO,\UPO}$\\
      return $(\err \geq r)$
    \\[6pt]
    \oraclev{$\QRYO(\qry_x)$}\\[2pt]
      if $\qry_x \in \mathcal{C}$ then return $\bot$\\
      $\setC \gets \setC \cup \{\qry_x\}$\\
      $a \gets \Qry[\HASHO_2](\pub, \qry_x)$\\
      if $a \neq \qry_x(\col)$ then $\err \gets \err + 1$\\
      return~$a$
    \\[6pt]
    \oraclev{$\UPO(\up_x)$}\\[2pt]
      $\setC \gets \emptyset$\\
      $a \gets \Qry[\hashbf[H]](\pub, \qry_x)$\\
      if $a \neq \qry_x(\col)$ and $\qry_x \in \setC$ then\\
      \tab $\err \gets \err-1$\\
      $\col \gets \col \union \{x\}$\\
      $\pub \gets \Up[\HASHO_2](\pub,\up_x)$\\
      return~$\bot$
    \\[6pt]
    \oraclev{$\HASHO_2(\salt,x)$}\\[2pt]
      $\hh \getsr [m]^2$; $\vv \gets \fff(\hh$)\\
      if $T[\salt,x]$ is defined then\\
      \tab $\vv \gets T[\salt,x]$\\
      $T[\salt,x] \gets \vv$;
      return $\vv$
  }
  {
    \underline{$\G_3(\advB)$}\\[2pt]
    \oraclev{$\QRYO(\qry_x)$}\\[2pt]
      $a \gets \Qry[\HASHO_3](\pub, \qry_x)$\\
      if $a \neq \qry_x(\col)$ then $\err \gets \err + 1$\\
      $\col \gets \col \union \{x\}$
      $\pub \gets \Up[\HASHO_2](\pub,\up_x)$\\
      return~$a$
  }
  {
    \oraclev{$\UPO(\up_x)$}\\[2pt]
      $a \gets \Qry[\HASHO_3](\pub, \qry_x)$\\
      if $a \neq \qry_x(\col)$ then $\err \gets \err + 1$\\
      $\col \gets \col \union \{x\}$
      $\pub \gets \Up[\HASHO_2](\pub,\up_x)$\\
      return~$\bot$
    \\[6pt]
    \oraclev{$\HASHO_i(\salt,x)$}\\[2pt]
      $\hh \getsr [m]^2$; $\vv \gets \fff(\hh$)\\
      return $\vv$
  }
  \caption{Games 0--3 for proof of Theorem~\ref{thm:bf-priv-salt-bound}.}
  \label{fig:bf-priv-salt-bound}
\end{figure}

\begin{proof} We first reduce from the $\erreps$ case to the $\erreps1$ case, which by Lemma 1 may scale the adversary's advantage only by a factor of $q_R$. The game~$\G_0$ is exactly equivalent to the $\erreps1$ experiment, so $\Adv{\errep1}_{\struct_s,r}(\advA) = \Prob{\G_0(\advA) = 1}$. In~$\G_1$ we split the hash oracle into three, giving the adversary access $\HASHO_1$ in both stages of the game, while $\HASHO_2$ is reserved for oracular use by $\Repx$, $\QRYO$, and $\UPO$. For any $\advA$ for~$\G_0$, there is $\advB$ for~$\G_1$ which produces the same advantage by simulating $\advA$. This adversary first creates its own table $R$ with all values initially undefined. When $\advA$ makes a query $w$ to $H$, $\advB$ returns $R[w]$ if that entry in the table is defined. Otherwise, if there are $\salt \in \bits^\lambda$, $j \in [k]$, and $x \in \bits^*$ such that $w = \langle\salt, j, x\rangle$, forward $(\salt,x)$ to $\HASHO_1$. For each $j \in [k]$, set $R[\langle\salt, j, x\rangle] = \vv_j$, where $\vv$ is the output of the $\HASHO_1$ oracle. If there is no such triple $\langle\salt, j, x\rangle$, just sample $r$ from $[m]$ uniformly and set $R[w] = r$. In either case, return $R[w]$ to $\advA$. When $\advA$ outputs its collection $\col$, $\advB$ outputs $\col$ as well. Any queries by $\advA$ to $\QRYO$ or $\UPO$ are forwarded to $\advB$'s corresponding oracle. The simulation is perfect because $\Rep[\hashbf[H]](\col)$ and $\Up[\hashbf[H]](\col,\up)$ are identically distributed to $\Rep[\HASHO_2](\col)$ and $\Up[\HASHO_2](\col,\up)$. Because we have a perfect simulation, $\Adv{\erreps1}_{\struct_s,r}(\advA) = \Prob{\G_1(\advA) = 1}$.

The game~$\G_2$ is the same as~$\G_1$ until $\bad_1$ is set, which occurs exactly when $\advB$ sends $(\salt^*,x)$ to $\HASHO_1$ for some $x$. In the first phase, there is again a $q_1/2^\lambda$ chance of the adversary guessing the salt. In the second phase, the random sampling used by $\HASHO_i$ ensures that each call the adversary makes to the $\HASHO_i$ oracle is independent of all previous calls. We therefore have a $q_2/2^\lambda$ chance of the adversary guessing the salt during this phase, for a total chance of $q_H/2^\lambda$ chance of the adversary guessing the salt at some point during the experiment. Then $\Adv{\erreps_1}_{\struct_s,r}(\advA) \le \Prob{\G_2(\advB) = 1} + q_H/2^\lambda$. Having taken this into account, we may now assume the adversary never guesses the salt.

We want to show that alternating between sequences of queries and sequences of updates is no better than making one long series of updates and then one long sequence of queries. There are three types of updates the adversary can make: updates to add elements that have been queried and found to be false positives; updates to add elements that have been queried and found not to be false positives; and updates to add elements that have not been queried yet. We may assume without loss of generality that the adversary never makes the first type of update, since doing so is never beneficial (it does not change the representation at all and decreases the number of errors the adversary has found).

Note that the choices of $\vv$ constructed by the $\HASHO_i$ oracles are independent of all previous queries. Because of this, any update of type 3 is equivalent to any other update of type 3; the probability of any bit being flipped by one update is the same as the probability of the bit being flipped by the other update. Similarly, any update of type 2 is equivalent to any other update of type 2, but is not the same as type 3 since the probability is conditioned on $\vv$ not being a false positive. We assume the worst case, namely that all updates are type 2 (i.e. at least one bit is flipped by each update).

Because the adversary never guesses the salt, $\HASHO_1$ simply functions as a random oracle. Furthermore, we can assume the adversary never adds an element of $\col$ to $\col$ and never makes a $\QRYO$ call for an element which is already in $\col$, since neither of these provides any additional information and neither affects the rest of the experiment in any way.

Now we move to the game~$\G_3$. Here each $\QRYO$ query also calls $\UPO$ to add that element to $\col$. Additionally, the penalty for adding known false positives is removed. To avoid penalizing the adversary by prematurely maxing out the number of elements in $\col$ because of added false positives, we also increase the maximum size of $\col$ from $n$ to $n+r$. Because the adversary (without loss of generality) stops after accumulating $r$ errors, only $r$ false positives will be added to $\col$ and so a maximum size of $n+r$ is sufficient to produce no penalty for the adversary. Furthermore, each $\UPO$ call is preceded by a $\QRYO$ call. Neither of these changes can produce a worse result for the adversary, so $\Prob{\G_2(\advB) = 1} \le \Prob{\G_3(\advB) = 1}$. Now, however, there is no longer any distinction between $\QRYO$ and $\UPO$ calls. All calls to either oracle are independent of each other and produce the same effect, querying and then updating $\col$. Each of these queries for false positives is at most as successful as a query to a Bloom filter with $n+r$ elements, so the adversary's probability of finding a false positive on any query is bounded above by the standard success rate for a Bloom filter with those parameters, $(1-e^{-k(n+r)/m})^k+O(1/(n+r))$. The adversary is required to produce $r$ errors over the course of $q_T+q_U$ queries, which by the binomial theorem gives an advantage bound of $\Prob{\G_3(\advB) = 1} \le \binom{q_T+q_U}{r}((1-e^{-k(n+r)/m})^k+O(1/(n+r)))^r$.

The full adversarial advantage is then
$$\Adv{\erreps_1}_{\struct_s,r}(\advA) \le q_R \cdot \left(\frac{q_H}{2^\lambda} + \binom{q_T+q_U}{r}((1-e^{-k(n+r)/m})^k+O(1/(n+r)))^r\right).$$
\missingqed
\end{proof}

The case of a Bloom filter whose representations are always public but which uses a private key in addition to a salt is almost identical. In practice the bound is actually somewhat stronger: the instance of $q_H$ in the bound is replaced with $q_R$, which means that the adversary which attempts to guess the salt must rely entirely on `online' queries as opposed to `offline' queries to $q_H$.

These two theorems together show that in order to guarantee correctness in the streaming setting, one must either ensure that representations are kept private from potential adversaries or use a secret key in addition to a salt when implementing a Bloom filter. In either case, some hidden information is necessary to protect the filter from an adversary.

\begin{theorem}[Correctness Bound for Keyed Bloom Filters]\label{thm:bf-key-bound}
Fix integers $k, m, n, \lambda, r\geq 0$, where $m \geq \lambda$, and let $F \colon \bits^\lambda \times \bits^* \to [m]$ be a function.
  For every $t, q_R, q_T \geq 0$ such that $q_T \geq r$, it holds that
  \begin{eqnarray*}
    \Adv{\erreps}_{\struct_\saltybloom,r}(t,&q_R,& q_T, q_U, q_H) \leq \\ && q_R \cdot
     \left[
      \frac{q_R}{2^\lambda} +
      {\dbinom{q_T+q_U}{r}} p(k, m, n+r)^r
    \right] \,.
\end{eqnarray*}
\end{theorem}

Despite the similarity of the bounds, the proof is somewhat different than in the private-representation salted-but-unkeyed case. The first thing we want to do is move from a pseudorandom function to a truly random function, about which we can produce bounds much more easily. This is problematic only if the adversary can easily distinguish between the pseudorandom function and a true random function, which for a good choice of pseudorandom function will not occur. Next, we note that for a true random function, the queries and updates in different representations will be independent unless the randomly-chosen salt $\salt$ is the same across two or more representations. The birthday bound gives us the chances of this occurring, but for sufficiently large salt size $\lambda$ the chance of this will also be small. Once we have reduced to a case where the filter operates on true random functions and salts never repeat, we are again left with a series of almost-independent queries where the adversary may be able to gain an advantage by following up a $\QRYO$ call with an $\UPO$ call. We take the same approach as in the previous case, turning $\QRYO$ and $\UPO$ into identical independent queries and then applying the Kirsch and Mitzenmacher upper bound for the largest set represented during the game.

\begin{figure}
  \twoCols{0.50}
  {
    \vspace{-7.5pt}
    \experimentv{$\G_0(\advA)$}
                  \hfill\diffplus{$\G_1$}\;\diffminus{$\G_0$}\\[2pt]
      $\setI \getsr \setP_r([q_T])$;
      $\setZ, \setQ \gets \emptyset$\\
      $\err_0, \ct, \ct' \gets 0$;
      $\ky \getsr \keys$\\
      $i \getsr \advA^{\REPO,\QRYO,\UPO}$;
      return $(\err_i \ge r)$
    \\[6pt]
    \oraclev{$\REPO(\col)$}\\[2pt]
      $\ct \gets \ct + 1$; $\err_\ct \gets 0$;
      $\col_\ct \gets \col$\\
      $\salt \getsr \bits^\lambda$;
      \diffminus{$\pub_\ct \gets \Repx[F_\ky](\col, \salt)$}\\
      \diffplus{$\pub_\ct \gets \Repx[\Rnd](\col, \salt)$}\\
      return $\pub_\ct$
    \\[6pt]
    \oraclev{$\QRYO(i,\qry)$}\\[2pt]
      $\ct' \gets \ct' + 1$\\
      \diffminus{$\res \gets \Qryx[F_\ky](\pub_i,\qry)$}\\
      \diffplus{$\res \gets \Qryx[\Rnd](\pub_i, \qry)$}\\
      if $\res \neq \qry(\col_i)$ then $\err_i\gets\err_i+1$\\
      return~$\res$
    \\[6pt]
    \oraclev{$\UPO(i,\up_x)$}\\[2pt]
      $\col_i \gets \up(\col_i)$\\
      \diffminus{$\pub_i \getsr \Upx[F_\ky](\pub_i,\up)$}\\
      \diffplus{$\pub_i \getsr \Upx[\Rnd](\pub_i,\up)$}\\
      $a \gets \Qry[\hashbf[H]](\pub, \qry_x)$\\
      if $a \neq \qry_x(\col)$ and $a \in \setC$ then\\
      \tab $\err_i \gets \err_i-1$\\
      return~$\pub_i$
    \\[6pt]
    \algorithmv{$\Rnd(x)$}\\[2pt]
      if $T[x]$ is undefined then $T[x] \getsr [m]$\\
      return $T[x]$
    \\
    \hspace*{-4pt}\rule{1.049\textwidth}{.4pt}
    \\[2pt]
    \oraclev{$\REPO(\col)$}
                            \hfill\diffplus{$\G_2$}\;\diffminus{$\G_1$}\hspace*{0pt}\\[2pt]
      $\ct \gets \ct + 1$; $\err_\ct \gets 0$;
      $\col_\ct \gets \col$\\
      \diffminus{$\salt \getsr \bits^\lambda$}
      \diffplus{$\salt \getsr \bits^\lambda \setminus \setZ$; $\setZ \gets \setZ
      \union \{\salt\}$}\\
      $\pub_\ct \gets \Repx[\Rnd](\col, \salt)$\\
      return $\pub_\ct$
  }
  {
    \vspace{-7.5pt}
    \oraclev{$\REPO(\col)$}
                    \hfill\diffplus{$\G_3$}\;\diffminus{$\G_2$}\hspace*{0pt}\\[2pt]
      $\ct \gets \ct + 1$; $\err_\ct \gets 0$;
      $\col_\ct \gets \col$\\
      $\salt \getsr \bits^\lambda \setminus \setZ$;
      $\setZ \gets \setZ \union \{\salt\}$\\
      $\pub_\ct \gets \Repx[\Rnd](\col, \salt)$\\
      \diffplusbox{
        for each $\qry \in \setQ$ do\\
        \tab $a \gets \Qryx[\Rnd](\pub_\ct, \qry)$\\
        \tab if $a \ne \qry(\col_\ct)$ then $\err_\ct \gets \err_\ct +1$
      }
      return $\pub_\ct$
    \\[6pt]
    \oraclev{$\QRYO(i, \qry)$}\\[2pt]
       $\ct' \gets \ct' + 1$\\
      \diffminusbox{
        $\res \gets \Qryx[\Rnd](\pub_i, \qry)$\\
        if $\res \neq \qry(\col_i)$ then $\err_i\gets\err_i+1$\\
        return~$\res$
      }
      \diffplusbox{
        \foreach{j}{1}{\ct}\\
        \tab $a_j \gets \Qryx[\Rnd](\pub_j, \qry)$\\
        \tab if $\qry\not\in\setQ$ and $a_j \ne \qry(\col_j)$\\
        \tab\tab then $\err_j \gets \err_j + 1$\\
        $\setQ \gets \setQ \union \{\qry\}$\\
        return~$a_i$
      }
    \\[6pt]
    \oraclev{$\UPO(i, \up)$}\\[2pt]
      $\col_i \gets \up(\col_i)$\\
      $\pub_i \getsr \Upx[\Rnd](\pub_i,\up)$\\
      $a \gets \Qry[\hashbf[H]](\pub, \qry_x)$\\
      if $a \neq \qry_x(\col)$ and $qry_x \in \setC$ then\\
      \tab $\err \gets \err-1$\\
      return~$\pub_i$
    \\[3pt]
    \hspace*{-4pt}\rule{1.049\textwidth}{.4pt}
    \\[2pt]
    \oraclev{$\QRYO(i, \qry_x)$},\hfill\diffplus{$\G_5$}\;{$\G_4$}\hspace*{0pt}\\[2pt]
      $\ct' \gets \ct' + 1$\\
      \foreach{j}{1}{\ct}\\
      \tab $a_j \gets \Qryx[\Rnd](\pub_j, \qry)$\\
      \tab if \diffplus{$\ct' \in \setI$ and} $\qry\not\in\setQ$ and $a_j \ne \qry(\col_j)$\\
      \tab\tab then $\err_j \gets \err_j + 1$\\
      \diffplus{if $\ct' \in \setI$ then} $\setQ \gets \setQ \union \{\qry\}$\\
      $\col_i \gets \up_x(\col_i)$\\
      $\pub_i \getsr \Upx[\Rnd](\pub_i,\up)$\\
      return~$a_i$
    \\[2pt]
    \oraclev{$\UPO(i,\up_x)$}\\[2pt]
      $\QRYO(i, \qry_x)$; return~$\pub_i$\\
  }
  \caption{Games 0--5 for the proof of
  Theorem~\ref{thm:bf-key-bound}.}
  \label{fig:bf-prf-correct}
\end{figure}

\begin{proof}We define algorithms $\Repx[f]: \bits^* \times \bits^\lambda \to \bits^*$, $\Qryx[f]: \bits^* \times \mathcal{Q} \to \mathcal{R}$, and $\Upx[f]: \bits^* \times \mathcal{U} \to \bits^*$ for arbitrary functions $f: \bits^* \to [m]$. The algorithm $\Repx[f]$ takes a set and salt, returning a representation of the Bloom filter constructed as usual using the given salt, but with calls to $F_K$ replaced with calls to $f$. Note that the salt is included in the output representation. Similarly, $\Qryx[f]$ and $\Upx[f]$ perform queries and updates on representations using $f$. in place of $F_K$.

First we move to the game~$\G_0$, which does not change the semantics of the usual correctness experiment. We assume without loss of generality that the adversary does not make the same query twice between updates, so there is no need to keep track of $\col$. We instead have sets $\setI$, $\setZ$, $\setQ$ not defined in the normal game. As yet these sets have no effect on the game. Instead of $\Rep_K(\cdot)$, the game uses $\Repx[F_K](\cdot,\salt)$ for freshly-sampled $\salt$, but these algorithms have identical behavior.

Next consider the game~$\G_1$. Here we replace $F_K$ entirely with a true random function $\Rnd$ which is lazily evaluated as necessary when $\REPO$, $\QRYO$, and $\UPO$ are called. The advantage of the adversary is then $\Adv{\errep}_{\struct_\prfbloom,r}(\advA) = \Adv{\prf}_F(\advB) + \Prob{\G_1(\advA)=1}$.

We want to ensure that salts do not repeat, so~$\G_2$ provides a game where the $\salt$ is randomly chosen exclusively from salts which have not been previously used. The game is idenitcal to~$\G_1$ until a salt repeats, which using the birthday bound will occur with probability $q_R^22^{-\lambda}$. Therefore $\Adv{\errep}_{\struct_\prfbloom,r}(\advA) \le \Adv{\prf}_F(\advB) + q_R^22^{-\lambda} + \Prob{\G_2(\advA)=1}$.

We then revise the game to~$\G_3$, where the adversary gets credit for queries $\QRYO(i,\qry)$ which are false positives for any $\pub_j$, regardless of the actual argument $i$ given to $\QRYO$. The oracle simply iterates through all representations the adversary has created and increments $\err_j$ for any $j$ where $\qry(\col_j)$ disagrees with $\Qryx[\Rnd](\pub_j,\qry)$. The adversary's view does not change as a result of this modification, and its advantage may only increase as additional opportunities to increment each $\err_i$ occur, so that $\Adv{\errep}_{\struct_\prfbloom,r}(\advA) \le \Adv{\prf}_F(\advB) + q_R^22^{-\lambda} + \Prob{\G_3(\advA)=1}$.

Next, in~$\G_4$ we modify $\QRYO$ and $\UPO$ to perform the same function. Each one tests the given element for false positive status across all extant representations, and then updates only the queried representation with that element. As in the case of the salted Bloom filter, we must remove the penalty for adding a false positive and increase the maximum size of the filter from $n$ to $n+r$ so that this change can never reduce the adversary's chances of success.  Because all queries are independent of each other, adding a known non-false-positive is again the optimal choice for the adversary under any circumstance. Furthermore, note that while the adversary may not be aware of its score in each representation, we can assume without loss of generality that it halts if it has sent several $\QRYO$ calls to the same representation and $r$ of those have returned a false positive. This means that no more than $r$ false positives will be added to any single representation. Then $\Adv{\errep}_{\struct_\prfbloom,r}(\advA) \le \Adv{\prf}_F(\advB) + q_R^22^{-\lambda} + \Prob{\G_4(\advA)=1}$.

Finally, in~$\G_5$ we keep track of how many $\QRYO$ calls have previously been made, and use $\setQ$ to keep track of those queries with indices in $\setI$. Only these $r$ queries can increment the error counters, but they do so if they cause an error in any of the currently-existing $\pub_j$. To win, the adversary must therefore return an $i$ so that all queries in $\setQ$ produce errors when directed at $\pub_i$. Assuming without loss of generality that the adversary does not add any elements to the set which it knows to be false positives, this condition is sufficient as well as necessary. Given that an adversary is successful in~$\G_3$, it is successful in~$\G_4$ with probability at least $\binom{q_T+q_U}{r}^{-1}$ since that is the chance of randomly selecting $r$ successful queries from the total collection of $q_T+q_U$ queries. Then we have $\Adv{\errep}_{\struct_\prfbloom,r}(\advA) \le \Adv{\prf}_F(\advB) + q_R^22^{-\lambda} + \binom{q_T+q_U}{r}\Prob{\G_4(\advA)=1}$. But since the salts are unique for each representation, the probability of winning in each representation is independent of the probability of winning in the others, and the so $\Prob{\G_4(\advA)=1}$ is just $q_R$ times the standard Kirsch and Mitzenmacher bound. Substituting this result into our earlier bound, we arrive at the final result of
$$\Adv{\errep}_{\struct_\prfbloom,r}(\advA) \le \Adv{\prf}_F(\advB) + \frac{q_R^2}{2^\lambda} + q_R\binom{q_T+q_U}{r}\left((1-e^{-k(n+r)/m})^k+O(1/(n+r))\right)^r.$$
\missingqed
\end{proof}

Next, we consider the case of a counting Bloom filter. In this setting, an update to add an element increments each of the counters mapped to by the hash functions, with some maximum ceiling $c$ above which the counter cannot increase. A Bloom filter would then just be a counting Bloom filter with a maximum counter value of 1, but we make the additional change that a counting Bloom filter can perform delete operations which work by hashing the element and decrementing the counters at those indices.

\begin{figure}
  \boxThmBFSaltCorrect{0.48}
  {
    \underline{$\G_0(\advA)$}\\[2pt]
      $\col \getsr \advA^H$; $\setC \gets \emptyset$; $\err \gets 0$\\
      $\pub \getsr \Rep[\hashbf[H]](\col)$\\
      $\bot \getsr \advA^{H,\QRYO,\UPO}$\\
      return $(\err \geq r)$
    \\[6pt]
    \oraclev{$\QRYO(x)$}\\[2pt]
      if $x \in \mathcal{C}$ then return $\bot$\\
      $\setC \gets \setC \union \{x\}$\\
      $a \gets \Qry[\hashbf[H]](\pub, x)$\\
      if $a \neq [x \in \col]$ then $\err \gets \err + 1$\\
      return~$a$
    \\[6pt]
    \oraclev{$\UPO(x,b)$}\\[2pt]
      $\setC \gets \emptyset$\\
      $a \gets \Qry[\hashbf[H]](\pub, \qry_x)$\\
      if $x \in \setC$ and $b = a \neq [x \in \col]$ then\\
      \tab $\err \gets \err-1$\\
      if $b = 1$ then\\
      \tab $\col \gets \col \union \{x\}$\\
      else\\
      \tab $\col \gets \col \setminus \{x\}$\\
      $\pub \gets \Up[\hashbf[H]](\pub,\up_{x,b})$\\
      return~$\bot$
    \\[4pt]
    \hspace*{-4pt}\rule{1.043\textwidth}{.4pt}
    \\[5pt]
    \oraclev{$\HASHO_1(\salt,x)$} \hfill\diffplus{$\G_2$}\;{$\G_1$}\hspace*{3pt}\\
      $\hh \getsr [m]^2$; $\vv \gets \fff(\hh$)\\
      if $\salt = \salt^*$ then\\
      \tab $\bad_1 \gets 1$; \diffplus{return $\vv$}\\
      if $T[\salt,x]$ is defined then $\vv \gets T[\salt,x]$\\
      $T[\salt,x] \gets \vv$;
      return $\vv$
  }
  {
    \underline{$\G_1(\advB)$}\\[2pt]
      $\salt^* \getsr \bits^\lambda$;
      $\col \getsr \advB^{\HASHO_1}$\\
      $\pub \gets \Repx[\HASHO_2](\col, \salt^*)$\\
      $\setC \gets \emptyset$;
      $\err \gets 0$\\
      $\bot \getsr \advB^{\HASHO_1,\QRYO,\UPO}$\\
      return $(\err \geq r)$
    \\[6pt]
    \oraclev{$\QRYO(\qry_x)$}\\[2pt]
      if $\qry_x \in \mathcal{C}$ then return $\bot$\\
      $\setC \gets \setC \cup \{\qry_x\}$\\
      $a \gets \Qry[\HASHO_2](\pub, \qry_x)$\\
      if $a \neq \qry_x(\col)$ then $\err \gets \err + 1$\\
      return~$a$
    \\[6pt]
    \oraclev{$\UPO(\up_x)$}\\[2pt]
      $\setC \gets \emptyset$\\
      $a \gets \Qry[\hashbf[H]](\pub, \qry_x)$\\
      if $a \neq \qry_x(\col)$ and $\qry_x \in \setC$ then\\
      \tab $\err \gets \err-1$\\
      $\col \gets \col \union \{x\}$\\
      $\pub \gets \Up[\HASHO_2](\pub,\up_x)$\\
      return~$\bot$
    \\[6pt]
    \oraclev{$\HASHO_2(\salt,x)$}\\[2pt]
      $\hh \getsr [m]^2$; $\vv \gets \fff(\hh$)\\
      if $T[\salt,x]$ is defined then\\
      \tab $\vv \gets T[\salt,x]$\\
      $T[\salt,x] \gets \vv$;
      return $\vv$
  }
  {
    \underline{$\G_3(\advB)$}\\[2pt]
    \oraclev{$\QRYO(\qry_x)$}\\[2pt]
      $a \gets \Qry[\HASHO_3](\pub, \qry_x)$\\
      if $a \neq \qry_x(\col)$ then $\err \gets \err + 1$\\
      $\col \gets \col \union \{x\}$
      $\pub \gets \Up[\HASHO_2](\pub,\up_x)$\\
      return~$a$
  }
  {
    \oraclev{$\UPO(\up_x)$}\\[2pt]
      $a \gets \Qry[\HASHO_3](\pub, \qry_x)$\\
      if $a \neq \qry_x(\col)$ then $\err \gets \err + 1$\\
      $\col \gets \col \union \{x\}$
      $\pub \gets \Up[\HASHO_2](\pub,\up_x)$\\
      return~$\bot$
    \\[6pt]
    \oraclev{$\HASHO_i(\salt,x)$}\\[2pt]
      $\hh \getsr [m]^2$; $\vv \gets \fff(\hh$)\\
      return $\vv$
  }
  \caption{Games 0--3 for proof of Theorem~\ref{thm:count-bf-bound}.}
  \label{fig:count-bf-bound}
\end{figure}

Because this structure has no private key, we can again use the result that the adversary's advantage in $\erreps$ is at most $q_R$ times their advantage in $\erreps1$. As in the case of an ordinary Bloom filter, we then want to reduce the counting Bloom filter to a structure using random functions rather than a concrete hash algorithm. Before the adversary makes its initial $\REPO$ query, it has no information about the filter that will be constructed, and so the adversary cannot perform better than random chance at guessing the hash. Given up to $q_H$ queries to $\HASHO$, the probability the adversary succeeds in this way, guessing the hash before the representation is created, is $q_H/2^\lambda$.

%%%

Since the adversary only has access to deletion, there are effectively four types of updates they can make: removing a false positive, removing a true positive, or removing a (true or false) negative. Attempting to remove an element which the filter believes is not present in the set will do nothing, so we may assume without loss of generality that the adversary does not do this. While removing a true positive clearly cannot increase the chances of getting a false positive, since counters are only decremented and not incremented, it is possible for removing a true positive to cause a false negative.

In the next game, we make three changes to the experiment. First, $\QRYO$ tests for potential false positives not in the current representation of the set, but instead in the initial representation constructed by $\Rep$. Next, each $\UPO$ call directly calls $\QRYO$, so that deleting an element which was originally a false positive increments the error counter in addition to producing possible false negatives. Finally, deletions do not merely decrement the associated counters, but actually reset them to 0.

First, note that repeated $\UPO$ and $\QRYO$ calls on elements outside the original set $\col$ are never helpful. With only deletions as updates, a true negative can never become a false positive, and since deletions zero out the associated counters repeated $\UPO$ calls on the same element do nothing. Additionally, because $\UPO$ checks against the original filter rather than the current filter, $\UPO$ calls do not decrease the false positive probability of further $\UPO$ calls. Since any false positives which could be found by $\QRYO$ can also be found by $\UPO$, in this new game there is no reason for an adversary to prefer a $\QRYO$ call on an element outside of the original set to an $\UPO$ call. In fact, there is no reason for an adversary to prefer making both a $\QRYO$ and $\UPO$ call on a single element outside $\col$ over just making an $\UPO$ call, since both produce the same result. We may therefore assume without loss of generality that all $\QRYO$ and $\UPO$ calls are made to different elements. Because $\UPO$ calls consist only of deletions, they can only increase the probability of false negatives and can only decrease the probability of false positives.

The adversary wins if it accumulates $r$ errors, and we may assume without loss of generality that it halts upon achieving this goal. The $r$ errors may accumulate over each of the $q_U$ calls to $\UPO$ as well as the $q_T$ calls to $\QRYO$. The $\UPO$ calls only produce false positives, while $q_T$ can produce false positives or false negatives. The probability of a false positive is bounded by the Kirsch and Mitzenmacher result of $(1-e^{kn/m})^k + O(1/n)$, while that of a false negative is given by the bound of Guo et al.

These can be split between false positives and false negatives. There may be up to $q_U + q_T$ chances to find a false positive, with the success probability given by the Kirsch and Mitzenmacher bound for a filter with $n$ elements, and up to $q_T$ chances to find a false negative after $q_U$ elements have been removed from the filter.

Suppose the errors consist of $f_p$ false positives and $f_n$ false negatives. The probability of this is given by $\binom{q_U + q_T}{f_p}p(k,n,m)^{f_p}\binom{q_T}{f_n}q(k,n,q_U,m)^{f_n}$.

% with insertion, it is also possible to insert an arbitrary element already in the set, insert an arbitrary element not already in the set, insert known false negatives, and insert known true negatives

% inserting an element already in the set may get rid of false negatives; if it does so, it may also introduce false positives
% inserting an element not in the set may get rid of false negatives and/or introduce false positives
% inserting a known false negative removes the false negative and may remove other false negatives, but may also introduce false positives
% inserting a known true negative may remove false negatives, but may also introduce false positives

% are known false negatives equivalent to each other, so that adding one is indistinguishable from adding another?
% no: consider {x,y} being tested for false negativity where x is found to be a false negative while y is not; then we remove another element and find that y is a false negative; x is likely to be missing more bits than y because the removal of the additional element may have decremented bits of x
% for creating false positives, the best that can happen is

% problem: the adversary learns about the filter from its queries, can adjust accordingly
% adversary may not need to guess salt at all: suppose they add x twice and remove y; if x is now a false positive, they know two of y's hashes collide
% adversary cannot guess what might happen when querying/updating on a new element, but can learn about elements that have already been inserted/removed/queried
% consider an element that has been queried but not inserted/deleted at any point; further insertions are no more likely to make this element a false positive than an element that has not been queried, while further deletions are not more likely to make this element a false positive at all; we may therefore assume that a query to an element which has not been previously manipulated is also to an element which has not been previously queried
% querying an element that has been previously inserted/deleted gives the adversary information of the form "this element shares at least one bit with {}, does not share more than one bit with {}, etc."
% the worst case is that the adversary gets to choose all of these factors, so that it can produce $k$ false negatives from every element, though this still requires an additional update and $k$ additional queries

% additionally, adversary can say "a is a false positive for S but not for T, therefore S and T are unlikely to share bits" or "a is a false positive for S and for T, therefore S and T must share at least k bits"

% advantage of an adversary in the single-private-rep salted mutable case is no greater than the advantage of an adversary in a multiple-(strictly-)private-rep unsalted immutable case
% by "strictly private" we mean \REPO does not return the representation and the adversary does not get access to \REVO
% what security properties does this have in general? ex. what is the security of an immutable unsalted Bloom filter with a strictly private representation?
% Bloom filter: the usual attacks don't work, but...

% want to get to a game where each query is actually independent of previous queries

% we can handle insertions alone or deletions alone; the question is how they interact
% insert an element, make queries, possibly delete element
% OR delete an element (false negative), make queries, possibly insert element
% false negatives / discovered false negatives...
% assume that the adversary can use queries to locally reconstruct a copy of the filter as it currently exists, and the positioning of previously-queried elements; however, they cannot predict how future additions/queries will work
% if removal of some elements will produce false negatives, or addition of other elements will produce

% Conjecture: the adversary's chances of finding a false positive are no better than in the case of an ordinary Bloom 
% this is certainly true in the deletion-only case
% in the case of insertion and deletion, consider an adversary which simply ignores all deletions and all insertions that do not increase the maximum set size seen so far

% on the other hand, if the adversary is only trying to produce false negatives, adding more elements to the filter is unhelpful since it can only lower the chances of finding a false negative... unless you later turn that element into a false negative
% however, this is unecessary unless q_T > n, so that you can mess with all the original elements and then begin adding in new elements to mess with

% at worst, the adversary learns all previously-queried hash values exactly... which allows replication of a Bloom-filter-like attack

\subsection{Attack on Counting Filters}

First, $\advA$ chooses an arbitrary set $\col$ of size $n$ and calls $\REPO$ to get a representation of $\col$. The adversary then chooses $q_U$ distinct elements not in $\col$ and makes a series of $\UPO$ calls to remove each of these from the set. Finally, the adversary chooses up to $q_T$ elements from $\col$ and makes a series of $\QRYO$ calls to test for membership of these elements in the set.

\begin{tabular}{|l|} 
 \hline
 \underline{adversary $\advA$:}\\
 $\col \gets [n]$\\
 $\REPO(\col)$\\
 for $i \in [q_U]$ do\\
 \tab $\UPO(0,\up_{i+n}')$\\
 for $j \in [q_T]$ do\\
 \tab $\QRYO(0,\qry_{j})$\\
 return 0\\
 \hline
\end{tabular}

\subsection{Cuckoo Filters}

Consider the case of a salted cuckoo filter with private representations. By `salted', we mean that both the hashing and the fingerprinting makes use of a salt chosen on a per-represntation basis. Because the structure is unkeyed, we can reduce to the single-representation case, whose advantage is lower by a factor of at most $q_R$.

As with the case of a Bloom filter, the goal is to move to a game where the salted hash function is replaced by a true random function. Doing so requires that the adversary never be able to guess the hash function. We split the hash oracle into two oracles, giving the adversary direct access to $\HASHO_1$ while the other oracles make use of $\HASHO_2$. For any $\advA$ for $\G_0$, there is $\advB$ for $\G_1$ which produces the same advantage by simulating $\advA$. [...]

The game $\G_2$ is the same as $\G_1$ until the $\mathrm{bad}$ flag is set, which occurs exactly when $\advB$ sends $(\salt^*,x)$ to its $\HASHO_1$ oracle for some $x$. Since the $\HASHO$ oracles use random sampling, there is only a $q_H/2^\lambda$ chance of the adversary guessing the salt during the experiment. So $\Adv{\erreps_1}_{\struct_s,r}(\advA) \le \Prob{\G_2(\advB) = 1} + q_H/2^\lambda$. Having separated out the $q_H/2^\lambda$ term, we may now assume that the adversary never guesses the salt.

Because membership tests rely on objects having the same fingerprint, and a single bucket may contain multiple copies of the same fingerprint, deleting $n$ false positives can only produce up to $n$ false negatives, one per fingerprint removed. Unless the filter becomes full, it is therefore never helpful for an adversary to delete a false positive, since it cannot increase the number of errors. We may then assume, subject to the condition that the adversary never has to worry about the filter becoming full, that the adversary only makes insertion updates. Furthermore, we may assume the adversary never inserts elements which are already in the set, or elements which have already been queried and found to be false positives, since these also cannot increase the number of errors.

We want to show that alternating between sequences of queries and updates is no better than making updates first and queries afterwards. We have shown that the adversary only makes two types of updates: inserting unqueried elements which are not in the set and inserting known true negatives.
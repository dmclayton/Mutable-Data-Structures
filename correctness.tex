\newcommand{\REVO}{\mathbf{Reveal}}
\newcommand{\HASHO}{\oraclefont{Hash}}
\newcommand{\diffplus}[1]{\fbox{#1}}
\newcommand{\diffplusbox}[1]{\fbox{\parbox{\dimexpr\textwidth-2\fboxsep-2\fboxrule\relax}{#1}}}
\newcommand{\diffminus}[1]{\colorbox{lightgray}{#1}}
\newcommand{\diffminusbox}[1]{\colorbox{lightgray}{\parbox{\dimexpr\textwidth-2\fboxsep-2\fboxrule\relax}{#1}}}
\newcommand{\hh}{\vectorfont{h}}
\newcommand{\fff}{\schemefont{Fn}}
\newcommand{\Rnd}{\schemefont{Rand}}
\newcommand{\Repx}{\Rep1}
\newcommand{\Qryx}{\Qry1}
\newcommand{\Upx}{\Up1}
\newcommand{\Ans}{\varfont{Ans}}
\newcommand{\setE}{\mathcal{E}}
\newcommand{\Resp}{\varfont{Resp}}
\def\ticks(#1,#2){\procfont{T}_{\hspace*{-1.5pt}#1}({#2})}
\newcommand{\highlighto}[1]{\colorbox{lightgray}{$\scriptstyle #1$}}
\newcommand{\highlight}[1]{\colorbox{lightgray}{$\displaystyle #1$}}
\begin{figure}[t]
  \twoColsNoDivide{0.48}
  {
     \raisebox{-5pt}{\experimentv{$\Exp{\errep}_{\struct,r}(\advA)$ {\colorbox{lightgray}{$\Exp{\erreps}_{\struct,r}(\advA)$}}}}\\[2pt]
      $\setP \gets \emptyset$\\
      $\ct \gets 0$ \\
      $\key \getsr \keys$\\
      $i \getsr \advA^{\REPO,\UPO,\QRYO\highlighto{,\REVO}}$\\
      \colorbox{lightgray}{if $i \in \setP$ then return 0} \\[2.0pt]
      return $[\sum_\qry \err_i[\qry] \geq r]$ 
    \\[6pt]
    \oraclev{$\REPO(\col)$}\\[2pt]
      $\ct\gets\ct+1$ \\
      $\pub_\ct \getsr \Rep_\key(\col)$\\
      $\setC_\ct \gets \emptyset$\\
      $\col_\ct \gets \col$\\
      $\mathrm{rv} \gets \pub_\ct$ \highlight{;\mathrm{rv} \gets \top} \\
      return $\mathrm{rv}$

      \medskip

    \oraclev{$\REVO(i)$}\\[2pt]
     $\setP \gets \setP \cup \{i\}$ \\
      return $\pub_i$
  }
  {
    \oraclev{$\UPO(i, \up)$}\\[2pt]
%      $X \gets \col^{v[i]}_i
%      $v[i] \gets v[i]+1$\\
      $X \gets \col_i$ \\
      $\col_i \gets \up(X)$\\
      $\pub_i \getsr \Up_\key(\pub_i, \up)$\\
      $\setC_i \gets \emptyset$\\
      for $\qry$ in $\err_i$ do\\
      \tab $a \gets \Qry_K(\pub_i, \qry)$\\
      \tab if $\err_i[\qry] > d(a,\qry(\col_i))$ then\\
      \tab\tab$\err_i[\qry] \gets d(a,\qry(\col_i))$\\
      $\mathrm{rv} \gets \pub_i$ \highlight{;\mathrm{rv} \gets \top}\\
      return $\mathrm{rv}$
      \medskip

    \oraclev{$\QRYO(i, \qry)$}\\[2pt]
      if $\qry \in \setC_i$ then return $\bot$\\
      $\setC_i \gets \setC \union \{\qry\}$\\
      $a \gets \Qry_K(\pub_i, \qry)$\\
      if $\err_i[\qry] < d(a,\qry(\col_i))$ then\\
      \tab$\err_i[\qry] \gets d(a,\qry(\col_i))$\\
      return $a$
  }
  \caption{Two notions of adversarial correctness. The $\errep$ notion captures correctness when the representation is always known to the adversary, while the $\erreps$ notion captures correctness when the representation is secret.}
  \vspace{6pt}\hrule
  \label{fig:security}
\end{figure}

We define two adversarial notions of correctness given by a pair of related experiments for a mutable data structure $\struct$ and error capacity $r$, one in which the representations of the true data are public ($\errep$), and one in which they are private ($\erreps$).  We will describe the former, as the latter is a closely related derivative.  Throughout, we assume that the adversary~$\advA$ does not make any pointless queries. 

Both experiments aim to capture the total weight of the errors caused by the adversary, at any point in time, with respect to the current data objects $\col_i$ and their representations $\pub_i$.  Because we consider mutable data objects and representations, the notion of ``current'' is defined by calls to the $\REPO$ and $\UPO$ oracles.  Specifically, for each~$\col_i$, both experiments maintain a set~$\setC_i$ that is initially set to empty (when~$\col_i$ is first assigned via a~$\REPO$-query), and it is reset to empty whenever~$\col_i$ is updated via an $\UPO$-query.  This is capturing the fact that applying a non-empty update function~$\up$ to~$\col_i$ produces a new data object.

To track errors, both experiments maintain an array $\err_i[]$ for every data object~$\col_i$ that has been defined.   Initially, $\err_i[]$ is implicitly assigned the value of~$\undef$ at every index.  For purposes of value comparison, we adopt the convention that $\undef < n$ for all $n \in \mathbb{R}$.
%
Now, the array~$\err_i$ is indexed by query functions~$\qry$, and the value of $\err_i[\qry]$ is the ``weight'' of the error caused by~$\qry$, with respect to the \emph{current} data object~$\col_i$ and \emph{current} representation~$\pub_i$ (of~$\col_i$).  
%
The value of~$\err_i[\qry]$ is updated within the $\QRYO$- and $\UPO$-oracles, but observe that $\err_i[\qry] = \undef$ until $(i,\qry)$ is queried to the $\QRYO$-oracle.  Intuitively, a representation~$\pub_i$ of data object~$\col_i$ cannot surface errors until it is queried.

When~$\QRYO(i,\qry)$ executes, the value in $\err_i[\qry]$ is overwritten iff the error caused by~$\qry$ is larger than the existing value of $\err_i[\qry]$.  The first time $(i,\qry)$ is queried to~$\QRYO$ this is guaranteed\footnote{\tsnote{Assuming no ``pointless'' queries are made by the adversary.}}.  Since the set $\setC_i$ is used to prevent (WLOG) the adversary from repeating a query~$\QRYO(i,\qry)$ for a given~$\col_i$, increases to the value of $\err_i[\qry]$ may only be made ``across'' updates to~$\col_i$.  This may seem to be overly conservative, as an error-heavy~$\col_i$ may become less so after an update.  But we account for this within the $\UPO$-oracle.
In particular, calls to the $\UPO$-oracle may only \emph{decrease} the value of~$\err_i[\qry]$.  

When a query $\UPO(i,\up)$ is made, the oracle first updates the data object~$\col_i$ and its corresponding representation.  The set~$\setC_i$ is reset to empty, because the data object~$\col_i$ is ``new'' again.
%
Now, for each defined value~$\err_i[\qry]$, we reevaluate the error that \emph{would} be caused by the previously asked~$\qry$ w.r.t. the newly updated $\col_i$ and $\pub_i$.  If the existing value of $\err_i[\qry]$ is larger than the error that~$\qry$ would cause to w.r.t. the newly updated~$\col_i$ and $\pub_i$, then we overwrite $\err_i[\qry]$ with the smaller value.  Doing so insures that the array~$\err_i$ does not overcredit the attacker for errors against the current data object and representation.

\tsnote{The next paragraph is a placeholder; it's too informal as written, but gets the point across for future edits.}
Let us give an example to help make concrete these definitional choices.  Say that the adversary queries a set~$\col = \{1,2,3\}$ to the $\REPO$-oracle, and learns that~$4$ is a false-positive value for the representation~$\pub$ of~$\col$.  If it adds~$4$ to~$\col$ via an update query, then it is no longer a false-positive.  Our definition insures that this is properly accounted.

\subsection{Example Data Structures}

In general, each probabilistic data structure has some bound on the error size per query, assuming the adversary is not fully adaptive. In each case we want to show that allowing the adversary full adaptivity does not significantly increase the error rate. Generally, we have a data object space $\mathcal{D} \subseteq 2^\mathcal{X}$, a collection of subsets of some universe $\mathcal{X}$, and a response space $\mathcal{R} = \{0,1\}$ with the usual metric of $d(m,n) = |m-n|$. Then the query space consists of indicator functions $\qry_x$ for $x \in \mathcal{X}$, so that $\qry_x(\col)$ is 1 if and only if $x \in \col$. The update space at least consists of insertions, and may also include deletions.

A typical case occurs with standard Bloom filters~\cite{bloomfilter}. Since there are no false negatives, the size of the error a non-adaptive adversary is expected to create per query is simply equal to the false positive rate, which is on the order of $(1-e^{-\frac{kn}{m}})^k$ for an $m$-bit array with $k$ hash functions storing up to $n$ values. Compressed Bloom filters~\cite{xxx} operate in the same way, with a false positive rate which must also take into account the degree of compression. The maximum amount we can compress a Bloom filter is determined by the probability that a given bit in the filter is 0, which is $\left(1-\frac{1}{m}\right)^kn$. This is closely approximated by $p = e^{-\frac{kn}{m}}$. For a given $p$, an optimal compressor will reduce an $m$-bit filter to $mH(p)$ bits, where $H$ is the entropy function $H(p) = -p\log p - (1-p)\log(1-p)$. Because of this, in order to compress the original $m$-bit filter down to $z$ bits we must have $z \ge mH(p)$. Note that an ordinary Bloom filter has false positive rate $(1-p)^k$; in the compressed case we instead have a false positive rate of $(1-p)^{-\frac{z \ln p}{nH(p)}}$.

Cuckoo filters~\cite{xxx} are a straightforward extension of this notion increasing $\mathcal{U}$ to include deletion operations, but fortunately the notions of correctness are still straightforward given that queries are simply testing for set membership. Counting bloom filters~\cite{xxx} both allow deletion and broaden the data object space to $\mathcal{D} \subseteq \N^X$.

Depending on the implementation, count-min sketch~\cite{xxx} may similarly use a data object space $\mathcal{D} \subseteq N^X$ or may use $\mathcal{D} \subseteq \Z^X$. In the former case we have two further sub-cases, the case where all updates increment the value associated with $x \in X$ and the case where updates may either increment or decrement the associated value (but not below 0). Additionally, count-min sketch supports multiple different types of queries, in each case yielding a response in $\mathbb{Z}$. For a point query, the difference between the query and the true value is bounded by $n\epsilon$ with probability $1-\delta$, where $n$ is the sum of the true frequencies of the stream elements. The maximum error of a single query is simply $n$, in the case that an element has never actually been added to the set but has incorrectly had all its counters incremented each of $n$ times another element has been added, so the expected non-adaptive error size is bounded above by $n\epsilon(1-\delta)+n\delta = n(\delta+\epsilon-\delta\epsilon)$.  Similarly, we find that the error size of an inner product query is bounded above by $n_1n_2\epsilon(1-\delta)+n_1n_2\delta = n_1n_2(\delta+\epsilon-\delta\epsilon)$. Range queries are tricky and I'm not quite sure on the precise bound in terms of $\delta$ and $\epsilon$.

Bloomier filters~\cite{xxx} are designed to represent arbitrary functions instead of set membership, providing a larger response space $\mathcal{R}$ than the binary $\{0,1\}$ used with Bloom filters and altering the data object space to a set of the form $\mathcal{D} \subseteq \mathcal{R}^\mathcal{X}$. One of the response values is $\bot$ to indicate that value has not been associated with any element of $\mathcal{R}$, and the only type of error that can occur is that a value which should return $\bot$ instead returns some other element of $\mathcal{R}$. Again we have a situation where `false positives' are the only type of error which can occur. Update queries allow for changes in the element of $\mathcal{R}$ associated with each element of $\mathcal{X}$, though changing an element's associated value to or from $\bot$ is not permitted: neither insertion nor deletion is possible with this structure.

Stable Bloom filters~\cite{xxx} are an example of a structure in the Bloom filter family which our notions cannot work with easily. One way in which they are unique is in featuring a probabilistic update algorithm, causing objects in the filter to probabilistically decay over time. While our syntax accounts for that, the more significant difference is that a stable Bloom filter only has good accuracy guarantees once it has seen enough updates to `stabilize'. Our current security notions do not encompass this type of conditional accuracy guarantee.

\subsection{Correctness Notion Equivalence}

Let $A$ be an adversary for the security notion in figure 1. We construct an adversary $B$ for the security notion in figure 4 which simulates $A$, forwarding any queries $A$ makes to its own oracles. As $B$ does so, it also keeps a table $\err_i$ for each representation, associating any $\qry$ passed to $\QRYO$ with the resulting error that query causes. Note that $B$ can easily compute this, since it knows both $\qry$ and the set $\col$ being represented, and so can compute the expected value of the query, $\qry(\col)$, on its own. Whenever $B$ makes an update query, it clears the table $\err_i$. If at any point the sum of the values in any $\err_i$ is at least $r$, 

The adversary $B$ maintains a counter $\ct$ of how many representations have been created. Whenever $A$ makes a $\REPO(\col)$ query, $B$ increments $\ct$, initializes $\col_\ct$ to $\col$ and $\err_\ct$ to 0, constructs an empty table $\Resp_\ct$, forwards the query to its own oracle, and (in the public representation case only) returns the response from this oracle to $A$. Whenever $A$ makes an $\UPO(i,\up)$ query, $B$ sets $\err_i$ to 0, $\Resp$ to an empty table, and $\col_i$ to $\up(\col_i)$ before forwarding the query to its own oracle and (again only in the public representation case) returning the response to $A$. Whenever $A$ makes a $\QRYO(i,\qry)$ query, $B$ checks whether $\Resp[i,\qry]$ is defined. If so, $B$ simply returns that value to $A$. Otherwise, $B$ forwards the query to its own oracle, sets $\Resp[i,\qry]$ to the returned value, increments $\err_i$ by the difference between the returned value and $\qry(\col_i)$, and then returns the result of the query to $A$. Finally, any $\REVO$ or $\HASHO$ queries are simply forwarded to $B$'s oracle, and their responses are returned to $A$.

\subsection{Correctness Proofs \& Attacks}

First, we note that any structure which is insecure in the immutable case is also insecure in the mutable case. Any adversary in the immutable case is identical to a corresponding adversary in the mutable case which simply never makes use of the $\UPO$ oracle. From this we know that standard (unsalted, unkeyed) Bloom filters cannot ensure correctness in the mutable case.

\begin{lemma}[\errep1 implies \errep for keyless structures]\label{lemma:errep}
  Let $\struct = (\Rep, \Qry, \Up)$ be a data structure with key
  space $\{\emptystr\}$. For every $t, q_R, q_T, q_U, q_H, r\geq 0$, it holds that
  \[
    \Adv{\errep}_{\struct,r}(t, q_R, q_T, q_U, q_H) \leq
    q_R\cdot\Adv{\errep1}_{\struct,r}(O(f(t)), q_T, q_U, q_H) \,,
  \]
  where $f(t) = t + (q_R-1)\ticks(\Rep,t) + q_T\ticks(\Qry,t) + q_U\ticks(\Up,t)$.
\end{lemma}

\begin{proof}For a fixed $r \ge 0$, let $A$ be an $\errep$ adversary which runs in $t$ time steps and makes $q_R$ $\REPO$ queries, $q_T$ $\QRYO$ queries, $q_U$ $\UPO$ queries, and $q_H$ RO queries. We construct an adversary $B$ for $\errep1$ as follows.

First, $B$ initializes a counter $ct \gets 0$ and a set $\setC \gets \emptyset$, and samples $q \getsr [q_R]$. Next $B$ executes $A$, simulating the answers to its oracle queries as follows. When $A$ asks the query $\REPO(\col)$, $B$ sets $ct \gets ct + 1$ and stores $\col_{ct} \gets \col$. Then, if $ct = q$, $B$ forwards $\col$ to its own $\REPO$ oracle and returns the resulting $\pub$ to $A$. Otherwise, $B$ computes and returns $\pub_{ct} \getsr \Rep(\col)$. When $A$ asks for the query $\QRYO(i,\qry)$, $B$ first checks if $(i,\qry) \in \setC$ and returns $\bot$ if this condition holds. Otherwise $B$ forwards $(i,\qry)$ to its $\QRYO$ oracle and returns $a$ if $i = q$, and reutrns $\Qry(\pub_i,\qry)$ otherwise. Similarly, when $A$ makes an $\UPO(i,\up)$ query, $B$ forwards $(i,\up)$ to its $\UPO$ oracle if $i = q$ and evaluates $\Up(\pub_i,\up)$ otherwise. Finally, queries from $A$ to its RO are simply forwarded to $B$'s RO. When $A$ halts and outputs $j$, $B$ does the same.

If $j = q$, then $B$ wins if $A$ does, since all queries from $A$ to $\pub_j$ were forwarded to $B$'s $\QRYO$ oracle. Given that $q$ is sampled uniformly from the range $[q_R]$, it follows that

$$\Adv{\errep}_{\struct,r}(t, q_R, q_T, q_U, q_H) \leq q_R\cdot\Adv{\errep1}_{\struct,r}(O(f(t)), q_T, q_H).$$

Note that $B$ makes at most $q_T$ queries to $\QRYO$ and $q_H$ queries to its RO. Since $A$ runs in $t$ time steps and writing a bit takes 1 time step, the input length to any $\Rep$, $\Qry$, or $\Up$ evaluated by $B$ is at most $t$ bits. Hence, adversary $B$ runs in time $O(t+(q_R-1)\ticks(\Rep,t)+q_T\ticks(\Qry,t)+q_U\ticks(\Up,t))$.\hfill\qed
\end{proof}

Say that a structure is {\em invertible} if for every representation $\mathsf{pub}$ and update $\mathsf{up} \in \mathcal{U}$ there is $\mathsf{up}' \in \mathcal{U}$ such that $\Prob{\textsc{Up}(\textsc{Up}(\mathsf{pub},\mathsf{up}),\mathsf{up}' = \mathsf{pub})} = 1$.  For example, the updates for both counting Bloom filters and count-min sketches are invertible, by consequence of being deterministic and having both insertion and deletion operations. Each also has the additional feature of having a natural choice of {\em empty} structure such that every other structure can be constructed by starting with the empty structure and performing a corresponding sequence of update operations.

There is a similar equivalence between $\erreps1$ and $\erreps$; the reasoning is identical to that in the above proof.

\begin{lemma}[Salts do not affect \errep for invertible structures]\label{lemma:noinvsalt}
  Let $\struct = (\Rep, \Qry, \Up)$ be a data structure with a salt randomly initialized at runtime and $\struct'$ be the same structure using a fixed value from the salt space in place of a randomized salt. For every $t, q_R, q_T, q_U, q_H, r\geq 0$, it holds that
  \[
    \Adv{\errep}_{\struct,r}(t, q_R, q_T, q_U, q_H) \leq
    \Adv{\errep}_{\struct',r}(O(t), 1, q_T, q_U+2n(q_R+q_T+q_U), q_H) \,,
  \]
  where $n$ is the longest minimal sequence of update operations needed to generate a data structure in the space.
\end{lemma}
\begin{proof}
Consider an adversary $A$ in the case of a non-salted data structure which makes $q_R$ queries to $\REPO$ and $q_T$ queries to $\QRYO$. We construct an adversary $B$ for the salted case which produces the same errors as follows. First, $B$ initializes a counter $ct$ to 0 and calls the $\REPO$ oracle on the empty set, receiving an empty representation $\pub$ together with the salt used to create the representation. Then $B$ runs $A$, answering its oracle queries as follows. Whenever $A$ makes a query of the form $\REPO(\col)$, $B$ sets $ct \gets ct + 1$ and calls $\UPO$ repeatedly on $\pub$ to transform it into a representation of $\col$. Then $B$ returns the modified $\pub$ to $A$, stores $\col_{ct} \gets \col$, and performs the opposite updates in reverse order to return to the original empty representation $\pub$. If $A$ makes a query of the form $\QRYO(i,\qry)$, $B$ calls $\UPO$ repeatedly to transform $\pub$ into a representation of $\col_i$ and then returns the result of querying its own oracle with $\QRYO(1, \qry)$. Then once again $B$ performs the inverse updates to transform $\pub$ back into the original empty representation. If $A$ queries for $\UPO(i, \up)$, $B$ sets $\col_i \gets \up(\col_i)$ and again uses $\UPO$ queries to transform $\pub$ into a representation of $\col_i$, returns the value of $\pub$, and then performs opposite $\UPO$ queries to return $\pub$ to the empty representation. Finally, $B$ forwards any of $A$'s RO queries to its own RO.

In general this may use as many as $2n(q_R+q_T+q_U)$ update queries, where $n$ is the longest minimal sequence of update operations needed to generate a data structure in the space. Furthermore $B$ succeeds if $A$ does, so the adversaries' advantages are equal.\hfill\qed
\end{proof}

Note that in a counting filter, $n$ is equal to the maximum supported multiset size, and in count-min sketch $n$ is equal to the sum of the absolute values of the elements' frequences. While this means a potentially large number of update queries are needed in the general case, for an unkeyed structure it turns out that far fewer updates are required. In the above proof, the only reason we needed to undo each sequence of $\UPO$ queries was because the following query might involve the representation of a different structure. But in the case of a salted but unkeyed structure, our initial lemma shows that for any adversary there is an adversary making only a single $\REPO$ query with an advantage differing by a factor of at most $q_R$. For an adversary which only ever makes one $\REPO$ query, we never need to reset the representation to the empty representation, and in the above simulation we can omit all $\UPO$ queries except those needed for $A$'s single $\REPO$ query and any $\UPO$ queries made by $A$ itself.

\begin{lemma}[Salts do not affect \errep for unkeyed structures with public representation]\label{lemma:nosalt}
  Let $\struct = (\Rep, \Qry, \Up)$ be a data structure with a salt randomly initialized at runtime and $\struct'$ be the same structure using a fixed value from the salt space in place of a randomized salt. For every $t, q_R, q_T, q_U, q_H, r\geq 0$, it holds that
  \[
    \Adv{\errep}_{\struct,r}(t, q_R, q_T, q_U, q_H) \leq
    \Adv{\errep}_{\struct',r}(O(t), q_R, r, n, q_H) \,,
  \]
  where $n$ is the longest minimal sequence of update operations needed to generate a data structure in the space.
\end{lemma}
\begin{proof}
Consider an adversary $A$ for a non-salted data structure which makes $q_R$ queries to $\REPO$ and $q_T$ queries to $\QRYO$. We construct an adversary $B$ for the salted case which produces the same errors by simulating $A$. The adversary $B$ maintains a counter $\ct$ initialized to 0. Whenever $A$ makes a $\REPO(\col)$ query, $B$ first forwards a $\REPO(\emptyset)$ query to its own $\REPO$ oracle, stores the salt $\salt_\ct$ and set $\col_\ct$, initializes an error counter $\err_\ct$ to 0 and a set $\setE_\ct$ to $\emptyset$, and increments $\ct$. Then $B$ simply computes the actual representation of $\col$ using the salt $\salt_\ct$ and returns that to $A$. Whenever $A$ makes a $\QRYO(i,\qry)$ query, $B$ computes the result itself. If it would cause an error, $\err_i$ is incremented appropriately and $\qry$ is added to $\setE_i$, and then $B$ returns the result of the query to $A$. Whenever $A$ makes a $\UPO(i, \up)$ query, $B$ resets $\err_i$ to 0 and again computes the result itself before returning it to $A$. Finally, $B$ simply forwards any $\HASHO$ queries to its own oracle. When $A$ halts and outputs $i$, $B$ performs a series of $\UPO(i, \up)$ queries to transform $\pub_i$ into a representation of its stored $\col_i$ and then calls the queries in $\setE_\ct$. Because the salt remains unchanged when the set is updated, these will induce errors for $B$ just as they did for $A$, and $B$ will win if and only if $A$ does.
\end{proof}

For a concrete example, we present an attack against salted counting Bloom filters.

\begin{center}
  \begin{tabular}{ | c | c | c | c | c | }
    \hline
    Salt & Key & Private Representation & Irreversible Updates & Correctness \\ \hline
    0 & 0 & 0 & 0 & Attack 1 \\ \hline
    0 & 0 & 0 & 1 & Attack 1 \\ \hline
    0 & 0 & 1 & 0 & Attack 1 \\ \hline
    0 & 0 & 1 & 1 & Attack 1 \\ \hline
    0 & 1 & 0 & 0 & Attack 2 \\ \hline
    0 & 1 & 0 & 1 & Attack 2 \\ \hline
    0 & 1 & 1 & 0 & Attack 2 \\ \hline
    0 & 1 & 1 & 1 & Attack 2 \\ \hline
    1 & 0 & 0 & 0 & Attack 3 \\ \hline
    1 & 0 & 0 & 1 & Attack 3 \\ \hline
    1 & 0 & 1 & 0 & ? \\ \hline
    1 & 0 & 1 & 1 & Theorem~\ref{thm:bf-priv-salt-bound} \\ \hline
    1 & 1 & 0 & 0 & Attack 4 \\ \hline
    1 & 1 & 0 & 1 & Theorem~\ref{thm:bf-key-bound} \\ \hline
    1 & 1 & 1 & 0 & ? \\ \hline
    1 & 1 & 1 & 1 & Theorem~\ref{thm:bf-key-bound} \\
    \hline
  \end{tabular}
\end{center}

\subsection{Attacks}

\begin{enumerate}
 \item The adversary chooses a maximally large test set $\col$ and simulates $\Rep$ to produce a representation $\pub$. The adversary then simulates $\Rep$ for many arbitrarily chosen singleton sets disjoint from $\col$, checking each one to see if its element is a false positive for $\pub$. Once it has accumulated $r$ false positives, call $\REPO$ on $\col$, call $\QRYO$ for each false positive found, and return 1.
 \item The adversary chooses a maximally large test set $\col$ and calls $\REPO$ to produce a representation $\pub$. The adversary then calls $\REPO$ for many arbitrarily chosen singleton sets disjoint from $\col$, using $\QRYO$ on each to determine if its element is a false positive for the representation constructed for $\col$. Once it has accumualated $r$ false positives, return 1.
 \item The adversary chooses a maximally large test set $\col$ and calls $\REPO$ to receive a representation $\pub$ together with the salt $\salt$. Using the known salt, the adversary simulates $\Rep$ for many arbitrarily chosen singleton sets disjoint from $\col$, checking each one to see if its element is a false positive for $\pub$. Once it has accumulated $r$ false positives, call $\QRYO$ for each such false positive and return 1.
 \item The adversary chooses a test set $\col_0$ and a target set $\col_1$, performing a search on representable subsets of $\col_0$ represented as a tree ordered by $\subseteq$. Moving up or down in the tree is accomplished using $\UPO$, and at each node $\QRYO$ is called for each element of $\col_1$ to determine which are false positives. Once $r$ false positives are found, halt and return 1.
\end{enumerate}

\subsection{Proofs}

Our first secure data structure is the salted Bloom filter, which at the time a representation is created chooses a salt $\salt$ which it will use for all further queries and updates. In order for security to be guaranteed, we must ensure that the representation, and in particular the salt, is kept secret from the adversary.

\begin{theorem}[Correctness Bound for Private-Representation Salted Bloom Filters]\label{thm:bf-priv-salt-bound}
Fix integers $k, m, n, \lambda, r\geq 0$, let $H \colon \bits^* \to [m]$ be a function, and let $\struct_s = \SBF[\hashbf[H],k,m,n,\lambda]$ in the private-representation setting.
  For every $t, q_R, q_T, q_H \geq 0$ such that $q_T \geq r$, it holds that
  \begin{eqnarray*}
    \Adv{\erreps}_{\struct_\saltybloom,r}(t,&q_R,& q_T, q_U, q_H) \leq \\ && q_R \cdot
     \left[
      \frac{q_H}{2^\lambda} +
      {\dbinom{q_T+q_U}{r}} \left( (1-e^{-k(n+r)/m})^k + O(1/(n+r)) \right)^r
    \right] \,,
\end{eqnarray*}
where $H$ is modeled as a random oracle.
\end{theorem}

This proof first reduces to the single-representation case, which as shown in Lemma 1 will reduce the adversary's advantage by at most a factor of $q_R$. The main idea behind the proof is to remove the adversary's adaptivity a step at a time. We isolate the possibility of the adversary guessing the salt, which would allow it to mount its own offline attack on the filter without relying on the $\QRYO$ oracle. If the adversary does not guess the salt, the outputs of the $\REPO$, $\QRYO$, and $\UPO$ oracles are unpredictable to the adversary, producing uniformly randomly distributed bits to set (for $\REPO$ and $\UPO$) or to check (for $\QRYO$). Under the assumption that the adversary does not predict the salt, queries made to distinct elements are independent of each other. The only remaining issue is that the adversary can potentially gain an advantage by testing whether some object $x$ is a false positive for the filter, and then updating the filter to include $x$ only if the test query returned `false'. An analysis shows that this is now (once imperfect pseudorandom functions and salt collisions have been dealt with) the only way for the adversary to gain an advantage over making queries to an immutable Bloom filter. Because this adaptive strategy introduces tricky conditional possibilities, we cannot compute an exact value for the adversary's advantage. Instead, we move to an alternate scenario where each $\QRYO$ also produces a free update and every $\UPO$ first performs a free query. This makes $\QRYO$ and $\UPO$ calls indistinguishable, so that the adversary is effectively making a series of independent random queries that each have a chance to increment the error counter. Because the size of the filter can only increase, the probability of a false positive from any one of these queries is bounded above by the probability of a false positive on the final maximally-sized filter, a probability which is given by the Kirsch and Mitzenmacher bound.

\begin{figure}
  \boxThmBFSaltCorrect{0.48}
  {
    \underline{$\G_0(\advA)$}\\[2pt]
      $\col \getsr \advA^H$; $\setC \gets \emptyset$; $\err \gets 0$\\
      $\pub \getsr \Rep[\hashbf[H]](\col)$\\
      $\bot \getsr \advA^{H,\QRYO,\UPO}$\\
      return $(\err \geq r)$
    \\[6pt]
    \oraclev{$\QRYO(\qry_x)$}\\[2pt]
      if $\qry_x \in \mathcal{C}$ then return $\bot$\\
      $\setC \gets \setC \union \{\qry_x\}$\\
      $a \gets \Qry[\hashbf[H]](\pub, \qry_x)$\\
      if $a \neq \qry_x(\col)$ then $\err \gets \err + 1$\\
      return~$a$
    \\[6pt]
    \oraclev{$\UPO(\up_x)$}\\[2pt]
      $\setC \gets \emptyset$\\
      $a \gets \Qry[\hashbf[H]](\pub, \qry_x)$\\
      if $\qry_x \in \setC$ and $a \neq \qry_x(\col)$ then\\
      \tab $\err \gets \err-1$\\
      $\col \gets \col \union \{x\}$\\
      $\pub \gets \Up[\hashbf[H]](\pub,\up_x)$\\
      return~$\bot$
    \\[4pt]
    \hspace*{-4pt}\rule{1.043\textwidth}{.4pt}
    \\[5pt]
    \oraclev{$\HASHO_1(\salt,x)$} \hfill\diffplus{$\G_2$}\;{$\G_1$}\hspace*{3pt}\\
      $\hh \getsr [m]^2$; $\vv \gets \fff(\hh$)\\
      if $\salt = \salt^*$ then\\
      \tab $\bad_1 \gets 1$; \diffplus{return $\vv$}\\
      if $T[\salt,x]$ is defined then $\vv \gets T[\salt,x]$\\
      $T[\salt,x] \gets \vv$;
      return $\vv$
  }
  {
    \underline{$\G_1(\advB)$}\\[2pt]
      $\salt^* \getsr \bits^\lambda$;
      $\col \getsr \advB^{\HASHO_1}$\\
      $\pub \gets \Repx[\HASHO_2](\col, \salt^*)$\\
      $\setC \gets \emptyset$;
      $\err \gets 0$\\
      $\bot \getsr \advB^{\HASHO_1,\QRYO,\UPO}$\\
      return $(\err \geq r)$
    \\[6pt]
    \oraclev{$\QRYO(\qry_x)$}\\[2pt]
      if $\qry_x \in \mathcal{C}$ then return $\bot$\\
      $\setC \gets \setC \cup \{\qry_x\}$\\
      $a \gets \Qry[\HASHO_2](\pub, \qry_x)$\\
      if $a \neq \qry_x(\col)$ then $\err \gets \err + 1$\\
      return~$a$
    \\[6pt]
    \oraclev{$\UPO(\up_x)$}\\[2pt]
      $\setC \gets \emptyset$\\
      $a \gets \Qry[\hashbf[H]](\pub, \qry_x)$\\
      if $a \neq \qry_x(\col)$ and $\qry_x \in \setC$ then\\
      \tab $\err \gets \err-1$\\
      $\col \gets \col \union \{x\}$\\
      $\pub \gets \Up[\HASHO_2](\pub,\up_x)$\\
      return~$\bot$
    \\[6pt]
    \oraclev{$\HASHO_2(\salt,x)$}\\[2pt]
      $\hh \getsr [m]^2$; $\vv \gets \fff(\hh$)\\
      if $T[\salt,x]$ is defined then\\
      \tab $\vv \gets T[\salt,x]$\\
      $T[\salt,x] \gets \vv$;
      return $\vv$
  }
  {
    \underline{$\G_3(\advB)$}\\[2pt]
    \oraclev{$\QRYO(\qry_x)$}\\[2pt]
      $a \gets \Qry[\HASHO_3](\pub, \qry_x)$\\
      if $a \neq \qry_x(\col)$ then $\err \gets \err + 1$\\
      $\col \gets \col \union \{x\}$
      $\pub \gets \Up[\HASHO_2](\pub,\up_x)$\\
      return~$a$
  }
  {
    \oraclev{$\UPO(\up_x)$}\\[2pt]
      $a \gets \Qry[\HASHO_3](\pub, \qry_x)$\\
      if $a \neq \qry_x(\col)$ then $\err \gets \err + 1$\\
      $\col \gets \col \union \{x\}$
      $\pub \gets \Up[\HASHO_2](\pub,\up_x)$\\
      return~$\bot$
    \\[6pt]
    \oraclev{$\HASHO_i(\salt,x)$}\\[2pt]
      $\hh \getsr [m]^2$; $\vv \gets \fff(\hh$)\\
      return $\vv$
  }
  \caption{Games 0--3 for proof of Theorem~\ref{thm:bf-priv-salt-bound}.}
  \label{fig:bf-priv-salt-bound}
\end{figure}

\begin{proof} We first reduce from the $\erreps$ case to the $\erreps1$ case, which by Lemma 1 may scale the adversary's advantage only by a factor of $q_R$. In the $\erreps1$ scenario we may assume without loss of generality that the adversary $\advA$ never makes use of the $\REVO$ oracle, since doing so would prevent the adversary from having any possibility of winning. The game~$\G_0$ is simply the $\erreps1$ experiment with the $\REVO$ oracle omitted, so $\Adv{\errep1}_{\struct_s,r}(\advA) = \Prob{\G_0(\advA) = 1}$. In~$\G_1$ we split the hash oracle into three, giving the adversary access $\HASHO_1$ in both stages of the game, while $\HASHO_2$ is reserved for oracular use by $\Repx$, $\QRYO$, and $\UPO$. For any $\advA$ for~$\G_0$, there is $\advB$ for~$\G_1$ which produces the same advantage by simulating $\advA$. This adversary first creates its own table $R$ with all values initially undefined. When $\advA$ makes a query $w$ to $H$, $\advB$ returns $R[w]$ if that entry in the table is defined. Otherwise, if there are $\salt \in \bits^\lambda$, $j \in [k]$, and $x \in \bits^*$ such that $w = \langle\salt, j, x\rangle$, forward $(\salt,x)$ to $\HASHO_1$. For each $j \in [k]$, set $R[\langle\salt, j, x\rangle] = \vv_j$, where $\vv$ is the output of the $\HASHO_1$ oracle. If there is no such triple $\langle\salt, j, x\rangle$, just sample $r$ from $[m]$ uniformly and set $R[w] = r$. In either case, return $R[w]$ to $\advA$. When $\advA$ outputs its collection $\col$, $\advB$ outputs $\col$ as well. Any queries by $\advA$ to $\QRYO$ or $\UPO$ are forwarded to $\advB$'s corresponding oracle. The simulation is perfect because $\Rep[\hashbf[H]](\col)$ and $\Up[\hashbf[H]](\col,\up)$ are identically distributed to $\Rep[\HASHO_2](\col)$ and $\Up[\HASHO_2](\col,\up)$. Because we have a perfect simulation, $\Adv{\erreps1}_{\struct_s,r}(\advA) = \Prob{\G_1(\advA) = 1}$.

The game~$\G_2$ is the same as~$\G_1$ until $\bad_1$ is set, which occurs exactly when $\advB$ sends $(\salt^*,x)$ to $\HASHO_1$ for some $x$. In the first phase, there is again a $q_1/2^\lambda$ chance of the adversary guessing the salt. In the second phase, the random sampling used by $\HASHO_i$ ensures that each call the adversary makes to the $\HASHO_i$ oracle is independent of all previous calls. We therefore have a $q_2/2^\lambda$ chance of the adversary guessing the salt during this phase, for a total chance of $q_H/2^\lambda$ chance of the adversary guessing the salt at some point during the experiment. Then $\Adv{\erreps_1}_{\struct_s,r}(\advA) \le \Prob{\G_2(\advB) = 1} + q_H/2^\lambda$. Having taken this into account, we may now assume the adversary never guesses the salt.

We want to show that alternating between sequences of queries and sequences of updates is no better than making one long series of updates and then one long sequence of queries. There are three types of updates the adversary can make: updates to add elements that have been queried and found to be false positives; updates to add elements that have been queried and found not to be false positives; and updates to add elements that have not been queried yet. We may assume without loss of generality that the adversary never makes the first type of update, since doing so is never beneficial (it does not change the representation at all and decreases the number of errors the adversary has found).

Note that the choices of $\vv$ constructed by the $\HASHO_i$ oracles are independent of all previous queries. Because of this, any update of type 3 is equivalent to any other update of type 3; the probability of any bit being flipped by one update is the same as the probability of the bit being flipped by the other update. Similarly, any update of type 2 is equivalent to any other update of type 2, but is not the same as type 3 since the probability is conditioned on $\vv$ not being a false positive. We assume the worst case, namely that all updates are type 2 (i.e. at least one bit is flipped by each update).

Because the adversary never guesses the salt, $\HASHO_1$ simply functions as a random oracle. Furthermore, we can assume the adversary never adds an element of $\col$ to $\col$ and never makes a $\QRYO$ call for an element which is already in $\col$, since neither of these provides any additional information and neither affects the rest of the experiment in any way.

Now we move to the game~$\G_3$. Here each $\QRYO$ query also calls $\UPO$ to add that element to $\col$. Additionally, the penalty for adding known false positives is removed. To avoid penalizing the adversary by prematurely maxing out the number of elements in $\col$ because of added false positives, we also increase the maximum size of $\col$ from $n$ to $n+r$. Because the adversary (without loss of generality) stops after accumulating $r$ errors, only $r$ false positives will be added to $\col$ and so a maximum size of $n+r$ is sufficient to produce no penalty for the adversary. Furthermore, each $\UPO$ call is preceded by a $\QRYO$ call. Neither of these changes can produce a worse result for the adversary, so $\Prob{\G_2(\advB) = 1} \le \Prob{\G_3(\advB) = 1}$. Now, however, there is no longer any distinction between $\QRYO$ and $\UPO$ calls. All calls to either oracle are independent of each other and produce the same effect, querying and then updating $\col$. Each of these queries for false positives is at most as successful as a query to a Bloom filter with $n+r$ elements, so the adversary's probability of finding a false positive on any query is bounded above by the standard success rate for a Bloom filter with those parameters, $(1-e^{-k(n+r)/m})^k+O(1/(n+r))$. The adversary is required to produce $r$ errors over the course of $q_T+q_U$ queries, which by the binomial theorem gives an advantage bound of $\Prob{\G_3(\advB) = 1} \le \binom{q_T+q_U}{r}((1-e^{-k(n+r)/m})^k+O(1/(n+r)))^r$.

The full adversarial advantage is then
$$\Adv{\erreps_1}_{\struct_s,r}(\advA) \le q_R \cdot \left(\frac{q_H}{2^\lambda} + \binom{q_T+q_U}{r}((1-e^{-k(n+r)/m})^k+O(1/(n+r)))^r\right).$$

\hfill\qed
\end{proof}

The case of a Bloom filter whose representations are always public but which uses a private key in addition to a salt is almost identical. In practice the bound is actually somewhat stronger: the instance of $q_H$ in the bound is replaced with $q_R$, which means that the adversary which attempts to guess the salt must rely entirely on `online' queries as opposed to `offline' queries to $q_H$.

These two theorems together show that in order to guarantee correctness in the streaming setting, one must either ensure that representations are kept private from potential adversaries or use a secret key in addition to a salt when implementing a Bloom filter. In either case, some hidden information is necessary to protect the filter from an adversary.

\begin{theorem}[Correctness Bound for Keyed Bloom Filters]\label{thm:bf-key-bound}
Fix integers $k, m, n, \lambda, r\geq 0$, where $m \geq \lambda$, and let $F \colon \bits^\lambda \times \bits^* \to [m]$ be a function.
  For every $t, q_R, q_T \geq 0$ such that $q_T \geq r$, it holds that
  \begin{eqnarray*}
    \Adv{\erreps}_{\struct_\saltybloom,r}(t,&q_R,& q_T, q_U, q_H) \leq \\ && q_R \cdot
     \left[
      \frac{q_R}{2^\lambda} +
      {\dbinom{q_T+q_U}{r}} \left( (1-e^{-k(n+r)/m})^k + O(1/(n+r)) \right)^r
    \right] \,.
\end{eqnarray*}
\end{theorem}

Despite the similarity of the bounds, the proof is somewhat different than in the private-representation salted-but-unkeyed case. The first thing we want to do is move from a pseudorandom function to a truly random function, about which we can produce bounds much more easily. This is problematic only if the adversary can easily distinguish between the pseudorandom function and a true random function, which for a good choice of pseudorandom function will not occur. Next, we note that for a true random function, the queries and updates in different representations will be independent unless the randomly-chosen salt $\salt$ is the same across two or more representations. The birthday bound gives us the chances of this occurring, but for sufficiently large salt size $\lambda$ the chance of this will also be small. Once we have reduced to a case where the filter operates on true random functions and salts never repeat, we are again left with a series of almost-independent queries where the adversary may be able to gain an advantage by following up a $\QRYO$ call with an $\UPO$ call. We take the same approach as in the previous case, turning $\QRYO$ and $\UPO$ into identical independent queries and then applying the Kirsch and Mitzenmacher upper bound for the largest set represented during the game.

\begin{figure}
  \twoCols{0.50}
  {
    \vspace{-7.5pt}
    \experimentv{$\G_0(\advA)$}
                  \hfill\diffplus{$\G_1$}\;\diffminus{$\G_0$}\\[2pt]
      $\setI \getsr \setP_r([q_T])$;
      $\setZ, \setQ \gets \emptyset$\\
      $\err_0, \ct, \ct' \gets 0$;
      $\ky \getsr \keys$\\
      $i \getsr \advA^{\REPO,\QRYO,\UPO}$;
      return $(\err_i \ge r)$
    \\[6pt]
    \oraclev{$\REPO(\col)$}\\[2pt]
      $\ct \gets \ct + 1$; $\err_\ct \gets 0$;
      $\col_\ct \gets \col$\\
      $\salt \getsr \bits^\lambda$;
      \diffminus{$\pub_\ct \gets \Repx[F_\ky](\col, \salt)$}\\
      \diffplus{$\pub_\ct \gets \Repx[\Rnd](\col, \salt)$}\\
      return $\pub_\ct$
    \\[6pt]
    \oraclev{$\QRYO(i,\qry)$}\\[2pt]
      $\ct' \gets \ct' + 1$\\
      \diffminus{$\res \gets \Qryx[F_\ky](\pub_i,\qry)$}\\
      \diffplus{$\res \gets \Qryx[\Rnd](\pub_i, \qry)$}\\
      if $\res \neq \qry(\col_i)$ then $\err_i\gets\err_i+1$\\
      return~$\res$
    \\[6pt]
    \oraclev{$\UPO(i,\up_x)$}\\[2pt]
      $\col_i \gets \up(\col_i)$\\
      \diffminus{$\pub_i \getsr \Upx[F_\ky](\pub_i,\up)$}\\
      \diffplus{$\pub_i \getsr \Upx[\Rnd](\pub_i,\up)$}\\
      $a \gets \Qry[\hashbf[H]](\pub, \qry_x)$\\
      if $a \neq \qry_x(\col)$ and $a \in \setC$ then\\
      \tab $\err_i \gets \err_i-1$\\
      return~$\pub_i$
    \\[6pt]
    \algorithmv{$\Rnd(x)$}\\[2pt]
      if $T[x]$ is undefined then $T[x] \getsr [m]$\\
      return $T[x]$
    \\
    \hspace*{-4pt}\rule{1.049\textwidth}{.4pt}
    \\[2pt]
    \oraclev{$\REPO(\col)$}
                            \hfill\diffplus{$\G_2$}\;\diffminus{$\G_1$}\hspace*{0pt}\\[2pt]
      $\ct \gets \ct + 1$; $\err_\ct \gets 0$;
      $\col_\ct \gets \col$\\
      \diffminus{$\salt \getsr \bits^\lambda$}
      \diffplus{$\salt \getsr \bits^\lambda \setminus \setZ$; $\setZ \gets \setZ
      \union \{\salt\}$}\\
      $\pub_\ct \gets \Repx[\Rnd](\col, \salt)$\\
      return $\pub_\ct$
  }
  {
    \vspace{-7.5pt}
    \oraclev{$\REPO(\col)$}
                    \hfill\diffplus{$\G_3$}\;\diffminus{$\G_2$}\hspace*{0pt}\\[2pt]
      $\ct \gets \ct + 1$; $\err_\ct \gets 0$;
      $\col_\ct \gets \col$\\
      $\salt \getsr \bits^\lambda \setminus \setZ$;
      $\setZ \gets \setZ \union \{\salt\}$\\
      $\pub_\ct \gets \Repx[\Rnd](\col, \salt)$\\
      \diffplusbox{
        for each $\qry \in \setQ$ do\\
        \tab $a \gets \Qryx[\Rnd](\pub_\ct, \qry)$\\
        \tab if $a \ne \qry(\col_\ct)$ then $\err_\ct \gets \err_\ct +1$
      }
      return $\pub_\ct$
    \\[6pt]
    \oraclev{$\QRYO(i, \qry)$}\\[2pt]
       $\ct' \gets \ct' + 1$\\
      \diffminusbox{
        $\res \gets \Qryx[\Rnd](\pub_i, \qry)$\\
        if $\res \neq \qry(\col_i)$ then $\err_i\gets\err_i+1$\\
        return~$\res$
      }
      \diffplusbox{
        \foreach{j}{1}{\ct}\\
        \tab $a_j \gets \Qryx[\Rnd](\pub_j, \qry)$\\
        \tab if $\qry\not\in\setQ$ and $a_j \ne \qry(\col_j)$\\
        \tab\tab then $\err_j \gets \err_j + 1$\\
        $\setQ \gets \setQ \union \{\qry\}$\\
        return~$a_i$
      }
    \\[6pt]
    \oraclev{$\UPO(i, \up)$}\\[2pt]
      $\col_i \gets \up(\col_i)$\\
      $\pub_i \getsr \Upx[\Rnd](\pub_i,\up)$\\
      $a \gets \Qry[\hashbf[H]](\pub, \qry_x)$\\
      if $a \neq \qry_x(\col)$ and $qry_x \in \setC$ then\\
      \tab $\err \gets \err-1$\\
      return~$\pub_i$
    \\[3pt]
    \hspace*{-4pt}\rule{1.049\textwidth}{.4pt}
    \\[2pt]
    \oraclev{$\QRYO(i, \qry_x)$},\hfill\diffplus{$\G_5$}\;{$\G_4$}\hspace*{0pt}\\[2pt]
      $\ct' \gets \ct' + 1$\\
      \foreach{j}{1}{\ct}\\
      \tab $a_j \gets \Qryx[\Rnd](\pub_j, \qry)$\\
      \tab if \diffplus{$\ct' \in \setI$ and} $\qry\not\in\setQ$ and $a_j \ne \qry(\col_j)$\\
      \tab\tab then $\err_j \gets \err_j + 1$\\
      \diffplus{if $\ct' \in \setI$ then} $\setQ \gets \setQ \union \{\qry\}$\\
      $\col_i \gets \up_x(\col_i)$\\
      $\pub_i \getsr \Upx[\Rnd](\pub_i,\up)$\\
      return~$a_i$
    \\[2pt]
    \oraclev{$\UPO(i,\up_x)$}\\[2pt]
      $\QRYO(i, \qry_x)$; return~$\pub_i$\\
  }
  \caption{Games 0--5 for the proof of
  Theorem~\ref{thm:bf-key-bound}.}
  \label{fig:bf-prf-correct}
\end{figure}

\begin{proof}We define algorithms $\Repx[f]: \bits^* \times \bits^\lambda \to \bits^*$, $\Qryx[f]: \bits^* \times \mathcal{Q} \to \mathcal{R}$, and $\Upx[f]: \bits^* \times \mathcal{U} \to \bits^*$ for arbitrary functions $f: \bits^* \to [m]$. The algorithm $\Repx[f]$ takes a set and salt, returning a representation of the Bloom filter constructed as usual using the given salt, but with calls to $F_K$ replaced with calls to $f$. Note that the salt is included in the output representation. Similarly, $\Qryx[f]$ and $\Upx[f]$ perform queries and updates on representations using $f$. in place of $F_K$.

First we move to the game~$\G_0$, which does not change the semantics of the usual correctness experiment. We assume without loss of generality that the adversary does not make the same query twice between updates, so there is no need to keep track of $\col$. We instead have sets $\setI$, $\setZ$, $\setQ$ not defined in the normal game. As yet these sets have no effect on the game. Instead of $\Rep_K(\cdot)$, the game uses $\Repx[F_K](\cdot,\salt)$ for freshly-sampled $\salt$, but these algorithms have identical behavior.

Next consider the game~$\G_1$. Here we replace $F_K$ entirely with a true random function $\Rnd$ which is lazily evaluated as necessary when $\REPO$, $\QRYO$, and $\UPO$ are called. The advantage of the adversary is then $\Adv{\errep}_{\struct_\prfbloom,r}(\advA) = \Adv{\prf}_F(\advB) + \Prob{\G_1(\advA)=1}$.

We want to ensure that salts do not repeat, so~$\G_2$ provides a game where the $\salt$ is randomly chosen exclusively from salts which have not been previously used. The game is idenitcal to~$\G_1$ until a salt repeats, which using the birthday bound will occur with probability $q_R^22^{-\lambda}$. Therefore $\Adv{\errep}_{\struct_\prfbloom,r}(\advA) \le \Adv{\prf}_F(\advB) + q_R^22^{-\lambda} + \Prob{\G_2(\advA)=1}$.

We then revise the game to~$\G_3$, where the adversary gets credit for queries $\QRYO(i,\qry)$ which are false positives for any $\pub_j$, regardless of the actual argument $i$ given to $\QRYO$. The oracle simply iterates through all representations the adversary has created and increments $\err_j$ for any $j$ where $\qry(\col_j)$ disagrees with $\Qryx[\Rnd](\pub_j,\qry)$. The adversary's view does not change as a result of this modification, and its advantage may only increase as additional opportunities to increment each $\err_i$ occur, so that $\Adv{\errep}_{\struct_\prfbloom,r}(\advA) \le \Adv{\prf}_F(\advB) + q_R^22^{-\lambda} + \Prob{\G_3(\advA)=1}$.

Next, in~$\G_4$ we modify $\QRYO$ and $\UPO$ to perform the same function. Each one tests the given element for false positive status across all extant representations, and then updates only the queried representation with that element. As in the case of the salted Bloom filter, we must remove the penalty for adding a false positive and increase the maximum size of the filter from $n$ to $n+r$ so that this change can never reduce the adversary's chances of success.  Because all queries are independent of each other, adding a known non-false-positive is again the optimal choice for the adversary under any circumstance. Furthermore, note that while the adversary may not be aware of its score in each representation, we can assume without loss of generality that it halts if it has sent several $\QRYO$ calls to the same representation and $r$ of those have returned a false positive. This means that no more than $r$ false positives will be added to any single representation. Then $\Adv{\errep}_{\struct_\prfbloom,r}(\advA) \le \Adv{\prf}_F(\advB) + q_R^22^{-\lambda} + \Prob{\G_4(\advA)=1}$.

Finally, in~$\G_5$ we keep track of how many $\QRYO$ calls have previously been made, and use $\setQ$ to keep track of those queries with indices in $\setI$. Only these $r$ queries can increment the error counters, but they do so if they cause an error in any of the currently-existing $\pub_j$. To win, the adversary must therefore return an $i$ so that all queries in $\setQ$ produce errors when directed at $\pub_i$. Assuming without loss of generality that the adversary does not add any elements to the set which it knows to be false positives, this condition is sufficient as well as necessary. Given that an adversary is successful in~$\G_3$, it is successful in~$\G_4$ with probability at least $\binom{q_T+q_U}{r}^{-1}$ since that is the chance of randomly selecting $r$ successful queries from the total collection of $q_T+q_U$ queries. Then we have $\Adv{\errep}_{\struct_\prfbloom,r}(\advA) \le \Adv{\prf}_F(\advB) + q_R^22^{-\lambda} + \binom{q_T+q_U}{r}\Prob{\G_4(\advA)=1}$. But since the salts are unique for each representation, the probability of winning in each representation is independent of the probability of winning in the others, and the so $\Prob{\G_4(\advA)=1}$ is just $q_R$ times the standard Kirsch and Mitzenmacher bound. Substituting this result into our earlier bound, we arrive at the final result of
$$\Adv{\errep}_{\struct_\prfbloom,r}(\advA) \le \Adv{\prf}_F(\advB) + \frac{q_R^2}{2^\lambda} + q_R\binom{q_T+q_U}{r}\left((1-e^{-k(n+r)/m})^k+O(1/(n+r))\right)^r.$$

\hfill\qed
\end{proof}

Consider the case of a counting Bloom filter. In this setting, an update to add an element increments each of the counters mapped to by the hash functions, with some maximum ceiling $c$ above which the counter cannot increase. A Bloom filter would then just be a counting Bloom filter with a maximum counter value of 1, but we make the additional change that a counting Bloom filter can perform delete operations which hashes the element, checks to make sure all the counters corresponding to the hashes are nonzero, and then decrements them.

Note that if an adversary constructs a representation with $n$ elements and discovers a false positive, it can increase the error counter up to $c$ total times by re-adding the same $n$ elements $c$ times each using $\UPO$ calls.

Beyond this, is there any point to adding elements multiple times? Note that if adding an element once does not produce a false positive, adding it again will not produce a false positive either, since the exact same bits will be set again

% public rep
%   immutable
%     no salt
%       no key: insecure (old paper)
%       key: insecure because adversary can just ask for representations and then find false positives itself
%     salt
%       no key: semi-secure (old paper)
%       key: secure (old paper)
%   mutable
%     no salt
%       no key: insecure because immutable case is insecure
%       key: insecure because immutable case is insecure
%     salt
%       no key: adversary asks for empty representation, gets salt, then proceeds with attack as in the unsalted case
%       key: secure (Theorem 2)
% private rep
%   immutable
%     no salt
%       no key: same as public-rep case because construction of representations is deterministic
%       key: same as public-rep case because construction of representations is deterministic
%     salt
%       no key: at least as secure as public rep case; is it on par with public rep secret key + salt?
%       key: at least as secure as public rep case
%   mutable
%     no salt
%       no key: same as public-rep case because construction of representations is deterministic
%       key: same as public-rep case because construction of representations is deterministic
%     salt
%       no key: secure (Theorem 1)
%       key: secure because public rep case is secure
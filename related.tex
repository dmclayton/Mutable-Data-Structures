\paragraph{Comparison with Naor-Yogev}
As previously noted, Naor and Yogev~\cite{naor2015bloom} were the first to
formalize adversarial correctness of Bloom filters.  Our work extends theirs
significantly in several directions. First, we consider abstract data
structures, rather than only set-membership structures.  Even with respect to
the specific case of correctness for set-membership structures, our work offers
several advantages as compared to the Naor-Yogev treatment.
%
One, our syntax distinguishes between the (secret) key and the public portion of
a data structure, an important distinction that is missing in their work.
%
Two, the Naor-Yogev definition of correctness allows the adversary to make
several queries, some of which may produce incorrect results; the attacker then
succeeds if it outputs a \emph{fresh} query that causes an error. This
separation seems arbitrary, and we propose instead a parameterized definition in
which the attacker succeeds if it can cause a certain number of (distinct)
errors during its entire execution.
%
Three, Naor and Yogev analyze the correctness of a new Bloom filter variant of
their own design. In contrast, we are mainly interested in analyzing existing,
real-world constructions to understand their security.

\paragraph{Other related works}
There is a long tradition in computer science of designing structures that
concisely (but probabilistically) represent data so as to support some set of
queries, and each of these structures has its own interesting security
characterisitcs~\cite{chazelle2004bloomier,cormode2005improved,DP08a,DF03,fredman1984storing,mironov2011sketching}.

We have already mentioned the ubiquity of Bloom filters in support of efficient
network communication and computing protocols.  They also find use in
security-critical environments, including spam filters, (distributed)
denial-of-service attack detection, and deep packet
inspection~\cite{tarkoma2012theory}.  Recently, Bloom filters were proposed as a
means of efficient certificate-revocation list (CRL)
distribution~\cite{larisch2017crlite}, a crucial component of public-key
infrastructures.

% Correctness attacks
Correctness of data structures in adversarial settings is well-motivated in the
security literature and in practice.
%
Perhaps the earliest published attack on the correctness of a data structure was due to
Lipton and Naughton~\cite{lipton1993clocked} who showed that timing analysis of
record insertion in a hash table allows an adversary to adaptively choose
elements so as to increase look-up time, effectively degrading a service's
performance.
%
Crosby and Wallach~\cite{crosby2003denial} exploited hash collisions to increase
the average URL load time in Squid, a web proxy used for caching content in
order to reduce network bandwidth.
%
More recently, Gerbet \etal~\cite{gerbet2015power} described \emph{pollution
attacks} on Bloom filters, whereby an adversary inserts a number of
adaptively-chosen elements with the goal of forcing a high false-positive rate.
Although some of their attacks exploit weak (i.e., non-cryptographic) hash
functions (as do~\cite{crosby2003denial}), their methodology is effective even
for good choices of hash functions.
%
They suggest revised parameter choices for Bloom filters (i.e., filter length and
number of hashes) in order to cope with their attacks.


\ignore{
Finally, we note that the dictionary construction considered in
Section~\ref{sec:dict} bares resemblance (at least structurally) to
\emph{garbled Bloom filters}, a tool used recently for efficient private-set
intersection~\cite{dong2013when,rindal2017improved}.
}

% NOTE(all) Below are notes and references we considered adding to related work.
\if{0}{
  Correctness in adversarial settings has been considered for broader ranges of
  data structures.  Mironov, Naor, and Segev~\cite{mironov2011sketching} studied
  a setting in which non-colluding parties interact with a third-party
  \emph{referee} in order to compute a function of their data: For example,
  whether their sets are equal, or the approximate size of their intersection.
  The parties, which share a common reference string, but otherwise do not
  communicate, send a concise \emph{sketch} of their data to the referee, who
  performs the computation and publishes the result  The adversary is modeled as
  a malicious party attempting to skew the result.
  %
  \cpnote{It would be interesting to see if there's a connection between our
  notion of correctness and their setting.}
}\fi

\if{0}{
  \emph{Secure indexes}, proposed by Eu-Jin Goh~\cite{goh2003secure}, structure
  a document so that it can be searched by keyword if the querying party has a
  special \emph{trapdoor} for the keyword. The party issuing trapdoors has a
  secret key.  \jnote{I'm not sure the work of Goh is super relevant. Or, if it
  is, then so is any searchable encryption scheme.}
  %
  \cpnote{I agree ... I included it since it was cited in the
  survey~\cite{tarkoma2012theory} as an example of a ``secure'' Bloom filter
variant.}
}\fi

\if{0}{
  Other security notions for data structures, beyond correctness and privacy,
  have been considered.  For example, \emph{authenticated data
  structures}~\cite{tamassia2003authenticated} allow a trusted third party to
  certify the validity of a query on a data set maintained by an untrusted
  server.
}\fi

\if{0}{
  We recommend reading the Naor-Yogev paper for a survey of related work and a
  discussion of related papers. Here we mention a few additional practical
  works, but stress that this only scratches the surface.
  %
  \jnote{Rather random collection of papers using Bloom filters and variants. I
  removed it for now, since it's not clear that they have any particular
relevance to us. I kept only the refs that seemed directly relevant.}
  %
  As previously mentioned, Bloom filters and their relatives are some of the most
  widely used data structures supporting set-membership queries. As examples,
  Hbase, the open-source implementation of Google's BigTable storage
  system~\cite{chang2008bigtable}, a Hadoop-based NoSql database designed to
  handle large datasets, includes an implementation of Bloom filters and
  counting Bloom filters, and he Squid proxy~\cite{fan2000summary} uses a Bloom
  filter as a ``summary'' of the set of URLs in its cache in order to improve
  latency for web-object retrieval. Reynolds and
  Vahdat~\cite{reynolds2003efficient} proposed an efficient distributed search
  engine that can be used to search for files containing a particular keyword.
  Their search engine maps the keywords of each file into a Bloom filter; a
  look-up of the keyword in the Bloom filter tells whether the node has files
  containing that keyword or not. Stochastic Fair Blue~\cite{feng2001stochastic}
  uses counting Bloom filter to manage non-responsive TCP traffic.
}\fi

\if{0}{
  \cite{gao2006internet} is an application of BFs for detecting pollution
  attacks on web caches.
  %
  \heading{Related work: attacks}
  \tsnote{Brought these back into the text just to help Chris get up to speed.}
  \begin{itemize}
    \item Niedermeyer et al., ``Cryptanalysis of Basic Bloom Filters Used for
      Privacy-Preserving Record Linkage'', breaking privacy of
      secret-hash-function Bloom filters. \tsnote{Journal of Privacy and
      Confidentiality, 2014}

    \item Gerbet, Kumar and Lauradoux, ``The power of evil choices in bloom
      filters''. \tsnote{DSN'15: Looks like a real goldmine of related work!}

    \item Crosby and Wallach, ``Denial of Service via Algorithmic Complexity
      Attacks'' \tsnote{Gives attacks on Squid}

    \item Gao et al., ``Internet Cache Pollution Attacks and Countermeasures''
  \end{itemize}

  %\ignore{
  \heading{Related work: definitions(?)}
  \begin{itemize}
    \item Nojima and Kadobayashi, ``Cryptographically Secure Bloom Filters''.
      \tsnote{Gives some security definitions for privacy. Quick scan, not super
      clear what they achieve. The definition of client-privacy (Definition 1) for
      example, makes no sense to me.  Actually, likewise for server-privacy
      (Definition 2).  Both seem vague and thoroughly underspecified.}

    \item Naor and Yogev

    \item Eujin Goh, ``Secure Indexes'' \tsnote{A secure index can
      be used for set membership.  Builds a secret-key data
      structure (an Index) that allows searching for keyword~$w$
      if one holds the trapdoor $T_w$ for~$w$, where the trapdoor
      depends on the secret key.  Main construction uses
      traditional Bloom filters and a PRF.  Construction appears
      quite inefficient, needing a very long secret key, turning a
      keyword~$w$ into a bunch of PRF outputs, and then storing
      each of these PRF outputs in the BF.  Haven't read the full
      analysis; don't know if this was ever published. }
      \jnote{Never published. I think this work uses Bloom filters
      for encrypted search; I don't remember the paper having much
      to say about Bloom filters themselves.}
  \end{itemize}

  \heading{Related work: constructions}
  \begin{itemize}
    \item Bellovin and Cheswick, ``Privacy-Enhanced Searches Using Encrypted Bloom
    Filters''.

  \item Kerschbaum , ``Public-Key Encrypted Bloom Filters with
    Applications to Supply Chain Integrity''.

  \item S\"{a}rell\"{a} et al., ``BloomCasting: Security in Bloom Filter Based Multicast''.

  \item Dong, Chen,
      Wen, ``When Private Set Intersection Meets Big Data: An Efficient and
      Scaleable Protocol'' \tsnote{``garbled bloom filters'', which actually store
      the set element by storing~$k$ xor-shares, one at each of the~$k$ hash
      indices (with care for reusing shares if hash collisions occur); also
      and``oblivious bloom intersection''}\tsnote{If the filter and the hash
      functions are public, there is a naive attack that works for some
      interesting parameters.}

  \item Tarkoma, Rothenberg, Lagerspetz ``Theory and Practice of Bloom Filters in Distributed
      Systems''
      %
      \tsnote{Great high-level coverage.  Only found preprint version though.}
      \cpnote{{ieeexplore.ieee.org/iel5/9739/6151681/05751342.pdf}}

    \item Durham, Kantarcioglu, Xue, Kuzu, Malin ``Composite Bloom Filters for
      Secure Record Linkage'' \tsnote{Per-field BFs, sampled and composed into
      single BF that is then permuted by a secret random permutation.  No clear
      statement of the problem that is being solved.  Should pull full version and
      get details.}
  \end{itemize}

  \heading{Related work: tangential}
  \begin{itemize}
    \item Chang and Mitzenmacher ``Privacy Preserving Keyword Searches on Remote Encrypted Data''.

    \item Mitzenmacher and Vadhan. ``Why Simple Hash Functions Work: Exploiting
      the Entropy in a Data Stream''.

    \item Dodis et al. ``Fuzzy Extractors: How to Generate Strong Keys from
      Biometrics and Other Noisy Data'' \tsnote{Introduces ``secure sketches'',
      which is a representation of a single-element set that is information
      theoretically private (up to some function of the min-entropy of the
      element); only tangentially related to ``sketches'' as defined in the Bloom
      filter literature.}
  \end{itemize}
}\fi

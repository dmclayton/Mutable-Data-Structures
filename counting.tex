\label{sec:count}
\begin{figure}
  \twoColsNoDivide{0.33}
  {
    \underline{$\Rep^R_K(\col)$}\\[2pt]
      $\salt \getsr \bits^\lambda$ \com{Choose a salt $\salt$}\\
      $\pub \gets \langle \zeroes(m), \salt\rangle$\\
      for $x \in \col$ do \\
        $\tab \pub \gets \Up^R_K(\pub, \up_{x,1})$\\
        $\tab$if $\pub = \bot$ then return $\bot$\\
      return $\pub$
    \\[6pt]
      \underline{$\Qry^R_K(\langle \v.M, \salt\rangle,x)$}\\[2pt]
      $\v.X \gets R_K(\salt \cat x)$\\
      for $i \in \v.X$ do\\
        $\tab$if $\v.M[i] = 0$ then return 0\\
      return 1
  }
  {
    \underline{$\Up^R_K(\langle \v.M, \salt\rangle, \up_{x,b})$}\\[2pt]
      if $c \geq n$ then return $\bot$\\
      $\v.M' \gets \v.M$;
      $\v.X \gets R_K(\salt \cat x)$\\
      for $i$ in $\v.X$ do\\
      $\tab$ $a \gets \v.M'[i]$\\
      $\tab$ if $a = 0 \wedge b < 0$ then return $\bot$\\
      $\tab \v.M'[i] \gets \v.M'[i] + b$\\
      $\v.M \gets \v.M'$\\
      return $\langle \v.M, \salt \rangle$
  }
  \caption{Keyed structure $\countbloom[R,\ell,\lambda]$ given by
  $(\Rep^R,\Qry^R,\Up^R)$ is used to define the $\ell$-thresholded version of a
  counting filter. The parameters are a function $R:
  \keys\by\bits^* \to [m]^k$ and integers $\ell, \lambda \geq0$. A concrete scheme
  is given by a particular choice of parameters. The function $\hw'$, used to
  determine if the filter is full, is defined in Section~\ref{sec:prelims}.}
  \label{fig:cbf-def}
\end{figure}

Counting filters are a modified version of Bloom filters. Like ordinary
Bloom filters, counting filters are only designed to handle set membership
queries, but counting filters are designed to
allow for deletion as well as insertion~\cite{fan2000summary}.
%
Our construction
$\countbloom[R,\ell,\lambda]$ defined in Figure~\ref{fig:cbf-def} involves an
$\ell$-thresholded version of this structure. The traditional description of
a counting filter involves a parameter $n$ describing the size of the input
multiset, whereas our $\ell$ describes the maximum number of nonzero counters. As
in the case of ordinary Bloom filters, this does not significantly change the operation
of the filter in a non-adversarial setting, since for random input multisets the
number of nonzero counters is closely related to the number of elements in the
multiset.
%
In the presence of an adversary, however, we expect $\ell$-thresholding to
provide better security bounds than the traditional definition would provide by
handicapping pollution attacks and similar adversarial strategies.

We will show that, in the \errep\ setting, the counting filter is
insecure regardless of whether $\ell$-thresholding is used or not, and
regardless of the details of the behavior of the hash function or PRF used to
insert and delete elements. On the other
hand, we show that \erreps\ security is achievable even under the assumption
that a salted but unkeyed hash function is used, i.e. $\lambda > 0$ but $\keys =
\{\emptyset\}$.

\heading{Error function for frequency queries}
%
Unlike with a Bloom filter, counting filters must account
for two different types of errors: false positives and false negatives. To be as
general as possible, we define a parametrized error function~$\delta$ for
positive $\delta^+, \delta^- \in \R$ as
\begin{equation}
  \delta(x, y) =
  \begin{cases}
    0 & \text{if}\ x = y \\
    \delta^+ & \text{if}\ x = 1, y = 0 \\
    \delta^- & \text{if}\ x = 0, y = 1
  \end{cases}
\end{equation}
This means that false positives are given a weight of $\delta^+$ while false
negatives are given a weight of $\delta^-$, and correct responses are given a
weight of 0.

\subsection{Insecurity of public counting filters}\label{sec:pub-count-bad}
Unlike in the Bloom filter case, good security bounds cannot be achieved for a
counting filter in the \errep\ setting, even if salts and/or private keys are
used. The insecurity is due to the relative power of the $\UPO$ oracle compared
to the Bloom filter, which adds and subtracts numeric values stored in the structure rather
than seeing only the effects of a bitwise-OR.
Because of this, the adversary can mount
an attack similar to the target-set coverage attack for a Bloom filter even if a
PRF is used for
hashing. First, the adversary calls $\REPO(\emptyset)$ to get an empty
representation. The adversary can then call $\UPO$ to insert an element into the
set, see exactly what the outputs of each of the hash functions are, and then
call $\UPO$ again to delete the element. By doing this repeatedly, an adversary
can determine the outputs of the PRF for $u$ different inputs using $2u$ calls
to $\UPO$. The combination of public representation with the insertion
and deletion operations effectively provides an oracle for the secretly-keyed
PRF. Once a sufficiently large number of PRF outputs has been determined,
the adversary can construct the test and target set used for the target-set
coverage attack (Section~\ref{sec:bad-bfs}). The adversary then calls $\UPO$ several more times to insert
each element of the test set into the filter, and then
each element of the target set will be overestimated.

In actual use, this specific attack may not be feasible for the adversary.
However, as long as the filter is public, the adversary can easily determine the
exact results of inserting or deleting any element just by seeing which counters
are incremented or decremented. For this reason it is not enough that the
function used to perform queries and updates is impossible for the adversary to
simulate, since the adversary can build a lookup table just by watching the
filter as it is updated. Instead, we must require that the filter itself be kept
secret from the adversary.

\subsection{Security of private, $\ell$-thresholded counting filters}

\begin{theorem}\label{thm:counting-erreps}
Fix integers $q_R,q_T,q_U,q_H,q_V, r, t \geq 0$, let $p_\ell = ((\ell+1)/m)^k$,
and let
$r' = \lfloor r/\max(\delta^+,k\delta^-) \rfloor$. For all such
$q_R,q_T,$ $q_U,q_H,q_V,r$, and~$t$, if $r' > p_\ell q_T$ then
  \begin{equation*}
  \begin{aligned}
   \Adv{\erreps}_{\Pi,\delta,r}(O(t),\,&q_R,q_T,q_U,q_H,q_V) \leq q_R \cdot \left[\frac{q_H}{2^\lambda} + e^{r'-p_\ell q_T}\left(\frac{p_\ell q_T}{r'}\right)^{r'}\right],
  \end{aligned}
\end{equation*}
where $H$ is modeled as a random oracle.
\end{theorem}
The proof is similar to the case of count min-sketch, but has additional
difficulties because the deletion operation is somewhat more complex to deal
with in the case of a counting filter. In particular, because the filter contains
single bits rather than counters, the adversary may be able
to `delete' an element incorrectly believed to be in the set in order to induce
false negatives. We therefore delay the proof until after the proof
of Theorem~\ref{thm:scms-erreps-th} in Section~\ref{sec:sketch-proof} for count min-sketches. The two proofs are
in fact quite similar due to similarity of the two structures, and in particular
the fact that the allowed updates are the same.

So we find that, unlike Bloom filters, there is no simple tweak that can be performed
to a counting filter to provide good \errep\ security bounds. In particular, it
does not achieve security even in the immutable setting, and adding a secret key
does not help.
%
However, the bound above shows that in settings where filters can be assumed
secret it is possible to prove an upper bound on the number of overestimates an
adversary can cause. In particular, we recommend the combination of random
per-representation salts and $\ell$-thresholding in order to mitigate possible
attacks in the \erreps\ setting.

False negatives impact the bound more
than false positives due to the scaling factor of $k$ that appears in $r'$
{\bf\color{red} OH NO, THIS~$r'$ IS THE FIRST THING ON THIS LINE!!!!},
which indicates that applications seeking to minimize false negatives will
require larger filters than those seeking to minimize false positives. This is
distinctly different than in the non-adaptive setting, where false positives are
much more common in counting filters than false negatives, and therefore much
more relevant in determining the minimum size of the filter.

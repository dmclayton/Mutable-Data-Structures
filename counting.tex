\begin{figure}
  \twoColsNoDivide{0.22}
  {
    \underline{$\Rep^R_K(\col)$}\\[2pt]
      $\salt \getsr \bits^\lambda$\\
      $\pub \gets \langle 0^m, \salt, 0\rangle$\\
      for $x \in \col$ do \\
        $\tab \pub \gets \Up^R_K(\pub, \up_{x,1})$\\
        $\tab$if $\pub = \bot$ then return $\bot$\\
      return $\pub$
    \\[6pt]
      \underline{$\Qry^R_K(\langle M, \salt, c \rangle,x)$}\\[2pt]
      $X \gets R_K(\salt \cat x)$\\
      for $i \in X$ do\\
        $\tab$if $M[i] = 0$ then return 0\\
      return 1
  }
  {
    \underline{$\Up^R_K(\langle M, \salt, c \rangle, \up_{x,b})$}\\[2pt]
      if $c \geq n$ then return $\bot$\\
      $M' \gets M$;
      $X \gets R_K(\salt \cat x)$\\
      for $i$ in $X$ do\\
      $\tab$ if $M'[i] = 0$ and $b < 0$ then return $\bot$\\
      $\tab M'[i] \gets M'[i] + b$\\
      $M \gets M'$;
      return $\langle M, \salt, c+b \rangle$
  }
  \caption{Keyed structure $\countbloom[R,n,\lambda]$ given by
  $(\Rep^R,\Qry^R,\Up^R)$ is used to define Bloom filter variants used to
  rerpresent sets of at most~$n$ elements. The parameters are a function $R:
  \keys\by\bits^* \to [m]^k$ and integers $n, \lambda \geq0$. A concrete scheme
  is given by a particular choice of parameters.}
  \label{fig:cbf-def}
\end{figure}

Counting filters are a modified version of Bloom filters which are designed to
allow for deletion as well as insertion. Unlike the count min-sketch structure,
counting filters are designed to handle set membership queries rather than
frequency queries. Despite this, the two structures are closely related in terms
of security properties.

\heading{Error function for frequency queries}
%
Unlike with a Bloom filter or count min-sketch, counting filters must account
for two different types of error: false positives and false negatives. To be as
general as possible, we define a parametrized error function~$\delta$ for
$\delta^+, \delta^- \in \R$ as
\begin{equation}
  \delta(x, y) =
  \begin{cases}
    0 & \text{if}\ x = y \\
    \delta^+ & \text{if}\ x = 1, y = 0 \\
    \delta^- & \text{if}\ x = 0, y = 1
  \end{cases}
\end{equation}
This means that false positives are given a weight of $\delta^+$ while false
negatives are given a weight of $\delta^-$, and correct responses are given a
weight of 0.

\subsection{Insecurity of Public Counting Filters}
Any counting filter construction necessarily fails to satisfy \errep\
correctness for the same reasons as in the case of count min-sketch. In
particular, the adversary can call $\REPO(\emptyset)$ to receive an empty
representation, insert an element $x$, observe which counters are incremented by
this insertion, and then delete $x$. By doing this repeatedly, the adversary can
gain information about which elements overlap with which combinations of other
elements, and can therefore mount the same target-set coverage attack as in the
count min-sketch case. In fact, the attack is slightly easier to pull off, since
once a good test set is found the adversary need only insert each element of the
set once in order to cause the elements of the target set to become false
positives.

\subsection{Private Thresholded Counting Filters}

\begin{theorem}
\end{theorem}
The proof is somewhat similar to that of Theorem~\ref{thm:scms-erreps-tex},
but a major difference comes from the difference in $\Qry$. In particular,
a string which is an element of the underlying data object cannot
produce an error through being overestimated, but it can produce an error
through being \emph{underestimated} if the adversary causes false negatives.
\begin{proof}
  \begin{figure*}
\twoCols{0.47}
{
  \vspace{-7pt}
  \experimentv{$\game_{0}(\advA)$}\hfill\diffplus{$\game_1$}\\[2pt]
    $M^* \gets \bot$;
    $\setS \gets \emptyset$;
    $\salt^* \getsr \bits^\lambda$\\
    $\advB^{\REPO,\QRYO,\UPO,\HASHO_1}$;
    return $\big[\sum_x \err[x] \geq r\big]$
  \\[6pt]
  \oraclev{$\HASHO_c(\salt \cat x)$}\\[2pt]
    $\vv \getsr [m]^k$\\
    if $\salt=\salt^*$ and $c = 1$ then \com{Caller is~$\advB$}\\
    \tab $\bad_1 \gets 1$; \diffplus{return $\vv$}\\
    if $T[Z,x] = \bot$ then $\vv \gets T[Z,x]$\\
    $T[Z,x] \gets \vv$; return $\vv$
  \\[6pt]
  \oraclev{$\QRYO(\qry_x)$}\\[2pt]
    $X \gets \HASHO_3(\salt^* \cat x)$;
    $a \gets \infty$;
    $\setS \gets \setS \cup \{x\}$\\
    for $i$ in $[1..k]$ do\\
      $\tab a \gets \min(a, M[i][X[i]])$\\
    if $\err[x] < \delta(a,\qry_x(\col^*))$ then
          $\err[x] \gets \delta(a,\qry_x(\col^*))\diffplus{+k}$\\
    return $a$
  \\[6pt]
  \oraclev{$\REPO(\col)$}\\[2pt]
    for $i$ in $[1..k]$ do\\
      $\tab M^*[i] \gets 0^m$\\
    $\setS^* \gets \col$\\
    for $x \in \col$ do\\
    $\tab\UPO(\up_x)$\\
    return $\top$
}
{
  \vspace{-7pt}
  \oraclev{$\UPO(\up_{x,b})$}\\[2pt]
    if $w'(M^*) > \ell$ then return $\top$\\
    $M' \gets M^*$\\
    for $i$ in $[1..k]$ do\\
      $\tab$ if $M'[i][X[i]] = 0$ and $b < 0$ then return $\top$\\
      $\tab M'[i][X[i]] \gets M'[i][X[i]] + b$\\
    $M^* \gets M'$\\
    if $\err[x] \neq \bot$ then\\
      $\tab a \gets \QRYO(\qry_x)$\\
      $\tab\err[x] \gets \min(\delta(a,\qry_x(\col^*)),err[x])$\\
    $\setS^* \gets \up_{x,b}(\setS^*)$;
    return $\top$
  \vspace{6pt}\hrule\vspace{3pt}
  \oraclev{$\REPO(\col)$}\hfill\diffplus{$\game_3$}\\[2pt]
    for $i$ in $[1..k]$ do\\
      $\tab M^*[i] \gets 0^m$\\
    $\setS^* \gets \col$\\
    for $x \in \col$ do\\
    $\tab\UPO(\up_x)$\\
    \diffplusbox{for $i$ in $[1..k]$ do\\
      $\tab$while $w(M[i]) < \ell+1$ do\\
        $\tab\tab j \getsr [m]$;
        $M[i][j] \gets 1$}
    return $\top$
}
\caption{Games 0--3 for proof of Theorem~\ref{thm:scms-erreps-th}.}
\label{fig:sbf-erreps/games}
\end{figure*}

As with the proof of Theorem~\ref{thm:sbf-errep-immutable}, we derive a bound in
the \erreps1 case and then use Lemma~\ref{thm:lemma1} to move from \erreps1 to
the more general \erreps case. Because we are in the \erreps1 case, we may
assume without loss of generality that the adversary does not call $\REVO$,
since revealing the only representation automatically prevents the adversary
from winning.

We begin with a game~$\game_0$ which has identical behavior to the \erreps1
experiment for a counting filter. As in the proof of
Theorem~\ref{thm:sbf-errep-immutable}, we have a
$\bad_1$ flag that gets set if the adversary ever calls $\HASHO_1$ with the
actual salt used by the representation. By a very similar argument, we can
move to~$\game_1$, where the behavior is different only when the $\bad_1$ flag
is set, with a bound of
\begin{equation}
  \Prob{\game_0(\advA)=1} \leq
    q_H/2^\lambda + \Prob{\game_1(\advA)=1} \,.
\end{equation}

Unlike in the case of a count min-sketch, it is entirely possible for deletions
to benefit the adversary in this game. In particular, if $x$ is found to be a
false positive, deleting $x$ may cause up to $k$ elements of $\col$ to become
false negatives. We therefore move to a game~$\game_2$ where the adversary gets
credit for either a single false positive or for $k$ false negatives whenever it
finds a false positive, but where the adversary cannot delete any false
positives that it finds. We let $r' = \lfloor r/\max(\delta^+,k\delta^-)\rfloor$
represent the number of false positives the adversary has to find in~$\game_2$
in order to win. In order to prevent the adversary from being penalized by the
filter becoming full too early, we also raise the thresold from $\ell$ to
$\ell+r'$ in~$\game_2$. Now for any $\advA$ for~$\game_1$, we can construct
$\advB$ for~$\game_2$ that simulates $\advA$, keeping track of all query
responses and forwarding all oracle queries in the natural way, except that
calls to delete false positives are ignored. Since $\UPO$ never fails for
$\advB$ due to the increased threshold, and since $\advB$ gets automatic credit
for any false negatives that might have been caused by deleting false positives,
$\advB$ succeeds whenever $\advA$ does, i.e.
$\Prob{\game_1(\advA)=1} \le \Prob{\game_2(\advB) = 1}$.

Since the remaining deletions do not cause errors, we can use the same argument
as in the proof of Theorem~\ref{thm:scms-erreps-th} to reduce from $\advB$ to an
adversary $\advC$ which does not make deletions at all. In~$\game_3$, we further
reduce from a counting filter to a normal Bloom filter by capping each of the
counters in the filter at 1. Since no deletions are performed, a counter
in~$\game_3(\advC)$ is nonzero if and only if the same counter
in~$\game_2(\advC)$ is nonzero. So $\QRYO$ behaves the same in~$\game_3$ as it
did in~$\game_2$, and $\Prob{\game_2(\advC)=1} \le \Prob{\game_3(\advC) = 1}$.

Note that~$\game_3$ is actually simulating an ordinary Bloom filter, since all
`counters' in the filter are restricted to the range $\bits$, there are no
deletions, and any insertions just set the corresponding bits to 1. In fact,
this game is identical to~$\game_2$ in the proof of
Theorem~\ref{thm:sbf-erreps-th} except that the adversary need only accumulate
$r'$ errors instead of $r$ errors and the threshold is $\ell+r'$ instead of
$\ell$. An identical argument allows us to reach the binomial bound of
\begin{equation}
   \Prob{\game_3(\advC)=1} \le
     \sum_{i=r'}^{q_T} \binom{q_T}{i}p_\ell^i(1-p_\ell)^{q_T-i} \,,
\end{equation}
where $p_\ell$ is now defined to be $((\ell+k+r')/m)^k$. Then the standard
Chernoff bound, along with Lemma~\ref{thm:lemma1}, yields the final bound of
\begin{equation}
   \Adv{\erreps}_{\Pi,\delta,r}(\advA) \leq
     q_R \cdot \left[\frac{q_H}{2^\lambda} + e^{r'-p_\ell q_T}\left(\frac{p_\ell q_T}{r'}\right)^{r'}\right]s.
\end{equation}
\end{proof}
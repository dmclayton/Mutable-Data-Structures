Data structures that provide an approximate representation of their inputs, such
as Bloom filters or count min-sketches, are relied upon in a wide variety of
systems. A certain amount of error is usually tolerable, and the designers of
these structures often prove rigorous bounds on how much error can be expected,
so long as the inputs are chosen independently of the randomness used to
construct it. There are any number of settings in which this assumption can
be violated by an adversary attacking the system.
%
This work provides a provable security treatment of probabilistic data
structures in adversarial environments. Our syntax is applicable to a wide
variety of structures, and our security notions capture the performance of a
structure under attack in a variety of settings.
%
We apply our definitional viewpoint to the study of a variety of widely used
data structures that aim to provide a compact, but approximate representation.
We demonstrate that, when carefully specified, such structures can provide good
performance in the presence of adversaries.


\ignore{
  This work initiates the study of abstract data structures from a cryptographic
  perspective.  We first establish a precise syntax that captures a broad
  class of real-world data structures.  We then treat the
  \emph{correctness} and \emph{privacy} of data structures
  as security properties, and establish formal security notions for
  each.  Loosely, our notion of correctness captures an (adaptive) adversary's ability to cause
  a data structure to err in the course of responding to a set of supported
  queries, and our two privacy notions neatly capture what a data structure leaks about
  the data it represents.

  We use our formalisms to explore the security of the widely used
  Bloom filter~\cite{bloom1970space} and some important variants.
  %
  We find, for example, that the security of Bloom filters depends
  crucially on whether or not the underlying hash functions are known
  by the adversary prior to the filter being constructed.
  %
  We also study a real-world mechanism for privacy-preserving record
  linkage (over hospital databases).  Our notions provide a crisp view of the
  (in)security of this Bloom-filter-based mechanism.
  %
  To demonstrate the broader applicability of our definitions, we move
  from data structures supporting set-membership queries (only) to
  dictionary data structures.  Concretely, we analyze the ``Bloomier
  filter''~\cite{chazelle2004bloomier}, which provides a compact
  representation of a key/value store.
}
